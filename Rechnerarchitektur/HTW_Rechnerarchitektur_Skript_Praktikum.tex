\part{Praktikum}
\setCustomSlidePath{Praktikum/RA_043_P00D}

\chapter{Einführung}
\section{Mikrocontroller}
\slideScale{2}
\subsection{Unterscheidung und Architektur}
\slideScale{3}

\unimptnt{
\subsection{Taktung}
\slideScale{4}
\subsection{Typische Peripherie}
\slideScale{5}
}
\subsection{Einsatzbeispiele}
\slideScale{6}
\subsection*{Computerklassen/-typen}
\begin{itemize}
\item Wegwerfcomputer
\item Microcontroller, embedded Computer
\item Mobile Computer
\item Personalcomputer
\item Server
\item Großrechner
\end{itemize}

\subsection{Speicherbedarf}
\slideScale{7}
\slideScale{8}

\unimptnt{
\chapter{Fallbeispiel Atmel AVR}
\section{AVR und AVR-Architektur im Überblick}
\subsection{Allgemeines}
\slideScale{9}
\subsection{CPU und Archtitektur}
\slideScale{10}
\slideScale{11}
\slideScale{12}
\subsection{CPU zu Statusregister SREG}
\slideScale{13}
\subsection{Interne Perepherie}
\slideScale{14}

\section{Speichertypen}
\subsection{Speicher}
\slideScale{15}
\slideScale{16}
\subsection{Stack}
\slideScale{17}

\section{Beispielcontroller}
\subsection{Typenbeispiele}
\slideScale{18}
\subsection{Blockschaltbild und Besonderheiten}
\slideScale{19}
\subsection{Gehäuse und Anschlüsse}
\slideScale{20}

\section{Entwicklungswerkzeuge}
\subsection{Demoplattform AVR Butterfly}
\slideScale{21}
\subsection{Hardware}
\slideScale{22}
\subsubsection{Experimentalplattform STK500/501}
\slideScale{23}
\slideScale{24}
\subsection{Software}
\slideScale{25}

\section{Baugruppen und deren Nutzung und Programmierung}
\subsection{System- und Takt-Anschlüsse}
\slideScale{26}
\subsection{Bidirektionale Ports}
\slideScale{27}
\slideScale{28}
\slideScale{29}
\slideScale{30}
uint8\_t für 8-Bit unsigned Integer - gut, da AVR 8-Bit ist.
\slideScale{31}
\slideScale{32}
\slideScale{33}
}
Wichtig: Bit-Tricks !!\\
1<<3 : 3. Bit wird auf 1 gesetzt; |= (bitweises Oder) belässt alle Bits gleich, bis auf das 3. (in dem Bsp.), das wird 1.\\
\textasciitilde() (bitweise Negation); \&= belässt alle Bits gleich, bis auf das 3., das wird 1.\\
\^{}= (exklusives Oder) fürs Invertieren
\unimptnt{
\slideScale{34}
\slideScale{35}
\slideScale{36}
\slideScale{37}
\slideScale{38}

\subsection{Alternative Nutzung der Universalports}
\slideScale{39}
\slideScale{40}
\slideScale{41}
\slideScale{42}

\subsection{Interrupts}
\subsubsection{Grundprinzip}
\slideScale{43}

\subsubsection{Interruptquellen}
\slideScale{44}
\subsubsection*{Beispiel ATmega128}
\slideScale{45}
\slideScale{46}
\subsubsection{Anforderungen an Interrupt-Routinen}
\slideScale{47}
Interrupt in C: mit globalen Variablen
\subsubsection{Beispiel für Interruptnutzung}
\slideScale{48}
Hinweis Atmel-Board: 0 ist "1" und 1 ist "0" (im Sinne, dass bei 0 die Lampe leuchtet)
\slideScale{49}
\slideScale{50}
\slideScale{51}

\subsection{Timer/Counter}
\slideScale{52}
\subsubsection{Prescaler (Vorteiler)}
\slideScale{53}
\subsubsection{Betriebsmodi}
\slideScale{54}

\subsubsection{Normal Mode}
\slideScale{55}
\subsubsection{CTC-Mode}
\slideScale{56}
\subsubsection{PWM-Modus}
\slideScale{57}
\subsubsection{Beispiel 1 für Timer/Counter-Nutzung}
\slideScale{58}
\subsubsection{Beispiel 2 für Timer/Counter-Nutzung}
\slideScale{59}
\slideScale{60}
\slideScale{61}
\slideScale{62}
\slideScale{63}
}