
\unimptnt{
%\slides{01}{2}
\section*{Streiflicht: Konrad Zuse}
\slides{01}{3}
\slides{01}{4}
\section*{Streiflicht: John von Neumann}
\slides{01}{5}

\section{Grundidee von von Neumann}
\slides{01}{6}
}
\section{Grundarchitekturen von Rechnern/Computern}
\slides{01}{7}
Havard-Architektur: Hardware-Aufwand größer durch mehr Busse

\section{Von-Neumann-Architektur}
\slides{01}{8}
Im Speicher sind auf den ersten Blick von außen Daten und Befehle nicht voneinander zu unterschieden.
\slides{01}{9}

\subsection{Kurzfassung Von-Neumann-Rechner Prinzip}
\slides{01}{10}
\paragraph{Blockschaltbild}
\slides{01}{11}
% Folie 12: Bild Wikipedia

\section{Befehlssequenz}
\slides{01}{13}
Für Sprungbefehle werden für performante Ausführung Sprungvorhersagen benötigt.

\section{Operandenadressierung}
\slides{01}{14}

\section{Architekturverfeinerung 1: Befehlszähler}
\slides{01}{15}
\unimptnt{
$\Rightarrow$ Von-Neumann Rechner mit Befehlszähler
\slides{01}{16}
}

\section{Klassischer Universalspeicher}
\subsection{Speicherwerk}
\slides{01}{17}
\subsection{Rechenwerk}
\slides{01}{18}
\unimptnt{
\subsubsection{Blockschaltbild einer ALU}
\slides{01}{19}
}
\subsubsection{Flags/Bits des Statusregisters}
\slides{01}{20}
$\mu P$: Mikroprozessor
\unimptnt{
\subsubsection*{Bsp.: Erzeugung des Bedingungsbits}
\slides{01}{21}
}
\subsection{Steuerwerk}
\slides{01}{22}
\subsection{CPU}
\slides{01}{23}
\subsection{E/A-Werk und Bus}
\slides{01}{24}

\section{Architekturverfeinerung 2: Einführung von Registern}
\slides{01}{25}
\unimptnt{
\subsection{Von-Neumann-Rechner mit Register-Files}
\slides{01}{26}
}

\section{Architekturverfeinerung 3: Erweiterung um relative Adressierung}
\slides{01}{27}

\section{Universalrechner: Instruktionstypen / Befehlssatz}
\slides{01}{29}

\paragraph{Beispielrechner: einfacher Von-Neumann-Beispielrechner}
\slides{01}{28}

\subsection{Instruktionssatz}
\slides{01}{30}

\subsection{Befehlszyklus}
\slides{01}{31}

\unimptnt{
\subsubsection*{Bsp.: Interpretationszyklus des Beispielrechners}
\slides{01}{32}
}

\section{Abweichungen moderner Rechner von den Von-Neumann-Prinzipien}
\slides{01}{33}