% Header aus der Vorlage
\documentclass[a4paper,11pt, footheight=26pt
%,twoside
]{scrreprt}
\usepackage[head=23pt]{geometry}	% head=23pt umgeht Fehlerwarnung, dafür größeres "top" in geometry
\geometry{a4paper, top=30mm, bottom=22mm,headsep=10mm, footskip=12mm
, left=20mm, right=20mm
%, inner=27mm, outer=13mm
}

\setcounter{secnumdepth}{3}	% zählt auch subsubsection
\setcounter{tocdepth}{3}	% Inhaltsverzeichnis bis in subsubsection

% Zeile 2 (,twoside) und 7 (inner=...) für eine Druckversion (doppelseitig) ent-kommentieren (Rand für Hefter)

% Input inkl. Umlaute, Silbentrennung
\usepackage[T1]{fontenc}
\usepackage[utf8]{inputenc}
\usepackage[ngerman]{babel}
\usepackage{csquotes}	% Anführungszeichen
\usepackage{eurosym}

% HTW Corporate Design: Arial (Helvetica)
\usepackage{helvet}
\renewcommand{\familydefault}{\sfdefault}

% Style-Aufhübschung
\usepackage{soul, color}	% Kapitälchen, Unterstrichen, Durchgestrichen usw. im Text
\usepackage{scrlayer-scrpage}	% Kopf-/Fußzeile
%\usepackage{titleref}
\usepackage[perpage]{footmisc}	% Fußnotenzählung Seitenweit, nicht Dokumentenweit
\renewcommand*{\thefootnote}{\fnsymbol{footnote}}	% Fußnoten-Symbole anstatt Zahlen
\renewcommand*{\titlepagestyle}{empty} % Keine Seitennummer auf Titelseite

% Mathe usw.
\usepackage{amssymb}
\usepackage[fleqn]{amsmath}	% fleqn: align-Umgebung rechtsbündig
\usepackage{xcolor}
\usepackage{esint}	% Schönere Integrale, \oiint vorhanden
\everymath=\expandafter{\the\everymath\displaystyle}	% Mathe Inhalte werden weniger verkleinert
\usepackage{wasysym}	% mehr Symbole, bspw \lightning
% Auch arcus-Hyperbolicus-Funktionen
\DeclareMathOperator{\arccot}{arccot}
\DeclareMathOperator{\arccosh}{arccosh}
\DeclareMathOperator{\arcsinh}{arcsinh}
\DeclareMathOperator{\arctanh}{arctanh}
\DeclareMathOperator{\arccoth}{arccoth} 
% Mathe in Anführungszeichen:
\newsavebox{\mathbox}\newsavebox{\mathquote}
\makeatletter
\newcommand{\mq}[1]{% \mathquotes{<stuff>}
  \savebox{\mathquote}{\text{"}}% Save quotes
  \savebox{\mathbox}{$\displaystyle #1$}% Save <stuff>
  \raisebox{\dimexpr\ht\mathbox-\ht\mathquote\relax}{"}#1\raisebox{\dimexpr\ht\mathbox-\ht\mathquote\relax}{''}
}
\makeatother

% tikz usw.
\usepackage{tikz}
\usepackage{pgfplots}
\pgfplotsset{compat=1.11}	% Umgeht Fehlermeldung
\usetikzlibrary{graphs}
%\usetikzlibrary{through}	% ???
\usetikzlibrary{arrows}
\usetikzlibrary{arrows.meta}	% Pfeile verändern / vergrößern: \draw[-{>[scale=1.5]}] (-3,5) -> (-3,3);
\usetikzlibrary{automata,positioning} % Zeilenumbruch im Node node[align=center] {Text\\nächste Zeile} automata für Graphen
\usetikzlibrary{matrix}
\usetikzlibrary{patterns}	% Schraffierte Füllung
\tikzstyle{reverseclip}=[insert path={	% Inverser Clip \clip
	(current page.north east) --
	(current page.south east) --
	(current page.south west) --
	(current page.north west) --
	(current page.north east)}
% Nutzen: 
%\begin{tikzpicture}[remember picture]
%\begin{scope}
%\begin{pgfinterruptboundingbox}
%\draw [clip] DIE FLÄCHE, IN DER OBJEKT NICHT ERSCHEINEN SOLL [reverseclip];
%\end{pgfinterruptboundingbox}
%\draw DAS OBJEKT;
%\end{scope}
%\end{tikzpicture}
]	% Achtung: dafür muss doppelt kompliert werden!
\usepackage{graphpap}	% Grid für Graphen
\tikzset{every state/.style={inner sep=2pt, minimum size=2em}}

% Tabular
\usepackage{longtable}	% Große Tabellen über mehrere Seiten
\usepackage{multirow}	% Multirow/-column: \multirow{2[Anzahl der Zeilen]}{*[Format]}{Test[Inhalt]} oder \multicolumn{7[Anzahl der Reihen]}{|c|[Format]}{Test2[Inhalt]}
\renewcommand{\arraystretch}{1.3} % Tabellenlinien nicht zu dicht
\usepackage{colortbl}
\arrayrulecolor{gray}	% heller Tabellenlinien

% Nützliches
\usepackage{verbatim}	% u.a. zum auskommentieren via \begin{comment} \end{comment}
\usepackage{tabto}	% Tabs: /tab zum nächsten Tab oder /tabto{.5 \CurrentLineWidth} zur Stelle in der Linie
\NumTabs{6}	% Anzahl von Tabs pro Zeile zum springen
\usepackage{listings} % Source-Code mit Tabs
\usepackage{lstautogobble} 
\usepackage{enumitem}	% Anpassung der enumerates
\setlist[enumerate,1]{label=\arabic*.)}	% global andere Enum-Items
\renewcommand{\labelitemiii}{$\scriptscriptstyle ^\blacklozenge$} % global andere 3. Item-Aufzählungszeichen
\usepackage{letltxmacro} % neue Definiton von Grundbefehlen
% Nutzen:
%\LetLtxMacro{\oldemph}{\emph}
%\renewcommand{\emph}[1]{\oldemph{#1}}

% Einrichtung von lst
\lstset{
basicstyle=\ttfamily, 
mathescape=true, 
%escapeinside=^^, 
autogobble, 
tabsize=2,
basicstyle=\footnotesize\sffamily\color{black},
frame=single,
rulecolor=\color{lightgray},
numbers=left,
numbersep=5pt,
numberstyle=\tiny\color{gray},
commentstyle=\color{gray},
keywordstyle=\color{green},
stringstyle=\color{orange},
morecomment=[l][\color{magenta}]{\#}
%showspaces=false,
showstringspaces=false,
breaklines=true,
literate=%
    {Ö}{{\"O}}1
    {Ä}{{\"A}}1
    {Ü}{{\"U}}1
    {ß}{{\ss}}1
    {ü}{{\"u}}1
    {ä}{{\"a}}1
    {ö}{{\"o}}1
    {~}{{\textasciitilde}}1
}
\usepackage{scrhack} % Fehler umgehen
\def\ContinueLineNumber{\lstset{firstnumber=last}} % vor lstlisting. Zum wechsel zum nicht-kontinuierlichen muss wieder \StartLineAt1 eingegeben werden
\def\StartLineAt#1{\lstset{firstnumber=#1}} % vor lstlisting \StartLineAt30 eingeben, um bei Zeile 30 zu starten
\let\numberLineAt\StartLineAt

% BibTeX
\usepackage[backend=bibtex, bibencoding=ascii]{biblatex}	% BibTeX
\usepackage{makeidx}
%\makeglossary
%\makeindex

% Grafiken
\usepackage{graphicx}
\usepackage{epstopdf}	% eps-Vektorgrafiken einfügen

% pdf-Setup
\usepackage{pdfpages}
\usepackage[bookmarks,%
bookmarksopen=false,% Klappt die Bookmarks in Acrobat aus
colorlinks=true,%
linkcolor=black,%
citecolor=red,%
urlcolor=green,%
]{hyperref}

% Titel, Autor usw. werden vor dem Anfang des Dokuments in einem Rutsch definiert…
\newcommand{\DTitel}[1]{\newcommand{\Dokumententitel}{#1}}
\newcommand{\DUntertitel}[1]{\newcommand{\Dokumentenuntertitel}{#1}}
\newcommand{\DAutor}[1]{\newcommand{\Dokumentenautor}{#1}}
\newcommand{\DNotiz}[1]{\newcommand{\Dokumentennotiz}{#1}}
\newcommand{\DSign}[1]{\newcommand{\Dokumentensignatur}{#1}}
\DSign{\footnotesize{\textcolor{darkgray}{Mitschrift von\\ \Dokumentenautor}}}
\newcommand{\Autorformat}[1]{\textcolor{darkgray}{Mitschrift von #1}}
% … Deswegen folgendes erst Nach Dokumentenbeginn ausführen:
\AtBeginDocument{
	\hypersetup{
		pdfauthor={\Dokumentenautor},
		pdftitle={HTW Dresden | \Dokumententitel - \Dokumentenuntertitel},
	}
	\automark[section]{section}
	\automark*[subsection]{subsection}
	\pagestyle{scrheadings}
	\ihead{\includegraphics[height=1.7em]{\workingdir LaTeX_master/HTW-Logo.eps}}
	\ohead{\Dokumententitel}
	\cfoot{\pagemark}
	\ofoot{\Dokumentensignatur}
	% Titelseite
	\title{\includegraphics[width=0.35\textwidth]{\workingdir LaTeX_master/HTW-Logo.eps}\\\vspace{0.5em}
	\Huge\textbf{\Dokumententitel} \\\vspace*{0,5cm}
	\Large \Dokumentenuntertitel \\\vspace*{4cm}}
	\author{\Autorformat{\Dokumentenautor} \vspace*{1cm}\\\Dokumentennotiz}
}

%% EIGENE BEFEHLE

% Arbeitsordner (in Abhängigkeit vom Master)
\newcommand{\workingdir}{../}

% Farbdefinitionen
\definecolor{red}{RGB}{180,0,0}
\definecolor{green}{RGB}{75,160,0}
\definecolor{blue}{RGB}{0,75,200}
\definecolor{orange}{RGB}{255,128,0}
\definecolor{yellow}{RGB}{255,245,0}
\definecolor{purple}{RGB}{75,0,160}
\definecolor{cyan}{RGB}{0,160,160}
\definecolor{brown}{RGB}{120,60,10}

\definecolor{itteny}{RGB}{244,229,0}
\definecolor{ittenyo}{RGB}{253,198,11}
\definecolor{itteno}{RGB}{241,142,28}
\definecolor{ittenor}{RGB}{234,98,31}
\definecolor{ittenr}{RGB}{227,35,34}
\definecolor{ittenrp}{RGB}{196,3,125}
\definecolor{ittenp}{RGB}{109,57,139}
\definecolor{ittenpb}{RGB}{68,78,153}
\definecolor{ittenb}{RGB}{42,113,176}
\definecolor{ittenbg}{RGB}{6,150,187}
\definecolor{itteng}{RGB}{0,142,91}
\definecolor{ittengy}{RGB}{140,187,38}

% Textfarbe ändern
\newcommand{\tred}[1]{\textcolor{red}{#1}}
\newcommand{\tgreen}[1]{\textcolor{green}{#1}}
\newcommand{\tblue}[1]{\textcolor{blue}{#1}}
\newcommand{\torange}[1]{\textcolor{orange}{#1}}
\newcommand{\tyellow}[1]{\textcolor{yellow}{#1}}
\newcommand{\tpurple}[1]{\textcolor{purple}{#1}}
\newcommand{\tcyan}[1]{\textcolor{cyan}{#1}}
\newcommand{\tbrown}[1]{\textcolor{brown}{#1}}

% Umstellen der Tabellen Definition
\newcommand{\mpb}[1][.3]{\begin{minipage}{#1\textwidth}\vspace*{3pt}}
\newcommand{\mpe}{\vspace*{3pt}\end{minipage}}

\newcommand{\resultul}[1]{\underline{\underline{#1}}}
\newcommand{\parskp}{$ $\\}	% new line after paragraph
\newcommand{\corr}{\;\widehat{=}\;}
\newcommand{\mdeg}{^{\circ}}

\newcommand{\nok}[2]{\begin{pmatrix}#1\\#2\end{pmatrix}}	% n über k
\newcommand{\mtr}[1]{\begin{pmatrix}#1\end{pmatrix}}	% Matrix
\newcommand{\dtr}[1]{\begin{vmatrix}#1\end{vmatrix}}	% Determinante (Betragsmatrix)
\renewcommand{\vec}[1]{\underline{#1}}	% Vektorschreibweise
\newcommand{\imptnt}[1]{\colorbox{red!30}{#1}}	% Wichtiges

% Definition von Titel, Autor usw.
\DTitel{Mathematik I}
\DUntertitel{Vorlesungsskript}
\DAutor{Falk-Jonatan Strube}
\DNotiz{Vorlesung von Herrn Michael Meinhold\\ \& Prof. Dr. Fabian Schwarzenberger}

\begin{document}

\maketitle
\newpage
\tableofcontents
\newpage

\chapter{Elementare Grundlagen}

\section{Aussagen und Grundzüge der Logik}
%\input{HTW_Mathe1_Skript_sec1_1}

\section{Mengen}\label{sec:Mengen}
%\input{HTW_Mathe1_Skript_sec1_2}

\section{Zahlen}
%\input{HTW_Mathe1_Skript_sec1_3}

\section{Reellwertige Funktionen einer reellen Veränderlichen}
%\subsection{Elementare Funktionen (Teil 1)} \label{1.4.1}
\subsubsection{Polynome}
\paragraph{Def. 1:} \parskp
$y=f(x)=a_n\; x^n+a_{n-1}\;x^{n-1}+...+a_2\; x^2+a_1\; x+a_0$ mit $(a_0, ..., a_n \in \mathbb{R}, x\in \mathbb{R})$ heißt ganze rationale Funktion oder \emph{Polynom} vom Grad $n$ (falls $a_n \not = 0$).\\
Zur Beschreibung der Funktionswerte zweckmäßig: HORNER-Schema\\
(vgl. Stellenwertsysteme  \ref{Stellenwertsysteme})\\
\begin{tabular}{r | c c c c c c}
 & $a_n$ & $a_{n-1}$ & $ a_{n-2}$ & ... & $a_1$ & $a_0$\\
$x_0$ &  & $b_{n-1}\cdot x_0$ & $b_{n-2}\cdot x_0$ & ... & $b_1 \cdot x_0$ & $b_0\cdot x_0$ \\
\hline
 & $\underbrace{\boxed{b_{n-1}}}_{=a_n}$ & $\boxed{b_{n-2}}$ & $\boxed{b_{n-3}}$ & ... & $\boxed{b_0}$ & $f(x_0)=r_0$\\
\end{tabular}\\
Polynomdivision: $\frac{f(x)}{x-x_0}=b_{n-1}\;x^{n-1}+b_{n-2}\;x^{n-2}+...+b_1\; x + b_0 + \frac{r_0}{x-x_0}$

\subparagraph{Bsp. 1:} \parskp
$f(x)=x^5-2x^3+x^2-6$, \quad $x_0=3$, \quad ges: $f(x_0)$, \quad$\frac{f(x)}{x-x_0}$\\
\begin{tabular}{r| c c c c c c}
 & $1$ & $0$ & $-2$ & $1$ & $0$ & $-6$\\
$x_0=3$ &  & $3$ & $9$ & $21$ & $66$ & $198$ \\
\hline
 & $1$ & $3$ & $7$ & $22$ & $66$ & $192$\\
\end{tabular}\\
$f(x):(x-3)=x^4+3 x^3 + 7x^2+22x + 66 + \frac{192}{x-3}$

\paragraph{Satz 1:} \parskp
Es sei $f(x)=p_n(x) = a_n \; x^n + ... + a_0$ ein Polynom vom Grad $n$ (d.h. $a_n\not = 0$). Dann besitzt $f$ (in $\mathbb{C}$) genau $n$ Nullstellen $x_1,...,x_n$ und es gilt: $f(x) = a_n (x-x_1)\cdot (x-x_2)\cdot ... \cdot (x-x_n)$. (Zerlegung in Linearfaktoren)

\subparagraph{Diskussion:}
\begin{enumerate}
\item Falls in der Linearfaktorzerlegung der Faktor $(x-x_0)$ genau $k$-mal ($1\geq k \geq n$) vorkommt, so heißt $x_0$ \emph{$k$-fache Nullstelle} (Nullstelle der Vielfachheit $k$).
\item Nichtreelle Nullstellen sind möglich, sie treten stets paarweise als konjugiert komplexe Zahlen auf ($x_0, \overline{x_0})$). In diesem Falle Zusammenfassung der Linearfaktoren zu reellen quadratischen Faktoren möglich: $(x-x_0)\cdot (x-\overline{x_0})=x^2-(x_0 + \overline{x_0})x + x_0 \cdot \overline{x_0}=x^2-2\cdot Re(x_0) \cdot x + |x_0|^2$
\item Falls $a_0, a_1, ... , a_n$ ganze Zahlen sind, dann sind ganzzahlige Nullstellen Teiler von $a_0$ (falls vorhanden).
\item Allgemeine Methoden zur Nullstellenbestimmung später (Kap. 3 \ref{sec:3})
\end{enumerate}

\subparagraph{Bsp. 2:} \parskp
$p(x) = x^4 + x^3 -5x^2 + x-6$, \quad ges: Nullstellen\\
Durch Probieren $\boxed{x_1=2}$\\
mit HORNER-Schema:\\
\begin{tabular}{r | c c c c c l}
 & 1 & 1 & -5 & 1  & -6 & \\
$x_1=2$ & & 2 & 6 & 2 & 6 &\\
\hline
 & 1 & 3& 1 & 3 &$\boxed{0}$ & $\curvearrowright p(x) = (x-2)\cdot \underbrace{(x^3+3x^2+x+3)}_{\text{durch Probieren }x_2=-3}$\\
$x_2=-3$ & & -3 & 0 & -3 & & \\
\hline 
 & 1 & 0 & 1 &\boxed{0}& & $\curvearrowright p(x) = (x-2)(x+3)\underbrace{(x^2+1)}_{x_{3,4}=\pm i}$\\ 
\end{tabular}\\
$\curvearrowright$ Zerlegung: $p(x)= (x-2)(x+3)(x-i)(x+i)$

\subsubsection{Gebrochen rationale Funktionen}

\paragraph{Def. 2:} \parskp
$y=f(x) = \frac{p(x)}{q(x)}=\frac{a_m\; x^m + a_{m-1}\; x^{m-1}+...+a_1 \; x+a_0}{b_n \; x^n+b_{n-1}\; x^{n-1} + ... + b_1 \; x + b_0}$ mit $a_m \not = 0$, $b_n \not = 0$ und $ Db(f)=\{x\in \mathbb{R}| q(x)\not = 0\}$ heißt gebrochenrationale Funktion. f heißt \emph{echt gebrochen}, falls $m<n$ und \emph{unechtgebrochen}, falls $m\geq n$.

\subparagraph{Diskussion:}
\begin{itemize}
\item Wir nehmen ohne Beschränkung der Allgemeinheit an, dass Zähler- und Nennerpolynom keine gemeinsamen Nullstellen besitzen (ansonsten: Kürzen gemeinsamer Linearfaktoren von Zähler und Nenner [unter Beachtung des Definitionsbereichs])
\item Die Nullstellen des Nennerpolynoms heißen \emph{Polstellen} der gebrochen rationalen Funktion (bei Polstelle $x_P$: $|f(x)|\rightarrow \infty$ für $x\rightarrow x_P$).
\item Die Nullstellen des Zählerpolynoms sind die Nullstellen von $f$.
\item Verhalten von $f(x)$ bei $k$-facher reeller Nullstelle oder Polstelle:\\
\fbox{Vorzeichenwechsel $\Leftrightarrow$ $k$ ungerade}
\item Polynomdivision $p(x): q(x) = \underbrace{s(x)}_{Polynom}+\underbrace{\frac{r(x)}{q(x)}}_{echt gebrochen}$\\
$y=a(x)$ ist die sogenannte Asymptote
\end{itemize}
\subparagraph{Bsp. 3:}\parskp
$y=\frac{x^3+2x^2}{x^2-x-2}=\frac{x^2(x+2)}{(x+1)(x-2)}=x+3+\frac{5x+6}{x^2-x-2}$\\
Daraus lassen sich leicht erste Werte der Kurvendiskussion ableiten:
\begin{itemize}
\item Nullstellen: $x_{1,2}=0\; \text{(doppelt)}, x_3=-2$
\item Polstellen: $x=-1,\, x=2$ (einfach $\Rightarrow$ Vorzeichenwechsel)
\item Asymptote: $y=x+3$\\
Schnittstellen und Asymptoten: $5x+6=0 \curvearrowright x=-1,2$
\end{itemize}
\subsubsection{Trigonometrische Funktion}
Übliche Definition der trigonometrische Funktionen (Kreisfunktionen wie $\sin() \cos()$ usw.):
\paragraph{Def. 3:} \parskp
Eine Funktion $y=f(x)$ heißt periodisch, wenn es eine Zahl $p>0$ gibt mit $f(x)=f(+p)$ (für alle $x \in Db(f)$). Die kleinste positive Zahl $p$ mit dieser Eigenschaft heißt Periode $f$.
\paragraph{Def. 4:} (Symmetrieeigenschaft)\\
Eine Funktion $y=f(x)$ heißt:
\begin{enumerate}
\item gerade (symmetrische zur y-Achse), wenn $f(-x)=f(x)$ für alle $x\in Db(f)$ gilt.
\item ungerade (punktsymmetrisch zum Ursprung), wenn $f(-x)=-f(x)$ für alle $x\in Db(f)$ gilt.
\end{enumerate}
\subparagraph{Diskussion:} \parskp
Einige Funktionen
\begin{tabular}{c c c c}
Funktion & $Db$ & Periode & Symmetrie\\
\hline
$\sin x$ & $\mathbb{R}$ & $2\pi$ & ungerade\\
$\cos x$ & $\mathbb{R}$ & $2\pi$ & gerade\\
$\tan x$ & $\mathbb{R}\setminus \left\lbrace\frac{\pi}{2}+k\pi | k\in \mathbb{Z}\right\rbrace$ & $\pi$ & ungerade\\
$\cot x$ & $\mathbb{R}\setminus \left\lbrace k\pi | k\in \mathbb{Z}\right\rbrace$ & $\pi$ & ungerade\\
\end{tabular}\\
Einige wichtige Formeln:
\begin{itemize}
\item $\sin^2 x+\cos^2 x = 1$
\item $\tan x = \frac{\sin x}{\cos x}$
\item $\cot x=\frac{1}{\tan x}$
\item $\sin 2x=2\sin x \cos x$
\item $\cos 2x=2 \cos^2x-1=1-2\sin^2x$
\end{itemize}
\subsubsection{Exponentialfunktion}
$y=f(x)=a\; (a>0, x\in \mathbb{R})$
\begin{itemize}
\item Wichtig: Potenzgesetze, z.B. $a^{x_1}\cdot a^{x_2}=a^{x_1+x_2}$ usw.
\item Besondere Bedeutung besitzt die Funktion $y=e^x$ ($x\in \mathbb{R}$ mit $e=\lim_{x\to \infty} \left( 1+\frac{1}{n}\right)^n=2,7182...$
\end{itemize}
\subsubsection{Hyperbelfunktion}
\paragraph{Def. 5:}Hyperbolicus
\begin{itemize}
\item $y=\cosh x := \frac{1}{2}(e^x+e^{-x}) \quad  (x \in \mathbb{R})$
\item $y=\sinh x := \frac{1}{2}(e^x-e^{-x}) \quad  (x \in \mathbb{R})$
\item $y=\tanh x := \frac{\sinh x}{\cosh x} \quad  (x \in \mathbb{R})$
\item $y=\coth x := \frac{1}{\tanh x} \quad  (x \not = 0)$
\end{itemize}
Wichtig: $\cosh^2 x-\sinh^2 x = 1$\\
Fourier: $y=\cosh x$ ist gerade, alle anderen Hyperbelfunktionen ungerade.
\subsection{Umkehrfunktionen}
\begin{itemize}
\item Zur Erinnerung: $y=f(x), x\in Db(f)$ heißt injektiv (umkehrbar eindeutig), wenn es zu jedem Bild $y \in Wb(f)$ genau ein Urbild $x\in Db(f)$ mit $y=f(x)$ gibt. D.h.:\\
$\underbrace{y}_{\in Wb(f)} \to \underbrace{x}_{\in Db(f)} =: f^{-1}(y)$\\
Die dadurch erklärte Funktion $f^{-1}$ („f oben -1“) ist die Umkehrfunktion von $f$.\\
Es gilt: $\boxed{Db\left(f^{-1}\right) = Wb(f)}, \; Wb\left( f^{-1}\right)= Db(f)$
\item Bilden der Umkehrfunktion zu $y=f(x)$, $x \in Db(f)$:
\begin{enumerate}
\item Auflösen der Funktionsgleichung nach $x$: $x=:f^{-1}(y)$ (falls dies eindeutig möglich ist, andernfalls existiert $f^{-1}$ nicht!)
\item Oft erfolgt anschließend eine Vertauschung von $x$ und $y$:\\
$y=f^{-1}(x)$, $x \in Db(f^{-1})=Wb(f)$.\\
Vertauschung entspricht geometrisch Spiegelung an der Geraden $x=y$, vgl. Bsp. 4.
\end{enumerate}
\end{itemize}
\subparagraph{Bsp. 4:} \parskp
$y=f(x)=\sqrt{x}+2 \qquad , x \in [0,\infty)$
\begin{enumerate}
\item Auflösen nach $x$: $y-2=\sqrt{x}\Rightarrow x=(y-2)^2=f^{-1}(y)$
\item Vertauschen von $x$ und $y$: $y=f^{-1}(x)=(x-2)^2, Db(f^{-1})=Wb(f)=[2,\infty)$\\
\begin{tikzpicture}[scale=.7]
\draw  [-latex](-0.5,0) -- (5.5,0) node[right]{$x$};
\draw [-latex] (0,-0.5) -- (0,5.5) node[above]{$y$};
\draw [domain=0:5, samples=150, green] plot (\x, {sqrt(\x)+2}) node[below]{$y=f^{-1}(x)$};
\draw [domain=2:4.24, purple] plot (\x, {(\x+-2)^2}) node[left]{$y=f(x)$};
\draw [blue] (-.25,-.25) node[below left]{$y=x$} -- (5,5);
\draw (2,-.1) node [below] {2}--(2,.1);
\draw (-.1,2) node [left] {2}--(.1,2);
\end{tikzpicture}\\
! $Db(f^{-1})$ nur $[2,\infty)$, obwohl $(x-2)^2$ für alle $x \in \mathbb{R}$ erklärt ist.
\end{enumerate}

\paragraph{Def. 6:} \parskp
Die reellwertige Fkt. $y=f(x)$ heißt
\begin{enumerate}[label=\alph*.)]
\item \emph{streng monoton wachsend}, falls $x_1<x_2 \Rightarrow f(x_1)<f(x_2)$ gilt.
\item \emph{monoton wachsend} (=nicht fallend), falls $x_1<x_2 \Rightarrow f(x_1) \leq f(x_2)$ gilt für alle $x_1, x_2 \in Db(f)$.
\item Analog: \emph{Streng monoton fallend} bzw \emph{monoton fallend} (=nicht wachsend).
\end{enumerate}

\paragraph{Satz 2:} \parskp
$f$ streng monoton $\Rightarrow$ $f$ ist injektiv (d.h. $f^{-1}$ existiert)

\subsection{Elementare Funktionen (Teil 2)}
\subsubsection{Wurzel- und Logarithmusfunktionen}\label{1.4.3}
\paragraph{Def. 7:} \parskp
$y=x^{\frac{1}{n}}=\sqrt[n]{x} \qquad (x\in [0,\infty), n \in \mathbb{N}^*)$ ist die Umkehrfunktion zu $y=x^n \quad (x \in [0, \infty))$

\subparagraph{Diskussion: }
\begin{enumerate}
\item Im Bereich der reellen Zahlen ist $\sqrt[n]{x}$ nur für $x\geq 0 $ erklärt, der Funktionswert ist selber nicht negativ.
\item Lässt man in $x^{\frac{1}{3}}$ negative $x$ zu (etwa $\sqrt[3]{-8}=-2$), so ergeben sich Widersprüche:\\
z.B.: $\sqrt[3]{-8}=-2 \Rightarrow -2=-8^{\frac{1}{3}}=(-8)^{\frac{2}{6}}=\left((-8)^2\right)^{\frac{1}{6}}=64^{\frac{1}{6}}=2$
\item Es gilt $\sqrt{x^2}=|x|$ für alle $x \in \mathbb{R}$
\end{enumerate}

\paragraph{Def. 8:}\parskp
$y=log_a (x) \quad (a>0\wedge a \not = 1, x\in (0,\infty))$ ist Umkehrfunktion zu $y=a^x \quad (x \in \mathbb{R})$.\\
Speziell: \begin{itemize}
\item $lg(x):= log_{10}(x)$
\item $ln(x):= log_e (x)$
\end{itemize}
\begin{tikzpicture}[scale=.7]
\draw (1,-.1) node [below] {1}--(1,.1);
\draw (-.1,1) node [left] {1}--(.1,1);
\draw [dashed] (0,1) -- (1,1) -- (1,0);

\draw  [-latex](-3.5,0) -- (5.5,0) node[right]{$x$};
\draw [-latex] (0,-3.5) -- (0,5.5) node[above]{$y$};
\draw [domain=-3:1.6, samples=150] plot (\x, {2.7^\x}) node[right]{$y=a^x$};
\draw [domain=0.05:4, samples=150, green] plot (\x, {ln(\x)}) node[right]{$y=log_a(x)$};

\draw [blue] (-.25,-.25) -- (5,5);
\end{tikzpicture}
\subparagraph{Diskussion:}
\begin{enumerate}
\item Log-Gesetze:
\begin{align*}
log_a(x \cdot y) &= log_a x + log_a y\\
log_a\left(\frac{x}{y}\right) &= log_a x - log_a y\\
log_a(x^{\alpha})&=\alpha \; log_a x\\
log_c b&=\frac{log_a b}{log_a c}
\end{align*}
\item Es gilt $x=a^{log_a x}$ \qquad ($f(f^{-1}(x))=x \forall y \in Db(f^{-1})$)
\item Ferner gilt $a^x=e^{ln(a^x)}=e^{x\cdot ln\;a}$
\end{enumerate}
\subsubsection{Arcusfunktionen}
Vorbetrachtung: $y=f(x)=sin\; x (x \in \mathbb{R})$ ist nicht injektiv, also existiert keine Umkehrfunktion.\\
Aber: $y=\sin(x)$, eingeschränkt auf z.B. $\left[-\frac{\pi}{2}, \frac{\pi}{2}\right]$ ist injektiv und damit umkehrbar.\\
\begin{tikzpicture}[scale=.7]
\draw  [-latex](-6.5,0) -- (6.5,0) node[right]{$x$};
\draw [-latex] (0,-1.5) -- (0,1.5) node[above]{$y$};
\draw [domain= -6:6]plot [smooth] (\x, {sin(\x*66)}) node[right]{$y=\sin(x)$};
\draw [dashed, green] (-1.4,-1.2) node [below] {$-\frac{\pi}{2}$}-- (-1.4,1.2);
\draw [dashed, green] (1.4,-1.2) node [below] {$\frac{\pi}{2}$}-- (1.4,1.2);
\draw [green] (-1.4,0.05) -- (1.4,0.05);
\end{tikzpicture}
\begin{tikzpicture}[scale=1]
\draw  [-latex](-2,0) -- (2,0) node[right]{$x$};
\draw [-latex] (0,-2) -- (0,2) node[above]{$y$};
\begin{scope}[yscale=-1,xscale=1]
\draw [domain= -1.37:1.37, rotate=90]plot [smooth] (\x, {sin(\x*66)}) node [below left]{$y=\arcsin(x)$};
\end{scope}
\draw (-1,-0.1) node [below]{-1} -- (-1,0.1);
\draw (1,-0.1) node [below]{1} -- (1,0.1);
\draw (-0.1,1.37) -- (0.1,1.37) node[right]{$\tfrac{\pi}{2}$};
\draw (-0.1,-1.37) -- (0.1,-1.37) node[right]{$-\tfrac{\pi}{2}$};
\end{tikzpicture}
\paragraph{Def. 9:} \parskp
Umkehrfunktionen\\
\renewcommand{\arraystretch}{1.6}
\begin{tabular}{c | c | c | c l}
 & Db & Wb & Umkehrfunktion von ...\\
\hline
$y=arcsin\; x$ & $[-1,1]$ & $\left[-\frac{\pi}{2}, \frac{\pi}{2}\right]$ & $y=sin\;x$ & $ -\frac{\pi}{2}\leq x \leq \frac{\pi}{2}$\\
$y=arccos\; x $ & $[-1,1]$ & $[0,\pi]$ & $y=cos\; x$ & $0\leq x \leq \pi$\\
$y=arctan\; x$ & $\mathbb{R}$ & $\left(-\frac{\pi}{2}, \frac{\pi}{2}\right)$ & $y=tan \; x $ & $-\frac{\pi}{2}<x<\frac{\pi}{2}$\\
$y=arccot\; x $ & $\mathbb{R}$ & $(0,\pi)$ & $y=cot\; x$ & $0< x <\pi$\\
\end{tabular}
\renewcommand{\arraystretch}{1.3}

\subparagraph{Bsp. 5:}\parskp
Gesucht sind alle Lösungen der folgenden Gleichung: $tan(2x)=y$\\
Es sei $2x \in \left( -\frac{\pi}{2}+k \cdot \pi, \frac{\pi}{2} + k \cdot \pi\right)$, mit $k \in \mathbb{Z}$.\\
$y=tan(2x)=tan(2x-k\cdot \pi)$ mit $2x-k\pi \in \left(-\frac{\pi}{2}, \frac{\pi}{2}\right)$\\
$\Rightarrow 2x - k \pi = arctan(y) \Rightarrow \resultul{x = \frac{arctan(y)+k\pi}{2}}$

\subsubsection{Areafunktionen}

\paragraph{Def. 10:} \parskp
Die Umkehrfunktionen der Hyperbelfunktionen\\
\begin{tabular}{c | c | c | c l}
 & Db & Wb & Umkehrfunktion von ...\\
\hline
$y=\arcsinh\; x$ & $\mathbb{R}$ & $\mathbb{R}$ & $y=\sinh\;x$ & $ x \in \mathbb{R}$\\
$y=\arccosh\; x $ & $[1,\infty)$ & $[0,\infty)$ & $y=cosh\; x$ & $x\geq 0$\\
$y=\arctanh\; x$ & $(-1,1)$ & $\mathbb{R}$ & $y=tanh \; x $ & $x \in \mathbb{R}$\\
$y=\arccoth\; x $ & $\mathbb{R}\setminus[-1,1]$ & $\mathbb{R}\setminus\{0\}$ & $y=coth\; x$ & $x \not = 0$\\
\end{tabular}\\
Aus der Def. der Hyberbelfunktionen (Def. 5) folgt:\\
$\arcsinh\; x = ln (x+\sqrt{x^2+1})$\\
$\arctanh\; x=\frac{1}{2}ln\left(\frac{1+x}{1-x}\right)$\\
$\arccosh\; x = ln (x+\sqrt{x^2+1}$\\
$\arccoth\; x = \frac{1}{2}ln\left(\frac{x+1}{x-1}\right)$

\section{Lineare Algebra}
%\input{HTW_Mathe1_Skript_sec1_5}

\newpage
\chapter{Folgen, Reihen, Grenzwerte}
%\section{Zahlenfolgen}
%\subsection{Grenzwerte von Zahlenfolgen}
\paragraph{Def. 1:} \parskp
Es sei $n_0 \in \mathbb{N}$. Eine Funktion $f$ mit $Db(f)=\{u\in \mathbb{N}|n\geq n_0\}$ und $Wb(f) \subset \mathbb{R}$ heißt reelle Zahlenfolge.\\
Schreibweise: \\
$a_n=f(n)$ \qquad ($n \in Db(f)$)\\
$\left(a_n\right)_{n\geq n_0}=\left(a_{n_0}, a_{n_0+1}, a_{n_0+2}, ...\right)$\\
oft $n_0=0$ oder $n_0=1$.
\subparagraph{Bsp. 1:} 
\begin{enumerate}[label=\alph*.)]
\item $a_n=(-1)^n\cdot n \quad (n \in \mathbb{N})\\
(a_n)=(0,-1,2,-3,4,...)$
\item $a_0=-1,\; a_n=n\cdot a_{n-1} \quad (n \in \mathbb{N}^*)$ \quad (rekursive Def.)\\
$(a_n)=(-1,-1,-2,-6,-24,...), \; a_n = -n!$
\item $a_n=\frac{3}{10}+\frac{3}{10^2}+...+\frac{3}{10^n} \quad (n\in \mathbb{N}^*)$\\
$(a_n)=(0.3, 0.33, 0.333, ... )$
\item $a_n=1+(-1)^n\frac{1}{n^2} \quad (n \in \mathbb{N}^*)$\\
$(a_n)=\left( \frac{5}{4}, \frac{8}{9}, \frac{17}{16}, \frac{24}{25},...\right)$
\end{enumerate}
\paragraph{Def. 2:} 
\begin{itemize}
\item $(a_n)$ heißt \emph{konvergent}, wenn es eine Zahl $a \in \mathbb{R}$ gibt mit folgender Eigenschaft:\\
Zu jedem $\varepsilon > 0$ existiert eine natürliche Zahl $n_0(\varepsilon)$, sodass für alle $n \geq n_0(\varepsilon)$ gilt: $|a_n-a|< \varepsilon$.
\item Die Zahl $a$ heißt \emph{Grenzwert} von $(a_n)$.\\
Schreibweisen:\\
$\boxed{a=\lim_{n\rightarrow \infty}(a_n)}$ oder $\boxed{a_n \underset{\color{gray}n\rightarrow\infty}{\longrightarrow} a}$
\item $(a_n)$ heißt \emph{divergent}, falls $(a_n)$ nicht konvergent ist.
\end{itemize}
\subparagraph{Diskussion}
\begin{enumerate}
\item Für $\varepsilon>0$ heißt $U_{\varepsilon}(a):=(a-\varepsilon, a+\varepsilon)$ (offenes Intervall) \emph{$\varepsilon$-Umgebung von $a$}.\\
\begin{tikzpicture}
\draw [-latex] (-5,0) -- (5,0);
\draw (-3,0) node{(};
\draw (-3,0) node[below]{$a-\varepsilon$};
\draw (3,0) node{)};
\draw (3,0) node[below]{$a+\varepsilon$};
\draw (0,0.1) -- (0,-0.1);
\draw (0,0) node[below] {$a$};
\draw [red, thick] (-3,0)--(3,0);
\draw [red] (-1,0) node [above] {$U_{\varepsilon}(a)$};
\end{tikzpicture}\\
$\left( \lim_{n\rightarrow \infty}a_n=a\right)\equiv \left( \forall\varepsilon>0 \; \exists n_0(\varepsilon) \; \forall n\geq n_0(\varepsilon) \; a_n \in U_{\varepsilon}(a)\right)$\\
d.h. für jedes (noch so kleine) $\varepsilon$, liegen ab einem bestimmten (von $\varepsilon$ abhängigen) Index $n_0(\varepsilon)$ alle Glieder $a_n(n\geq n_0(\varepsilon))$ in $U_{\varepsilon}(a)$.
\item Im Bsp. 1 sind:\\
konvergente Folgen: \\
c.) mit $\underset{n\rightarrow\infty}{a_n}=\frac{1}{3}$\\
d.) mit $\underset{n\rightarrow\infty}{a_n}=1$\\
divergente Folgen: a.) und b.)
\item Ist $\underset{n\rightarrow\infty}{a_n}=0$, so heißt $(a_n)$ \emph{Nullfolge}.
\end{enumerate}
\paragraph{Def. 3:} \parskp
$(a_n)$ heißt:
\begin{itemize}
\item \emph{streng monoton wachsend}, falls für jedes $n$ gilt: $a_n<a_{n+1}$.
\item \emph{monoton wachsend}, falls für jedes $n$ gilt: $a_n\leq a_{n+1}$.
\item \emph{streng monoton fallend}, falls für jedes $n$ gilt: $a_n>a_{n+1}$.
\item \emph{monoton fallend}, falls für jedes $n$ gilt: $a_n\geq a_{n+1}$.
\end{itemize}
\paragraph{Def. 4:} \parskp
$(a_n)$ heißt beschränkt, wenn es eine Konstante $C>0$ gibt mit $|a_n|\leq C$ für alle $n$.
\subparagraph{Diskussion:}
\begin{enumerate}
\item $(a_n)$ beschränkt \\
$\Leftrightarrow \exists c>0 \; \forall n \quad |a_n| \leq C \\
\Leftrightarrow \exists c_1\in \mathbb{R}\; \exists c_2 \in \mathbb{R}\; \forall n \quad c_1 \leq a_n \leq c_2$
\item Folgen aus Bsp. 1:\\
\begin{tabular}{l l l l}
&Folge & Monotonie & Beschränktheit\\
a.)& $a_n=(-1)^n\cdot n$ & --  & --\\
b.)& $a_n=-n!$ & streng monoton fallend (ab $n=1$) & --\\
c.)& $a_n=\tfrac{3}{10}+\tfrac{3}{10^2}+...+\tfrac{3}{10^n}$ & streng monoton wachsend & $0,3\leq a_n < \tfrac{1}{3}$\\
d.) & $a_n=1+(-1)^n\tfrac{1}{n^2}$ & -- & $0\leq a_n <\tfrac{5}{4}$
\end{tabular}
\end{enumerate}
\paragraph{Satz 1:} \parskp
Jede konvergente folge ist beschränkt.
\paragraph{Satz 2:} \parskp
Jede monotone und beschränkte Folge ist konvergent.
\paragraph{Def. 5:} \parskp
$(a_n)$ heißt \emph{bestimmt konvergent} gegen $\begin{cases}
+\infty\\
- \infty
\end{cases}$, falls gilt: $\forall c \in \mathbb{R} \; \exists n_0(c) \; \forall n \geq n_0(c) \begin{cases}
a_n>c\\
a_n<c
\end{cases}$.\\
Schreibweise: $\boxed{\lim_{n\to \infty} a_n = \begin{cases}
+\infty\\
- \infty
\end{cases}}$
\subparagraph{Bsp. 2:} 
\begin{enumerate}[label=\alph*.)]
\item aus Bsp. 1c.): $a_n=\frac{3}{10}+...+\frac{3}{10^n}, \; (a_n)$ monoton wachsend und beschränkt $\Rightarrow (a_n)$ ist konvergent, $\lim_{n\to \infty}a_n = \frac{1}{3}$. 
\item aus Bsp. 1b.): $a_n=-n!$, $(a_n)$ monoton fallend und unbeschränkt $\Rightarrow$ $(a_n)$ ist bestimmt divergent, $\lim_{n\to \infty}a_n=-\infty$
\end{enumerate}
\subparagraph{Diskussion:}\parskp
Eine divergente Folge, die nicht bestimmt divergent ist, heißt \emph{unbestimmt divergent}.\\
Bpsw. Folge aus Bsp. 1a.) $a_n=(-1)^n\cdot n$.\bigskip\\
Einige wichtige Grenzwerte:
\begin{enumerate} [label=\alph*.)]
\item $\lim_{n\to \infty} \left(1+\frac{1}{n}\right)^n=e=2.71...$ (EULERsche Zahl)
\item $\lim_{n\to\infty} \sqrt[n]{n}=1$
\item $\lim_{n\to\infty} \frac{\ln n}{n}=0$
\item $\lim_{n \to \infty} \sqrt[n]{a}=1 \quad (a>0)$
\end{enumerate}
\paragraph{Satz 4:} Rechenregeln (Grenzwertsätze)\\
$(a_n)$ und $(b_n)$ seien zwei konvergente Folgen mit $\lim_{n\to \infty}=a, \; \lim_{n\to \infty} = b$. Dann gilt:
\begin{itemize}
\item $\lim_{n\to \infty}(a_n+b_n)=a+b$
\item $\lim_{n\to \infty}(c\cdot a_n)=c \cdot a$
\item $\lim_{n\to\infty} (a_n \cdot b_n) = a \cdot b$
\item $\lim_{n \to \infty} \left(\frac{a_n}{b_n}\right) = \frac{a}{b} \qquad (b_n\not = 0, b\not = 0)$
\end{itemize}
\subparagraph{Bsp. 3:}
\begin{enumerate} [label=\alph*.)]
\item $a_n=\frac{2n^2-1}{3n^2+n} \qquad (n=1,2,3,...)$\\
$a_n=\frac{n^2\left(2-\frac{1}{n^2}\right)}{n^2\left(3+\frac{n}{n^2}\right)}=\frac{2-\frac{1}{n^2}}{3+\frac{1}{n}}=\frac{\lim_{n\to\infty}\left(2-\frac{1}{n^2}\right)}{\lim_{n\to\infty}\left(3+\frac{1}{n}\right)}=\resultul{\frac{2}{3}}$\\
\fbox{Ausklammern der höchsten Potenzen in Zähler und Nenner}
\item $a_n=n\cdot \left(\sqrt{n^2+1}-n\right)$ \\
(in Klammern: „$\infty-\infty$“ $\curvearrowright$ Erweitern mit 3. binomischer Formel)\\
$a_n=\frac{n\cdot \left( \sqrt{n^2+1}-n\right) \cdot \tgreen{\left(\sqrt{n^2+1}+n\right)}}{\tgreen{\left(\sqrt{n^2+1}+n\right)}}
=\frac{n\cdot (n^2+1-n^2)}{n\cdot \sqrt{1+\frac{1}{n^2}}+n}
=\frac{n\cdot 1 }{n\left(\sqrt{1+\frac{1}{n^2}}+1\right)}
\underset{n\to \infty}{\longrightarrow} \frac{1}{2}$ \\
oder:
$\lim_{n\to\infty} = \resultul{\frac{ 1}{2}}$
\item $a_n=\frac{\sin n}{n} \qquad \left(0\leq |a_n|=\frac{|\sin n|}{n}\leq \frac{1}{n}\right)\Rightarrow\lim_{n\to\infty}a_n=0$\bigskip\\
Allgemein: $(a_n)$ beschränkt und $(b_n)$ bestimmt divergent $\Rightarrow\lim_{n\to\infty}\frac{a_n}{b_n}=0$.
\end{enumerate}

\subsection{Lineare Rekursionsgleichungen (Differenzengleichungen)}
\begin{itemize}
\item Allgemeine Form einer Rekursionsgleichung $k$-ter Ordnung:\\
$x_n=f(n,x_{n-1},x_{n-2},...,x_{n-k}) \qquad (k \geq 1,\; n\geq n_0+k)$
\item Wir betrachten nur \emph{lineare Rekursionsgleichungen mit konstanten Koeffizienten} (d.h. $a_j$ nicht von $n$ abhängig):\\
$\boxed{x_n=a_1 x_{n-1}+a_2 x_{n-2}+...+a_k a_{n-k}+h_n}\quad k\geq 1, \; a_k \not = 0, \; n\geq n_0 +k$\\
$x_n$ gesucht, $a_1,a_2,...,a_k, h_n \; (n\geq n_0)$ bekannt.
\item Indexverschiebung möglich:\\
$\boxed{x_{n+k}=a_1 x_{n+k-1}+a_2 x_{n+k-2}+...+a_k a_{n}+h_{n+k}}\quad (n\leq n_0)$\\
Wichtig ist die Differenz zwischen höchstem und niedrigstem Index von $x$ (=Ordnung der Rekursionsgleichung).
\item Da die größen $x_n, x_{n-1}, x_{n-2},...$ auch durch $x_n$ und die Differenzen $\Delta x_n:=x_n-x_{n-1}, \Delta^2x_n:=\Delta x_n-\Delta x_{n-1}=x_n-2x_{n-1}+x_{n-2}, ...$ ausgedrückt werden können, ist der Name \emph{Differenzengleichung} sehr verbreitet.
\item Die Differenzengleichung (aus erstem Punkt) heißt homogen, falls $h_n=0$ (für alle $n$), sonst inhomogen.
\end{itemize}
Zur Lösung von der Differenzengleichung (aus erstem Punkt oberhalb):
\begin{enumerate}
\item Allgemeine Lösung: $\boxed{x_n=x_n^{(h)}+x_n^{(p)}}$, dabei ist $x_n^{(h)}$ die \emph{allgemeine Lösung der zugehörigen homogenen Gleichung} $\boxed{x_n=a_1 x_{n-1}+...+a_k x_{n-k}}$ und $x_n^{(p)}$ eine \emph{partikuläre (spezielle) Lösung der inhomogenen Gleichung}.
\item Es gibt $k$ Lösungen $x_n^{(1)},...,x_n^{(h)}$ der homogenen Gleichung, so dass gilt:\\
$x_n^{(h)}=c_1 x_n^{(1)}+...+c_k x_n^{(h)}$\\
Diese erhält man mit Hilfe der Lösungen der charakteristischen Gleichung:\\
$\lambda^k=a_1 \lambda^{k-1}+a_2 \lambda^{k-2}+...+a_{k-1} \lambda+a_k$\\
Dise ergibt sich aus dem Ansatz:\\
$x_n^{(h)}= \lambda^n \; (\lambda \not = 0)\\
\Rightarrow \lambda^n = a_1 \lambda^{n-1}+...+a_k \lambda^{n-k} \quad |: \lambda^{n-k}\\
\Rightarrow $ Bei $k$ verschieden Lösungen $\lambda_1, ..., \lambda_2$ ergibt sich $x_n^{(h)}=c_1 \lambda_1^n+...+c_k \lambda_k^n$, falls z.B. $\lambda$ 2-fach auftritt, dann: $x_n^{(h)}=c_1\lambda_1^n+c_2\lambda_1^n\cdot n +...$
\item Für die Partikulärlösung $x_n^{(p)}$ führen spezielle Ansätze zum Ziel:\\
\begin{tabular}{ 
p{\dimexpr0.25\columnwidth-2\tabcolsep-1.5\arrayrulewidth} | >{\raggedright}
p{\dimexpr0.2\columnwidth-2\tabcolsep-1.5\arrayrulewidth} | >{\raggedright}
p{\dimexpr0.4\columnwidth-2\tabcolsep-1.5\arrayrulewidth}}
Inhomogenität $h_n$ & Bedingung & Ansatz für $x_n^{(p)}$\tabularnewline
\hline
Polynom in $n$ (Grad $r$) & $\lambda = 1$ ist keine$^{*)}$ Lösung von $\lambda^k$ & Polynom vom gleichen Grade mit unbestimmten Koeffizienten\tabularnewline
\hline
Potenzfunktion $b^n$ & $\lambda = b$ ist keine$^{*)}$ Lösung von $\lambda^k$ & $x_n^{(p)}=A \cdot b^n$
\end{tabular}\\
$^{*)}$ bei $\xi$-facher Lösung ist der Ansatz mit $n^\xi$ zu multiplizieren\\
Unbestimmte Koeffizienten $A, ...$ durch Einsetzen in die inhomogene Gleichung und Koeffizientenvergleich ermitteln.
\item Die $k$ Koeffizienten $c_1,...,c_k$ in der allgemeinen Lösung können durch die Anfangsbedingungen $(AB)$ (Vorgabe der ersten $k$ Glieder von $(x_n)$) ermittelt werden.\\
Es sind also folgende Schritte durchzuführen:
\begin{enumerate}[label=\Alph*)]
\item Allgemeine Lösung $x_n^{(h)}$ der homogenen Gleichung ermitteln
\item eine spezielle Lösung $x_n^{(p)}$ der inhomogenen Gleichung ermitteln
\item $x_n = x_n^{(h)}+x_n^{(p)}$
\item $AB$ erfüllen
\end{enumerate}
\end{enumerate}
\subparagraph{Bsp. 4:} $x_{n+1}=2 x_n +3 \qquad n\geq 0, \; x_0 =1$\\
Erste Glieder: $1,\; 5,\; 13,\; 29, ...$\\
Typ: Lineare Differenzengleichung 1. Ordnung\\
Lösung:
\begin{enumerate}[label=\Alph*)]
\item homogene Gleichung $x_{n+1}=2x_n$ (charakteristische Gleichung $\lambda_1 =2$)\\
$\lambda_1 =2 \; \Rightarrow \; x_n^{(h)}=C\cdot 2^n$
\item $h_n=3$ (Polynom des $0$-ten Grades). Ansatz: $x_n^{(p)}=A$ (Einsetzen in Ausgangsgleichung)\\
$A=2\cdot 2A+3 \; \Rightarrow \; A=-3 \;\Rightarrow\; x_n^{(p)}=-3$
\item $x_n = x_n^{(h)}+x_n^{(p)}=C\cdot 2^n-3$
\item $AB$: $n=0 \; \Rightarrow \; x_0 = 1 = C \cdot 2^0 \; \Rightarrow \; C=4$
\end{enumerate}
Also: $x_n=4 \cdot 2^n -3$
\subparagraph{Bsp. 5:} $x_{n+2}=x_{n+1}+2x_n \qquad n\geq 0, \; x_0 =2, \;x_1=3$\\
Erste Glieder: $2,\; 3, \; 7, \; 13,\; 27,\; 53, ...$\\
Typ: lineare homogene Dz.-Gleichung 2. Ordnung
\begin{enumerate}
\item[A)] Schritt A liefert bereits die allgemeine Lösung (B und C entfallen):
$\lambda^2=\lambda+2 \;\Rightarrow\; \lambda_1 = -1, \; \lambda_2 =2$\\
$\Rightarrow\; x_n=x_n^{(h)}=C_1 \cdot (-1)^n + C_2\cdot 2^n$
\item[D)] $AB$ erfüllen:\\
$C=\frac{5}{3},  \; C_1=\frac{1}{3} \begin{cases}
n=0 \; &\Rightarrow \; x_0 =2 = C_1+C_2\\
n=1 \; &\Rightarrow \; x_1 =3 = -C_1+2C_2
\end{cases}$
\end{enumerate}
Also: $x_n=\frac{1}{3}(-1)^n+\frac{5}{3}\cdot 2^n$
\subparagraph{Diskussion:} Bei einer homogenen linearen Dz.-Gleichung 2. Ordnung können folgende Fälle auftreten:
\begin{itemize}
\item $\lambda_1, \; \lambda_2$ reel und verschieden:\\
$\Rightarrow x_n = x_n^{(h)}= C_1 \lambda_1^n+C_2 \lambda_2^n$ (vgl. Bsp. 5)
\item $\lambda_1 = \lambda_2$ (reelle Doppellösung):\\
$\Rightarrow x_n = x_n^{(h)}= C_1 \lambda_1^n+C_2 n \lambda_2^n = \lambda_1^n(C_1+C_2\cdot n)$
\item $\lambda_{1,2} = i \pm iv \; (v \not = 0)$ homogene komplexe Lösung:\\
$\Rightarrow x_n = x_n^{(h)}= C_1 \lambda_1^n+C_2 \lambda_2^n$ (wie im 1. Fall, die Koeffizienten $C_1$ und $C_2$ sind aber im allgemeinen komplex, $x_n$ selbst ist aber wieder reell)
\end{itemize}
Reeller Ansatz ist mit Hilfe der Formeln von EULER und MOIVRE möglich:\\
$\lambda_1^n= \left(r\cdot e^{i\varphi}\right)^n = r^n \cdot e^{i\cdot n \cdot \varphi} = r^n(\cos(n\varphi)+i\cdot \sin(n\varphi))$\\
$\lambda_2^n= \left(r\cdot e^{-i\varphi}\right)^n = r^n \cdot e^{-i\cdot n \cdot \varphi} = r^n(\cos(n\varphi)-i\cdot \sin(n\varphi))$\\
Damit reeller Ansatz:\\
$x_n=x_n^{(h)}=K_1 r^n \cos(n\varphi)+K_2 r^n \sin(n \varphi)$\\
Bemerkung: Falls Rechner mit komplexer Arithmetik vorhanden, so ist direkt die Formel aus 1. Fall bequemer.

\subsection{Unendliche Reihen} 
\subsubsection{Grundbegriffe}\label{2.1.3}
\paragraph{Def. 6:} 
Gegeben sei die Zahlenfolge $(a_n)_n \geq n_0, \; n\in \mathbb{N}$. Die Zahlenfolge $(S_n)_n \geq n_0$ mit $S_{n_0}:=a_{n_0}, \; S_{n_0 + 1} := a_{n_0}+a_{n_0 +1}, \; S_{n_0 +2} := a_{n_0} + a_{n_0 +1} + a_{n_0 +2}, \; ..., \; S_n=a_{n_0}+a_{n_0 +1} +...+a_n$ (\emph{Partialsumeenfolge}) heißt \emph{unendliche Reihe}.\\
Bezeichnung: $\boxed{\sum_{n=n_0}^{\infty} a_n}$
\begin{itemize}
\item Die Zahlen $a_n$ heißen Glieder der Reihe, die Zahlen $S_n$ heißen Partialsummen der Reihe
\item Ist die Reihe konvergent, d.h. die Folge $(S_n)$ ist konvergent, so heißt $s:=\lim_{n\to \infty}S_n=: \sum_{n=n_0}^{\infty} a_n$ die Summe der Reihe
\item Die Reihe heißt (bestimmt oder unbestimmt) divergent, wenn die Partialsummen die entsprechende Eigenschaft haben.
\end{itemize}
Bemerkung: Oft $n_0=0$ oder $=1$
\subparagraph{Bsp. 6:} $a_n=a q^n$ mit $a\not = 0, q \not = 0, n=0,1,2,...$\\
$(a_n)=(a, aq, aq^2, aq^3,...)$ (\emph{geometrische Zahlenfolge})\\
$(S_n)=\sum_{n=0}^\infty aq^n$ (\emph{geometrische Reihe})\\
$= (\underbrace{a}_{s_0}, \underbrace{a+aq}_{s_1}, \underbrace{a+ aq+ aq^2}_{s_2}, ...)$\\
$S_n=a + aq + aq^2 + aq^3+...+aq^n \quad | \cdot q$\\
$S_nq=aq+aq^2+aq^3+aq^4+...+aq^{n+1}$\\
Beide Zeilen voneinander abgezogen:\\
$S_n-S_nq = a - aq^{n+1}\\
S_n(1-q) = a - aq^{n+1} \quad | : (1-q) \text{ falls }q\not = 1\\
\boxed{S_n = a\cdot \frac{1-q^{n+1}}{1-q}}$ (\emph{Summenformel für die endliche geometrische Reihe} mit Anfangsglied $a$ und $n+1$ Summanden)\\
$\Rightarrow \lim_{n\to \infty}S_n=\frac{a}{a-q}$ falls $|q|<1$
$\Rightarrow$ Summe der unendlichen geometrischen Reihe:\\
$\boxed{\sum_{n=0}^\infty a q^n=\frac{a}{1-q}}$ für $|q|<1$.\\
z.B. $0,\overline{72}=0,727272...=\frac{72}{100}+\frac{72}{10.000}+\frac{72}{1.000.000}+...=\frac{72}{99}=\frac{8}{11}$

\subparagraph{Bsp. 7:} $\sum_{n=1}^\infty \frac{1}{n}=1+\frac{1}{2}+\frac{1}{3}+...$ heißt \emph{harmonische Reihe}. Offensichtlich ist $(S_n)$ streng monoton wachsend. Man kann zeigen, dass $(S_n)$ nicht beschränkt ist. Aus Satz 3 folgt: die harmonische Reihe ist bestimmt divergent.\\
Schreibweise: $\sum_{n=1}^\infty \frac{1}{n}=\infty$

\paragraph{Def. 7:} Die Reihe $\sum_{n=n_0}^\infty a_n$ heißt
\begin{enumerate}[label=(\alph*)]
\item absolut konvergent, falls $\sum_{n=n_0}^\infty|a_n|$ konvergent ist.
\item bedingt konvergent, falls $\sum_{n=n_0}^\infty a_n$ konvergent, aber $\sum_{n=n_0}^\infty |a|$ nicht konvergent ist.
\end{enumerate}

\paragraph{Satz 5:} $\sum_{n=n_0}^\infty a_n$ absolut konvergent $\Rightarrow$ $\sum_{n=n_0}^\infty a_n$ konvergent.
\subparagraph{Diskussion:}
\begin{enumerate}
\item Die Umkehrung gilt im Allgemeinen nicht. Es gibt konvergente Reihen, die nicht absolut konvergieren. Z.B. $\sum_{n=1}^\infty (-1)^{n-1}\frac{1}{n}=1-\frac{1}{2}+\frac{1}{3}-\frac{1}{4}+...$
\item Für Reihen mit nicht-negativen Gliedern ($a_n \geq 0$ für alle $n$) ist absolute Konvergenz identisch mit (gewöhnlicher) Konvergenz. Für solche Reihen gilt entweder $\sum_{n=n_0}^\infty a_n<\infty$ [(absolut) konvergent] oder $\sum_{n=n_0}^\infty a_n = \infty$ [bestimmt divergent].
\end{enumerate}

\subsubsection{Konvergenzkriterien}
\begin{enumerate} [label= \arabic*., wide]
\item Notwendiges Konvergenzkriterium
\paragraph{Satz 6:} $\sum_{n=n_0}^\infty a_n$ konv. $\Rightarrow$ $\lim_{n\to \infty} a_n = 0$\\
bzw.: $(a_n)$ konvergiert $\Rightarrow$ $\sum_{n=0}^\infty a_n$ divergiert 
\subparagraph{Beweis:} $a_n = S_n - S_{n-1}$\\
$\Rightarrow \lim_{n\to \infty}a_n = \lim_{n\to\infty}S_n - \lim_{n\to\infty}S_{n-1}=s-s=0$
\subparagraph{Bemerkung:} 
\begin{enumerate}
\item Bedingung $\lim_{n\to\infty}a_n = 0$ ist notwendig, aber nicht hinreichend. Z.B. $a_n=\frac{1}{n}$, dann $\lim_{n\to\infty}a_n=0$ aber $\sum_{n=1}^\infty a_n=\infty$
\item Anwendung des Satzes meist in logisch äquivalenter Form: $\lim_{n \to \infty} a_n \not = 0 \Rightarrow \sum_{n=n_0}^\infty a_n$ divergiert
\end{enumerate}

\subparagraph{Bsp. 8:} $\sum_{n=1}^\infty \left(\frac{n}{10n-1}\right)^{50}$\\
$a_1=1,94\cdot 10^{-48}, a_2=1,3 \cdot 10^{-49}, ...$\\
$\lim_{n\to\infty}a_n=\lim_{n\to\infty}\left(\frac{n}{10-\frac{1}{n}}\right)^{50}\not = 0$\\
$\Rightarrow$ Reihe divergent (sogar besimmt divergent, da alle $a_n \geq 0$)

\item Hinreichendes Kriterien
\begin{enumerate}[label=(\Alph*),wide]
\item \emph{Leibnitzkriterium für alternierende Reihen}
\paragraph{Satz 7:} Sei $(b_n)$ Folge mit 
\begin{itemize}
\item $b_n \geq b_{n+1} > 0$ für alle $n \in \mathbb{N}$
\item $\lim_{n\to\infty} b_n = 0$
\end{itemize}
Dann ist $\sum_{n=0}^\infty(-1)^n b_n=b_0-b_1+b_2-b_3+...$ konvergent. D.h. wenn die Beträge $b_n$ der Glieder einer alternierenden Reihe mit $a_n = (-1)^n b_n$ eine Nullfolge bilden, dann ist die Reihe konvergent.\\
Weiter gilt: $|s-S_n| \leq | a_{n+1} |$\\
Also ist der Fehler bei der Approximation von $s$ durch $S_n$ beschränkt durch den Beträg von $a_{n+1}$.

\subparagraph{Bsp. 9:} $\sum_{n=1}^\infty (-1)^{n-1} \frac{1}{n}=1-\frac{1}{2}+\frac{1}{3}-\frac{1}{4}+...$ (alternierende harmonische Reihe)\\
$s_1=1, \; s_2=0,5, \; s_3\approx 0,83, \; s_4\approx 0,583, \; s_5\approx 0,78, \; s_6\approx 0,62$\\
\begin{tikzpicture}[scale=.9]
\draw (0,0) -- (16,0);
\foreach \x in {0,1}{
	\draw (\x*16, 0.5) -- (\x*16,-0.5) node[below]{$\x$};
}
\foreach \x in {0.5,0.6,0.7,0.8,0.9}{
	\draw (\x*16, 0.2) -- (\x*16,-0.2) node[below]{$\x$};
}

\node[above, purple] at (16,0.5) {$s_1$};
\node[above, purple] at (0.5*16,0.2) {$s_2$};
\node[above, purple] at (0.83*16,0.2) {$s_3$};
\node[above, purple] at (0.583*16,0.2) {$s_4$};
\node[above, purple] at (0.783*16,0.2) {$s_5$};

\node[above, purple] at (0.69*16,0.2) {$s$};
\draw[purple, thick] (0.69*16, 0.2) -- (0.69*16,-0.2);

\draw [-latex, purple] plot[smooth, tension=1] coordinates {(16,1.2) (11.8,2.2) (8,1)};
\draw [-latex, purple] plot[smooth, tension=1] coordinates {(8,0.8) (10.4,1.4) (0.83*16,1)};
\draw [-latex, purple] plot[smooth, tension=1] coordinates {(0.83*16,0.8) (11.2,1.2) (0.583*16,1)};
\draw [-latex, purple] plot[smooth, tension=1] coordinates {(0.583*16,0.8) (10.8,1) (0.783*16,0.8)};
\end{tikzpicture}\\
Man kann zeigen: $s=\ln 2 = 0,6931$
\item \emph{Verleichskriterien für Reihen mit nicht-negativen Gliedern}
\paragraph{Satz 8:} (Majoranten-Kriterium)\\
Seien $(a_n)_{n\geq n_0}, (b_n)_{n\geq n_0}$ Folgen mit $0 \leq a_n \leq b_n$ für alle $n\geq n_1 \geq n_0$ und $\sum_{n=n_0}^\infty b_n < \infty$ (d.h. konvergent). \\
Dann $\sum_{n=n_0}^\infty a_n < \infty$ (d.h. konvergent).

Die Reihe $\sum _{n=n_0}^\infty b_n$ heißt dann konvergente Majorante zur Reihe $\sum_{n=n_0}^\infty a_n$.\\
Beweisidee: $0\leq a_n \leq b_n \rightsquigarrow 0 \leq \sum _{n=n_0}^\infty a_n \leq \sum _{n=n_0}^\infty b_n \leq \infty $
\paragraph{Satz 9:} (Minoranten-Kriterium)\\
Seien $(a_n)_{n\geq n_0}, (b_n)_{n\geq n_0}$ Folgen mit $0 \leq b_n \leq a_n$ für $n\geq n_1 \geq n_0$ und $\sum_{n=n_0}^\infty b_n = \infty$ (d.h. divergent)\\
Dann $\sum_{n=n_0}^\infty a_n = \infty$ (also auch divergent)\\
Die Reihe $\sum_{n=n_0}^\infty b_n$ heißt divergente Minorante der Reihe $\sum_{n=n_0}^\infty a_n$.\bigskip\\
Eine nützliche Vergleichsreihe für die Anwendung der Sätze 8 und 9 ist:\\
$\boxed{\sum_{n=1}^\infty \frac{1}{n^\lambda}= \begin{cases}
\text{konvergent} & \text{für }\lambda > 1\\
\text{divergent} & \text{für }\lambda \leq 1
\end{cases}}$
\subparagraph{Bsp. 10:} Man untersuche das Konvergenzverhalten der folgenden Reihe:
\begin{enumerate}[label=\alph*.)]
\item $\sum_{n=1}^\infty \underbrace{\frac{1}{n^2-n+1}}_{a_n}$ (Vermutung: Verhalten wie $\sum \frac{1}{n^2}$ wegen der Dominanz der höchsten Potenz)\\
Wir versuchen eine konvergente Majorante zu finden.\\
$a_n=\frac{1}{n^2-n+1}\geq \frac{1}{n^2-\frac{n^2}{2}}$ (wegen $n\geq \frac{n^2}{2}$ für $n\geq 2$)\\
Somit $\frac{1}{n^2-\frac{n^2}{2}}= \frac{2}{n^2}=:b_n$\\
Mit der Vergleichsreihe gilt: $\sum_{n=1}^\infty b_n = 2 \sum_{n=1}^\infty \frac{1}{n^2}$ ist konvergent.\\
$\overset{\text{Satz 8}}{\Longrightarrow} \sum_{n=0}^\infty a_n$ ist konvergent, sogar absolut konvergent, da $a_n\geq 0 \; (n \in \mathbb{N})$
\item $\sum_{n=1}^\infty \frac{n^2+4}{n^3+n^2+31}$ (Vermutung: divergent, da Verhalten wie $\sum \frac{n^2}{n^3}=\sum \frac{1}{n}$)\\
Wir versuchen divergente Minoranten zu finden.\\
$a_n = \frac{n^2+4}{n^3+n^2+31}\geq ... \geq \frac{1}{3n}=: b_n$ (für $n\geq 4$)\\
Wieder gilt mit der Vergleichsreihe: $\sum b_n$ divergent. Also folgt mit Satz 9: $\sum_{n=1}^\infty a_n$ divergent.
\end{enumerate}
\item \emph{Quotienten- und Wurzelkritierien}
\paragraph{Satz 10:} (Quotientenkriterium)\\
Sei $(a_n)_{n\geq n_0}$ eine Folge, so gilt:\\
$\boxed{\lim_{n\to\infty}\left|\frac{a_{n+1}}{a_n}\right|\begin{cases}
<1\\
>1
\end{cases}\Rightarrow \sum_{n=n_0}^\infty a_n \text{ ist}\begin{cases}
\text{absolut konvergent}\\
\text{divergent}
\end{cases}}$
\paragraph{Satz 11:} (Wurzelkriterium)\\
Sei $(a_n)_{n\geq n_0}$ eine Folge, so gilt:\\
$\boxed{\lim_{n\to\infty} \sqrt[n]{|a_n|} \begin{cases}
<1\\
>1
\end{cases}\Rightarrow \sum_{n=n_0}^\infty a_n \text{ ist}\begin{cases}
\text{absolut konvergent}\\
\text{divergent}
\end{cases}}$
\subparagraph{Bemerkung:} Falls in Satz 10 oder 11 $\lim ... =1$ gilt, so ist mit diesem Kriterium keine Konvergenzaussage möglich.

\subparagraph{Bsp. 11:}
\begin{enumerate}[label=\alph*.)]
\item $\sum_{n=2}^\infty \underbrace{\left( \frac{1}{\ln(n)}\right)^n \cdot (-1)^n}_{a_n}$\\
Wegen $\sqrt[n]{|a_n|}=\frac{1}{\ln n} \overset{n\to\infty}{\longrightarrow} 0 $ liefert das Wurzelkriterium, dass die Reihe absolut konvergent ist.
\item $\sum_{n=1}^\infty \frac{(-1)^n (2n)!}{(n!)^2}$\\
Wegen $\left|\frac{a_{n+1}}{a_n}\right|=\frac{\frac{(2(n+1))!}{((n+1)!)^2}}{\frac{(2n)!}{(n!)^2}}=\frac{(2n+2)!}{((n+1)!)^2}\cdot \frac{(n!)^2}{(2n)!}=\frac{(2n+2)(2n+1)(n!)^2}{(n+1)^2 (n!)^2}=\frac{(2n+2)(2n+1)}{(n+1)^2}
=\frac{4n^2+4n+2n+2}{n^2+2n+1}\overset{n\to\infty}{\longrightarrow} 4$\\
Daher ist die Reihe divergent.
\end{enumerate}
\end{enumerate}
\end{enumerate}

\subsubsection{Rechenregeln}
\begin{itemize}
\item $\sum_{n=n_0}^\infty a_n$ und $\sum_{n=n_0}^\infty$ konvergent mit Summe $a$ und $b$, dann gilt: 
\begin{itemize}
\item $\sum_{n=n_0}^\infty (a_n+b_n)= a + b$
\item $\sum_{n=n_0}^\infty c\cdot a_n= c\cdot a$
\end{itemize}
\item $\sum_{n=n_0}^\infty a_n$ absolut konvergent $\Leftrightarrow$ die Glieder $a_n$ lassen sich beliebig umordnen, ohne dass sich die Summe ändert.
\item $\sum_{n=n_0}^\infty a_n$ und $\sum_{n=n_0}^\infty b_n$ absolut konvergen mit Summen $a$ und $b$, dann gilt:
\begin{itemize}
\item $\left(\sum_{i=0}^\infty a_i\right)\cdot \left( \sum_{j=0}^\infty b_j\right)=\sum_{i=0}^\infty  \sum_{j=0}^\infty a_i b_j=a\cdot b \qquad \left( =\sum_{n=0}^\infty  \sum_{i=0}^\infty a_i b_{n-i} \quad \text{Cauchy-Produkt}\right)$
\end{itemize}
\end{itemize}

\section{Grenzwerte und Stetigkeit von Funktionen}
\subsection{Grenzwerte von Funktionen}
\paragraph{Def. 1:} Es sei $x_0 \in \mathbb{R}$ und es existiere eine Umgebung $U(x_0)$ mit $U(x_0)\{x_0\}\subseteq Db(f)$.\\
\begin{tikzpicture}[scale=1]
\draw[-latex] (0.5,0) -- (8.5,0);
\path[pattern= north east lines, pattern color=green] (1.2,0.2) rectangle (3.2,-0.2);
\path[pattern= north east lines, pattern color=green] (4,0.2) rectangle (5.4,-0.2);
\path[pattern= north east lines, pattern color=green] (6.4,0.2) rectangle (7.8,-0.2);
\draw (4.6,0.4) -- (4.6,-0.4) node[below]{$x_0$};
\draw [orange, thick] (4.2,0.01) -- (5,0.01);
\draw [orange, thick] plot[smooth, tension=1] coordinates {(4.3,0.25) (4.2,0) (4.3,-0.25)};
\draw [orange, thick] plot[smooth, tension=1] coordinates {(4.9,0.25) (5,0) (4.9,-0.25)};
\node [above right, orange] at (4.6,0.3) {$U(x_0)$};
\node [right] at (8,0.5) {korrekt};
\end{tikzpicture}\\
\begin{tikzpicture}[scale=1]
\draw[-latex] (0.5,0) -- (8.5,0);
\path[pattern= north east lines, pattern color=green] (2.2,0.2) rectangle (4.52,-0.2);
\path[pattern= north east lines, pattern color=green] (4.68,0.2) rectangle (6.6,-0.2);
\draw (4.6,0.4) -- (4.6,-0.4) node[below]{$x_0$};
\draw [orange, thick] (4.2,0.01) -- (5,0.01);
\draw [orange, thick] plot[smooth, tension=1] coordinates {(4.3,0.25) (4.2,0) (4.3,-0.25)};
\draw [orange, thick] plot[smooth, tension=1] coordinates {(4.9,0.25) (5,0) (4.9,-0.25)};
\node [above right, orange] at (4.6,0.3) {$U(x_0)$};
\node [right] at (8,0.5) {korrekt};
\draw  (4.6,0) ellipse (0.2 and 0.2);
\end{tikzpicture}\\
\begin{tikzpicture}[scale=1]
\draw[-latex] (0.5,0) -- (8.5,0);
\path[pattern= north east lines, pattern color=green] (1.6,0.2) rectangle (3.8,-0.2);
\path[pattern= north east lines, pattern color=green] (6.4,0.2) rectangle (8,-0.2);
\draw (4.6,0.4) -- (4.6,-0.4) node[below]{$x_0$};
\draw [orange, thick] (4.2,0.01) -- (5,0.01);
\draw [orange, thick] plot[smooth, tension=1] coordinates {(4.3,0.25) (4.2,0) (4.3,-0.25)};
\draw [orange, thick] plot[smooth, tension=1] coordinates {(4.9,0.25) (5,0) (4.9,-0.25)};
\node [above right, orange] at (4.6,0.3) {$U(x_0)$};
\node [right] at (8,0.5) {falsch};
\end{tikzpicture}\\
$\lim_{x\to x_0} f(x)=\lambda : \Leftrightarrow$ Für jede Folge $(x_n)$ mit $x_n\in Db(f)$, $x_n \not = x$ (für alle $n$) und $\lim_{n\to\infty}x_n=x_0$ gilt $\lim_{n\to\infty}f(x_n)=a$.\\
Anschaulich: $f(x)$ strebt gegen $a$, wenn $x$ gegen $x_0$ strebt.
\subparagraph{Bemerkung:} Die Stelle $x_0$ muss \emph{nicht} selbst zum Definitionsbereich gehören.
\subparagraph{Bsp. 1:}
\begin{itemize}
\item $\lim_{x\to 0} \frac{\sin(x)}{x}$\\
\begin{tikzpicture}[scale=4]

\draw (0,0) -- (1,0) -- (1,1) -- cycle;
\fill  (0,0) circle (0.015) node [below]{$M$};
\fill  (1,0) circle (0.015) node [below right]{$A$};
\fill  (1,1) circle (0.015) node [above right]{$C$};
\fill  (0.707,0.707) circle (0.015) node [above left]{$B$};

\draw [ {[-]} ] (0,-0.2) -- (1,-0.2) node[pos=.5, below]{$1$};
\draw [ {[-]} ] (-0.1,0.1) -- (0.607,0.807) node[pos=.5, above left]{$1$};
\draw (1,0) arc (0:45:1);

\draw (0.25,0) arc (0:45:0.25) node [left, pos=.4]{$x$};
\draw [green] (0,-.03) -- (0.707, -0.03) node[below, pos=.5]{$\cos x$};
\draw [orange] (0.707,0) -- (0.707, 0.707) node[left, pos=.5]{$\sin x$};
\draw [brown] (1.03,0) -- (1.03, 1) node[right, pos=.5]{$\tan x$};
\draw [blue] (0.707,0.707)-- (1,0);
\end{tikzpicture}\\
$F_{\vartriangle MAB}\leq F_{Sektor\; MAB} \leq F_{\vartriangle MAC}$\\
$\frac{1}{2}\sin x < \frac{1}{2} x < \frac{1}{2} \tan x \quad |\cdot \frac{2}{\sin x}$\\
$\Leftrightarrow 1 < \frac{x}{\sin x} < \frac{1}{\cos x}$\\
$\Leftrightarrow 1 > \frac{\sin x}{x} > \cos x$\\
$\Rightarrow \lim_{x\to 0} \frac{\sin(x)}{x}=1$
\end{itemize}
Analog zu Grenzwertsätzen für Zahlenfolgen gilt:
\paragraph{Satz 1:} Es gelte $\lim_{x\to x_0} f(x) = a$ und $\lim_{x\to x_0} g(x) = b$. Dann:
\begin{itemize}
\item $\lim_{x\to x_0} (f(x)+g(x) = a+b$
\item $\lim_{x\to x_0} c \cdot f(x) = c \cdot a$
\item $\lim_{x\to x_0} (f(x) \cdot g(x)) = a \cdot b$
\item $\lim_{x\to x_0}\frac{f(x)}{g(x)}=\frac{a}{b} \quad $ (falls $b\not = 0$)
\end{itemize}
\subparagraph{Bsp. 2:}
\begin{enumerate}[label=\alph*.)]
\item $\lim_{x\to 0} \frac{3x^3-7x+4}{3 \cos x} = \frac{4}{3}$
\item $\lim_{x\to 3} \frac{x^2-x-6}{x-3}=\mq{\frac{0}{0}}$ Satz nicht anwendbar.\\
$= \lim_{x\to 3} \frac{(x-3)(x+2)}{(x-3)} = \lim_{x\to 3} x+2 = 5$\\
(andere Möglichkeit mit $\mq{\frac{0}{0}}$ umzugehen lernen wir später)
\end{enumerate}
\paragraph{Def. 2:} 
\begin{enumerate}[label = \alph*.)]
\item rechtseitiger Grenzwert:\\
$\lim_{x\searrow x_0} f(x)=a : \Leftrightarrow $ für jede Folge $(x_n)$ mit $x_n \in Db(f)$ und $x_n > x_0$ und $\lim_{n\to \infty} x_n =x_0$ gilt $\lim_{n\to \infty} f(x_n)=a$.\\
Andere Schreibweise: $\lim_{x\searrow x_0}=\lim_{x\to x_0+0}$
\begin{tikzpicture}[scale=2]
\draw [-latex] (-0.8, 0) -- (1.7, 0);
\draw (0, 0.1) -- (0,-0.1) node[below]{$x_0$};
\draw (1, 0.1) -- (1,-0.1)node[below]{$x$};
\draw [-latex] (0.9,0.15) -- (0.4,0.15);
\end{tikzpicture}
\item linkseitiger Grenzwert:\\
$\lim_{x\nearrow x_0} f(x) =a :\Leftrightarrow$ analog rechtsseitiger Grenzwert
\item $\lim_{x\to \infty} f(x) = a :\Leftrightarrow$ für jede Folge $(x_n)$ mit $x_n \in Db(f)$ und $\lim_{x\to \infty} x_n = \infty$ gilt $\lim_{n\to \infty} f(x_n)=a$.
\item $\lim_{x\to \infty} f(x) =a :\Leftrightarrow$ analog s.o.
\end{enumerate}
\subparagraph{Diskussion:} Uneigentliche Grenzwerte:\\
Wir schreiben 
 $\lim_{\bullet} f(x) 0 \begin{cases}
 \infty\\
 -\infty
 \end{cases}$  bei bestimmter Divergenz der Funktionswerte für:\\
$\bullet \begin{cases}
x \to x_0\\
x \nearrow x_0\\
x \searrow x_0\\
x\to \infty\\
x\to -\infty
\end{cases}$

\paragraph{Satz 2:}\parskp
$\lim_{x\to x_0} f(x) =a \Leftrightarrow \lim_{x\nearrow x_0} f(x)=\lim_{x\searrow x_0}=a$

\subparagraph{Bsp. 3:} (einseitiger Grenzwert)\\
$f(x)=\begin{cases}
-x^2 & \text{ für } x<0\\
\sqrt{x+1} & \text{ für } x\geq 0
\end{cases}$\\
\begin{tikzpicture}
\draw [thick, domain=0:3] plot[smooth] (\x,{sqrt(\x+1)});
\draw [thick, domain=-1.5:0] plot[smooth] (\x,{-(\x*\x)});
\draw (0,0) circle(0.1);
\fill (0,1) circle(0.1);

\draw [-latex] (-3.2,0) -- (3.2,0) node [below right] {$x$};
\draw [-latex] (0,-2.2) -- (0,2.2) node [above right] {$y$};
\end{tikzpicture}\\
$\lim_{x\nearrow 0}f(x) = 0, \; \lim_{x\searrow 0}f(x)=1$\\
$\Rightarrow \lim_{x\to 0}f(x)$ existiert nicht!

\paragraph{Bsp. 4:} \parskp
$\lim_{x\to \infty} x \cdot \sin \left(\frac{4}{x}\right)= \mq{\infty \cdot 0}$\\
$\overset{u=\frac{4}{x}}{=}\lim_{u\searrow 0}\frac{4}{u} \sin(u) = 4$
\paragraph{Bsp. 5:} \parskp
$\lim_{x\nearrow \frac{\pi}{2}} \tan x = \infty$\\
$\lim_{x\searrow \frac{\pi}{2}} \tan x = -\infty$\\
\begin{tikzpicture}[scale = 0.5]
\draw plot [domain=(3.15-1.3):(3.15+1.3)] (\x, {tan(\x r)} );
\draw plot [domain=(-1.3):(1.3)] (\x, {tan(\x r)} );
%\draw [dashed] ((0.5*3.14),-3) -- ((0.5*3.14),3); 
\draw [dashed] (1.56, -4) -- (1.56, 4);
\draw [thick] (1.56, 0.5) -- (1.56, -0.5) node [below left]{$\frac{\pi}{2}$};

\draw [-latex] (-3.2*2,0) -- (3.2*2,0) node [below right] {$x$};
\draw [-latex] (0,-2.2*2) -- (0,2.2*2) node [above right] {$y$};
\end{tikzpicture}
\subsection{Stetigkeit von Funktionen}
\paragraph{Def. 3:} Sei $f: Db(f)\to \mathbb{R}, \; Db(f) \subseteq \mathbb{R}$ eine Funktion und $x_0 \in Db(f)$ gegeben.\\
Es heißt $f$:
\begin{enumerate}[label=\alph*.)]
\item stetig in $x_0$ falls $\lim_{x \to x_0} f(x) = f(x_0)$ gilt\\
(also $\lim_{x\to x_0} f(x)=f(\lim_{x\to x_0}x)$, d.h. Limes und Funktion kann vertauscht werden).\\
\begin{tikzpicture}[scale = 0.5];
\draw  plot[smooth, tension=.7] coordinates {(1,.5) (2,2) (4,4) (5,4.5)} node[right] {$f$};
\fill (2,2) circle (.1);
\draw [latex-]( 2.5, 2.8)--(3,3.3);
\draw[dashed] (2,0) node[below]{$x_0$}-- (2,2) -- (0,2) node[left]{$f(x_0)$};
\draw[dashed] (4,0) node[below]{$x$} -- (4,4) -- (0,4) node[left]{$f(x)$};

\draw [-latex] (-1.4,0) -- (6.4,0) node [below right] {$x$};
\draw [-latex] (0,-0.4) -- (0,4.4) node [above right] {$y$};
\draw [-latex] (3.5,-0.5) -- (2.5,-0.5);
\draw [-latex] (-0.5,3.5) -- (-0.5,2.5);
\end{tikzpicture}
\item linksseitig stetig in $x_0$, falls $\lim_{x\nearrow x_0}f(x)=f(x_0)$.
\item rechtsseitig stetig in $x_0$, falls $\lim_{x\searrow x_0}f(x)=f(x_0)$.
\end{enumerate}

\subparagraph{Bsp. 6:}
\begin{enumerate}[label=\alph*.)]
\item $f_1(x)=\begin{cases}
\frac{\sin x}{x} & x\not = 0 \\
0 & x = 0
\end{cases}$ ist in $x_0=0$ nicht stetig, da $\lim_{x\to 0} f(x) = 1 \not = 0 = f(0)$.\\
Aber $\overset{\sim}{f}_1(x)=\begin{cases}
f(x) & x\not = 0\\
1 & x= 0
\end{cases}$ ist in $x_0=0$ stetig.\\
Bezeichnung: hebbare Unstetigkeit.\\
\begin{tikzpicture}[scale = 0.5]
\draw [-latex] (-6.4,0) -- (6.4,0) node [below right] {$x$};
\draw [-latex] (0,-4.4) -- (0,4.4) node [above right] {$y$};
\draw plot[domain=-5:-0.2, smooth] (\x, {4.12*sin(50*\x)/\x});
\draw plot[domain=0.2:5, smooth] (\x, {4.12*sin(50*\x)/\x});
\draw  (0,3.6) circle (0.2) node[above right]{$1$};
\draw (3.6,0.2) -- (3.6,-0.2) node[below]{$\pi$};
\draw (-3.6,0.2) -- (-3.6,-0.2) node[below]{$-\pi$};
\end{tikzpicture}
\item $f_2(x)=\begin{cases}
\arctan \left(\frac{1}{x}\right) & x\not = 0\\
0 & x = 0
\end{cases}$ ist unstetig in $x_0=0$, da $\lim_{x\nearrow 0}f_2(x) \not = f_2(0) \not = \lim_{x\nearrow 0} f_2(x)$\\
Bezeichnung: endlicher Sprung.\\
\begin{tikzpicture}[scale = 0.5]
\draw [-latex] (-6.4,0) -- (6.4,0) node [below right] {$x$};
\draw [-latex] (0,-4.4) -- (0,4.4) node [above right] {$y$};
\draw plot[domain=-5:-0.01, smooth] (\x, {-2.7^(\x)-1});
\draw plot[domain=0.01:5, smooth] (\x, {2.7^(-\x)+1});
\draw  (0,2) circle (0.2);
\draw  (0,-2) circle (0.2);
\fill  (0,0) circle (0.2);
\end{tikzpicture}
\item $f_3(x)=\begin{cases}
\frac{1}{x} & x\not = 0\\
0 & x=0
\end{cases}$ ist unstetig in $x_0=0$, da $\lim_{x\nearrow 0}f_3(x) = \infty \not = f_3(0)$.\\
\begin{tikzpicture}[scale = 0.5]
\draw [-latex] (-6.4,0) -- (6.4,0) node [below right] {$x$};
\draw [-latex] (0,-4.4) -- (0,4.4) node [above right] {$y$};
\draw plot[domain=-5:-0.3, smooth] (\x, {1/\x});
\draw plot[domain=0.3:5, smooth] (\x, {1/\x});
\draw (2,0.2) -- (2,-0.2) node[below]{$x$};
\draw [-latex] (1.5,-0.2) -- (0.5,-0.2);
\end{tikzpicture}
\item $f_3(x)=\begin{cases}
\sin \frac{1}{x} & x \not = 0\\
1 & x=0
\end{cases}$ ist unstetig in $x_0=0$, da der Grenzwert $\lim_{x\to 0}\sin \frac{1}{x}$ nicht existiert.\\
\begin{tikzpicture}[scale = 0.5]
\draw [-latex] (-10.4,0) -- (10.4,0) node [below right] {$x$};
\draw [-latex] (0,-4.4) -- (0,4.4) node [above right] {$y$};
\def \sc {1000};
\draw plot[domain=-10:-1, smooth, samples=100] (\x, {3.6*sin(\sc/\x)});
\draw plot[domain=-1:-0.5, smooth, samples=100] (\x, {3.6*sin(\sc/\x)});
\draw plot[domain=1:10, smooth, samples=100] (\x, {3.6*sin(\sc/\x)});
\draw plot[domain=0.5:1, smooth, samples=100] (\x, {3.6*sin(\sc/\x)});
\fill [rounded corners=1pt] (-0.5,-3.6) rectangle (0.5,3.6);
\end{tikzpicture}
\end{enumerate}
\paragraph{Def. 4:} Die Funktion $f: DB(f) \to \mathbb{R}, \; Db(f) \subseteq \mathbb{R}$ heißt
\begin{enumerate}[label=\alph*.)]
\item \emph{in einem Intervall $I \subset Db(f)$ stetig}, falls $f$ an jeder inneren Stelle $x_0 \in I$ stetig ist und in evtl. zu $I$ gehörenden Randpunkten einseitig stetig ist.
\item \emph{stetig}, falls $f$ in allen Punkten $x_0\in Db(f)$ stetig ist.
\end{enumerate}
\subparagraph{Bemerkung:} Jede der in \ref{1.4.1} und \ref{1.4.3} betrachteten Funktionen ist stetig.
\subparagraph{Bsp. 7:} $f: \mathbb{R}\setminus \{0\} \to \mathbb{R}, \; f(x)=\frac{1}{x}$ ist stetig.

\paragraph{Satz 3:} Sind $f$ und $g$ stetig in $x_0$, so sind auch $c_1 \cdot f + c_2 \cdot g, \; f\cdot g$ und $\frac{f}{g}$(falls $g(x_0)\not = 0$) stetig in $x_0$.

\paragraph{Satz 4:} (Stetigkeit und Verknüpfungen)\\
Seien $g: Db(g) \to \mathbb{R}$ und $f: Db(f) \to \mathbb{R}$ Funktionen mit $Wb(g)\subseteq Db(f)$, dann gilt:\\
Ist $g$ stetig in $x_0$ und $f$ stetig in $g(x_0)$, so ist $f\circ g:Db(g) \to \mathbb{R}, \; (f\circ g)(x) = f(g(x))$ stetig in $x_0$.

\paragraph{Satz 5:} (Zwischenwertsatz)\\
Sei $f: Db(f) \to \mathbb{R}, \; Db(f)\subseteq \mathbb{R}$ stetig auf $[a,b]  Db(f)$. Falls $f(a) \cdot f(b) < 0$ (also haben unterschiedliche Vorzeichen), so gilt $\exists x^*\in [a,b]$ mit $f(x^*)=0$\\
\begin{tikzpicture}[scale = 0.5]
\draw [-latex] (-6.4,0) -- (6.4,0) node [below right] {$x$};
\draw [-latex] (0,-4.4) -- (0,4.4) node [above right] {$y$};

\fill  (0.5,-2.5) circle (0.1);
\fill  (5.5,3.5) circle (0.1);
\draw  plot[smooth, tension=.7] coordinates {(0.5,-2.5) (1.5,-1.5) (2,-0.5) (3,0) (3.5,2) (4.5,3) (5.5,3.5)};

\draw (0.75,-0.5) -- (0.5,-0.5) -- (0.5,0.5) -- (0.75,0.5) node[above]{$a$};
\draw (5.5,0.5) node[above]{$b$} -- (5.75,0.5) -- (5.75,-0.5) -- (5.5,-0.5);
\draw (3,0.25) -- (3,-0.25) node[below]{$x^*$};
\draw (0.25,-2.5) -- (-0.25,-2.5) node[left]{$f(a)$};
\draw (0.25,3.5) -- (-0.25,3.5) node[left]{$f(b)$};
\end{tikzpicture}
\paragraph{Satz 6:} Sei $f: Db(f) \to \mathbb{R}, \; Db(f) \subseteq \mathbb{R}$ stetig auf $[a,b]$. Dann nimmt $f$ auf $[a,b]$ Minimum und Maximum an.

\subparagraph{Diskussion:}
\begin{enumerate}[label=\alph*.)]
\item $f(x) = \tan x$ nimmt auf $\left( - \frac{\pi}{2}, \frac{\pi}{2}\right)$ kein Maximum an.\\
ABB21
\item $f(x) = \begin{cases}
\arctan \frac{1}{x} & x \in [-1,1]\setminus \{0\}\\
0 & x = 0
\end{cases}$ nicht stetig und nimmt kein Maximum auf $[-1,1]$ an.\\
\begin{tikzpicture}[scale = 0.5]
\draw [-latex] (-6.4,0) -- (6.4,0) node [below right] {$x$};
\draw [-latex] (0,-4.4) -- (0,4.4) node [above right] {$y$};
\draw plot[domain=-5:-0.01, smooth] (\x, {-2.7^(\x)-1});
\draw plot[domain=0.01:5, smooth] (\x, {2.7^(-\x)+1});
\draw  (0,2) circle (0.2);
\draw  (0,-2) circle (0.2);
\fill  (0,0) circle (0.2);
\end{tikzpicture}
\end{enumerate}

\section{Potenzreihen}
\paragraph{Def.:} Sei $(a_n)$ eine Zahlenfolge und $x_0 \in \mathbb{R}$ heißt $\boxed{\sum_{n=0}^\infty a_n (x-x_0)^n}$ Potenzreihe mit dem Mittelpunkt $x_0$.
\subparagraph{Diskussion:} 
\begin{itemize}
\item Für jedes feste $x \in \mathbb{R}$ ist die Potenzreihe eine feste Reihe.
\item Konvergenzbereich $K:=\{x\in \mathbb{R} | \text{Potenzreihe ist konvergent}\}$
\item Für jedes $x\in K$ existiert der Summenwert der Potenzreihe. Die Funktion $f: K \to \mathbb{R}$ mit $f(x) = \sum_{n=0}^\infty a_n (x-x_0)^n$ heißt Grenzfunktion der Potenzreihe.
\end{itemize}
Zur Bestimmung des Konvergenzbereichs nutz man Satz 10 und 11 aus \ref{2.1.3} und erhält absolute Konvergenz in einem um $x_0$ liegendem Konvergenzintervall $I:=(x_0-r, x_0+r)$.\\
Wie $r$ bestimmt wird liefert:

\paragraph{Satz 1:} Sei $(a_n)$ Zahlenfolge mit $r:=\lim_{n\to \infty} \left| \frac{a_n}{a_{n+1}}\right|=\lim \frac{1}{\sqrt[n]{|a_n|}}$ existiert.\\
Dann ist $\sum_{n=0}^\infty a_n (x-x_0)^n\begin{cases}
\text{absolut konvergent} & \text{für }x\in \mathbb{R} \text{ mit }|x-x_0|<r\\
\text{divergent} & \text{für }x\in \mathbb{R} \text{ mit }|x-x_0|>r
\end{cases}$.

\subparagraph{Diskussion:} 
\begin{itemize}
\item Verwechslungsgefahr:
\begin{itemize}
\item Satz 10 und 11 betrachten (Zahlen-)Reihen $\sum_{n=0}^\infty a_n$
\item Satz 1 betrachtet Potenzreihen $\sum_{n=0}^\infty a_n (x-x_0)^n$, wobei $a_n$ ein Faktor vor $(x-x_0)^n$ ist.
\end{itemize}
\item Falls der Grenzwert $r$ aus Satz 1 nicht existiert, so gibt es trotzdem einen Konvergenzradius.\\
Den gilt es auf andere Weise zu betrachten/ermitteln.
\item Satz 1 sagt nichts über das Verhalten an den Randpunkten aus $\rightarrow$ gesonderte Untersuchung nötig.
\end{itemize}
\subparagraph{Bsp. 1:} 
\begin{enumerate}[label=\alph*.)]
\item $\sum_{n=1}^\infty \frac{x^n}{n}$, d.h. $x_0=0$, $a_n=\frac{1}{n}$, $n=1,2,...$\\
$r= \lim_{n\to\infty} \frac{1}{\sqrt[n]{\left|\frac{1}{n}\right|}}=\lim_{n\to \infty}=\frac{1}{\frac{1}{\sqrt[n]{n}}}=\lim_{n\to\infty}\sqrt[n]{n}=1$\\
$\Rightarrow$ Konvergenzintervall $I=(-1,1)$\\
Randpunkte:\\
$x=-1: \sum_{n=1}^\infty \frac{(-1)^n}{n}$ bedingt konvergent (alternierenden harmonische Reihe)\\
$x=1: \sum_{n=1}^\infty \frac{1}{n}$ divergent\\
$\Rightarrow$ Konvergenzbereich: $K=[-1,1)$
\item $\sum_{n=0}^\infty \frac{x^n}{n!}$, d.h. $x_0=0$, $a_n=\frac{1}{n!}$\\
$\left| \frac{a_n}{a_{n+1}}\right| = \frac{\frac{1}{n!}}{\frac{1}{(n+1)!}}=\frac{(n+1)!}{n!}=n+1\overset{n\to \infty}{\longrightarrow} \infty$\\
$\Rightarrow r = \infty$\\
d.h. die Reihe ist absolut konvergent für alle $x \in \mathbb{R}$.\\
Bezeichnung: \emph{beständige Konvergenz}
\item $\sum_{n=0}^\infty \frac{x^{2n}}{(2n)!}=1+\frac{x^2}{2!}+\frac{x^2}{4!}+...\quad$ d.h. $x_0 = 0$, $a_n=\begin{cases}
\frac{1}{n} & n \text{ gerade}\\
0 & n \text{ungerade}
\end{cases}$\\
Satz 1 ist aber nicht unmittelbar anwendbar.\\
Substitution $u:=x^2$ liefert aber $\sum_{n=0}^\infty \frac{u^n}{(2n)!}$ mit $u_0=0$, $b_n=\frac{1}{(2n)!}$ ($\sum b_n (u-u_0)^n$)\\
$\left| \frac{b_n}{b_{n+1}}\right|=\frac{(2n+2)!}{(2n)!}=(2n+2)\cdot (2n+1) \overset{n\to \infty}{\longrightarrow} \infty$\\
$\Rightarrow r_u=\infty$ (Konvergenzradius für die Substituierte Reihe)\\
$\Rightarrow r_x=\mq{\sqrt{\infty}\,}= \infty$ (Konvergenzradius für die untersuchte Funktion)\\
Im Konvergenzbereich $K$ wird dadurch eine Potenzreihe eine Funktion dargestellt, die Grenzfunktion (siehe vorhergehende Diskussion).
\end{enumerate}
\subparagraph{Bsp. 2:}
\begin{enumerate}[label=\alph*.)]
\item $\sum_{n=0}^\infty x^n = \frac{1}{1-x}$ für $x\in (-1,1)$ (geometrische Reihe)
\item $\sum_{n=0}^\infty \frac{x^n}{n!}=e^x$ für $x\in \mathbb{R}$ (Beweis später)
\end{enumerate}
\paragraph{Satz 2:} Die \emph{Grenzfunktion} jeder Potenzreihe ist \emph{im Konvergenzbereich stetig}.

\chapter{Differentialrechnung für Funktionen einer reellen Variablen}
\section{Grundbegriffe}
\emph{Tangentenproblem}\\
ABB38\\
Gegeben: $y=f(x)$\\
Gesucht: Tangente im Punkt $(x_0, f(x_0))$
\begin{itemize}
\item Zunächst \tgreen{Sekante} durch $(x_1, f(x_1))$ und $(x_0, f(x_0))$
\item Dann betrachten wir $x_1 \to x_0$
\item Damit geht \tgreen{Sekante} über in die \torange{Tangente}.\\
Außerdem geht \tgreen{$\varphi$} in \torange{$\alpha$} über.
\end{itemize}
$\tan \alpha = \lim_{\varphi \to \alpha} \tan \varphi = \lim_{x_1\to x_0} \underbrace{\frac{f(x_1)-f(x_0)}{x_1-x_0}}_{\text{Differenzenquotient}}$

\paragraph{Def. 1:} Die Funktion $f:Db(f) \to \mathbb{R}$ heißt an der Stelle $x_0$ (mit $U(x_0)\subseteq Db(f)$) differenzierbar, falls der Grenzwert $\boxed{f'(x_0):=\lim_{x\to x_0} \frac{f(x)-f(x_0)}{x-x_0}}$ existiert.\\
$f'(x_0)$ heißt dann \emph{1. Ableitung} von $f$ an der Stelle $x_0$.

\subparagraph{Diskussion:}
\begin{itemize}
\item $f'(x_0)=\lim_{h\to 0} \frac{f(x_0+h)-f(x_0)}{h}$
\item Gleichung der Tangente in $(x_0, f(x_0)) $ ist $t(x)=f(x_0)+f'(x_0)(x-x_0)$ ($t: \mathbb{R}\to \mathbb{R}$)
Anstieg der Tangente ist als $m=\tan \alpha = f'(x_0)$
\item $f$ in $x_0$ differenzierbar bedeutet es existiert eine eindeutige Tangente an die Kurve in dieser Stelle.\\
z.B. ist $f: \mathbb{R}\to \mathbb{R}, \; f(x) = |x|$ in $x_0=0$ nicht differenzierbar:\\
ABB39
\end{itemize}

\paragraph{Satz 1:} Ist $f: \mathbb{R}\to \mathbb{R}$ in $x_0$ differenzierbar, so ist $f$ in $x_0$ stetig.\\
Beweis:\\
Sei $f$ in $x_n$ differenzierbar und $(x_n)$ eine beliebige Folge mit $x_n\to x_0$. Dann gilt:\\
$\lim_{n\to \infty}\frac{f(x_n)-f(x_0)}{x_n-x_0}$ existiert.\\
$\Rightarrow \exists\; K > 0$ mit $\left| \frac{f(x_n)-f(x_0)}{x_n-x_0} \right| = \frac{|f(x_n)-f(x_0)|}{|x_n-x_0|}\leq K$\\
$\Rightarrow |f(x_n) - f(x_0) | \leq K \cdot |x_n - x_0| \overset{n\to \infty}{\longrightarrow} 0$\\
$\Rightarrow \lim_{n\to \infty} f(x_n)=f(x_0) \Rightarrow f$ ist stetig.

\paragraph{Def. 2:} Eine Funktion $f: Db(f)\to \mathbb{R}$\\
$Db(f)\subseteq \mathbb{R}$ heißt
\begin{enumerate}[label=\alph*.)]
\item differenzierbar im Interval $I \subseteq Db(f)$, falls $f$ an jeder inneren Stelle $x_0\in I$ differenzierbar ist und in eventuellen Randpunkten einseitig differenzierbar ist.\\
d.h. $\lim_{x\nearrow x_r} \text{ bzw. } \lim_{x\searrow x_r} \frac{f(x)-f(x_r)}{x-x_r}$ existiert
\item differenzierbar, wenn $f$ in jedem Punkt $x_0 \in Db(f)$ differenzierbar ist.
\end{enumerate}
\emph{Schreibweise:}\\
Die resultierende Funktion bezeichnen wir mit \\
$f': Db(f')\to \mathbb{R}, f'(x)=\lim_{h\to 0} \frac{f(x+h)-f(x)}{h}$\\
wobei $Db(f')$ aus allen Punkten $x \in Db(f)$ besteht für welche der genannte Grenzwert existiert.

\paragraph{Def. 3:} Sei $f: Db(f) \to \mathbb{R}, Db(f) \subseteq \mathbb{R}$. Wir definieren rekursiv die $n$-te Ableitung von $f$ an der Stelle $x_0$ mittels\\
$f^{(n)}(x_0):= \left(f^{(n-1)}\right)'(x_0) \quad n=1,2,3,...$\\
wobei $f^{(0)}(x_0)=f(x_0)$ (unter der Voraussetzung, dass die jeweilige Ableitung existiert).

\subparagraph{Bsp. 1:} $f: \mathbb{R}\to \mathbb{R}, \; f(x): x^n, \; n \in \mathbb{N}$
\begin{align*}
\frac{f(x+h)-f(x)}{h}&=\frac{1}{h}\left((x+h)^n-x^n\right)\\
&=\frac{1}{h}\left(x^n+\nok{n}{1}\cdot x^{n-1}h+\nok{n}{2}x^{n-2}h^2+...+\nok{n}{n}h^n-x^n\right)\\
&\overset{h\to 0}{\longrightarrow}n\cdot x^{n-1}
\end{align*}
d.h. $f$ ist auf $\mathbb{R}$ differenzierbar. $f'(x)=n\cdot x^{n-1}$.

\subparagraph{Bsp. 2:} $f: \mathbb{R} \to \mathbb{R}$, $f(x):= \sin(x)$
\begin{align*}
\frac{f(x+h)-f(x)}{h}&=\frac{\sin(x+h)-\sin(x)}{h} \qquad |\; \sin x-\sin y=2\cos\frac{x+y}{2}\cdot\\
&= \frac{2\cdot \cos \frac{2x+h}{2}\cdot \sin \frac{h}{2}}{h}\\
&= \frac{\cos \left( x+\frac{h}{2}\right) \cdot \sin \frac{h}{2}}{\frac{h}{2}} \qquad |\; \frac{\sin \frac{h}{2}}{\frac{h}{2}} \overset{h\to 0}{\longrightarrow}1\\
&= \cos x
\end{align*}
Also $f'(x)=\cos x$.\\
\emph{Bemerkung:} Ableitung der wichtigsten Grundfunktionen findet man in Formelsammlungen.\\
Zur Ableitung zusammengesetzter Funktionen lernen wir im später weitere Ableitungsregeln kennen.

\subsection{Das Differential} \parskp
ABB 49\\
$\mathrm{d}y=h\cdot \tan \alpha = f \cdot f'(x_0)$
\paragraph{Def. 4:} 
\begin{enumerate}[label=\alph*.)]
\item $\mathrm{d}y:= f'(x_0)\cdot h$ heißt das zur Stelle $x_0$ und dem Zuwachs $h=\Delta x$ gehörende \emph{Differential} von $f$.
\item $\Delta y := f(x_0+h)-f(x_0)$ heißt die zur Stelle $x_0$ und dem Zuwachs $h=\Delta x$ gehörende \emph{Differenz} von $f$.
\end{enumerate}
\subparagraph{Diskussion}
\begin{enumerate}
\item $\Delta y$ ist die Änderung der Funktion $f$, wenn $x$ in $x+h$ übergeht; $\mathrm{d}y$ ist die entsprechende Änderung wenn statt $f$ die Tangente an der Stelle $x_0$ betrachtet wird (Linearisierung).
\item Für kleine Zuwächse $\Delta x$ gilt: $\Delta y \approx \mathrm{d}y$\\
d.h. $\Delta y \approx f'(x_0)\cdot \Delta x$ für kleines $\Delta x$ (nutzt man in der Fehlerrechnung)
\item Sei $y=f(x)=x \Rightarrow \mathrm{d}y=\mathrm{d}x=1\cdot h$ also $\boxed{h=\Delta x = \mathrm{d}x}$
\item Damit $f'(x)=\frac{\mathrm{d}y}{\mathrm{d}x}$\\
Also: 1. Ableitung = Differentialquotient\\
andere Schreibweise: $f'(x)=\frac{d}{\mathrm{d}x}f(x)$
\item Höhere Ableitungen:\\
$f^{(n)}(x)=\frac{d^n y}{\mathrm{d}x^n}=\frac{d^n}{\mathrm{d}x^n}f(x)$
\end{enumerate}
\section{Differentiationsregeln}
\paragraph{Satz 1:} Falls die Ableitungen auf der rechten Seite existieren:
\begin{itemize}
\item $(C_1 u(x)+C_2 v(x))' = C_1 u'(x)+C_2 v'(x)$ (Linearität)
\item $(u(x) \cdot v(x))' = u'(x)v(x)+v'(x)u(x)$ (Produktregel)
\item $\left( \frac{u(x)}{v(x)}\right)' = \frac{u'(x)v(x)-v'(x)u(x)}{(v(x))^2}$ (Quotientenregel)
\end{itemize}

\subparagraph{Bsp. 1:}
\begin{enumerate}[label=\alph*.)]
\item $f(x)=7x^4+\sqrt[3]{x}+\frac{2}{\sqrt{x}} = 7x^4+x^{\frac{1}{3}}+2x^{-\frac{1}{2}} \quad (x>0)$\\
$\Rightarrow f'(x) = 28 x^3 + \frac{1}{3}x^{-\frac{2}{3}}-x^{\frac{3}{2}}=28x^3+\frac{1}{3\sqrt[3]{x^2}}-\frac{1}{\sqrt{x^3}}$
\item $f(x)=x\cdot \ln x \quad (x\geq 0)$\\
$\Rightarrow f'(x) = 1 \cdot \ln x + \frac{1}{x} \cdot x = \ln x + 1$ (Produktregel)
\item $f(x)=\frac{e^x}{x^2+2}$\\
$\Rightarrow f'(x)=\frac{e^x\cdot (x^2+2)-e^x\cdot 2x}{(x^2+2)^2}=\frac{e^x(x^2-2x+2)}{(x^2+2)^2}$ (Quotientenregel)

\end{enumerate}

\paragraph{Satz 2:} Seien $f: Db(f)\to \mathbb{R}, \; g: Db(g) \to \mathbb{R}$ Funktionen mit $Db(f) \subseteq \mathbb{R}, \; Db(g) \subseteq \mathbb{R}$ und 
\begin{itemize}
\item $g$ bei $x_0 \in Db(g)$ differenzierbar
\item $f$ bei $g(x_0) \in Db (f)$ differenzierbar
\end{itemize}
so gilt:\\
$(f\circ g)'(x_0)=f'(g(x_0))\cdot g'(x_0)$
\subparagraph{Diskussion:} $y=f( \underbrace{g(x)}_{u})=f(u)$ mit $u=g(x)$\\
Differentialschreibweise:\\
$y'=\frac{\mathrm{d}y}{\mathrm{d}x}=\frac{\mathrm{d}y}{du}\cdot \frac{du}{\mathrm{d}x}$ \quad (äußere Ableitung $\cdot$ innere Ableitung)

\subparagraph{Bsp. 2:}
\begin{enumerate}[label=\alph*.)]
\item $y=f(x)=\sin \underbrace{3x}_u$\\
$y'=\frac{\mathrm{d}y}{\mathrm{d}x}=\frac{\mathrm{d}y}{du}\cdot \frac{du}{\mathrm{d}x}=\cos u \cdot 3=3 \cos 3x$
\item $y=f(x)=2^{\tan(3x)} \quad \left(-\frac{\pi}{6}<x<\frac{\pi}{6}\right)$\\
Substitution: \\
$u:=\tan 3x$\\
$v:=3x$\\
$\Rightarrow y = 2^u, \; u = \tan v$\\
$\Rightarrow y' = \frac{\mathrm{d}y}{\mathrm{d}x}=\frac{\mathrm{d}y}{du} \cdot \frac{du}{dv} \cdot \frac{dv}{\mathrm{d}x} = 2^u\cdot \ln 2 \cdot (1+\tan^2 v) \cdot 3= 3\cdot 2^{\tan 3x} \cdot \ln 2 \cdot (1+\tan^2 3x)$
\end{enumerate}

\subparagraph{Bsp. 3:} (Logarithmische Differentiation)\\
$f(x)=x^{\sin x} \qquad x \in (0,\infty)$\\
Basis und Exponent hängen von $x$ ab!\\
Die Regeln $(x^a)'=a x^{a-1}$ bzw. $(a^x)'=a^x\cdot \ln a$ sind nicht unmittelbar anwendbar.\\
Betrachten: 
\begin{align*}
f(x)&= x^{\sin} \\
\ln(f(x))&=\sin x \cdot \ln x\\
\overset{\text{Ableiten}}{\Longrightarrow} \frac{1}{f(x)}\cdot f'(x)&=\cos x \cdot \ln x + \sin x \cdot \frac{1}{x}\\
\Rightarrow f'(x)&= f(x) \cdot (cos(x)\cdot \ln x + \sin x \frac{1}{x})\\
&= x^{\sin x} (\cos x \ln x + \frac{\sin x}{x}
\end{align*}

\paragraph{Satz 3:} Sei $f: (x_0-r, x_0+r) \to \mathbb{R}$, $f(x)=\sum_{n=0}^\infty a_n (x-x_0)^n$ Grenzfunktion einer Potenzreihe mit Kurvenradius $r$.\\
Dann gilt für alle $x \in (x_0-r, x_0+r)$: $f'(x) = \sum_{n=1}^\infty a_n \cdot n (x-x_0)^{n-1}$

\subparagraph{Bsp. 4:} $\frac{1}{1-x}=1+x+x^2+x^3+...= \sum_{n=0}^\infty x^n, \quad |x|<1$\\
$\left(\frac{1}{1-x}\right)' = 0+1+2x+3x^2+...=\sum_{n=1}^\infty n x^{n-1} \quad |x| <1$

\section{Anwendungen}
\subsection{Taylorsche Formel, Taylor-Reihe}

\emph{Problem:} „Komplizierte“ Funktionen $f$ soll in der Umgebung von $x_0$ durch ein Polynom $p_n$ $n$-ten Grades angenähert werden.\\
\emph{Ansatz:} $p_n(x)=a_0 + a_1 (x-x_0) + a_2(x-x_0)^2+...+ a_n (x-x_0)^n$\\
\emph{Forderung:} $p_n(x_0)= f(x_0), \; p_n'(x_0)=f'(x_0), \; p_n''(x_0)=f''(x_0), ...$\\
liefert: $p_n(x_0)=a_0, \; p_n'(x_0)=a_1, \; p_n''(x_0)=2a_2, ...$\\
und $a_k=\frac{f^{(k)}(x_0)}{k!}$.\\
Allgemein: $\boxed{p^{(k)}_n = k! a_k} \quad \text{für }k=0,1,...,n$
\paragraph{Def. 1:} Das Polynom $p_n(x)=f(x_0)+\frac{f'(x_0)}{1!}(x-x_0)+\frac{f''(x_0)}{2!}(x-x_0)^2+...+\frac{f^{(n)}(x_0)}{n!}(x-x_0)^n$ heißt \emph{Taylorpolynom} $n$-ten Grades mit Entwicklungsstelle $x_0$.

\subparagraph{Diskussion:} 
\begin{enumerate}
\item $p_n$ ist eine Näherung für $f$.\\
Fehler: $f(x)-p_n(x)=: R_n(x)$ heißt Restglied
\item Restglied ist im Allgemeinen umso kleiner, je kleiner $|x-x_0|$ ist und je größer $n$ ist.\\
ABB 54
\end{enumerate}
\paragraph{Satz 1:} Taylorsche Formel\\
Es sei $f$ in $[a,b]$ $(n+1)$-mal differenzierbar, sowie $x_0,x \in [a,b]$. Dann existiert ein $\xi$ zwischen $x_0$ und $x$ (d.h. $\xi = x_0 + \vartheta (x-x_0)$ mit $\vartheta \in (0,1)$) mit $R_n (x) = \frac{f^{n+1}(\xi)}{(n+1)!}(x-x_0)^{n+1}$: \emph{Restgliedform von Lagrange}.\\
Es gilt also $f(x) = \underbrace{\sum_{k=0}^n \frac{f^{(k)}(x_0)}{k!}(x-x_0)^k}_{p_n(x)} +\underbrace{\frac{f^{(n+1)}(x_0+\vartheta(x-x_0))}{(n+1)!}(x-x_0)^{n+1}}_{R_n(x)}$
\subparagraph{Diskussion:}
Spezialfall $n=0$: $f(x)=f(x_0)+f'(\xi)(x-x_0)$ (\emph{Mittelwertsatz der Differentialrechnung})\\
ABB 55\\
Satz sagt: es gibt zwischen $x_0$ und $x_1$ einen Punkt auf der Funktion, sodass die Senkante die Tangente dieses Punktes ist. \\
Umstellen liefert: $\underbrace{f'(\xi)}_{\text{Anstieg der Tangente}}= \underbrace{\frac{f(x)-f(x_0)}{x-x-}}_{\text{Anstieg der Sekante}}$

\subparagraph{Bsp. 1:} $f(x) = e^x$ \qquad $x\in \mathbb{R}$
\begin{align*}
f'(x) &= e^x = f''(x) = f'''(x) = ...\\
\overset{x_0=0}{\Longrightarrow} f'(0) &= 1 = f''(x) = f'''(x) = ... 
\end{align*}
$\Rightarrow e^x = \sum_{k=0}^n \frac{1}{k!}\cdot x^k + \frac{e^{\vartheta x}}{(n+1)!}x^{n+1} \quad 0 < \vartheta <1$\\
Wie gut ist diese Näherung?\\
Für $x=\frac{1}{10}=0,1$ und $n=4$ gilt: \\
$e^{0,1}=1+\frac{0,1}{1!}+\frac{0,1^2}{2!}+\frac{0,1^3}{3!}+\frac{0,1^4}{4!}+\underbrace{\frac{0,1^5}{5!}e^{\vartheta \cdot 0,1}}_{R_4(0,1)}$ für ein $\vartheta \in (0,1)$.\\
$\Rightarrow e^{0,1} = \underbrace{1 + 0,1 + 0,005 + 0,0001\overline{6} + 0,0000041\overline{6}}_{=1,1051708\overline{3}}+R_4(0,1)$
Abschätzen des $\vartheta$:
\begin{alignat*}{3}
8,\overline{3}\cdot 10^{-8} = \frac{0,1^5}{5!}= \frac{0,1^5}{5!}e^0&< \frac{0,1^5}{5!}e^{\vartheta \cdot 0,1} &&< \frac{0,1^5}{5!} e^{1\cdot 0,1} < \frac{0,1^5}{5!}\cdot 3 = 25 \cdot 10^{-8}\\
\Rightarrow 1,1051708\overline{3} +8,\overline{3} &\leq e^{0,1} &&\leq 1,1051708\overline{3}+25\cdot 10^{-8} \\
1,10517091\overline{6} &\leq e^{0,1} &&\leq 1,10517108\overline{3}
%&&& =1,10517091808...
\end{alignat*}
\subparagraph{Bsp. 2:} 
\begin{alignat*}{3}
f(x) &= \cos (x), \; x_0 = 0 &&\Rightarrow f(x_0)=1\\
f'(x)&=-\sin x && \Rightarrow f'(x_0) = 0\\
f''(x) &= -\cos x && \Rightarrow f''(x_0)=-1\\
f'''(x) &= \sin x && \Rightarrow f'''(x_0) = 0 \\
f^{(4)}(x) &= \cos x && \Rightarrow f^{(4)}(x_0) = 1 \\
&...
\end{alignat*}
$n=2m+1$
\begin{align*}
\cos x &= \underbrace{1}_{f(x_0)}+\underbrace{0}_{\frac{f'(x_0)}{1!}(x-x_0)}\underbrace{-\frac{x^2}{2!}}_{\frac{f''(x_0}{2!}(x-x_0)^2}+0+\frac{x^4}{4!}+...+ (-1)^m \frac{x^{2m}}{(2m)!}+0+R_{2m+1}\\
&= 1 + \frac{x^2}{2!} + \frac{x^4}{4!} - \frac{x^6}{6!}+...+(-1)^m\frac{x^{2m}}{(2m)!}+(-1)^{m+1}\cos (\vartheta x) \frac{x^{2m+2}}{(2m+2)!}
\end{align*}
ABB 56\\
Näherung: $\cos x \equiv 1-\frac{x^2}{2}$ für $|x| \ll 1$\\
Fehler: $|R_3(x)| \leq \frac{x^4}{4!}$\\
Bsp.:\\
$\cos 5^\circ = \cos \left( \frac{\pi}{36}\right) = \underbrace{1-\frac{\pi^2}{2 \cdot 36^2}}_{0,9961923}+ R_3$\\
...\\
$|R_2| \leq \frac{\pi^4}{36^4\cdot 24}= 2,416\cdot 10^{-6}$\\
genau gilt: $\cos 5^\circ = 0,99619 $ (auf 5 Stellen genau)

\subparagraph{Bsp. 3:} $f(x)=(1+x)^\alpha$ mit $\alpha \in \mathbb{R}\setminus \{0\}$
\begin{align*}
f'(x)&=\alpha (1+x)^{\alpha-1}\\
f''(x)&=\alpha (\alpha -1)(1+x)^{\alpha -2}\\
&...\\
f^{(k)}(x)&=\alpha (\alpha -1)(\alpha -2)\cdot ... \cdot (\alpha -k+1)(1+x)^{\alpha-k}\\
&=\nok{\alpha}{k}\cdot k! (1+x)^{\alpha-k}\\
\end{align*}
wir betrachten $x_0=0$\\
$f(0)=1$, $f'(0)=\alpha$, $f''(0)=\alpha(\alpha-1)$, …, $f^{(k)}(0)=\nok{\alpha}{k}k!$\\
Erinnerung:\\
$\nok{n}{k}=
\begin{cases}
\frac{n!}{k!(n-k)!} &\text{falls }n,k \in \mathbb{N}, \; k \leq n\\
\frac{n\cdot (n+1)\cdot...\cdot (n-k+1)}{k!} & \text{kann für beliebige }n\in \mathbb{R} \text{ ausgewertet werden.}
\end{cases}$\\
$\Rightarrow (1+x)^\alpha = \sum_{k=0}^n \nok{\alpha}{k} x^k+\nok{\alpha}{n+1}(1+\vartheta x)^{\alpha -n-1}x^{n+1}$ mit $\vartheta\in (0,1)$
\subparagraph{Bsp. 4:} $f(x)$… Polynom $n$-ten Grades\\
$\Rightarrow f^{(n+1)}(x)=0$ für $x \in \mathbb{R}$\\
$\Rightarrow R_n(x)=0$ für $x \in \mathbb{R}$\\
$\Rightarrow$ Taylorpolynom stellt $f$ exakt dar (Entwicklung nach Potenzen von $(x-x_0)$)

\subsubsection{Taylor Reihen}
\paragraph{Satz 2:} Es sei $f$ auf $U(x_0)$ beliebig oft differenzierbar und es gelte $\lim_{n\to\infty} R_n(x)=0$.\\
Dann gilt $\boxed{f(x)=\sum_{k=0}^\infty \frac{f^{(k)}(x_0)}{k!}(x-x_0)^k}$.\\
Denn: Taylor-Formel sagt $f(x)=\sum_{k=0}^n \frac{f^{(k)}(x_0)}{k!}(x-x_0)^k + R_n(x)$. Mit $n\to \infty$ folgt die Behauptung.

\subparagraph{Bsp. 5:} $e^x=\sum_{k=0}^n \frac{x^k}{k!}+R_n(x)$ (vgl. Bsp. 1)\\
Es gilt $\lim_{n\to \infty}R_n(x) = 0$ für alle $x\in \mathbb{R}$.\\
\emph{Beweis:} Sei $x\in \mathbb{R}$ fest. Wähle $n_0$ so, dass $q:= \frac{|x|}{n_0}<1$.\\
$\Rightarrow$ für $n> n_0$ gilt:
\begin{align*}
|R_n(x)| &= \left| e^{\vartheta x} \cdot \frac{x^{n+1}}{(n+1)!}\right| \leq e^{|\vartheta x|} \cdot \frac{x^{n+1}}{(n+1)!}\leq e^{|x|} \cdot \frac{x^{n+1}}{(n+1)!}\\
&<e^{|x|}\cdot \frac{|x|}{1}\cdot \frac{|x|}{2}\cdot ... \cdot \frac{|x|}{n_0} \underbrace{\cdot \frac{|x|}{n_0} \cdot...\cdot \frac{|x|}{n_0}}_{(n-n_0+1)\text{ Faktoren}}\\
&= e^{|x|}\cdot \frac{|x|^{n_0}}{n_0!}\cdot q^{n-n_0+1} \to 0 
\end{align*}
$\Rightarrow e^x=\sum_{k=0}^\infty \frac{x^k}{k!}$ für alle $x \in (-\infty, \infty)$
\subparagraph{Bsp. 6:} $\cos x = \sum_{k=0}^m (-1)^k \frac{x^{2k}}{(2k)!}+R_{2m+1}(x)$ (vgl. Bsp. 2)\\
Ähnlich wie in Bsp. 5 kann man zeigen $\lim_{n\to \infty}R_{2m+1}(x)=0$ für alle $x\in \mathbb{R}$.\\
$\Rightarrow \cos x = \sum_{k=0}^\infty (-1)^k \frac{x^{2k}}{(2k)!} \qquad x \in (-\infty, \infty)$\\
Analog: $\sin x = \sum_{k=0}^\infty (-1)^k \frac{x^{2k+1}}{(2k+1)!} \qquad x \in (-\infty, \infty)$

\subparagraph{Bsp. 7:} Restglieduntersuchung in Bsp. 3 führt zu:\\
$(1+x)^\alpha = \sum_{k=0}^\infty \nok{\alpha}{k}x^k \qquad |x|<1, \; \alpha \in \mathbb{R}$\\
z.B. für $\alpha = \frac{1}{2}$: 
\begin{align*}
\sqrt{1+x}&=1+\frac{1}{2}x-\frac{1}{8}x^2+\frac{1}{16}x^3-... \\
&\approx 1+\frac{1}{2}x \qquad \text{falls }|x|\ll 1
\end{align*}

\subsection{Grenzwertbestimmung mittels der Regel von l'Hopital}
\paragraph{Satz 3:} (Regel von l'Hopital)\\
Es gelte: 
\begin{enumerate}
\item $\lim_{x\to a} f(x) = 0$ und $\lim_{x\to a} g(x)=0$.
\item $\lim_{x\to a} \frac{f'(x)}{g'(x)}$ existiert (als eigentlicher und uneigentlicher Grenzwert).
\end{enumerate}
Dann folgt:
$\boxed{\lim_{x\to a} \frac{f(x)}{g(x)}=\lim_{x \to a} \frac{f'(x)}{g'(x)}}$ $\left(\text{Typ: }\mq{\frac{0}{0}}\right)$\\
Die gleiche Aussage gilt, wenn 1.) ersetzt wird durch
\begin{enumerate}
\item[1'.)] $\lim_{x\to a}f(x)=\pm \infty$, $\lim_{x\to a}g(x)=\pm \infty$ $\left(\text{Typ: }\mq{\frac{\infty}{\infty}}\right)$
\end{enumerate}
\emph{Beweis:} seien $f,g,f',g'$ stetig in $x_0$ und $g'(x_0)\not = 0$\\
Mittelwertsatz: $\frac{f(x)}{g(x)}=\frac{\overbrace{f(x_0)}^{0}+f'(\xi)(x-x_0)}{\underbrace{g(x_0)}_{0}+g'(\xi)(x-x_0)}=\frac{f'(\xi_1)}{g'(\xi_2)}\overset{x\to x_0}{\longrightarrow}\frac{f'(x_0)}{g'(x_0)}$

\subparagraph{Bsp. 8:}
\begin{enumerate}[label=\alph*.)]
\item $\lim_{x\to 1} \frac{\ln x}{x-1}=\mq{\frac{0}{0}}$\\
$\lim_{x\to 1} \frac{\frac{1}{x}}{1}=\lim_{x\to 1} \frac{1}{x}=1$\\
$\Rightarrow \lim_{x \to 1} \frac{\ln x}{x-1}=1$
\item $\lim_{x \to \infty} \frac{\ln x}{\sqrt{x}}=\mq{\frac{\infty}{\infty}}$\\
$\lim_{x\to \infty} \frac{\frac{1}{x}}{\frac{1}{2}x^{-\frac{1}{2}}}=\lim_{x \to \infty} \frac{2x^{\frac{1}{2}}}{x}=\lim_{x\to \infty} \frac{2}{\sqrt{x}}=0$\\
$\Rightarrow \lim_{x\to \infty} \frac{\ln x}{\sqrt{x}}=0$
\item $\lim_{x\to 0} \frac{x^2}{1-\cos x} = \mq{\frac{0}{0}}$\\
$\lim_{x\to 0}\frac{2x}{\sin x}=\mq{\frac{0}{0}}$\\
$\lim_{x\to 0}\frac{2}{\cos x}=2$\\
$\Rightarrow \lim_{x\to 1} \frac{\ln x}{x-1}=\lim_{x\to 0}\frac{2x}{\sin x}=2$\\
Regel also auch mehrfach hintereinander anwendbar.
\item $\lim_{x\to \infty} \frac{\sinh (x+1)}{\cosh x}=\mq{\frac{\infty}{\infty}}$\\
$\lim_{x\to \infty} \frac{\cosh(x+1)}{\sinh x}=\mq{\frac{\infty}{\infty}}$\\
$\lim_{x\to \infty} \frac{\sinh (x+1)}{\cosh x} = \mq{\frac{\infty}{\infty}}$\\
…\\
$\Rightarrow$ Satz nicht anwendbar, da 2.) nie erfüllt ist.\\
Aber: \\
$\lim_{n\to \infty} \frac{\sinh (x+1)}{\cosh x}=\lim_{x\to \infty} \frac{e^{x+1}-e^{-(x+1)}}{e^x+e^{-x}}=\lim_{x\to \infty} \frac{e^{x+1}\left(1-e^{-2(x+1)} \right)}{e^x \left( 1+ e^{-2x}\right)}=e \underbrace{\lim \frac{1-e^{-2(x+1)}}{1+e^{-2x}}}_{=1}=e$
\end{enumerate}
\subparagraph{Diskussion:}
\begin{enumerate}
\item Man beachte, dass der Anwendung von Satz 3 Zähler und Nenner einzeln differenziert werden (keine Quotientenregel)!
\item Falls $\lim_{x\to a} \frac{f'(x)}{g'(x)}$ nicht existiert, \emph{darf man nicht} schlussfolgern, dass $\lim_{x\to a} \frac{f(x)}{g(x)}$ nicht existiert (siehe Bsp. 9).
\end{enumerate}

\subparagraph{Bsp. 9:} $\lim_{x\to \infty} \frac{5x+\sin x}{3x-\cos x} = \mq{\frac{\infty}{\infty}}$\\
$\lim_{x\to\infty} \frac{5+\cos x}{3+\sin x}$ existiert nicht.\\
1.) erfüllt, 2.) nicht erfüllt $\Rightarrow$ Satz nicht anwendbar\\
Aber:\\
$\frac{5x+\sin x}{3x-\cos x}=\frac{x\left( 5 + \frac{\sin x}{x}\right)}{x\left(3-\frac{\cos x}{x}\right)}=\frac{5+\frac{\sin x}{x}}{3-\frac{\cos x}{x}}\overset{x \to \infty}{\longrightarrow} \frac{5}{3}$\\
Weitere unbestimmte Ausdrücke:\\
Durch Zurückführen auf $\mq{\frac{0}{0}}$ oder $\mq{\frac{\infty}{\infty}}$ lässt sich auch folgendes behandeln:\\
$\mq{0\cdot \infty}$: $f(x)\cdot g(x)$ als Doppelbruch schreiben, d.h. $\frac{f(x)}{\frac{1}{g(x)}}$ oder $\frac{g(x)}{\frac{1}{f(x)}}$ ist dann vom Typ $\mq{\frac{0}{0}}$ oder $\mq{\frac{\infty}{\infty}}$.\\
$\mq{\infty - \infty}$: Ausklammern $f(x)-g(x)=f(x)\left( 1 - \frac{g(x)}{f(x)}\right)$ oder falls Brüche vorliegen Hauptnenner bilden.\\
$\mq{0^0} / \mq{1^\infty} / \mq{\infty^0}$: Umformung 
\begin{align*}
\lim_{x \to a} f(x)^{g(x)} &= \lim_{x\to a} \exp\left(\ln\left(f(x)^{g(x)}\right)\right)\\
&= \lim_{x\to a} \exp\left(g(x) \ln f(x) \right)\\
&= \exp \left( \lim_{x\to a} \underbrace{g(x) \cdot \ln f(x)}_{\text{Typ }\mq{0\cdot \infty}}\right)\\
\end{align*}
\subparagraph{Bsp. 10:}
\begin{enumerate}[label=\alph*.)]
\item $\lim_{x\to 0} \tan x \cdot \cot 3x \overset{\mq{0 \cdot \infty}}{=}\lim_{x\to 0} \frac{\tan x}{\frac{1}{\cot 3x}}$\\
$=\lim_{x\to 0} \frac{\tan x}{\tan 3x}\overset{\mq{\tfrac{0}{0}}}{=}\lim_{x\to 0} \frac{1+\tan^2x}{3(1+\tan^23x)}=\resultul{\frac{1}{3}}$
\item $\lim_{x \to 0} \left( \frac{1}{\sin x}-\frac{1}{e^x-1}\right)\overset{\mq{\infty - \infty}}{=} \lim_{x\to 0} \frac{e^x-1-\sin x}{\sin x \cdot (e^x-1)}\overset{\mq{\tfrac{0}{0}}}{=}... \\
= \lim_{x\to 0} \frac{e^x - \cos x}{\cos x (e^x+1)+\sin (x) \cdot e^x}\\
\overset{\mq{\tfrac{0}{0}}}{=}\lim_{x \to 0}\frac{e^x + \sin x}{-\sin x \cdot  (e^x-1) + \cos (x) \cdot  e^x+\cos (x) \cdot  e^x + \sin(x) \cdot e^x }= \resultul{\frac{1}{2}}$
\item $\lim_{x\to 0 } (1-x)^{\frac{1}{x}} \overset{\mq{1^\infty}}{=}\lim_{x\to 0} \left( \ln\left( (1-x)^{\frac{1}{x}}\right)\right)=\underbrace{\lim_{x\to 0} \exp}_{\text{tauschen geht, da $\exp(\cdot)$ stetig ist}}\left(\frac{\ln(1-x)}{x}\right)\\
= \exp\underbrace{\left(\lim_{x\to 0}\frac{\ln (1-x)}{x}\right)}_{Typ \mq{\frac{0}{0}}}$\\
Denn: $\lim_{x\to 0} \frac{\ln (1-x)}{x}=\lim_{x\to 0} \frac{-\frac{1}{1-x}}{1}=\lim_{x\to 0} - \frac{1}{1-x}=-1$
\end{enumerate}

\subsection{Kurvendisskusion}\label{3.3.3}
\emph{Problemstellung:} Gegeben ist eine Funktion $f: Db(f) \to \mathbb{R}$ $Db(f) \subseteq \mathbb{R}$.\\
Dann ist der Graph der Funktion definiert durch: $\{ (x,f(x))\in \mathbb{R}^2| x \in Db(f)\}$.\\
Dieser Graph ist zu untersuchen auf 
\begin{enumerate}[label=\alph*.)]
\item Nullstellen
\item Stellen lokaler bzw. globaler Extrema
\item Wendestellen
\item Verhalten im Unendlichen, bzw. an den Randstellen des Definitionsbereichs $Db(f)$ und (falls vorhanden) bei Annäherung an Unstetigkeitsstellen.
\end{enumerate}
\subparagraph{Diskussion:} 
\begin{enumerate}
\item $x_0 \in Db(f)$ heißt \emph{Nullstelle $n$-ter Ordnung}, falls \\
$f(x_0) = f'(x_0)=...=f^{(n-1)}(x_0)=0 \wedge f^n (x_0) \not = 0$.\\
Zur Nullstellenbestimmung lernen wir bald das (iterative) Newton-Verfahren kennen.
\item \emph{Lokale Extrema} sind extremal bzgl. einer Umgebung der Extremstelle.\\
\emph{Globale Extrema} sind extremal bzgl. des gesamten Definitionbereichs, sie sind lokale Extram oder Funktionswerte in den Randpunkten.
\item \emph{Wendepunkte} sind Punkte, an denen die Kurve von konkav in konvex oder von konvex in konkav übergeht.\\
ABB 64\\
ABB 65
\item Einige einfache Zusammenhänge zwischen Eigenschaften der Kurve und der Ableitungen an der Stelle $x_0$ ($f$ sei auf $U(x_0)$ hinreichend oft differenzierbar).\\
\begin{tabular}{l c l }
$f'(x_0)<0$ & $\Rightarrow$ & $f$ in Umgebung von $x_0$ streng monoton fallend.\\
$f'(x_0)>0$ & $\Rightarrow$ & $f$ in Umgebung von $x_0$ streng monoton wachsend.\\
$f'(x_0)=0$ & $\Leftarrow$ & $f$ in $x_0$ lokal extremal.\\
\hline
$f''(x_0)<0$ & $\Rightarrow$ & $f$ in Umgebung von $x_0$ konkav.\\
$f''(x_0)>0$ & $\Rightarrow$ & $f$ in Umgebung von $x_0$ konvex.\\
$f''(x_0)=0$ & $\Leftarrow$ & $x_0$ Wendestelle.\\
\hline 
$f'(x_0)=0 \wedge f''(x_0)<0$ & $\Rightarrow$ & $f$ in $x_0$ lokal minimal\\
$f'(x_0)=0 \wedge f''(x_0)>0$ & $\Rightarrow$ & $f$ in $x_0$ lokal maximal\\
\end{tabular}
\item Problem: $f'(x_0)=0 \wedge f''(x_0)=0$ ?\\
Extremstelle oder Wendestelle oder was?
\end{enumerate}
\subsubsection*{Hinreichende Bedingungen für das Vorliegen von Extremstellen}
\paragraph{Satz 4:} Sei $f:Db(f) \to \mathbb{R} , \; Db(f) \subseteq \mathbb{R}$ eine in $x_0 \in Db(f)$ $n$-mal differenzierbare Funktion und sei $f^{(n)}$ stetig in $x_0$. Dann gilt falls $f'(x_0)=f''(x_0)=...=f^{(n-1)}(x_0)=0 \wedge f^{(n)}(x_0)\not = 0$:
\begin{enumerate}[label=\alph*.)]
\item $n=2,4,6,...$ (also gerade), so ist $x_0$ lokale Extremstelle (Maximum falls $f^{(n)}(x_0)<0$, Minimum falls $f^{(n)}(x_0)>0$).
\item $n=3,5,7,...$ (also ungerade), so ist $x_0$ eine Horizontal-Wendestelle (konvex$\to$konkav, falls $f^{(n)}(x_0)<0$; konkav$\to$konvex, falls $f^{(n)}(x_0)>0$).\\
ABB 66
\end{enumerate}
Beweis mittels Taylor-Formal.\\
Oft ist auch folgendes Kriterium nützlich:
\paragraph{Satz 4':} Sei $f:Db(f)\to \mathbb{R}, Db(f) \subseteq \mathbb{R}$ differenzierbar und $x_0\in Db(f)$, sowie $f'(x_0)=0$. Dann:
\begin{enumerate}[label=\alph*.)]
\item $f'$ wechselt bei $x_0$ das Vorzeichen $\begin{cases}
\text{von }+ \text{ auf }- \Rightarrow x_0 \text{ lokale Maximumstelle}\\
\text{von }- \text{ auf }+ \Rightarrow x_0 \text{ lokale Minimumstelle}
\end{cases}$
\item kein Vorzeichenwechsel $\Rightarrow x_0$ ist Horizontal-Wendestelle
\end{enumerate}
\subsubsection*{Hinreichende Bedingung für das Vorliegen einer Wendestelle}

\paragraph{Satz 5:} Sei $f: Db(f) \to \mathbb{R}, Db(f) \subseteq \mathbb{R}$ $n$-mal differenzierbar an $x_0$ und $f^{(n)}$ stetig in $x_0$. Dann gilt falls $f''(x_0)=f'''(x_0)=...=f^{(n-1)}(x_0)=0 \wedge f^{(n)}(x_0)\not = 0$ und
\begin{enumerate}[label=\alph*.)]
\item $n=3,5,7,... \Rightarrow x_0$ ist Wendestelle $\begin{cases}
f^{(n)}(x_0)<0 & konvex \Rightarrow konkav\\
f^{(n)}(x_0)>0 & kankav \Rightarrow konvex
\end{cases}$
\item $n=4,6,8,... \Rightarrow x_0$ keine Wendestelle, sondern sogenannte Flachstelle und Extremstelle, falls zusätzlich $f'(x_0) = 0$.\\
ABB 67
\end{enumerate}
Analog zu Satz 4 und 4' gibt es auch für Wendestellen ein alternatives hinreichendes Kriterium:
\paragraph{Satz 5':} Es sei $f$ eine 2 mal differenzierbare Funktion (in Umgebung von $x_0$), und es gelte $f''(x_0)=0$. Dann:
\begin{enumerate}[label=\alph*.)]
\item $f''$ wechselt bei $x_0$ das Vorzeichen $\begin{cases}
\text{von } + \text{ auf } -: \text{(konvex}\to \text{konkav) Wendestelle}\\
\text{von } - \text{ auf } +: \text{(konkav}\to \text{konvex) Wendestelle}\\
\end{cases}$
\item kein Vorzeichenwechsel $\Rightarrow$ keine Wendestelle (sondern Flachstelle)
\end{enumerate}
\emph{Bemerkung} (zu Satz 4' und 5'):\\
Vorzeichenwechsel von $f'$ bzw. $f''$ bei $x=x_0 \Leftrightarrow f'$ bzw. $f''$ hat bei $x_0$ Nullstelle ungerader Ordnung.

\subsection[Kurvendarstellungen]{Kurvendarstellungen, Tangenten- und Normalengleichungen, Krümmung}
\subsubsection{Darstellung ebener Kurven}
\begin{enumerate}
\item \emph{Explizite karthesische Darstellungen} $y=f(x)$\\
Wobei $f: \mathbb{R}\to \mathbb{R}$ (vgl. Abschnitt \ref{3.3.3}).
\item \emph{Implizite karthesische Darstellungen} $F(x,y)=0$\\
Für graphische Darstellung ungünstig. Unter bestimmten Voraussetzungen lässt sich $F(x,y)=0$ auflösen nach $y$ (oder $x$). Mehr dazu im Kapitel \ref{5} (Differentialrechnung für Funktionen mehrer Veränderlicher).
\item \emph{Parameter Darstellung} $x=x(t), y=y(t), t\in I$ (kurz PD)\\
vektorielle Form: $\vec{r}=\mtr{x//y}=\mtr{x(t)\\y(t)}, t \in I$
\subparagraph{Bsp. 13:}\parskp
$x=a \cos t$\\
$y=b \sin t$\\
$t\in [0,2\pi) \quad a,b>0$
\begin{align*}
t=0 &\Rightarrow x(0)=a, \; y(0)=0\\
t=\frac{\pi}{2} &\Rightarrow x\left(\frac{\pi}{2}\right)=0, \; y\left(\frac{\pi}{2}\right)=b\\
t=\pi &\Rightarrow x\left(\pi\right)=-a,\; y(\pi)=0
\end{align*}
Dies ergibt eine Ellipse.\\
ABB R1\\
Übergang zur Parameterfreien Darstellung: $t$ eleminieren.\\
$\frac{x}{a}=\cos t, \; \frac{y}{b}=\sin t \qquad|$ Quadrieren und Addieren\\
$\boxed{\frac{x^2}{a^2}+\frac{y^2}{b^2}=\cos^2t+\sin^2t=1}$
\subparagraph{Bsp. 14:} Kreis mit Mittelpunkt $M=(x_0, y_0) $ und Radius $R$.\\
PD bspw.: $x=x_0+r\cos t \qquad y=y_0+R \sin t \qquad t \in [0,2\pi)$\\
Parameterfreie Darstellung: \\
$(x-x_0)^2+(y-y_0)^2=R^2$\\
ABB R2
\item Explizite Darstellung in Polar-Koordinaten
\begin{itemize}
\item Darstellung eines Punktes in der Ebene\\
ABB 72\\
$x,y$ … karthesische Koordinaten\\
$r, \varphi$ … Polarkoordinaten von $P$ (analog Betrag und Argument einer komplexen Zahl) $r\geq 0, \; \varphi\in \mathbb{R}$\\
Umrechnung:
\begin{align*}
x &= r \cdot \cos \varphi\\
y &= r \cdot \sin \varphi
\end{align*}
\item Kurvendarstellung $r=r(\varphi) \quad , \; \varphi \in [\alpha, \beta]$\\
Bsp.: $r(\varphi)=2, \; \varphi\in [0,2\pi)$\\
ABB 73\\
Für jeden Winkel $\varphi \in [\alpha, \beta]$ die Strecke $r(\varphi)$ auf den $\varphi$ entsprechenden Strahl von $0$ abtragen.
\subparagraph{Bsp. 15:}$r=r(\varphi)=8 \cos \varphi \quad, \; \left(0\leq \varphi \leq \frac{\pi}{2}\right)$\\
\begin{tabular}{l | c c c c c c c c c c c c c}
$\varphi$ & $0^\circ$ & $15^\circ$ & $30^\circ$ & $45^\circ$ & $60^\circ$ & $75^\circ$ & $90^\circ$\\
\hline
$8 \cos \varphi$ & $8$ & $7,73$ & $6,92$ & $5,66$ & $4$ & $2,07$ & $0$
\end{tabular}\\
ABB R3
\end{itemize}
\end{enumerate}
\paragraph{Bemerkung}
\begin{itemize}
\item Übergang „explizite Darstellung $\to$ Parameterdarstellung“\\
$y=f(x), \; x\in [a,b]\\
\Rightarrow x=t, \; y=f(t), \; t\in [a,b]$ ($t$ als Parameter)
\item Übergang „explizite Polardarstellung $\to$ Parameterdarstellung“\\
$r=r(\varphi), \; \varphi\in [a,b]\\
\Rightarrow x=r(\varphi)\cos \varphi, \; y=r(\varphi) \sin \varphi , \; \varphi \in [a,b]$ ($\varphi$ als Parameter)
\end{itemize}
\subparagraph{Im Bsp. 15:} \parskp
$x=8\cos^2 \varphi$\\
$y=8\cos \varphi \sin \varphi \qquad \varphi \in \left[0,\frac{\pi}{2}\right]$\\
Parameterfreie Darstellung:
\begin{align*}
y^2&=64\underbrace{\cos^2 \varphi}_{\tfrac{x}{8}} \underbrace{\sin^2\varphi}_{1-\tfrac{x}{8}}\\
&=x(8-x)\\
&\Rightarrow x^2-8x+y^2=0\\
&\Rightarrow (x-4)^2+y^2=4^2
\end{align*}
(Halb-)Kreis mit Radius $4$ und Mittelpunkt $(4,0)$.
\subsubsection{Tangenten und Normalen ebener Kurven}
\begin{itemize}
\item Anstieg $y'$ einer in PD gegebener Kurve $x=x(t), \; y=y(t), \; t\in I$.\\
Dazu sei $y=f(x)$ die explizite karthesiche Form (ohne die Elimination von $t$ durchzuführen).\\
$\Rightarrow \frac{\mathrm{d}y}{dt}=\frac{\mathrm{d}y}{\mathrm{d}x}\cdot \frac{\mathrm{d}x}{dt}$ (Kettenregel)\\
In Anwendungen in $t$ oft die Zeit, üblicher Weise schreibt man dann:\\
$\frac{\mathrm{d}x}{dt}=:\dot{x} \quad \frac{\mathrm{d}y}{dt}=:\dot{y} \quad \Rightarrow y'=\frac{\dot{y}}{\dot{x}}$\\
$\frac{d^2x}{dt^2}=:\ddot{x} \dots $
\item Tangente im Punkt $P_0=(x_0,y_0), \; x_0 = x(t_0), \; y_0=y(t_0)$\\
ABB 74\\
(Ein) Richtungsvektor der Tangente in $x_0, y_0$ ist gegeben durch $\vec{t}=\mtr{\dot{x}(t_0)\\ \dot{y}(t_0)}$.\\
Für $\vec{n}=\vec{n}(t_0)=\mtr{-\dot{y}(t_0)\\ \dot{x}(t_0)}$ gilt $(\vec{t}, \vec{n})=0$. Also ist $\vec{n}\perp \vec{t}$ und $\vec{n}$ ist daher ein Richtungsvektor.\\
\begin{tabular}{L{.18} | l | L{.18} | l}
Kurve & $y=f(x),\;x\in I$ & $x=x(t)$\newline $y=y(t), \; t\in I$ & $r(\varphi), \; \varphi \in I$ \\
\hline 
Punkt \newline$P_0=(x_0,y_0)$ & $P_0=(x_0,f(x_0))$ & $P_0=(x(t_0),y(t_0))$& $P_0=(r(\varphi_0)\cdot \cos\varphi_0 , \; r(\varphi_0) \cdot \sin \varphi_0$\\
Anstieg $m=\tan\alpha$ in $P_0$ & $f'(x_0)$ & $\frac{\dot{y}(t_0)}{\dot{x}(t_0)}$ & $\frac{r'(\varphi_0)\sin\varphi_0 + r(\varphi_0)\cos \varphi_0}{r'(\varphi_0)\cos\varphi_0 - r(\varphi_0)\sin \varphi_0}$\\
Tangenten-\newline vektor $\vec{t}$ & $\mtr{1\\ f'(x_0)}$ & $\mtr{\dot{x}(t_0) \\ \dot{y}(t_0)}$ & $\mtr{r'(\varphi_0)\cos\varphi_0 - r(\varphi_0)\sin \varphi_0\\r'(\varphi_0)\sin\varphi_0 + r(\varphi_0)\cos \varphi_0}$\\
Normalen-\newline vektor $\vec{n}$ & $\mtr{-f'(x_0) \\ 1}$ & $\mtr{-\dot{y}(t_0)\\\dot{x}(t_0)}$ & $\mtr{-r'(\varphi_0)\sin\varphi_0 - r(\varphi_0)\cos \varphi_0\\ r'(\varphi_0)\cos\varphi_0 - r(\varphi_0)\sin \varphi_0}$ \\
\end{tabular}
\end{itemize}
Tangentengleichungen:\\
$y=y_0+m(x-x_0)$\\
$\vec{r}=\mtr{x\\y}=\mtr{x_0\\y_0}+s \cdot t\vec{t} \quad s\in \mathbb{R}$\\
Normalengleichungen:\\
$y=y_0-\frac{1}{m}(x-x_0)$\\
$\vec{r}=\mtr{x\\y}=\mtr{x_0\\y_0}+s\cdot \vec{n} \quad s \in \mathbb{R}$

\paragraph{Bsp. 16:} Für welche Werte des Parameters $\varphi$ ist die Tangente an die Kurve $r=r(\varphi)=a(1+\cos \varphi), \; \varphi \in [0,2\pi)$ parallel zur y-Achse?\\
Lösung: $r'(\varphi)=-a\sin \varphi$ mit der Bedingung $r'(\varphi)\cdot \cos \varphi - r(\varphi)\cdot \sin \varphi = 0$\\
$\Rightarrow -a \sin \varphi \cos \varphi - a(1+\cos \varphi)\cdot \sin \varphi = 0$\\
$\Rightarrow -a \sin \varphi(\cos \varphi + 1 + \cos \varphi) = 0$\\
$\Rightarrow \sin \varphi = 0 \vee \cos \varphi = -\frac{1}{2}$\\
$\Rightarrow \varphi_1=0^\circ, \; \varphi_2=180^\circ, \; \varphi_3 = 120^\circ, \; \varphi_4 = 240^\circ $\\
Allerdings entfällt $\varphi_2$, da $r'(\varphi_2)\sin\varphi_2+r(\varphi_2)\cos\varphi_2 = 0$

\subsubsection{Krümmung ebener Kurven}
ABB 75\\
Gegeben sei die Kurve $C$ und der feste Punkt $P_0=(x_0, y_0)$. Außerdem sind zwei Punkte $R$ und $S$ auf der Kurve gegeben. Durch 3 Punkte $P_0, \; R$ und $S$ im Allgemeinen eindeutig ein Kreis festgelegt.\\
Es sei $K$ die Grenzlage dieses Kreises, wenn $R$ und $S$ in $P_0$ übergeben.\\
Es heißt dann:\\
$K$… Krümmungskreis (Schmiegkreis)\\
$\varkappa$ (Kappa)… Krümmung\\
$\varrho$… Krümmungsradius mit $\varrho=\frac{1}{|\varkappa|}$\\
$M$… Mittelpunkt des Krümmungskreises $\overrightarrow{OM}=\mtr{x_0 \\ y_0}+\frac{1}{\varkappa}\cdot \frac{\vec{n}}{|\vec{n}|}$\\
Tabelle (Krümmungen)\\
\begin{tabular}{L{.2} | L{.3} | L{.199} | L{.3}}
Kurve & $y=f(x),\;x\in I$ & $x=x(t)$\newline $y=y(t), \; t\in I$ & $r(\varphi), \; \varphi \in I$ \\
\hline 
Krümmung $\varkappa$ in Punkt $P=(x,y)$ & $\varkappa=\frac{y''}{(1+(y')^2)^{\tfrac{3}{2}}}$ & $\varkappa=\frac{\dot{x}\ddot{y}-\ddot{x}\dot{y}}{(\dot{x}^2+\dot{y}^2)^{\tfrac{3}{2}}}$ & $\varkappa=\frac{r^2+2(r')^2-r\cdot r''}{(r^2+(r')^2)^{\tfrac{3}{2}}}$
\end{tabular}
\paragraph{Bsp. 17:} In welchem Punkt ist $f(x)=e^x$ am stärksten gekrümmt (d.h. maximiere $|\varkappa|$)\\
Lösung: $y'=e^x+y''$\\
$\varkappa=\frac{e^x}{(1+e^{2x}){\tfrac{3}{2}}}=|\varkappa|$\\
$\frac{d|\varkappa|}{\mathrm{d}x}=\frac{e^x(1+e^{2x})^{\tfrac{3}{2}}-e^x\cdot\frac{3}{2}(1+e^{2x})^{\tfrac{1}{2}}\cdot 2e^{2x}}{(1+e^{2x})^3}\overset{!}{=}0$\\
$\Rightarrow\underbrace{e^x(1+^{2x})^{\tfrac{1}{2}}}_{\not = 0}(1+e^{2x}-3e^{2x})=0$\\
$\Rightarrow 1-2e^{2x}=0$\\
$\Rightarrow x_1 = \frac{1}{2}\ln \tfrac{1}{2}=-\frac{1}{2}\ln 2 \qquad y_1=\sqrt{\frac{1}{2}}$\\
mit $\varkappa = \frac{\sqrt{\frac{1}{2}}}{\left(\sqrt{\frac{3}{3}}\right)^3}=\frac{2}{3\sqrt{3}}, \; \varrho=\frac{3\sqrt{3}}{2}$\\
ABB 76
\subsubsection{Raumkurven}
\begin{itemize}
\item \emph{Parameterdarstellung} $x=x(t),\;y=y(t), \;z=z(t), \;t \in I$\\
vektorielle Form: $\vec{r}=\vec{r}(t)=\mtr{x(t)\\y(t)\\z(t)}, \; t \in I, \; \vec{r}=\mtr{x\\ y \\ z}$
\item \emph{Tangente im Punkt} $P_0=(x(t_0),y(t_0),z(t_0))^T$\\
mit $\vec{r}(t_0)=\mtr{x(t_0)\\y(t_0)\\z(t_0)}, \; \dot{\vec{r}}(t)=\mtr{\dot{x}(t_0)\\\dot{y}(t_0)\\\dot{z}(t_0)}$ gilt $\vec{g}(s)=\vec{r}(t_0)+s\cdot \vec{\dot{r}}(t_0), \; s \in \RR$ ist die Tangente im Punkt $P_0$.
\item \emph{Physikalische Darstellung} $\vec{r}=\vec{r}(t), \; t \in I$ … Bewegung eines Massepunktes im Raum\\
$\vec{\dot{r}}(t_0)$ … Geschwindigkeit zur Zeit $t_0$\\
$\vec{\ddot{r}}(t_0)$ … Beschleunigung zur Zeit $t_0$
\item \emph{Krümmung} $\varkappa = \frac{|\dot{r}\times \ddot{r}|}{|\dot{r}|^3}$, Krümmungsradius $\varrho=\frac{1}{\varkappa}$
\end{itemize}
\paragraph{Bsp. 18:} (Schraubenlinie)\\
$\vec{r}=\vec{r}(t)=\mtr{a \cos (t)\\ a \sin (t) \\ \tfrac{h}{2\pi}t}\qquad t\geq 0, \; a>0, \; h>0$ ($h$ ist Abstand zwischen zwei Schraubenlinien)\\
Gesucht ist die Tangente in Punkt $P_0=(x(t_0), y(t_0), z(t_0))^T$ für $t_0=\frac{\pi}{2}$.\\
Tangente:
$\vec{g}(s)=\mtr{0 \\ a \\ \tfrac{h}{4}}+s\cdot \mtr{-a\\ 0 \\ \tfrac{h}{2\pi}}\qquad s \in R$\\ 
(da die y-Koordinate in $s\cdot \mtr{-a\\ 0 \\ \tfrac{h}{2\pi}}$ $0$ ist: $\vec{g}$ ist parallel zur x-z-Ebene)

\subsection{Newton-Verfahren zur Nullstellenbestimmung}
\paragraph{Satz 6:} Es sei $x^*$ eine Lösung der Gleichung $f(x)=0$. Für ein geeignetes Intervall $I=(x^*-r, x^*+r)$ gelte $f'(x)\not = 0$ und $\left|\frac{f(x)\cdot f''(x)}{(f'(x))^2}\right|\leq k < 1$ für alle $x \in I$.\\
Dann konvergiert für jeden Startwert $x_0\in I$ die mittels $x_{n+1}=x_n-\frac{f(x_n)}{f'(x_n)} \quad n=0,1,2,\cdots$ festgelegte Folge gegen $x^*$, d.h. $\lim_{n\to\infty}x_n=x^*$.\\
Außerdem gilt $|x^*-x_n|\leq \frac{k}{1-k}|x_{n+1}-x_n|\leq \frac{k^n}{1-k}|x_1-x_0|$.
\subparagraph{Diskussion:} 
\begin{itemize}
\item Geometrische Veranschaulichung:\\
ABB 78\\
Tangente in $P_0:$\\
$y=f(x_0)+f'(x_0)(x-x_0)$\\
$x_1$ … Nullstelle der Tangente\\
$0=f(x_0)+f'(x_0)(x_1-x_0)$\\
$x_1 = x_0-\frac{f(x_0)}{f'(x_0)}$\\
ABB R Newton 1.
\item Zur Wahl des Startwertes $x_0$:\\
Falls in $I$ gilt $f''(x)>0$, dann ist ein $x_0$ mit $f(x_0)>0$ günstig (bzw. bei $f''(x)<=$ ein $f(x_0)<0$).
\item Praktisches Vorgehen:\\
Abbruch falls $|x_{n+1}-x_n|<\varepsilon$.
\end{itemize}
\subparagraph{Bsp. 19:} Gesucht sind Lösungen von $f(x)=\cos (x)=x \Leftrightarrow x-\cos(x)=0$. Gesucht ist nun eine Nullstelle von $f$.\\
Start $x_0=0,8$ (nur ein Beispiel)\\
ABB 79\\
$f'(x)=1+\sin(x)$\\
$x_{n+1}=\frac{x_n-x_n-\cos(x_n)}{1+\sin(x_n)} \quad n=0,1,2, \dots$\\
\begin{tabular}{l | l}
$n$ & $x_n$\\
\hline 
$0$ & $0,8$\\
$1$ & $0,73985$\\
$2$ & $0,73908526$\\
$3$ & $0,73908513322$\\
$4$ & $0,73908513322\dots$\\
\end{tabular}\\
$\Rightarrow x^*=0,739085$\\
ABB R Newton 2. 

\chapter{Integralrechnung für Funktionen einer reellen Veränderlichen}
\section{Der Integralbegriff}
\subsection{Das bestimmte Integral}
\paragraph{Problem:} \parskp
\emph{Gegeben:} Kurve $y=f(x), \;x\in [a,b]$ und $f(x) \geq 0$.\\
\emph{Gesucht:} Flächeninhalt $I$ unter der Kurve\\
ABB 80\\
\emph{Vorgehen:} 
\begin{itemize}
\item Zerlegung $Z$ des Intervalls $[a,b]$:\\
$a=x_0<x_1<x_2<x_3<\dotsb<x_{n-1}<n_n=b$
\item In jedem Teilintervall Zwischenstelle $\xi_i\in [x_{in}, x_i]$ wählen. Dies ergibt die Zerlegung $Z^*$ ($Z$ mit Zwischenstellen).
\item $\Delta (Z^*):= \max_{i=1,\dots, n}(x_i-x_{i-1}=$ … Länge des größten Teilintervalls
\item Approximation von $I$ durch die Summe von Rechteckflächen: $$S(Z^*,f):=\sum_{i=1}^n f(\xi_i)(x_i-x_{i-1})$$
$S(Z^*,f)$ heißt Riemann-Summe. Sie ist abhängig von der Zerlegung $Z^*$.
\end{itemize}
\paragraph{Def. 1} Die Funtkion $f$ heißt (Riemann-)integrierbar über $[a,b]$ falls für jede Zerlegungsfolge $Z^*_\mu$ von $[a,b]$ mit $\lim_{\mu\to \infty}\Delta (Z^*_\mu)=0$ gilt: $\lim_{\mu\to\infty}S(Z^*_\mu, f) = I$. Die Zahl $I$ heißt dann bestimmtes Integral von $f$ über $[a,b]$. Bezeichnung: $i=\int_a^bf(x)\intd{x}$.
\subparagraph{Diskussion:}
\begin{itemize}
\item Def. 1 basiert auf der Forderung $f(x)\geq 0$. Falls $f(x)<0$ für alle $x\in [a,b]$, so gilt im Falle der Integrierbarkeit $\int_a^bf(x)\intd{x}<0$:\\
ABB 82\\
$\Rightarrow$ Flächeninhalt $F=\int_a^b|f(x)|\intd{x}=-\int_a^bf(x)\intd{x}$.
\item Man definiert:\\
$\int_a^af(x)\intd{x}:=0$\\
$\int_b^af(x)\intd{x}:= - \int_a^bf(x)\intd{x} \quad (b>a)$
\item Eigenschaften des bestimmten Integrals:\\
$\int_a^bf(x)\intd{x}=\int_a^cf(x)\intd{x}+\int_c^bf(x)\intd{x}$\\
für beliebige $a, b, c \in \RR$.
\item $\int_a^bc_1 u(x) + c_2 v(x) \intd{x} = c_1 \int_a^bu(x)\intd{x}+c_2 \int_a^b v(x) \intd{x}$ für $c_1, c_2 \in \RR$
\end{itemize}
\paragraph{Satz 1:} Es sei $f: [a,b] \to \RR$ stetig. Dann ist $f$ auf $[a,b]$ integrierbar. 
\subparagraph{Diskussion:} 
\begin{itemize}
\item Falls $f$ stückweise stetig ist, mit endlich vielen Sprungstellen, so ist $f$ ebenfalls integrierbar (Integration von Sprungstelle zu Sprungstelle).\\
ABB 93
\item Nicht integrierbar ist bspw. $f: [0,1] \to \RR, \; f(x) = \begin{cases}
1 & x \text{ irrational}\\
0 & x \text{ rational}
\end{cases}$
\end{itemize}
\subsection{Sammfunktion und unbestimmtes Integral}
\paragraph{Satz 2:} (Mittelwertsatz der Integralrechnung)\\
Sei $f: [a,b]\to \RR$ stetig. Dann existiert (mindestens) ein $\xi \in (a,b)$ mit: 
\[\int_a^b f(x)\intd{x}= f(\xi) (b-a)\]
Anschaulich:\\
ABB 94\\
Wir nennen $m=\frac{1}{b-a}\int_a^b f(x) \intd{x}$ den Integralmittelwert von $f$ auf $[a,b]$.\\
\emph{Integral mit variabler oberer Grenze:}\\
Wir betrachten $\int_a^x f(t) \intd{t} =: F(x)$\\
ABB 95
\paragraph{Satz 3:} Sei $f:[a,b]\to \RR$ stetig. Dann ist $F(x)=\int_a^x f(t)\intd{t}$ auf $[a,b]$ differenzierbar und es gilt:
\[F'(x)=f(x)\]
Beweis:\\
$\frac{F(x+h)-F(x)}{h}=\frac{\int_x^{x+h} f(t) \intd{t}}{h}\overset{\substack{\text{Satz 2}\\\text{(mit }\xi\in (x,x+h))}}{=}\frac{f(\xi)\cdot (x+h-x)}{h}=f(\xi)\overset{h\to 0}{\longrightarrow} f(x)$ da $f$ stetig.\\
$\Rightarrow F'(x) = f(x)$
\paragraph{Def. 2:} Die Funktion $F$ heißt \emph{Stammfunktion} von $f$ (auf $[a,b]$), wenn gilt $F'(x)=f(x)$.
\subparagraph{Diskussion:} Ist $F$ eine Stammfunktion, so ist auch $\tilde{F}$ mit $\tilde{F}(x)=F(x)+C$ eine Stammfunktion.
\paragraph{Def. 3:} Die Menge $\{F(x)+C|C\in \RR$ aller Stammfunktionen von $f$, wobei $F$ beliebige Stammfunktion von $f$ ist, heißt unbestimmtes Integral von $f$.\\
Bezeichunung: $\int f(x) \intd{x}=F(x)+C$
\subsection{Hauptsatz der Differential- und Integralrechnung (HDI)}
\paragraph{Satz 4:} Sei $f:[a,b]\to \RR$ stetig und $F$ beliebige Stammfunktion von $f$.
\[\int_a^b f(x) \intd{x}=\big[F(x)\big]_a^b=F(b)-F(a)\]
Beweis: Satz 3 liefert $F_1(x):=\int_a^x f(t) \intd{t}$ ist Stammfunktion von $f$. Also gilt $F(x)=F_1(x)+k$\\
$\Rightarrow F(b)-F(a)=F_1(b)+k-\underbrace{F_1(a)}_{=0}-k = \int_a^bf(t)\intd{t}$
\subparagraph{Diskussion:}
\begin{enumerate}
\item $\underbrace{\int_a^b f(x) \intd{x}}_{\substack{\text{Flächeninhaltsproblem,}\\\text{ Integralrechnung}}}=\underbrace{F(b)-F(a)}_{\substack{\text{Stammfunktion,}\\\text{Umkehrung der Differentialrechnung}}}$\\
Dieser Term ist also der Zusammenhang zwischen der Differential- und der Integralrechnung.
\item Symbolik: $\frac{\mathrm{d}F(x)}{\mathrm{d}x}=f(x) \Leftrightarrow \underbrace{\int \intd{F(x)}}_{F(x)+C}=\int f(x) \intd{x}$
\item Aus Tabellen zur Differentiation lassen sich Integrationsregeln ableiten.
\end{enumerate}
\subparagraph{Beispiele:}
\begin{anumerate}
\item $\frac{\intd{}}{\intd{x}}\cos x = - \sin x \\
\Leftrightarrow \int -\sin x \intd{x}=\cos x + C^* \quad |\cdot (-1)\\
\int \sin x \intd{x}= - \cos x + \underbrace{C}_{=-C^*}$
\item $\frac{\intd{}}{\intd{x}}x^{\alpha+1}=(\alpha+1)x^\alpha\\
\Leftrightarrow \int x^\alpha \intd{x} = \frac{1}{\alpha +1} x^{\alpha +1} + C$ (falls $\alpha \not = -1$)
\end{anumerate}
\section{Integrationsmethoden}
\subsection{Substitution}
Zu berechnen ist $\int f\big(g(x)\big)\cdot g'(x) \intd{x}$. Bekannt sei dabei die Stammfunktion $F$ von $f$. Dann gilt:
\[\int f\big( g(x) \big) g'(x) \intd{x}\overset{\substack{\text{Subst.}\\ u=g(x)}}{=}\int f(u) \intd{u}=F(u) + C =\overset{u=g(x)}{=}F\big(g(x)\big)+C \]
Substitution $u=g(x)$ impliziert $\frac{\intd{u}}{\intd{x}}=g'(x) \Rightarrow \intd{u}=g'(x) \intd{x}$.\\
\emph{Merke:} Anwendung dieser Methode ist zweckmäßig, wenn der Integrand das Produkt eine Verknüpfung zweier Funktionen mit der Ableitung der inneren Funktion ist und eine Stammfunktion für die äußere Funktion bekannt ist.
\paragraph{Bsp. 1:} $\int \frac{1}{x}\sqrt[3]{\ln x}\intd{x}\overset{\substack{u=\ln x\\\intd{x}=\tfrac{1}{x}\intd{u}}}{=}\int \underbrace{\sqrt[3]{u}}_{u^{\tfrac{1}{3}}}\intd{u}=\frac{3}{4}u^{\frac{4}{3}}+C=\frac{3}{4}(\ln x)^{\tfrac{4}{3}}+C$
\paragraph{Bsp. 2:}$\int x e^{-x^2}\intd{x}\overset{\substack{u=-x^2\\\intd{x}=-\tfrac{\intd{u}}{2x}}}{=}\int x e^u\frac{\intd{u}}{-2x}=-\frac{1}{2}\int e^u \intd{u}=-\frac{1}{2}e^u+C=-\frac{1}{2}e^{-x^2}+C$
\paragraph{Bsp. 3:} (Substitution bei bestimmten Integral)
\begin{itemize}
\item 1. Variante: Grenzen ersetzen\\
$I=\int_0^{\sqrt{8}}x\sqrt{1+x^2}\intd{x}\overset{\substack{u=1+x^2\\dx=\tfrac{\intd{u}}{2x}}}{=}\int_1^9 x \sqrt{u} \frac{\intd{u}}{2x}=\frac{1}{2}\int u^{\tfrac{1}{2}}\intd{u}= \left[\frac{1}{3}u^{\tfrac{3}{2}}\right]_1^9=\frac{1}{3}(27-1)=\frac{26}{3}$\\
Grenzen in Substitution einsetzen $u=1+x^2\Rightarrow u_{unt}=1+0^2=1 \quad u_{ob}=1+\sqrt{8}^2=9$
\item 2. Variante: Erst unbestimmtes Integral  lösen\\
$I=\int_0^{\sqrt{8}}x\sqrt{1+x^2}\intd{x}=\frac{1}{3}(1+x^2)^{\tfrac{3}{2}}+C$\\
Dann Grenzen einsetzen:\\
$I=\left[\frac{1}{3}(1+x^2)^{\tfrac{3}{2}}\right]_0^{\sqrt{8}}=\frac{1}{3}(27-1)=\frac{26}{3}$
\end{itemize}
\subparagraph{Bsp. 4:} $\int \frac{f'(x)}{f(x)}\intd{x}=\ln |f(x)|+C$\\
(Zähler = Ableitung des Nenners)\\
Nutze dazu die Substitution $u=f(x), \; \intd{x}=\frac{\intd{u}}{f'(x)}$\\
$\Rightarrow \int \dots = \int \frac{1}{u} \intd{u}=\ln | u| + C = \ln | f(x) | + C$
\paragraph{Bsp. 5:}(lineare Substitution)\\
Allgemein: $\int f(ax+b) \intd{x} \overset{\substack{u=ax+b\\\tfrac{du}{dx}=a}}{=} \int f(u) \frac{\intd{u}}{a}\overset{\substack{F: \text{ Stammfkt.}\\ \text{von } f}}{=}\frac{1}{a}\cdot F(u) + C$
\begin{anumerate}
\item $\int \cos (3x) = \frac{1}{3}\sin (3x) + C$
\item $\int e^{-2x} \intd{x}=\frac{1}{-2}e^{-2x}+C$
\item $\int (3x+4)^6 \intd{x}=\frac{1}{3}\cdot \frac{1}{7}(3x+4)^7+C$
\item $\int \sin \left( \frac{x}{2}+\pi\right) \intd{x}=2 \cdot  -\cos\left(\frac{x}{2}+\pi\right)+C$
\end{anumerate}

\subparagraph{Diskussion:} Neben diesen „natürlichen“ und leicht erkennbaren Substitutionen sind weiter Substitutionen durch die Einführung von „künstlichen“ Variablen möglich:\\
$\int f(x) \intd{x}\overset{\substack{x=f(t)\\\tfrac{\diffd{x}}{\diffd{t}}=\dot{\varphi}(t)}}{=}\int f(\varphi(t)) \cdot \dot{\varphi}(t) \intd{t}$\\
Dies entsprecht der Substitutionsregel, von rechts nach links gelesen. Falls die rechte Seite davon integrierbar ist (mit Stammfunktion $H$), dann:\\
$\int f(x) \intd{x}=H(t) + C = H\big(\varphi^{-1}(t)\big) + C$ \qquad (falls $\varphi^{-1}$ existiert)
\subparagraph{Bsp. 6:} \parskp
$\int \frac{1}{\sqrt{1+x^2}}\intd{x}\overset{\substack{x=\sinh(t)\\\tfrac{\diffd{x}}{\diffd{t}}=\cosh(t)\\\cosh^2(t)-\sinh^2(t)=1}}{=}\int \frac{1}{\cosh(t)}\cosh(t)\intd{t}=\int \intd{t}=t+C = \arcsinh (x) + C$\\
Für weitere geeignete Substitutionen siehe Integrationstabelle.

\subsection{Partielle Integration}
Produktregel der Differentiation:\\
$\frac{\diffd{}}{\diffd{x}}\big(u(x)\cdot v(x)\big) = u'(x)\cdot v(x)+u(x)\cdot v'(x)$\\
$\Rightarrow u(x) v(x) \int u'(x) v(x) \intd{x} + \int u(x) v'(x) \intd{x}$\\
$\Rightarrow \boxed{\int u(x)v'(x)\intd{x}=u(x)v(x)- \int u'(x) v(x) \intd{x} }$

\subparagraph{Bsp. 7:}
\begin{anumerate}
\item $\int \underbrace{x}_{u(x)} \underbrace{\sin(2x)}_{v'(x)} \intd{x}=\underbrace{x}_{u}\cdot \underbrace{-\frac{1}{2}\cos(2x)}_{v} - \int \underbrace{1}_{u'}\cdot \underbrace{-\frac{1}{2}\cos (2x)}_{v}\intd{x}=-\frac{x}{2}\cos(2x)+\frac{1}{4}\sin(2x) + C $\\
$u'(x) = 1 \qquad v(x) = -\frac{1}{2}\cos (2x)$
\item $\int \underbrace{x^3}_{v'}\underbrace{\ln x}_{u} = \frac{1}{4}x^4 \ln x - \int \frac{1}{x}\cdot \frac{1}{4}x^4 \intd{x} = \frac{1}{4} x^4 \ln x - \frac{1}{16} x^4 + C = \frac{1}{4}x^4 \left(\ln x - \frac{1}{4} \right) ++C$
\end{anumerate}
\emph{Merke:} Typische Anwendungsfälle für partielle Integration (mit $p(x)$ jeweils als $u$):
\begin{itemize}
\item $\int p(x) e^{ax} \intd{x}$
\item $\int p(x) \cos (ax) \intd{x}$
\item $\int p(x) \sin (ax) \intd{x}$
\end{itemize}
aber (mit $\ln(x)$ jeweils als $u$):
\begin{itemize}
\item $\int p(x) \cdot \ln (x) \intd{x}$
\item $\int x^\alpha \cdot \ln(x) \intd{x}$
\end{itemize}
\subparagraph{Bsp. 8:} \parskp
$\int \arctan (x ) \intd{x} \overset{\substack{u=\arctan(x)\\ v'=1}}{=}x \cdot \arctan
(x) - \int x \cdot \frac{1}{1+x^2}\intd{x}=x \cdot \arctan (x) - \frac{1}{2}\ln(|x^2+1|)+C$\\
$u'=\frac{1}{1+x^2} \qquad v = x$
\subsection{Integration gebrochen rationaler Funktionen}
Gegeben: Gebrochen rationale Funktion $f(x)=\frac{p(x)}{q(x)}$\\
Integration erfolgt in 5 Schritten:
\begin{enumerate}
\item Falls $f$ unecht gebrochen: Polynomdivision erhalten dann $f(x) =\underbrace{a(x)}_{\text{Polynom}} + \underbrace{\frac{r(x)}{q(x)}}_{\text{echt gebrochen}}$
\item Nullstellen von $q$ ermitteln. Dann Zerlegung $q$: \\
$q(x) = (x-\alpha_1)^{k_1}\cdot (x-\alpha_2)^{k_2}\cdot \dots \cdot (x^2+p_1+q_1)^{m_1}\cdot (x^2+p_2+q_2)^{m_2}\cdot \dots$\\
$k_i$: reelle Nullstellen \qquad $m_i$: nicht reell zerlegbar\\
Dabei kürzt man eventuelle gemeinsame Faktoren in $r$ und $q$ heraus.
\item \emph{Ansatz für die Partialbruchzerlegung}\\
$\frac{r(x)}{q(x)}= $Summe von Partialbrüchen\\
Jeden Faktor der Form $\begin{cases}
(x-\alpha)^k\\
(x^2+px+q)^m
\end{cases}$ der Gleichung entspricht der Anteil \\
$\begin{cases}
\frac{A_1}{x-\alpha}+\frac{A_2}{(x-\alpha)^2}+\dots + \frac{A_2}{(x-\alpha)^k}\\
\frac{B_1x+C_1}{x^2+px+q}+\frac{B_2 x + C_2}{(x^2+px+q)^2}+\dots + \frac{B_m x + C_m}{(x^2+px+q)^m}
\end{cases}$ in dieser Summe.
\subparagraph{Bsp. 9:}
\begin{align*}
f(x) &= \frac{x^2+4}{(x-1)^3(x+5)(x^2+2x+2)^2}\\
&=\frac{A}{x-1}+\frac{B}{(x-1)^2}+\frac{C}{(x-1)^3}+ \frac{D}{x+5}+ \frac{Ex+F}{x^2+2x+2}+\frac{Gx+H}{(x^2+2x+2)^2}
\end{align*}
Beachte: $x^2+2x+2$ ist reell nicht weiter zerlegbar, Nullstelle: $1\pm i$.
\item Ermittlung der Koeffizienten durch 
\begin{itemize}
\item Multiplikation des Ansatzes der Partialbruchzerlegung mit $q(x)$
\item Kombination der folgenden beiden Methoden
\begin{anumerate}
\item Einsetzen der reellen Nullstellen
\item Koeffizientenvergleich
\end{anumerate}
(falls $q$ nur reelle Nullstellen hat, recht Methode a.)
\end{itemize}
\item Integration der Partialbrüche
\begin{anumerate}
\item $\int \frac{1}{(x-\alpha)^j} \intd{x}=\begin{cases}
\ln(|x-\alpha|)+C & j=1\\
\frac{1}{1-j}(x-\alpha)^{1-j}+C & j=2,3,4,\dots
\end{cases}$
\item $\int \frac{3x+C}{(x^2+px+q)^j}\intd{x}=\int \frac{\tfrac{B}{2}(2x+q)}{(x^2+px+q)^j}+\frac{C-\tfrac{Bp}{2}}{(x^2+px+q)^j}\intd{x}$
\begin{itemize}
\item $\int \frac{2x+p}{(x^2+px+q)^2}\intd{x}$: Nutze Substitution.
\item $\int \frac{1}{(x^2+px+q)^j}\overset{\substack{\text{quadratische}\\\text{Ergänzung}}}{=}\int \frac{1}{\left(\left(x+\frac{p}{2}\right)^2+q-\frac{p^2}{4}\right)^j}\intd{x}\overset{u=x+\frac{p}{2}}{=}\int \frac{1}{\left(u^2+a^2\right)^j}\intd{x}$\\
$\begin{cases}
j=1 & \text{Stammfunktion siehe Merkblatt}\\
j>1 & \text{siehe weitere Formelsammlung (selten)}
\end{cases}$
\end{itemize}
\end{anumerate}
\end{enumerate}
\subparagraph{Bsp. 10:} $I=\int \frac{3x+4}{x^2+2x-3}\intd{x}$
\begin{itemize}
\item echt gebrochen
\item Nullstellen des Nenners: $x_1=-3, \; x_2 = 1$\\
$\Rightarrow x^2+2x-3=(x+3)(x-1)$
\end{itemize}
Ansatz für PBZ:
\begin{align*}
\frac{3x+4}{(x+3)(x-1)}&=\frac{A}{x+3}+\frac{B}{x-1} \qquad |\cdot (x+3)(x-1)\\
3x+4 &= A(x-1)+B(x+3)
\end{align*}
Einsetzen der NS: \\
$x_1: \qquad -5 = A\cdot (-4) \Rightarrow A = \frac{5}{4}$\\
$x_2: \qquad 7 = B \cdot 4 \Rightarrow B=\frac{7}{4}$
\begin{align*}
\Rightarrow I &= \int \frac{\frac{5}{4}}{x+3}\intd{x}+\int \frac{\frac{7}{4}}{x-1}\intd{x}\\
 &= \frac{5}{4} \ln (|x+3|) + \frac{7}{4}\ln(|x-1|)+C
\end{align*}
\subparagraph{Bsp. 11:} $I=\int \frac{7x^2-10x+37}{(x+1)(x^2-4x+13)}\intd{x}$
\begin{itemize}
\item echt gebrochen
\item Nenner reell nicht weiter zerlegbar (denn Nullstellen von $x^2-4x+13$ sind $x_{1/2}=2\pm3i$)
\item Partialbruchzerlegung:
\begin{align*}
\frac{7x^2-10x+37}{(x+1)(x^2-4x+13)}&=\frac{A}{x+1}+\frac{Bx+C}{x^2-4x+13} \quad |\cdot \text{Nenner}\\
7x^2-10x+37&=A(x^2-4x+13)+(Bx+C)(x+1)\\
&= (A+B)x^2+(-4A+B+C)x+(13A+C)
\end{align*}
\begin{tabular}{l l l}
Einsetzen der Nullstelle $x=-1$: & $54=18 A$ & $\Rightarrow A=3$\\
Koeffizientenvergleich $x^2$: & $7=A+B $ & $\Rightarrow B=4$\\
 $x^0$ & $13A+C)$ & $\Rightarrow C=-2$ 
\end{tabular}\\
$\Rightarrow \int \frac{7x^2-10x+37}{(x+1)(x^2-4x+13)}\intd{x}=\int \underbrace{\frac{3}{x+1}}_{a}+ \underbrace{\frac{4x-2}{x^2-4x+13}}_{b}\intd{x} $\\
$a: \; \int \frac{3}{x+1} \intd{x} = 3 \ln|x+1| +C_1$\\
$b: \; \int \frac{4x-2}{x^2-4x+13}\intd{x}=\int\frac{2(2x-4)-2+8}{x^2-4x+13}=2\underbrace{\int \frac{2x-4}{x^2-4x+13}}_{I_1}+\underbrace{\frac{6}{x^2-4x+13}}_{I_2}\intd{x}$\\
$I_1 = \int \frac{2x-4}{x^2-4x+13}\intd{x}\overset{\text{Subst.}}{=}\ln|x^2-4x+13| + C_2$\\
$I_2 = 6\int \frac{\intd{x}}{(x-2)^2+9}=6\int \frac{\intd{u}}{u^2+3^2}=6\cdot \frac{1}{3}\arctan\left(\frac{u}{3}\right)+C_3=2\cdot\arctan\left(\frac{x-2}{3}\right)+C_3$\medskip\\
$\Rightarrow I = 3\ln|x+1| + 2 \ln\left(x^2-4x+13\right)+2\arctan\left(\frac{x-2}{3}\right)+C_4$
\end{itemize}

\subsection{Integration von Potenzreihen}
\paragraph{Satz 1:} Es sei $f:(x_0-r, x_0+r)\to \RR$, $f(x)=\sum_{n=0}^\infty a_n (x-x_0)^n$ die Grenzfunktion der Potenzreihe (mit Konvergenzradius $r$). Dann ist $F: (x_0-r,x_0+r)\to \RR$, $F(x)=\sum_{n=0}^\infty \frac{a_n}{n+1}(x-x_0)^{n+1}$ eine Stammfunktion von $f$ (gliedweises Integrieren im Konvergenzintervall).
\subparagraph{Bsp. 12:} \parskp
$f(x)=\frac{1}{1+x^2}=1-x^2+x^4-x^6+\dots \quad |x|<1$ (siehe Übung, nutze geometrische Reihe)\\
$\Rightarrow \int \frac{1}{1+x^2}\intd{x}=x-\frac{x^3}{3}+\frac{x^5}{5}-\frac{x^7}{7}+\dots + C_1$\\
Wir wissen aber auch: $\int \frac{1}{1+x^2}\intd{x}=\arctan (x) + C_2$\\
$\Rightarrow \arctan(x) = x-\frac{x^3}{3}+\frac{x^5}{5}-\frac{x^7}{7}+\dots + C_3$ \quad mit $C_3=0$ (setze $x=0$ ein) und $|x|<1$.
\subparagraph{Bsp. 13:} Gesucht ist Stammfunktion zu $f(x)=e^{-x^2}$
\begin{align*}
F(x)&=\int_0^x e^{-t^2}\intd{t}=\int_0^x\left(1-t^2+\frac{(t^2)^2}{2!}-\frac{(t^2)^3}{3!}+\dots\right)\intd{t}\\
&= x - \frac{x^3}{3}+\frac{x^5}{5\cdot 2!}-\frac{x^7}{7\cdot 3!}+\frac{x^9}{9\cdot 4!}-\cdots 
\quad (x\in \RR)
\end{align*}
\subparagraph{Diskussion:} 
\begin{itemize}
\item $\int e^{-t^2}\intd{t}$ nicht geschlossen auswertbar.
\item Für nicht zu große $x$ ist die Reihendarstellung zur Auswertung von $F$ gut geeignet.\\
z.B. $\int_0^1 e^{-x^2}\intd{x}=\underbrace{1-\frac{1}{3}+\frac{1}{5\cdot 2!}-\frac{1}{7\cdot 3!}+\frac{1}{9\cdot 4!}-\frac{1}{11\cdot 5!}+\frac{1}{13\cdot 6!}-\frac{1}{15\cdot 7!}+\frac{1}{17\cdot 8!}}_{0,746824\cdots}-\dots$\\
$\left|\text{Fehler}\right| < \frac{1}{19\cdot 9!}=1,4504\cdot 10^{-7}$ (vgl. Satz über Leibnitz-Kriterium)
\end{itemize}

\section{Numerische Integration}
\emph{Ziel:} Berechne $I=\int_a^b f(x) \intd{x}$ falls Stammfunktion „kompliziert“ oder nicht elementar angebbar.\\
\emph{Prinzip:} 
\begin{anumerate}
\item Zerlegung von $[a,b]$ in $n$ gleichlange Teilintervalle der Länge $h=\frac{1}{n}(b-a)$\\
$\Rightarrow$ Teilpunkte sind $x_k=a+k\cdot h \quad (k=0,1,\dots,n)$, $y_k=f(x_k)$\\
ABB 101
\item Ersetze $f(x)$ über den Teilintervallen durch einfachere Funktionen.\\
z.B.: \begin{itemize}
\item lineare Funktionen \quad $\rightsquigarrow$ Trapez Regel
\item quadratische Funktionen \quad$\rightsquigarrow$ SIMPSON-Regel
\end{itemize}
ABB 102+3\\
Als Näherung für $I$ ergibt sich für die Simpson-Regel: \\
$I \approx S_n(h)=\frac{h}{3}\left(\tblue{(y_0+y_n)}+4\tgreen{(y_1+y_3+\dots+y_{n-1})}+2\tred{(y_2+y_4+\dots+y_{n-2})}\right)$\\
falls $n$ gerade ist.
\end{anumerate}
\subparagraph{Diskussion:}
\begin{enumerate}
\item Fehlerabschätzung:\\
$I=S_n(h)-\frac{h^4(b-a)}{180}f^{(4)}(\xi) \quad a < \xi < b$\\
(falls $f^{(4)}$ stetig in $[a,b]$)
\item Simpson-Regel ist für Polynome einschließlich Grad 3 exakt.
\item Praktische Durchführung: Schrittweitenhalbierung\\
Startwert: $S^{(1)}=S_n(h)$ für geeignetes $h$. $S^{(2)}=S_{2n}\left(\frac{h}{2}\right)$, $S^{(3)}=S_{4n}\left(\frac{h}{4}\right)$, usw. bis dich die Ziffern in gewünschter Genauigkeit nicht mehr ändern.
\end{enumerate}
\subparagraph{Bsp.:} $\int_0^1 e^{-x^2}\intd{x}$\\
$n=4, \; h=0,25$\\
\begin{tabular}{l l l l l}
$k$ & $x_k$ & \tblue{$y_0, y_n$} & \tgreen{$y_{2j+1}$} & \tred{$y_{2j}$}\\
\hline
0 & 0		& 1	&\\
1 & 0,25& 	&0,939413\\
2 & 0,5	&		&					& 0,778801\\
3 & 0,75&		&0,569783\\
4 & 1		& 0,367879\\
\hline 
 & &		1,367879 & 1,509196 & 0,778801
\end{tabular}\\
$S_4(0,25)=\frac{0,25}{3}\left(1,367879 + 4\cdot 1,509196 + 2\cdot 0,778801\right)$

\section{Uneigentliche Integrale}
\begin{itemize}
\item Vorbetrachtung:\\
Bisher $\int_a^b f(x) \intd{x}$ wobei $[a,b]$ endliches Integral auf $f$ stückweise stetig auf $[a,b]$ (daher beschränkt)
\item 2 Erweiterungen:
\begin{enumerate}
\item unendliches Intervall $(-\infty, b]$, $[a,\infty)$ oder $(-\infty, \infty)$
\item Funktion $f$ unbeschränkt (Unendlichkeits- bzw. Polstellen)
\end{enumerate}
\end{itemize}
\paragraph{Unendliches Intervall} (zu 1.)
\begin{anumerate}
\item $\int_{-\infty}^b f(x) \intd{x}:= \lim_{A\to \infty} \int_A^b f(x) \intd{x}$\\
analog:\\
$\int_a^{\infty} f(x) \intd{x} := \lim_{B \to \infty} \int_a^{B} f(x) \intd{x}$
\item $\int_{-\infty}^{\infty} f(x) \intd{x} := \int_{-\infty}^c f(x) \intd{x} + \int_c^{\infty} f(x) \intd{x}$ für beliebiges $c \in \RR$ (bpsw. $c=0$).
\end{anumerate}
\subparagraph{Diskussion:}
\begin{enumerate}
\item Falls die Grenzwerte existieren, so heißt das Integral konvergent, sonst divergent.
\item Ein berühmtes Beispiel ist die $\Gamma$-Funktion:\\
$\Gamma (x) = \int_0^{\infty} e^{-t} t^{x-1} \intd{t} \qquad (x >0 )$\\
Eigenschaft: $\Gamma (n) = (n-1)!$ falls $n \in \NN$
\end{enumerate}
\subparagraph{Bsp. 1:} 
\begin{anumerate}
\item $\int_0^{\infty} e^{-x} \intd{x} = \lim_{A \to \infty} \int_0^A e^{-x} \intd{x} = \lim_{A \to \infty} \left[ -e^{-x}\right]_0^A = \lim_{A \to \infty} \left( -e^{-A}+e^0\right) =1$
\item $\int_0^{\infty} \cos x \intd{x}= \lim_{A\to \infty} \int_0^A \cos x \intd{x} = \lim_{a\to \infty} \left[ \sin (x) \right]_0^A = \lim_{A\to \infty} \left( \sin A - \underbrace{\sin 0}_{=0}\right)$\\
Grenzwert existiert nicht $\Rightarrow$ Integral unbestimmt divergent.
\item $\int_1^{\infty} = \lim_{A\to \infty} \int_1^A \frac{1}{x} \intd{x} = \lim_{A\to \infty} \left[ \ln | x| \right]_1^A = \lim_{A\to \infty} \left( \ln A - \underbrace{\ln 1}_{0}\right) = \infty$\\
$\Rightarrow$ bestimmt divergent
\end{anumerate}
\paragraph{Unbeschränkter Integrand} (zu 2.)
\begin{anumerate}
\item bspw. Unendlichkeitsstellen bei $b$:\\
$\int_a^b f(x) \intd{x} := \lim_{\varepsilon \searrow 0} \int_a^{b-\varepsilon} f(x) \intd{x}$\\
ABB 104
\item falls Unendlichkeitsstelle $x_0$ im Inneren von $[a,b]$ liegt:\\
$\int_a^b f(x) \intd{x} := \int_a^{x_0} f(x) \intd{x} + \int_{x_0}^b f(x) \intd{x}$\\
… und nutzen nun a.):\\
$\lim_{\varepsilon\searrow 0} \int_a^{x_0-\varepsilon} f(x) \intd{x} + \lim_{\varepsilon\searrow 0} \int_{x_0 + \varepsilon}^b f(x) \intd{x}$
\end{anumerate}
\subparagraph{Bsp. 2:} \parskp
$\int_0^4 \frac{1}{\sqrt{x}}\intd{x} = \lim_{\varepsilon\searrow0} \left[2\sqrt{x}\right]_{\varepsilon}^4 = \lim_{\varepsilon \searrow 0} \left( 2 \sqrt{4}-2 \sqrt{\varepsilon}\right) = 4$
\paragraph{Unendliches Intervall und unbeschränkter Integrand} (Kombination von 1. und 2.)
\subparagraph{Bsp. 3} \parskp
$I = \int_1^{\infty} \frac{\intd{x}}{x \sqrt{x-1}}= \int_1^2 \frac{\intd{x}}{x\sqrt{x-1}}+ \int_2^{\infty} \frac{\intd{x}}{x\sqrt{x-1}}$\\
mit $\int \frac{\intd{x}}{x\sqrt{x-1}}= \underbrace{\dots}_{\text{Subst. }t=\sqrt{x-1}} = 2 \arctan \sqrt{x-1} + C$:
\begin{align*}
I &= \lim_{\varepsilon \searrow 0} \int_{1+\varepsilon}^2 \frac{\intd{x}}{x \sqrt{x-1}}+ \lim_{a\to \infty} \int_2^A \frac{\intd{x}}{x\sqrt{x-1}}\\
&= \lim_{\varepsilon\searrow0} \left[ 2\arctan \sqrt{x-1}\right]_{1+\varepsilon}^2 + \lim_{A\to \infty} \left[ 2 \arctan\sqrt{x-1}\right]_2^A\\
&= \lim_{\varepsilon\searrow 0} ( 2 \arctan 1 - 2 \arctan \sqrt{\varepsilon}) + \lim_{A\to\infty} (2 \arctan \sqrt{A-1}-2\arctan 1 )\\
&= \underbrace{\lim_{A \to \infty} 2 \arctan \sqrt{A-1}}_{\pi} - \underbrace{\lim_{\varepsilon \searrow} 2 \arctan \sqrt{\varepsilon}}_{0}\\
&= \pi
\end{align*}

\section{Anwendungen}
\subsection{Geometrische Anwendungen}
\subsubsection{Inhalte ebener Flächen}
\begin{itemize}
\item ABB 105\\
mit $a<b$ und $f(x) \geq 0$:\\
$F=\int_a^b f(x) \intd{x}$
\item ABB 106\\
mit $a<b<c$:\\
$F=\int_a^c |f(x)| \intd{x}=\left| \int_a^b f(x) \intd{x}\right| + \left| \int_b^c f(x) \intd{x}\right|= \int_a^b f(x) \intd{x} + \int_c^b f(x) \intd{x}$
\item ABB 107\\
mit $f(x)$: obere Funktion und $g(x)$: untere Funktion:\\
$F=\int_{x_1}^{x_2} f(x) - g(x) \intd{x}$
\end{itemize}
\subparagraph{Bsp. 1:} Gesucht ist der Flächeninhalt $F$ des von der Ellipse $\frac{x^2}{a^2}+\frac{y^2}{b^2}=1 \quad (a>0, \; b>0)$ begrenzten Bereichs.\\
ABB 108\\
$y= \pm b\sqrt{a-\frac{x^2}{a2}}$
\begin{align*}
F&=4 \cdot \int_0^a b \sqrt{1-\frac{x^2}{a}}\intd{x}=\frac{4b}{a}\int_0^a\sqrt{a^2-x^2}\intd{x}\overset{\substack{\text{Subst.}\\x=a\sin t}}{=}\frac{4b}{a}\left[\frac{1}{2}\left(x\sqrt{a^2-x^2}+a^2 \arcsin\frac{x}{a}\right)\right]_0^a\\
&=\frac{4b}{a}\frac{1}{2}a^2\underbrace{\arcsin 1}_{\frac{\pi}{2}}=\pi \cdot ab
\end{align*}

\subsubsection{Bogenlänge}
\paragraph{Bogenlänge ebener Kurven}\parskp
ABB 109\\
Kurve $K$ mit Parameterdarstellung. $x=x(t) \quad y = y(t) \quad t \in [\alpha, \beta]$
\begin{itemize}
\item Vorgehen: Approximieren durch Streckenzug, dann Verfeinerung\\
Länge des Streckenzugs: $\sum_{i=1}^n \overline{P_{i-1}P_i}=\sum_{i=1}^n \sqrt{(\Delta x_i)^2+(\Delta y_i)^2}$\\
ABB 110\\
$\Rightarrow \text{(Mittelwertsatz der Differentialrechnung} \sum_{i=1}^n \overline{P_{i-1}P_i}= \sum_{i=1}^n \sqrt{(\dot{x}(u_i))^2 + (\dot{y}(v_i))^2}\cdot \Delta t_i$\\
mit $u_i, v_i\in (t_{i-1},t_i)$\\
Verfeinerung:\\
$\int_{\alpha}^{\beta} \sqrt{(\dot{x}(t))^2 + (\dot{y}(t))^2}\intd{t}$
\end{itemize}
\subparagraph{Diskussion:}
\begin{itemize}
\item Bogenlänge der Kurve $\vec{r} = \vec{r}(t) = \mtr{x(t) \\ y(t)}$ zwischen $\alpha$ und $t$ ist\\
$s=\int_{\alpha}^t \underbrace{\sqrt{(\dot{x}(u))^2 + (\dot{y}(u))^2}}_{|\dot{r}(u)|}\intd{u} =: f(t)$\\
$\Rightarrow \frac{\diffd{s}}{\diffd{s}}= |\dot{r}(t)| \Rightarrow \diffd{s}=|\dot{r}(t)| \intd{t}=\sqrt{\dot{x}^2+\dot{y}^2}\intd{t}$ (heißt Bogenelement)
\item Tabelle (Bogenlänge ebener Kurven)\\
\begin{tabular}{l | l}
Kurvendarstellung & Bogenlänge $s$, Bogenelement $\diffd{s}$\\ 
\hline
$x=x(t),\; y=y(t),\; t \in [\alpha,\beta]$ & $s= \int_{\alpha}^{\beta} \underbrace{\sqrt{(\dot{x}(t))^2-(\dot{y}(t))^2}\intd{t}}_{\diffd{s}}$\\
$y =f(x),\; x\in [a,b]$ & $s=\int_a^b\underbrace{\sqrt{a+(f'(x))^2}\intd{x}}_{\diffd{s}}$\\
$x= g(y), \; y \in [c,d] $ & $s=\int_c^d\underbrace{\sqrt{1+(g'(y))^2}\intd{y}}_{\diffd{s}}$\\
$r=r(\varphi), \; \varphi\in [\alpha, \beta]$ & $s=\int_{\alpha}^{\beta}\underbrace{\sqrt{(r(\varphi))^2+(r'(\varphi))^2}\intd{\varphi}}_{\diffd{s}}$
\end{tabular}
\end{itemize}
\paragraph{Bogenlänge von Raumkurven} \parskp
Gegeben sei Kurve $K $mit Parameterdarstellung $x=x(t), \; y=y(t), \; z=z(t), \; t \in [\alpha
, \beta]$.\\
Die Bogenlänge berechnet sich dann mittels\\
$s=\int_{\alpha}^{\beta} \sqrt{(\dot{x}(t))^2+(\dot{y}(t))^2+(\dot{z}(t))^2}\intd{t}$
\subparagraph{Bsp. 2} (Schraubenlinie)\\
$\vec{r}= \vec{r}(t)=\mtr{a \cos t \\ a \sin t \\ \frac{h}{2\pi}t}, \; t \in [0,2\pi]$\\
$\dot{r}(t) = \mtr{-a \sin t \\ a \cos t \\ \frac{h}{2\pi}}$
\begin{align*}
s&= \int_0^{2\pi} \sqrt{\dot{x}^2+\dot{y}^2+\dot{z}^2}\intd{t}\\
&= \int_0^{2\pi} \sqrt{\underbrace{a^2 \sin^2t - a^2 \cos^2t}_{a^2}+\frac{h^2}{(2\pi)^2}}\intd{t}\\
&= 2\pi \sqrt{a^2+\frac{h^2}{4\pi^2}}\\
&= \sqrt{4\pi^2a^2+h^2}
\end{align*}
\subsubsection{Volumen von Rotationskörpern}
ABB 118
\begin{anumerate}
\item Gegeben:
\begin{itemize}
\item Kurve $K$ mit $y=f(x), \; x \in [a,b]$
\item Das Flächenstück $F_x$ zwischen Kurve und $x$-Achse rotiere um die $x$-Achse.
\end{itemize}
Gesucht:
\begin{itemize}
\item Volumen $V_x$ des dabei erzeugten Körpers
\end{itemize}
Es gilt: $V_x = \pi \cdot \int_a^b(f(x))^2\intd{x}$\\
(Idee: Approximation von $F_x$ durch Rechteckflächen \\
$\overset{Rotation}{\rightsquigarrow}$ Zylinderscheiben $V_{\alpha}\approx \sum_i \pi (f(\xi_i))^2\Delta x_i$\\
Grenzübergang: $V_x=\pi \int_a^b (f(x))^2\intd{x}$)
\subparagraph{Diskussion:}
\begin{enumerate}
\item Allgemein gilt $V_x=\pi \int_a^b y^2\intd{x} \quad (a<b)$
\item Falls K in Parameterform geggeben $x=x(t),\; y=y(t), \; \alpha ... t ... \beta$, dann ergibt sich \\
$V_x=\pi \int_{\alpha}^{\beta} (y(t))^2\cdot \underbrace{\dot{x}(t) \intd{t}}_{\diffd{x}}$, wobei $...$ (die Orientierung) so zu wählen ist, dass $a:= x(\alpha) < x(\beta) = b$ gilt. Unter Umständen kann $\alpha<\beta$ sein.
\end{enumerate}
\item Gegeben: Kurve $x=g(y), \; y\in [c,d]$\\
ABB 119\\
Gesucht: Volumen $V_y$ bei Rotation von $F_y$ um $y$-Achse.\\
$V_y=\pi \int_c^d x^2\intd{y} = \pi \int_c^d (g(y))^2\intd{y}$\\
Für Parameterdarstellung:\\
$V_y=\pi \int_{\alpha}^{\beta}(x(t))^2\cdot \underbrace{\dot{y}(t) \intd{t}}_{\diffd{y}}$, wobei $c=y(\alpha)<y(\beta)=d$
\subparagraph{Bsp. 3:} Gesucht ist das Volumen eines Rotationsparaboloids der Höhe $h$ und mit Basisradius $R$.\\
ABB 120\\
Kurve: $y=ax^2, \; h=aR^2 \Rightarrow a =\frac{h}{R^2}$
\begin{align*}
V_y&=\pi\int_0^h x^2\intd{y}=\pi\int_0^h \frac{y}{a}\intd{y} = \pi\int_0^h \frac{R^2}{h}y  \intd{y}\\
&= \frac{\pi R^2}{h}\int_0^h y \intd{y}=\frac{\pi R^2}{h}\left[\frac{1}{2}y^2\right]_0^h\\
&= \frac{\pi R^2}{h}\cdot \frac{1}{2}h^2=\frac{\pi R^2 h}{2}
\end{align*}
\end{anumerate}

\subsubsection{Mantelflächen von Rotationskörpern}
\begin{anumerate}
\item Gegeben: Kurve $K$ mit $y=f(x)\geq 0, \; x \in [a,b]$\\
Gesucht: Die von der Kurve $K$, bei Rotation um $x$-Achse erzeugte Rotationsfläche $M_x$.\\
ABB 121\\
$M_x=2\pi \int_a^b f(x) \sqrt{1+(f'(x))^2}\intd{x}$\\
(Approximation der Kurve $K$ durch Polygonzug\\
$\overset{Rotation}{\rightsquigarrow}$ Kegelstumpffläche\\
$M_x\approx \sum_i 2\pi f(\xi_i)\Delta s_i\\
\to 2\pi \int_a^b f(x)\underbrace{\sqrt{1+(f'(x))^2}\intd{x}}_{\diffd{s}}$ (Bogenelement))\\
Allgemein gilt: $M_x=2\pi \int_a^by\intd{s}$
\item Gegeben: Kurve $x=g(y) \geq 0, \; y\in [c,d]$\\
Gesucht: Mantelfläche bei Rotation um die $y$-Achse.\\
$M_y=2\pi \int_K x \intd{s}=2\pi \int_c^d g(y) \sqrt{1+(g'(y))^2}\intd{y}$\\
Für Parameterdarstellung:\\
$M_y=2\pi \int_{\alpha}^{\beta} x(t) \sqrt{(\dot{x}(t))^2+(\dot{y}(t))^2}\intd{t}$ mit $\alpha \leq t \leq \beta$
\subparagraph{Bsp. 4:} Kugeloberlfäche\\
ABB 122\\
Halbkreis $K$ soll um $x$-Achse rotiert werden.\\
$x=R\cdot \cos t\\
y = R \cdot \sin t\\
t \in [0,\pi]$\\
$\dot{x}(t) = -R \sin t \qquad \dot{y}(t) = R \cos t$
\begin{align*}
M_x&= 2\pi \int_K y\intd{s}=2\pi \int_0^{\pi} y(t) \sqrt{\dot{x}^2+\dot{y}^2}\intd{t}\\
&= 2\pi \int_0^{\pi} R \sin t \cdot R \intd{t} = 2\pi R^2 \left[-\cos t\right]_0^{\pi}\\
&= 4\pi R^2
\end{align*}
\end{anumerate}
\subsubsection{Fourier-Reihen}
Gegeben: Funktion $f(x), \; x \in [0,T]$\\
ABB 123\\
Gesucht: Reihendarstellung mit trigonometrischen Funktionen der Periode $T, \; \frac{T}{2}, \; \frac{T}{3}, \dots$\\
d.h. (mit $\omega= \frac{2\pi}{T}$):\\
$\cos (\omega x), \; \cos (2 \omega x) ,\; \cos (3 \omega x) , \dots$\\
$\sin (\omega x), \; \sin (2 \omega x) ,\; \sin (3 \omega x), \dots$\\
Ansatz ist daher:\\
$f(x) = \frac{a_0}{2}+\sum_{k=1}^a \left( a_k \cos (k\omega x) + b_k \sin (k\omega x) \right)$, wobei die Koeffizienten $a_k, \; b_k,\; k\geq 0$ zu ermitteln sind.\\
Motivation ist: Approximation von $f$ zum Zwecke der Speicherplatzreduzierung (Abgespeichert werden i.A. nur wenige der Koeffizienten $a_k$ und $b_k$. Das gilt auch dann, wenn $f$ in diskreter Form vorliegt, d.h. in Form von Messwerten $y_k$ an vielen Messstellen $x_k$.)
\end{document}
