\newcommand{\customDir}{../}

\RequirePackage{ifthen,xifthen}

% Input inkl. Umlaute, Silbentrennung
\RequirePackage[T1]{fontenc}
\RequirePackage[utf8]{inputenc}

% Arbeitsordner (in Abhängigkeit vom Master) Standard: .LateX_master Ordner liegt im Eltern-Ordner
\providecommand{\customDir}{../}
\newcommand{\setCustomDir}[1]{\renewcommand{\customDir}{#1}}
%%% alle Optionen:
% Doppelseitig (mit Rand an der Innenseite)
\newboolean{twosided}
\setboolean{twosided}{false}
% Eigene Dokument-Klasse (alle KOMA möglich; cheatsheet für Spicker [3 Spalten pro Seite, alles kleiner])
\newcommand{\customDocumentClass}{scrreprt}
\newcommand{\setCustomDocumentClass}[1]{\renewcommand{\customDocumentClass}{#1}}
% Unterscheidung verschiedener Designs: htw, fjs
\newcommand{\customDesign}{htw}
\newcommand{\setCustomDesign}[1]{\renewcommand{\customDesign}{#1}}
% Dokumenten Metadaten
\newcommand{\customTitle}{}
\newcommand{\setCustomTitle}[1]{\renewcommand{\customTitle}{#1}}
\newcommand{\customSubtitle}{}
\newcommand{\setCustomSubtitle}[1]{\renewcommand{\customSubtitle}{#1}}
\newcommand{\customAuthor}{}
\newcommand{\setCustomAuthor}[1]{\renewcommand{\customAuthor}{#1}}
%	Notiz auf der Titelseite (A: vor Autor, B: nach Autor)
\newcommand{\customNoteA}{}
\newcommand{\setCustomNoteA}[1]{\renewcommand{\customNoteA}{#1}}
\newcommand{\customNoteB}{}
\newcommand{\setCustomNoteB}[1]{\renewcommand{\customNoteB}{#1}}
% Format der Signatur in Fußzeile:
\newcommand{\customSignature}{\ifthenelse{\equal{\customAuthor}{}} {} {\footnotesize{\textcolor{darkgray}{Mitschrift von\\ \customAuthor}}}}
\newcommand{\setCustomSignature}[1]{\renewcommand{\customSignature}{#1}}
% Format des Autors auf dem Titelblatt:
\newcommand{\customTitleAuthor}{\textcolor{darkgray}{Mitschrift von \customAuthor}}
\newcommand{\setCustomTitleAuthor}[1]{\renewcommand{\customTitleAuthor}{#1}}
% Standard Sprache
\newcommand{\customDefaultLanguage}[1]{}
\newcommand{\setCustomDefaultLanguage}[1]{\renewcommand{\customDefaultLanguage}{#1}}
% Folien-Pfad (inkl. Dateiname ohne Endung und ggf. ohne Nummerierung)
\newcommand{\customSlidePath}{}
\newcommand{\setCustomSlidePath}[1]{\renewcommand{\customSlidePath}{#1}}
% Folien Eigenschaften
\newcommand{\customSlideScale}{0.5}
\newcommand{\setCustomSlideScale}[1]{\renewcommand{\customSlideScale}{#1}}
\setCustomTitle{Angewandte Sensorik}
\setCustomSubtitle{Vorlesungsskript}
\setCustomAuthor{Falk-Jonatan Strube}
\setCustomNoteB{Vorlesung von Prof. Dr.-Ing. habil. Gorbunoff}
%-- Prüfen, ob Beamer
\ifthenelse{\equal{\customDocumentClass}{beamer}}{
%%% TODO: andere Layouts für Beamer außer HTW
	\documentclass[ignorenonframetext, 11pt, table]{beamer}
	
	\usenavigationsymbolstemplate{}
	\setbeamercolor{author in head/foot}{fg=black}
	\setbeamercolor{title}{fg=black}
	\setbeamercolor{bibliography entry author}{fg=htworange!70}
	%\setbeamercolor{bibliography entry title}{fg=blue} 
	\setbeamercolor{bibliography entry location}{fg=htworange!60} 
	\setbeamercolor{bibliography entry note}{fg=htworange!60}  
	
	\setbeamertemplate{itemize item}{\color{black}$\bullet$}
	\setbeamertemplate{itemize subitem}{\color{black}--}
	\setbeamertemplate{itemize subsubitem}{\color{black}$\bullet$}
	\makeatother
	\setbeamertemplate{footline}
	{
	\leavevmode
	\def\arraystretch{1.2}
	\arrayrulecolor{gray}
	\begin{tabular}{ p{0.167\textwidth} | p{0.491\textwidth} | p{0.089\textwidth} | p{0.103\textwidth}}
	\hline
	\strut\insertshortauthor & \insertshorttitle & Slide \insertframenumber{}% / \inserttotalframenumber{}
	 & May 4, 2016\\
	\end{tabular}
	}
	\setbeamertemplate{headline}
	{
	\leavevmode
	\setlength{\arrayrulewidth}{1pt}
	\hspace*{2em}	
	\begin{tabular}{p{0.63\textwidth}}
	\rule{0pt}{3em}\normalsize{\textbf{\insertsection\strut}}\\
	\arrayrulecolor{htworange}
	\hline
	\end{tabular}
	\begin{tabular}{l}
	\rule{0pt}{4em}\includegraphics[width=3.25cm]{\customDir .LaTeX_master/HTW_GESAMTLOGO_CMYK.eps}\\
	\end{tabular}
	}
	\makeatletter	
}{	
	%-- Für Spicker einiges anders:
	\ifthenelse{\equal{\customDocumentClass}{cheatsheet}}{
		\documentclass[a4paper,10pt,landscape]{scrartcl}
		\usepackage{geometry}
		\geometry{top=2mm, bottom=2mm, headsep=0mm, footskip=0mm, left=2mm, right=2mm}
		
		% Für Spicker \spsection für Section, zur Strukturierung \HRule oder \HDRule Linie einsetzen
		\usepackage{multicol}
		\newcommand{\spsection}[1]{\textbf{#1}}	% Platzsparende "section" für Spicker
	}{	%-- Ende Spicker-Unterscheidung-if
		%-- Unterscheidung Doppelseitig
		\ifthenelse{\boolean{twosided}}{
			\documentclass[a4paper,11pt, footheight=26pt,twoside]{\customDocumentClass}
			\usepackage[head=23pt]{geometry}	% head=23pt umgeht Fehlerwarnung, dafür größeres "top" in geometry
			\geometry{top=30mm, bottom=22mm, headsep=10mm, footskip=12mm, inner=27mm, outer=13mm}
		}{
			\documentclass[a4paper,11pt, footheight=26pt]{\customDocumentClass}
			\usepackage[head=23pt]{geometry}	% head=23pt umgeht Fehlerwarnung, dafür größeres "top" in geometry
			\geometry{top=30mm, bottom=22mm, headsep=10mm, footskip=12mm, left=20mm, right=20mm}
		}
		%-- Nummerierung bis Subsubsection für Report
		\ifthenelse{\equal{\customDocumentClass}{report} \OR \equal{\customDocumentClass}{scrreprt}}{
			\setcounter{secnumdepth}{3}	% zählt auch subsubsection
			\setcounter{tocdepth}{3}	% Inhaltsverzeichnis bis in subsubsection
		}{}
	}%-- Ende Spicker-Unterscheidung-else
	
	\usepackage{scrlayer-scrpage}	% Kopf-/Fußzeile
	\renewcommand*{\thefootnote}{\fnsymbol{footnote}}	% Fußnoten-Symbole anstatt Zahlen
	\renewcommand*{\titlepagestyle}{empty} % Keine Seitennummer auf Titelseite
	\usepackage[perpage]{footmisc}	% Fußnotenzählung Seitenweit, nicht Dokumentenweit
}

% Input inkl. Umlaute, Silbentrennung
\RequirePackage[T1]{fontenc}
\RequirePackage[utf8]{inputenc}
\usepackage[english,ngerman]{babel}
\usepackage{csquotes}	% Anführungszeichen
\RequirePackage{marvosym}
\usepackage{eurosym}

% Style-Aufhübschung
\usepackage{soul, color}	% Kapitälchen, Unterstrichen, Durchgestrichen usw. im Text
%\usepackage{titleref}
\usepackage[breakwords, fit]{truncate}	% Abschneiden von Sätzen
\renewcommand{\TruncateMarker}{\,…}

% Mathe usw.
\usepackage{amssymb}
\usepackage{amsthm}
\ifthenelse{\equal{\customDocumentClass}{beamer}}{}{
\usepackage[fleqn,intlimits]{amsmath}	% fleqn: align-Umgebung rechtsbündig; intlimits: Integralgrenzen immer ober-/unterhalb
}
%\usepackage{mathtools} % u.a. schönere underbraces
\usepackage{xcolor}
\usepackage{esint}	% Schönere Integrale, \oiint vorhanden
\everymath=\expandafter{\the\everymath\displaystyle}	% Mathe Inhalte werden weniger verkleinert
\usepackage{wasysym}	% mehr Symbole, bspw \lightning
% Auch arcus-Hyperbolicus-Funktionen
\DeclareMathOperator{\arccot}{arccot}
\DeclareMathOperator{\arccosh}{arccosh}
\DeclareMathOperator{\arcsinh}{arcsinh}
\DeclareMathOperator{\arctanh}{arctanh}
\DeclareMathOperator{\arccoth}{arccoth} 
%\renewcommand{\int}{\int\limits}
%\usepackage{xfrac}	% mehr fracs: sfrac{}{}
\let\oldemptyset\emptyset	% schöneres emptyset
\let\emptyset\varnothing
%\RequirePackage{mathabx}	% mehr Symbole
\mathchardef\mhyphen="2D	% Hyphen in Math

% tikz usw.
\usepackage{tikz}
\usepackage{pgfplots}
\pgfplotsset{compat=1.11}	% Umgeht Fehlermeldung
\usetikzlibrary{graphs}
%\usetikzlibrary{through}	% ???
\usetikzlibrary{arrows}
\usetikzlibrary{arrows.meta}	% Pfeile verändern / vergrößern: \draw[-{>[scale=1.5]}] (-3,5) -> (-3,3);
\usetikzlibrary{automata,positioning} % Zeilenumbruch im Node node[align=center] {Text\\nächste Zeile} automata für Graphen
\usetikzlibrary{matrix}
\usetikzlibrary{patterns}	% Schraffierte Füllung
\usetikzlibrary{shapes.geometric}	% Polygon usw.
\tikzstyle{reverseclip}=[insert path={	% Inverser Clip \clip
	(current page.north east) --
	(current page.south east) --
	(current page.south west) --
	(current page.north west) --
	(current page.north east)}
% Nutzen: 
%\begin{tikzpicture}[remember picture]
%\begin{scope}
%\begin{pgfinterruptboundingbox}
%\draw [clip] DIE FLÄCHE, IN DER OBJEKT NICHT ERSCHEINEN SOLL [reverseclip];
%\end{pgfinterruptboundingbox}
%\draw DAS OBJEKT;
%\end{scope}
%\end{tikzpicture}
]	% Achtung: dafür muss doppelt kompliert werden!
\usepackage{graphpap}	% Grid für Graphen
\tikzset{every state/.style={inner sep=2pt, minimum size=2em}}
\usetikzlibrary{mindmap, backgrounds}
%\usepackage{tikz-uml}	% braucht Dateien: http://perso.ensta-paristech.fr/~kielbasi/tikzuml/

% Tabular
\usepackage{longtable}	% Große Tabellen über mehrere Seiten
\usepackage{multirow}	% Multirow/-column: \multirow{2[Anzahl der Zeilen]}{*[Format]}{Test[Inhalt]} oder \multicolumn{7[Anzahl der Reihen]}{|c|[Format]}{Test2[Inhalt]}
\renewcommand{\arraystretch}{1.3} % Tabellenlinien nicht zu dicht
\usepackage{colortbl}
\arrayrulecolor{gray}	% heller Tabellenlinien
\usepackage{array}	% für folgende 3 Zeilen (für Spalten fester breite mit entsprechender Ausrichtung):
\newcolumntype{L}[1]{>{\raggedright\let\newline\\\arraybackslash\hspace{0pt}}m{\dimexpr#1\columnwidth-2\tabcolsep-1.5\arrayrulewidth}}
\newcolumntype{C}[1]{>{\centering\let\newline\\\arraybackslash\hspace{0pt}}m{\dimexpr#1\columnwidth-2\tabcolsep-1.5\arrayrulewidth}}
\newcolumntype{R}[1]{>{\raggedleft\let\newline\\\arraybackslash\hspace{0pt}}m{\dimexpr#1\columnwidth-2\tabcolsep-1.5\arrayrulewidth}}
\usepackage{caption}	% Um auch unbeschriftete Captions mit \caption* zu machen

% Nützliches
\usepackage{verbatim}	% u.a. zum auskommentieren via \begin{comment} \end{comment}
\usepackage{tabto}	% Tabs: /tab zum nächsten Tab oder /tabto{.5 \CurrentLineWidth} zur Stelle in der Linie
\NumTabs{6}	% Anzahl von Tabs pro Zeile zum springen
\usepackage{listings} % Source-Code mit Tabs
\usepackage{lstautogobble} 
\ifthenelse{\equal{\customDocumentClass}{beamer}}{}{
\usepackage{enumitem}	% Anpassung der enumerates
%\setlist[enumerate,1]{label=(\arabic*)}	% global andere Enum-Items
\renewcommand{\labelitemiii}{$\scriptscriptstyle ^\blacklozenge$} % global andere 3. Item-Aufzählungszeichen
}
\newenvironment{anumerate}{\begin{enumerate}[label=(\alph*)]}{\end{enumerate}} % Alphabetische Aufzählung
\usepackage{letltxmacro} % neue Definiton von Grundbefehlen
% Nutzen:
%\LetLtxMacro{\oldemph}{\emph}
%\renewcommand{\emph}[1]{\oldemph{#1}}
\RequirePackage{xpatch}	% ua. Konkatenieren von Strings/Variablen (etoolbox)


% Einrichtung von lst
\lstset{
basicstyle=\ttfamily, 
%mathescape=true, 
%escapeinside=^^, 
autogobble, 
tabsize=2,
basicstyle=\footnotesize\sffamily\color{black},
frame=single,
rulecolor=\color{lightgray},
numbers=left,
numbersep=5pt,
numberstyle=\tiny\color{gray},
commentstyle=\color{gray},
keywordstyle=\color{green},
stringstyle=\color{orange},
morecomment=[l][\color{magenta}]{\#}
showspaces=false,
showstringspaces=false,
breaklines=true,
literate=%
    {Ö}{{\"O}}1
    {Ä}{{\"A}}1
    {Ü}{{\"U}}1
    {ß}{{\ss}}1
    {ü}{{\"u}}1
    {ä}{{\"a}}1
    {ö}{{\"o}}1
    {~}{{\textasciitilde}}1
}
\usepackage{scrhack} % Fehler umgehen
\def\ContinueLineNumber{\lstset{firstnumber=last}} % vor lstlisting. Zum wechsel zum nicht-kontinuierlichen muss wieder \StartLineAt1 eingegeben werden
\def\StartLineAt#1{\lstset{firstnumber=#1}} % vor lstlisting \StartLineAt30 eingeben, um bei Zeile 30 zu starten
\let\numberLineAt\StartLineAt

% BibTeX
\usepackage[backend=bibtex8, bibencoding=ascii,
%style=authortitle, citestyle=authortitle-ibid,
%doi=false,
%isbn=false,
%url=false
]{biblatex}	% BibTeX
\usepackage{makeidx}
%\makeglossary
%\makeindex

% Grafiken
\usepackage{graphicx}
\usepackage{epstopdf}	% eps-Vektorgrafiken einfügen
%\epstopdfsetup{outdir=\customDir}

% pdf-Setup
\usepackage{pdfpages}
\ifthenelse{\equal{\customDocumentClass}{beamer}}{}{
\usepackage[bookmarks,%
bookmarksopen=false,% Klappt die Bookmarks in Acrobat aus
colorlinks=true,%
linkcolor=black,%
citecolor=red,%
urlcolor=green,%
]{hyperref}
}

%-- Unterscheidung des Stils
\newcommand{\customLogo}{}
\newcommand{\customPreamble}{}
\ifthenelse{\equal{\customDesign}{htw}}{
	% HTW Corporate Design: Arial (Helvetica)
	\usepackage{helvet}
	\renewcommand{\familydefault}{\sfdefault}
	\renewcommand{\customLogo}{HTW-Logo}
	\renewcommand{\customPreamble}{HTW Dresden}
}{
% \renewcommand{\customLogo}{HTW-Logo.eps}
}

% Nach Dokumentenbeginn ausführen:
\AtBeginDocument{
	% Autor und Titel für pdf-Eigenschaften festlegen, falls noch nicht geschehen
	\providecommand{\pdfAuthor}{John Doe}
	\ifdefempty{\customAuthor} {} {\renewcommand{\pdfAuthor}{\customAuthor}}
	\providecommand{\pdfTitle}{}
	\providecommand{\pdfTitleA}{}
	\providecommand{\pdfTitleB}{}
	\providecommand{\pdfTitleC}{}	
	\ifdefempty{\pdfTitle}{
		\ifdefempty{\customPreamble} {} {\renewcommand{\pdfTitleA}{\customPreamble{} | }}
		\ifdefempty{\customTitle} {\renewcommand{\pdfTitleB}{No Title}} {\renewcommand{\pdfTitleB}{\customTitle}}
		\ifdefempty{\customSubtitle} {} {\renewcommand{\pdfTitleC}{ - \customSubtitle}}
	}{}
	
	\newcommand{\customLogoLocation}{\customDir .LaTeX_master/\customLogo}
	\hypersetup{
		pdfauthor={\pdfAuthor},
		pdftitle={\pdfTitleA\pdfTitleB\pdfTitleC},
	}
	\ifthenelse{\equal{\customDocumentClass}{beamer}}{
		\title{\customTitle}
		\author{\customAuthor}
	}{
		\automark[section]{section}
		\automark*[subsection]{subsection}
		\pagestyle{scrheadings}
		\ifthenelse{\equal{\customDocumentClass}{report} \OR \equal{\customDocumentClass}{scrreprt}}{
		\renewcommand*{\chapterpagestyle}{scrheadings}
		}{}
		%\renewcommand*{\titlepagestyle}{scrheadings}
		\ihead{\includegraphics[height=1.7em]{\customLogoLocation}}
		%\ohead{\truncate{4cm}{\customTitle}}
		\chead{\truncate{.5\textwidth}{\headmark}}
		\ohead{\customTitle}
		\cfoot{\pagemark}
		\ofoot{\customSignature}
		% Titelseite
		\title{
		\includegraphics[width=0.35\textwidth]{\customDir .LaTeX_master/\customLogo}\\\vspace{0.5em}
		\Huge\textbf{\customTitle}
		\ifdefempty{\customSubtitle} {} {\\\vspace*{0.7em}\Large \customSubtitle}
		\\\vspace*{5em}}
		\author{
		\ifdefempty{\customNoteA} {} {\customNoteA \vspace*{1em}}\\ 
		\ifdefempty{\customAuthor} {} {\customTitleAuthor}
		\ifdefempty{\customNoteB}{}{\vspace*{1em}\\\customNoteB}
		}
		
		\ifthenelse{\equal{\customDocumentClass}{cheatsheet}}{
			\pagestyle{empty}
			\setlist{nolistsep}
	%		\usepackage{parskip}	% Aufzählung Abstand
	%		\setlength{\parskip}{0em}
			\lstset{
	    belowcaptionskip=0pt,
	    belowskip=0pt,
	    aboveskip=0pt,
			tabsize=2,
			frame=none,
			numbers=none,
			showspaces=false,
			showstringspaces=false,
			breaklines=true,
			}
		}{}
	}
}

% Unterabschnitte
%\newtheorem{example}{Beispiel}%[section]
%\newtheorem{definition}{Definition}[section]
%\newtheorem{discussion}{Diskussion}[section]
%\newtheorem{remark}{Bemerkung}[section]
%\newtheorem{proof}{Beweis}[section]
%\newtheorem{notation}{Schreibweise}[section]
\RequirePackage{xcolor}

% Horizontale Linie:
\newcommand{\HRule}[1][\medskipamount]{\par
  \vspace*{\dimexpr-\parskip-\baselineskip+#1}
  \noindent\rule[0.2ex]{\linewidth}{0.2mm}\par
  \vspace*{\dimexpr-\parskip-.5\baselineskip+#1}}
% Gestrichelte horizontale Linie:
\RequirePackage{dashrule}
\newcommand{\HDRule}[1][\medskipamount]{\par
  \vspace*{\dimexpr-\parskip-\baselineskip+#1}
  \noindent\hdashrule[0.2ex]{\linewidth}{0.2mm}{1mm} \par
  \vspace*{\dimexpr-\parskip-.5\baselineskip+#1}}
% Mathe in Anführungszeichen:
\newsavebox{\mathbox}\newsavebox{\mathquote}
\makeatletter
\newcommand{\mq}[1]{% \mathquotes{<stuff>}
  \savebox{\mathquote}{\text{"}}% Save quotes
  \savebox{\mathbox}{$\displaystyle #1$}% Save <stuff>
  \raisebox{\dimexpr\ht\mathbox-\ht\mathquote\relax}{"}#1\raisebox{\dimexpr\ht\mathbox-\ht\mathquote\relax}{''}
}
\makeatother

% Paragraph mit Zähler (Section-Weise)
\newcounter{cparagraphC}
\newcommand{\cparagraph}[1]{
\stepcounter{cparagraphC}
\paragraph{\thesection{}-\Roman{cparagraphC} #1}
%\addcontentsline{toc}{subsubsection}{\thecparagraphC{} #1}
\label{\thesection-\thecparagraphC}
}
\makeatletter
\@addtoreset{cparagraphC}{section}
\makeatother


% (Vorlesungs-)Folien einbinden:
% Folien von einer Datei skaliert
\newcommand{\slideScale}[2][0.5]{\begin{center}
\includegraphics[page=#2, scale=#1]{\customSlidePath .pdf}
\end{center}}
% Folien von einer Datei mit fester Höhe
\newcommand{\slideHeight}[2][9.63cm]{\begin{center}
\includegraphics[page=#2, height=#1]{\customSlidePath .pdf}
\end{center}}
% Folien von einer Datei mit fester Breite
\newcommand{\slideWidth}[2][12.8cm]{\begin{center}
\includegraphics[page=#2, width=#1]{\customSlidePath .pdf}
\end{center}}
% Folien von mehreren nummerierten Dateien skaliert
\newcommand{\slidesScale}[3][0.5]{\begin{center}
\includegraphics[page=#3, scale=#1]{\customSlidePath #2.pdf}
\end{center}}
% Folien von mehreren nummerierten Dateien mit fester Höhe
\newcommand{\slidesHeight}[3][9.63cm]{\begin{center}
\includegraphics[page=#3, height=#1]{\customSlidePath #2.pdf}
\end{center}}
% Folien von mehreren nummerierten Dateien mit fester Breite
\newcommand{\slidesWidth}[3][12.8cm]{\begin{center}
\includegraphics[page=#3, width=#1]{\customSlidePath #3.pdf}
\end{center}}


%% EINFACHE BEFEHLE

% Abkürzungen Mathe
\newcommand{\EE}{\mathbb{E}}
\newcommand{\QQ}{\mathbb{Q}}
\newcommand{\RR}{\mathbb{R}}
\newcommand{\CC}{\mathbb{C}}
\newcommand{\NN}{\mathbb{N}}
\newcommand{\ZZ}{\mathbb{Z}}
\newcommand{\PP}{\mathbb{P}}
\renewcommand{\SS}{\mathbb{S}}
\newcommand{\cA}{\mathcal{A}}
\newcommand{\cB}{\mathcal{B}}
\newcommand{\cC}{\mathcal{C}}
\newcommand{\cD}{\mathcal{D}}
\newcommand{\cE}{\mathcal{E}}
\newcommand{\cF}{\mathcal{F}}
\newcommand{\cG}{\mathcal{G}}
\newcommand{\cH}{\mathcal{H}}
\newcommand{\cI}{\mathcal{I}}
\newcommand{\cJ}{\mathcal{J}}
\newcommand{\cM}{\mathcal{M}}
\newcommand{\cN}{\mathcal{N}}
\newcommand{\cP}{\mathcal{P}}
\newcommand{\cR}{\mathcal{R}}
\newcommand{\cS}{\mathcal{S}}
\newcommand{\cZ}{\mathcal{Z}}
\newcommand{\cL}{\mathcal{L}}
\newcommand{\cT}{\mathcal{T}}
\newcommand{\cU}{\mathcal{U}}
\newcommand{\cV}{\mathcal{V}}
\renewcommand{\phi}{\varphi}
\renewcommand{\epsilon}{\varepsilon}

% Farbdefinitionen
\definecolor{red}{RGB}{180,0,0}
\definecolor{green}{RGB}{75,160,0}
\definecolor{blue}{RGB}{0,75,200}
\definecolor{orange}{RGB}{255,128,0}
\definecolor{yellow}{RGB}{255,245,0}
\definecolor{purple}{RGB}{75,0,160}
\definecolor{cyan}{RGB}{0,160,160}
\definecolor{brown}{RGB}{120,60,10}

\definecolor{itteny}{RGB}{244,229,0}
\definecolor{ittenyo}{RGB}{253,198,11}
\definecolor{itteno}{RGB}{241,142,28}
\definecolor{ittenor}{RGB}{234,98,31}
\definecolor{ittenr}{RGB}{227,35,34}
\definecolor{ittenrp}{RGB}{196,3,125}
\definecolor{ittenp}{RGB}{109,57,139}
\definecolor{ittenpb}{RGB}{68,78,153}
\definecolor{ittenb}{RGB}{42,113,176}
\definecolor{ittenbg}{RGB}{6,150,187}
\definecolor{itteng}{RGB}{0,142,91}
\definecolor{ittengy}{RGB}{140,187,38}

\definecolor{htworange}{RGB}{249,155,28}

% Textfarbe ändern
\newcommand{\tred}[1]{\textcolor{red}{#1}}
\newcommand{\tgreen}[1]{\textcolor{green}{#1}}
\newcommand{\tblue}[1]{\textcolor{blue}{#1}}
\newcommand{\torange}[1]{\textcolor{orange}{#1}}
\newcommand{\tyellow}[1]{\textcolor{yellow}{#1}}
\newcommand{\tpurple}[1]{\textcolor{purple}{#1}}
\newcommand{\tcyan}[1]{\textcolor{cyan}{#1}}
\newcommand{\tbrown}[1]{\textcolor{brown}{#1}}

% Umstellen der Tabellen Definition
\newcommand{\mpb}[1][.3]{\begin{minipage}{#1\textwidth}\vspace*{3pt}}
\newcommand{\mpe}{\vspace*{3pt}\end{minipage}}

\newcommand{\resultul}[1]{\underline{\underline{#1}}}
\newcommand{\parskp}{$ $\\}	% new line after paragraph
\newcommand{\corr}{\;\widehat{=}\;}
\newcommand{\mdeg}{^{\circ}}

\newcommand{\nok}[2]{\binom{#1}{#2}}	% n über k BESSER: \binom{n}{k}
\newcommand{\mtr}[1]{\begin{pmatrix}#1\end{pmatrix}}	% Matrix
\newcommand{\dtr}[1]{\begin{vmatrix}#1\end{vmatrix}}	% Determinante (Betragsmatrix)
\renewcommand{\vec}[1]{\underline{#1}}	% Vektorschreibweise
\newcommand{\imptnt}[1]{\colorbox{red!30}{#1}}	% Wichtiges
\newcommand{\intd}[1]{\,\mathrm{d}#1}
\newcommand{\diffd}[1]{\mathrm{d}#1}
% für Module-Rechnung: \pmod{}

%\documentclass[a4paper,18pt]{scrreprt}

\usepackage[ngerman]{babel}
\usepackage{fontspec} 
\usepackage{unicode-math}
\usepackage[a4paper]{geometry}
\geometry{verbose,tmargin=2cm,bmargin=2cm,lmargin=2cm,rmargin=2cm,headheight=1cm,headsep=1cm,footskip=1cm}
\usepackage{graphicx}

\providecommand{\customTitle}{Titel}
\providecommand{\customAuthor}{Autor}

\title{\customTitle}
\author{\customAuthor}

\usepackage{xcolor}
\definecolor{darkblue}{RGB}{0,0,102}
\color{darkblue}

\providecommand{\imptnt}[1]{#1}
\usepackage{amsmath}
\usepackage{enumitem}	% Anpassung der enumerates

\usepackage{geometry}
\geometry{top=10mm, bottom=15mm, left=15mm, right=15mm}

\setmainfont{FJS} 
\setsansfont{FJS} 
\setmonofont{FJS}

%\setmathfont{FJS}
%\setmathfont[range=\mathup]  {FJS}
%\setmathfont[range=\mathbfup]{FJS}
%\setmathfont[range=\mathbfit]{FJS}
%\setmathfont[range=\mathit]  {FJS}

\usepackage{enumitem}	% Anpassung der enumerates

\usepackage{array}	% für folgende 3 Zeilen (für Spalten fester breite mit entsprechender Ausrichtung):
\newcolumntype{L}[1]{>{\raggedright\let\newline\\\arraybackslash\hspace{0pt}}m{\dimexpr#1\columnwidth-2\tabcolsep-1.5\arrayrulewidth}}
\newcolumntype{C}[1]{>{\centering\let\newline\\\arraybackslash\hspace{0pt}}m{\dimexpr#1\columnwidth-2\tabcolsep-1.5\arrayrulewidth}}
\newcolumntype{R}[1]{>{\raggedleft\let\newline\\\arraybackslash\hspace{0pt}}m{\dimexpr#1\columnwidth-2\tabcolsep-1.5\arrayrulewidth}}
\usepackage{caption}	% Um auch unbeschriftete Captions mit \caption* zu machen


\begin{document}

\maketitle
\newpage
\tableofcontents
\newpage

\chapter*{Vorbemerkung}
Prüfung: Keine Rechnung, kein Programmieren. Reine Wissensabfrage (Welche Sensoren für diesen Zweck, Vergleich, Vorteil/Nachteil, …)\\
Hilfsmittel: alles Handschriftliche zugelassen.

\setcounter{chapter}{-1}
\chapter{Einführung}

Elektronische Geräte (PC usw.) sind blind, brauchen Sensoren.

\section{Eingabegeräte}
Eingabegeräte: alle Geräte, über die einem Computer Informationen zugeführt werden können.
\begin{itemize}
\item \emph{Umfeldsensoren}: führen die Informationen über die Umgebung unmittelbar dem Computer zu (ohne menschliches einwirken): Temperatur, Druck, Schall, … $\Rightarrow$ Hauptthema der VL
\item \emph{User Interface Devices}: Informationseingabe durch Benutzer: Tastaturen, Touchscreens (kapazitiv, induktiv, …) usw.
\end{itemize}

\section{Ziel des Kurses}
\begin{itemize}
\item \emph{Sensorik} (Sensortechnik): Teilgebiet der Messtechnik, das sich mit Entwicklung und Einsatz von Sensoren befasst.
\item wichtigste physikalischen Prinzipien zur Erfassung nichtelektrischer Messgrößen
\item Funktionsweise ausgewählter Sensoren
\item Vor- und Nachteile jedes Typs
\item Anwendungshinweise für den zuverlässigen Sensorenbetrieb
\end{itemize}

Kapitel:
\begin{itemize}
\item Grundlagen der Messtechnik
\item Einführung in die Sensorik
\item Temperatursensoren
\item Drucksensoren
\item Durchflusssensoren
\item Binäre Positionssensoren
\end{itemize}

\chapter{Grundlagen der Messtechnik}

\section{Messgröße und Messwert}

\begin{itemize}
\item \imptnt{\emph{Messgröße}} ist ein messbares \imptnt{Merkmal} eines Objektes oder Prozess:
\begin{itemize}
\item Eigenschaft 
\item Vorgang
\item Zustand
\end{itemize}
$\Rightarrow$ Temperatur, Druck, Durchfluss, Position
\item \imptnt{\emph{Maßeinheit}}: eine durch Vereinbarung festgelegte \imptnt{Vergleichsgröße}
\item \emph{SI-System:} m, kg, s, A, K, mol, cd
\item \imptnt{\emph{Messwert}}: der gemessene Wert einer Messgröße 
$=ZW \cdot \mathrm{Einheit}$ 
%$=ZW * \mathrm{Einheit}$ 
(also Messgröße kombiniert mit Maßeinheit)
\item Messung: (Ablauf)
\begin{itemize}
\item Quelle (Träger eine Messgröße)
\item Erfassung der Größe \& Vergleich der Größe mit einer Einheit
\item Anzeige (des Messwerts)
\end{itemize}
\end{itemize}

\subsection{Normal}
\emph{Normal} (DIN 1390): Maßverkörperung, Messgerät, Referenzmaterial oder Messeinrichtung zum \linebreak 
Zweck eine Einheit festzulegen, zu verkörpern, zu bewahren oder zu reproduzieren.
\begin{itemize}
\item Internationale (nationale) Normale (Kopie der Ursprungs-Normale: bspw. Kopie Ur-Kilo)
\item Bezugsnormale (PTB) (für Betriebe usw.)
\item Gebrauchs-, Prüfnormale (für die konkrete Messung)
\item Lehren
\item …
\end{itemize}

\subsection{Messen}
\emph{Messen} ist der Vorgang, durch den ein Wert einer physikalischen Größe als Vielfaches eines Bezugswertes (Maßeinheit) ermittelt wird (Messen = Vergleichen).

\subsection{Messprinzipien}
\emph{Messprinzip}: eine charakteristische naturwissenschaftliche Erscheinung oder ein gesetzmäßiger Zusammenhang, der einer Messung zugrunde gelegt wird.

Physikalische Grundlage der Messung: über 100 physikalisch-chemisch-biologische Effekte.

\section{Messmethoden}
\emph{Messmethode}: „spezielle, vom Messprinzip unabhängige Art des Vorgehens bei der Messung.“ (DIN 1319-1)
\begin{itemize}
\item allgemeine Vorgehensweise bei der Durchführung von Messungen
\item nicht an eine physikalische Realisierung gebunden
\end{itemize}

\subsection{Methoden nach Art der Messgröße}
Bsp. Messung der Temperatur (die sich nicht ändert)
\begin{itemize}
\item \emph{statisch}: bestimmt wird eine zeitlich unveränderliche Messgröße nach einem Messprinzip, das nicht auf der zeitlichen Änderung anderer Größen beruht (Bsp.: Ablesen der Flüssigkeits-Säule im Thermometer)
\item \emph{dynamisch}: Messgröße ist entweder selbst zeitlich veränderlich oder ihr Wert ergibt sich aus zeitlichen Änderungen anderer Größen (Messung durch Einwirkung anderer Kräfte um die Reaktion heraus zu finden [und so zu messen]. Bsp. Halbwertszeit von Materialien lässt Rückschlüsse auf Temperatur zu)
\end{itemize}

\subsection{Methoden nach Art der Vergleichsgröße}
\begin{itemize}
\item \emph{direkte Messmethode}: der Wert der Messgröße wird durch quantitative Vergleich mit einer physikalischen gleichartigen Einheit ermittelt (Metermaß).
\item \emph{indirekte Messmethode}: der Messwert wird aus den Messdaten einer oder mehrerer anderer (direkt messbaren) Größen ermittelt, die mit der gesuchten Messgröße in einem definierten Zusammenhang steht (stehen) (Messung der Temperatur durch Messung der Ausdehnung der Flüssigkeit im Thermometer)
\item \emph{inkrementale Messmethode}: der Messwert wird von einem Bezugspunkt aus durch Addition/Subtraktion von kleinen Wertzuwachsen (Inkrementen) ermittelt (Geld zählen: Münzen (als Inkrement) in Säule stapeln, diese messen und damit den Wert bestimmen; vgl. auch Kugelmaus: gerasterte, inkrementale Messung der Veränderung der Kugelbewegung)
\end{itemize}

\subsection{Methoden nach Art der Vergleichsvorgehensweise}
\begin{itemize}
\item \emph{Ausschlagmethode}: Messgröße wird direkt oder über eine Zwischengröße in eine möglichst proportionale Ausgangsgröße umgewandelt (vglw. Küchenwaage mit Feder)\\
Messgröße $M$ $\to$ Übertragungsfunktion $f$ $\to$ Ausgangsgröße $A$\\
$\Rightarrow \boxed{M = f^{-1}(A)}$
\begin{itemize}[label=$+$]
\item keine Hilfsenergie benötigt
\item einfacher Aufbau
\item schnell
\end{itemize}
\begin{itemize}[label=-]
\item Energieentzug aus Messobjekt $\to$ Rückwirkung
\item Kennlinie (Übertragungsfunktion) des Messgerätes muss bekannt sein
\item starker Einfluss von Störgrößen
\end{itemize}
\item \emph{Kompensationsmethode}: Messgerät erzeugt eine bekannte variable Vergleichsgröße, um die Differenz zwischen beiden Größen gegen Null streben zu lassen.\\
ABB AS1\\
$M=S$\\
(Vergleich: alte Küchenwaage mit Schieberegler/Ausgleichsgewichten)
\begin{itemize}[label=$+$]
\item Variable Art der Kompensationsgröße
\item Reduzierung der Rückwirkung und Störeinflüsse
\item Nichtlinearität unkritisch
\item keine Kalibrierung notwendig
\item leichte Realisierung großer Messbereiche
\item hohe Genauigkeit möglich
\end{itemize}
\begin{itemize}[label=-]
\item größerer gerätetechnischer Aufwand
\item mehrstufig: in der Regel langsam
\item Hilfsenergie notwendig
\item Viele bekannte Vergleichsgrößen nötig
\end{itemize}
\end{itemize}
\paragraph{Beispiel Spannungsmessung}
\begin{itemize}
\item Ausschlagmethode\\
ABB AS2\\
Problem: braucht unendlich großen Widerstand, sonst verfälscht es die Messung
\item Kompensationsmethode\\
ABB AS3\\
Mit bekannter Spannung $U_{ref}$.\\
Strommessgerät: Ist Strom gleich 0? Dann sind beide Spannungen gleich.
\end{itemize}
weitere Methoden:
\begin{itemize}
\item Differenzierungsmethode
\item Substitutionsmethode
\end{itemize}

\subsection{Methoden nach Zeitablauf}
\begin{itemize}
\item \emph{kontinuierliche Messmethode}: die Größe wird ohne zeitliche Unterbrechung erfasst und auch angezeigt. (Vgl. Thermometer mit Flüssigkeit verändert sich stets/kontinuierlich)\\
ABB AS4
\item \emph{diskontinuierliche Messmethode}: die Messgröße wird nur zu bestimmten (diskreten) Zeitpunkten erfasst (abgetastet) und/oder angezeigt.\\
ABB AS5\\
Informationen werden begrenzt (aufs wesentliche), Zeit muss diskretisiert werden (in vernünftigen Maß: Nyquist-Shannon-Abtasttheorem $f_{Abtast}\geq f_{max}$)
\end{itemize}

\subsection{Methoden nach Wertebereich der Ausgangsgröße}
\begin{itemize}
\item \emph{analoge Messmethode}: die Messgröße wird durch eine eindeutige und stetige Anzeigegröße (Messwert) abgebildet.
\begin{itemize}[label=\textbullet]
\item kontinuierlicher Wertebereich
\end{itemize}
(Spannungsmesser mit Zeiger, Analoge Küchenwaage)
\item \emph{digitale Messmethode}: die Messgröße wird in Form einer in festgelegten Schritten quantisierten Anzeigegröße abgebildet.
\begin{itemize}[label=\textbullet]
\item diskreter Wertebereich
\end{itemize}
(digital Multimeter)
\end{itemize}

\section{Signale}
\begin{itemize}
\item \emph{Signal} in der Messtechnik: wenn man einer speziell ausgewählten \emph{zeitlich veränderlichen} physikalischen Größe (Signalträger, Informationsträger) eine Information zuordnet. \\
Hat 3 Elemente: Erzeuger, Träger, Empfänger (Kommunikation: Nachricht wird generiert (im Kopf) und Schall wird erzeugt (Sprechen) $\to$ Schall wird übertragen $\to$ Schall wird gehört (Ohr) und Schall-Information wird entschlüsselt (im Kopf))
\item \emph{Messsignal}: Signal mit einem (oder mehreren) zeitvariablen informationstragenden Parameter (Informationsparameter), der die Werte der Messgröße in eindeutiger und reproduzierbarer Weise abgebildet.
\item \emph{Rauschen}: zufällige zeitlich veränderliche Größe ohne nützliche Information.
\end{itemize}

\subsection{Informationsparameter}
Kontinuierlich:
\begin{itemize}
\item Signalträgerwert („Wert auf x-Achse“)
\item Amplitude
\item Frequenz
\item Phase
\end{itemize}
Diskret:
\begin{itemize}
\item Impulshöhe
\item Pulsbreite (Pulsbreite/-länge als Indikator für Wert)
\item Pulsfolge (Digitaltechnik: 001 hat andere Folge als 010)
\end{itemize}

\subsection{Signaltypen}
\begin{itemize}
\item \emph{Determiniertes Singal}\\
Der Signalwert ist zu jedem Zeitpunkt verfügbar
\begin{itemize}[label=$+$]
\item Information mit einmaliger Messung gewinnbar
\item[$-$] zu viel Information. Sie kann auch durch Störung unbrauchbar werden
\end{itemize}
\item Stochastisches Signal\\
regellos, zufällig schwankender Signalverlauf
\begin{itemize}[label=$+$]
\item Störungen machen sich nur stark reduziert bemerkbar, sie werden über die Messzeit integriert
\item[$-$] Information ist erst mit mehrmaligen Messungen zu gewinnen, das erfordert einen großen Zeitbedarf
\end{itemize}
\item \emph{Signalgemisch}\\
deterministisches Signal mit einem stochastischen Anteil (Rauschen)
\begin{itemize}
\item[$-$] Das Rauschen ist unerwünscht und muss unterdrückt oder ausgefiltert werden
\end{itemize}
\end{itemize}

\subsubsection{Nach Wertebereich}
\begin{itemize}
\item \emph{Wertkontinuierliches Signal}: kontinuierlicher Wertebereich
\begin{itemize}[label=$+$]
\item der Informationsparameter bildet adäquat die Messgröße ab
\item kann theoretisch beliebig viele Werte innerhalb seines Wertebereichs annehmen
\item[$-$] einfach zu stören
\end{itemize}
\item \emph{Wertdiskretes Signal}: diskontinuierlicher Wertebereich aus einer endlichen Anzahl von vordefinierten Werten
\begin{itemize}
\item[$+$] Störeinflüsse machen sich erst nach Überschreiten von Grenzwerten bemerkbar
\item[$-$] möglicher Informationsverlust (wenn Inkrement falsch gewählt wird)
\end{itemize}
\item \emph{binäres Signal}: der Wertebereich hat nur zwei Werte
\end{itemize}
\subsubsection{Nach zeitlichen Verlauf}
\begin{itemize}
\item \emph{kontinuierliches Signal}: IP kann zu jedem beliebigen Zeitpunkt seinen Wert ändern
\begin{itemize}[label=$+$]
\item jederzeit vorhanden: jederzeit ist der zeitliche Verlauf der Messwerten verfolgbar
\item[$-$] Störungen können jederzeit wirken
\item[$-$] Informationsmenge ist oft unnötig groß
\end{itemize}
\item \emph{diskontinuierliches Signal}: IP kann nur zu diskreten Zeitpunkten seinen wert ändern.
\begin{itemize}[label=$+$]
\item Störungen zwischen den Zeitpunkten der Parameterveränderungen können sich nicht auswirken
\item[$-$] Informationen stehen nur zu diskreten Zeitpunkten zur Verfügung (Informationsverlust)
\end{itemize}
\end{itemize}
\subsubsection{Kombinationen}
\begin{itemize}
\item Analogsignal
\item Digitalsignal (Zeit- und Wert-diskret)
\begin{itemize}[label=$+$]
\item hohe Störsicherheit
\item einfache Datenspeicherung
\item flexible Weiterverarbeitung
\item vielfältige Übertragungsmöglichkeiten
\end{itemize}
\item diskret-kontinuierlich ($\Delta A$: Amplituden-Quantisierungsintervall)
\item analog-diskontinuierlich ($\Delta t$: Zeit-Quantisierungsintervall)
\end{itemize}

\subsection{Normsignale}
Strom als Informationsträger ist störungssicher $\to$ Spannung ist leichter zu manipulieren als Stromstärke!
\begin{itemize}
\item \emph{Strom analog}:\\
$0\dots 20\; \mathrm{mA}$ (dead zero)\\
$4 \dots 20 \;\mathrm{mA}$ (live zero)
\item \emph{Spannung analog}:\\
$0\dots 10 \;\mathrm{mA}$ (dead zero)\\
$2 \dots 10 \;\mathrm{mA}$ (live zero)
\item \emph{Spannung digital}:\\
Spannungsbereiche an Transistoren für Definition von 1/0-Signalen (TTL, LVTTL, CMOS)
\end{itemize}
Unterscheidung dead/live zero um einen Kabelbruch von einem Null-Signal unterscheiden zu können. Bei normaler Temperatur Unterscheidung bspw. $0\;\mathrm{mA}$ dead zero und $4 \;\mathrm{mA}$ live zero.

\section{Messeinrichtung}
Gesamtheit aller zusammenhängender Funktionseinheiten (Glieder), die zum Zweck der Messung, Messdatenverarbeitung, Anzeige von einer oder mehreren Messgrößen benutzt werden [In Anlehnung an die Norm DIN 1319].

besteht aus:
\begin{itemize}
\item Messgeräte
\item Hilfsgeräte
\item Maßverkörperungen
\item chemische Reagenzien
\item …
\end{itemize}
Messanlage: dauerhaft installierte Messeinrichtung

\paragraph{Beispiel: Fieberthermometer}
\begin{itemize}
\item Eingangsgröße: Temperatur $\vartheta$
\item Eingang: Sensor Thermoelement PT100. Temperatur erzeugt Änderung des darin liegenden Widerstands 
\item Anpassung: Element erzeugt durch Spanngs-/Stromquelle vom Widerstand abhängige Signale
\item Verstärkung
\item AD-Wandler
\item $\mu$Controller (verarbeitet Signale; piepst, wenn Messung fertig)
\item Anzeige der Temperatur
\end{itemize}
\subsection{Messkette}
$\underbrace{\overset{\substack{\text{Mess-}\\\text{größe}}}{\longrightarrow} \text{ Aufnehmer (Sensor)}}_{\text{Prozess}}$ 
$\overset{\substack{\text{Mess-}\\\text{signal}}}{\longrightarrow}$ Messumformer 
$\overset{\substack{\text{Norm-}\\\text{signal}}}{\longrightarrow}$ Messgerät / Anzeige 
$\overset{\text{Ausgabe}}{\longrightarrow}$\\
offene Wirkungskette: Messkette

\section{Steuern}
Ein Vorgang in einem System, bei dem eine oder mehrere Größen als Eingangsgrößen (Führungsgrößen) andere Größen als Ausgangsgrößen aufgrund der dem System eigentümlichen Gesetzmäßigkeiten beeinflussen [DIN 19226].\\
Steuerung ist also ein Plan, wie eine veränderliche Größe beeinflusst werden soll.\\
Beispiel Steuerung: Wecker klingelt jeden Tag um 7 Uhr. In dem Fall steuert der Wecker die Person.\medskip

Stellglied (beeinflusst unmittelbar die Steuergröße) kann verstellt werden (Steuerglied+Steller erzeugen aus der Führungsgröße die \emph{Stellgröße}), um \emph{Steuergröße} (output) zu ändern. Der Sollwert (\emph{Führungsgröße}) für die Steuergröße wird vom \emph{Führungsglied} (als starrer Befehlsgröße) diktiert. Es gibt allerdings keine Rückmeldung der Steuergröße, \emph{Störgrößen} könnten also nicht bedacht werden.\\
$\underbrace{\text{Stellglied (Aktor)}}_{\substack{\text{Prozess}\\\text{(Steuergröße)}\\\text{über Steuerstrecke}}}$
$\overset{\substack{\text{Stell-}\\\text{größe}}}{\longleftarrow}$ Steller
$\underset{\substack{\text{(Norm-}\\\text{signal)}}}{\overset{\substack{\text{Stell-}\\\text{signal}}}{\longleftarrow}}$ Steuerglied 
$\overset{\substack{\text{Führungs-}\\\text{größe}}}{\longleftarrow}$ Führungsglied\\
offene Steuerkette\\
Nachteile:
\begin{itemize}
\item keine Rückmeldung
\item dynamische Eigenschaften des Prozesses müssen genau bekannt sein.
\end{itemize}

\section{Regeln}
Ein Vorgang, bei dem fortlaufend eine Größe (die zu regelnde Größe, die Regelgröße) erfasst, mit einer anderen Größe (der Führungsgröße) verglichen, und im Sinne einer Angleichung an die Führungsgröße beeinflusst wird [DIN 19226].
\begin{enumerate}
\item Messen: die Regelgröße wird direkt gemessen oder aus anderen Messgrößen berechnet.
\item Vergleichen: die Regelgröße wird mit der Führungsgröße verglichen und die Regelabweichung berechnet.
\item Regelglied + Steller erzeugt die Stellgröße aus der Regelabweichung
\item[\textbullet] mit der Stellgröße wird somit die Regelgröße wieder beeinflusst.
\item[\textbullet] eine Störgröße kann durch das Messen der Regelgröße wieder an den Steller zurück geführt werden.
\end{enumerate}
Aus dem Prozess (Regelgröße) führt somit eine offene Messkette als Ausgabe heraus und führt eine offene Steuerkette als Regelabweichung herein. So wird zwischen dem Ein- und Ausgang ein Vergleichsglied gesteckt, die aus der Rückführgröße (Ist-Wert) und der Führungsgröße (Soll-Wert, von Außen bestimmt) diese Regelabweichung berechnet. Im Prozess führt die Regelstrecke vom Eingang zum Ausgang.

So entsteht ein geschlossener \emph{Regelkreis}.

\section{Steuern oder Regeln?}
\paragraph{Steuerung}
\begin{itemize}
\item offene Wirkungskette
\item Prozesseigenschaften (statisch + dynamisch) müssen genau bekannt sein)
\item kann auf Störungen nicht reagieren
\item kein Soll-Ist-Vergleich
\item keine Messung notwending
\item Prozessstabilität wird nicht beeinflusst
\end{itemize}
ist sinnvoll, wenn:
\begin{itemize}
\item die Auswirkungen von Störgrößen vernachlässigbar klein sind
\item nur eine Störgröße auftritt, die nach Art und Verlauf bekannt ist
\item Störgrößenänderungen selten sind
\end{itemize}
\paragraph{Regelung}
\begin{itemize}
\item geschlossener Regelkreis
\item Prozesseigenschaften müssen nicht genau bekannt sein (Robustheit gegenüber Parameteränderungen)
\item kann Störungen ausregeln (Störkompensation)
\item Soll-Ist-Vergleich
\item Messung ist notwending
\item Der geschlossene Regelkreis kann instabil werden (bsp.: Mikrofon neben Lautsprecher $\to$ Pfeifen)
\end{itemize}
ist sinnvoll, wenn:
\begin{itemize}
\item veränderliche, unbekannte, nach Art und Größe verschiedene, nicht messbare Störgrößen auftreten können
\end{itemize}

\chapter{Sensorik}
\section{Sensor}
Der Sensor ist das primäre Element einer Messkette\\
Sensor = Aufnehmer, Messwertaufnehmer, Messwertgeber, Signalgeber, Fühler, Geber, Initiator, Detektor, Zelle, Wandler, Transducer\\
$\overset{\text{Messgröße}}{\longrightarrow} \underbrace{\underbrace{\underbrace{
\text{Aufnehmer}}_{\text{Sensorelement}}\overset{\text{Messsignal}}{\longrightarrow}
\text{Signalaufbereitung} \overset{\text{Normsignal}}{\longrightarrow}}_{\text{integrierter Sensor}}
\substack{\text{Signalverarbeitung}\\\text{und -auswertung}}\underset{\text{Steuersignal}}{\overset{\text{Ausgabe}}{\rightrightarrows}}}_{\text{intelligenter Sensor}}$
\section{Einteilung von Sensoren}
\begin{anumerate}
\item Funktionsweise
\begin{itemize}
\item selbstgenerierende Sensoren (keine externe Energiequelle)
\begin{itemize}
\item Spannungsausgang
\item Stromausgang
\item Ladungsausgang
\end{itemize}
\item modulierende Sensoren (externe Energiequelle nötig)
\begin{itemize}
\item Widerstand
\item Induktivität
\item Kapazität
\end{itemize}
\end{itemize}
\item Aktive und passive Sensoren
\begin{itemize}
\item Selbstgenerierende Sensoren\\
= aktive Sensoren (in dt. Literatur, in engl.: passive)
\item Modulierende Sensoren\\
= passive Sensoren (in dt. Literatur, in engl.: aktive)
\end{itemize}
Alternativ:
\begin{itemize}
\item passive: Aufnehmen vorhandener Signale\\
bspw. Kamera, Mikrofon, IR-Sensor
\item aktive: Stimulieren der Umwelt und Aufnehmen der Antwort\\
bspw. Ultraschallsensor, Laserscanner, Radar
\end{itemize}
\item Messprinzipien / Energieformen
\begin{itemize}
\item elektrisch
\item magnetisch
\item mechanisch
\item thermisch
\item Strahlung
\item chemisch/biologisch
\end{itemize}
\item Einzelne Messgrößen
\item Grad der Erfassung
\begin{itemize}
\item erfassende Sensoren (Binärsensoren)
\item messende Sensoren (Analog- und Digitalsensoren)
\end{itemize}
\item Kontaktierende und kontaktlose Sensoren
\item Absolute und relative Sensoren
\item Interne, externe Sensoren\\
typisch interner Sensor: Schmerz\\
typisch externer Sensor: Augen
\item Einsatzgebiet
\item Sensormaterialien
\item Betriebseigenschaften
\item Sondermerkmale
\item[] …
\end{anumerate}

\section{Statische Eigenschaften von Sensoren}
(keine Funktion der Zeit)\\
$\overset{x}{\longrightarrow}$Sensor$\overset{y}{\longrightarrow}$
\paragraph{(statische) Transferfunktion} (Empfindlichkeitskennlinie, Kennlinie):\\
Beziehung zwischen Eingangssignal $x$ und Ausgangssignal $y$.
$$\boxed{y=f(x)}$$
\subparagraph{Ideale Kennlinie} soll möglichst
\begin{itemize}
\item monoton
\item eindeutig
\item linear
\end{itemize}
sein (bspw. $f(x)=x$).\\
Reale Transferfunktionen weisen in der Regel nicht alle dieser Eigenschaften auf (ggf. quantisierung nötig).\\
Weiterhin ist der Messbereich (FSI: full scale input) und der Ausgabebereich (FSO) in der Regel begrenzt.
\subparagraph{Dezibel} $G[dB]=20 \log \frac{s}{s_0}$ (Einheit von FSI und FSO)\\
für $s_0$ wird dann die Normgröße für Kraft, Strom oder Spannung eingesetzt (Spannung: $1 \mu V$, Leistung: $1mV$, Schallpegel: $1\cdot 10^{-12}W/m^2$)

\subsection{Auflösung}
$\Delta x_{\text{min}} \Rightarrow$ fassbares $\Delta y$\\
Sprich: „der kleinste Schritt der Eingangsgröße liefert auch noch eine fassbare Ausgangsgröße“

\subsection{Sensitivität} (Empfindlichkeit)
$$S(x)=\frac{\diffd{y}(x)}{\diffd{x}}$$
Sensitivität abhängig von Steigung der Funktion:
\begin{itemize}
\item hohe Steigung: hohe Sensitivität
\item mittlere Steigung: mittlere Sensitivität
\item niedrige Steigung: niedrige Sensitivität
\end{itemize}
\subsection{Linearität} (Nichtlinearität)\\
Betrachtet wird die maximale Abweichung zwischen einer linearen Näherung einer nicht-linearen Funktion ($\Delta y_{\text{max}}$).\\
$\epsilon_{\text{max}}=\frac{\Delta y_{\text{max}}}{FSO}\cdot 100 \%$\\
ABB AS9

\subsection{Ansprechschwelle}
$x_{\text{min}}$\\
ABB AS6
\subsection{Totband}
ABB AS7
\subsection{Sättigung}
ABB AS8

\subsection{Wiederholgenauigkeit}
\subsubsection{Reproduzierbarkeit}
Maximales Intervall des Stimulus, bei welchem sich unter gleichen Laborbedingungen das Ausgangssignal nicht gleich ist.\\
ABB AS10\\
$\delta_R = \frac{\Delta x_{\text{max}}}{FSI}$

\subsubsection{Hysterese}
(Wenn man die Messung zwei mal von unterschiedlichen Richtungen ausführt $\to$ Abweichung in $y$-Wert abhängig von der Richtung)\\
ABB AS11

\subsection{Selektivität}
(Querempfindlichkeit)\\
Empfindlichkeit des Messwertes auf andere Größen
\begin{itemize}
\item Empfindlichkeitsänderung\\
ABB AS12
\item Nullpunktdrift\\
ABB AS13
\end{itemize}
Behebung: Kompensationssensor einsetzen.

\subsection{Rauschen}
ABB AS14

\subsection{Weitere}
\begin{itemize}
\item Langzeitstabilität
\item Lebensdauer
\item Exemplarstreuung
\item …
\end{itemize}

\section{Dynamische Eigenschaften von Sensoren}
\begin{itemize}
\item zeitlich veränderliche Messgrößen
\item diese seien adäquat (entsprechend der statischen Transferfunktion) und verzögerungsfrei weiterzugeben
\end{itemize}
Beispiel:\\
Energiespeicher
\begin{itemize}
\item Massen
\item elektrische Kapazitäten
\item Induktivitäten
\item Wärmekapazitäten
\item dissipative Energieverluste
\end{itemize}

\paragraph{Ansprechverhalten} Antwort eines Sensors auf ein zeitlich veränderliches Eingangssignal

\paragraph{Ordnung des Sensors} Anzahl vorhandener Energiespeicher (Ordnung der Differentialgleichung)
\begin{itemize}
\item 0. Ordnung: keine Zeitabhängigkeit $I(t) = U_0 \cdot \sigma (t)$
\item 1., 2.,… Ordnung: zeit- und frequenzabhängiges Ansprechverhalten
\end{itemize}

\subsection{Sprungantwort und Zeitkenngrößen}
Sprunganregung: $x(t) =\begin{cases}
0 &(t>0)\\
x_0 & (t \geq 0)
\end{cases}$\\
ABB AS15
\begin{itemize}
\item 0. Ordnung: $y(t\geq 0) = S(x_0)$
\item 1. Ordnung: (bspw. Kapazität, die auf-/entladen wird) $U_C(t) = U_0 \cdot \left\lbrace 1- \exp \left(-\frac{t}{\tau}\right)\right\rbrace$\\
$\tau = R \cdot C$ (Zeitkonstante)\\
ABB AS16\\
Halbwertszeit: $t_{0,5}$\\
Anstiegszeit (rise time): $t_r = t_{0,9}$ (also $t_{90\%}$, wann $y_{st}$ zu $90\%$ erreicht ist)\\
Einstellzeit: $t_E = 3 \tau$
\end{itemize}
\subsubsection{Dämpfung}
\begin{itemize}
\item $\delta$ Dämpfungskonstante
\item $\omega_0$ Eigenkreisfrequenz
\end{itemize}
Dämpfungsgrade:
\begin{itemize}
\item $\delta = 0$: ungedämpfte Schwingung%
\item $\frac{\delta}{\omega_0}<1$: Schwingfall
\item $\frac{\delta}{\omega_0}\geq 1$: Kriechfall (stationärer Wert wird ohne Überschwingung erreicht)
\end{itemize}
ABB AS18
\begin{itemize}
\item Totzeit $t_T$
\item Verzögerungszeit $t_V$
\item Ausgleichszeit $t_A = \tau$
\item Einstellzeit $t_E$ 
\end{itemize}
Das Toleranzband ist vorher zu bestimmen!
\begin{itemize}
\item $\delta_{opt}=0,7 \dots 0,8 \cdot \omega_0$
\item $\frac{\Delta y}{y_{st}}=\pm 4,5 \%$
\end{itemize}
\subsection{Frequenzkenngrößen}
Antwort auf ein sinusförmiges Eingangssignal: $x(t) = x_0 \cos (\omega t)$ \quad ($\omega$: Erregerfrequenz)
\begin{itemize}
\item stationärer (eingeschwungener) Zustand: $\lim_{t\to \infty} y(t) = \hat{y}_{st}(\omega) \cdot \cos (\omega t + \varphi)$\\
Nicht alle spektralen Komponenten (Frequenzen) können amplituden- und phasentreu übertragen werden.\\
\emph{stationärer Amplitudengang} (magnitude ratio): $M=\frac{\hat{y}_{st}(\omega)}{\hat{y}_{st} (\omega \to 0)}$\\
\emph{stationärer Phasengang}: $\varphi$\\
Sensoren 0. Ordnung: amplitudentreu bei $M=1$, phasentreu bei $\varphi = 0$
\end{itemize}
\subsubsection{Frequenzkenngrößen von Sensoren 1. Ordnung}
ABB AS19\\
Grenzfrequenz: $\omega_G$\\
$M=0,707$ \quad $\varphi=45^\circ$\\
Grenzfrequenz wird so angelegt, dass $\omega/\omega_G=1$
\begin{itemize}
\item $\omega/\omega_G \ll 1$:\\
$M \approx 1$\\
$\varphi \approx 0$
\item $\omega / \omega_G \gg 1$:\\
$M \to 0$\\
$\varphi \to 90^\circ$
\end{itemize}
Bandbreite: Frequenzbereich mit $M$ zwischen $0,707\dots 1,414 (\pm 3dB)$

\subsubsection{Frequenzkenngrößen von Sensoren 2. Ordnung}
ABB 20\\
$\zeta = \frac{\delta}{\omega_0}$
\begin{itemize}
\item quasistatische Anregung:\\
$\omega \ll \omega_0$: $M\approx 1$, $\varphi \approx 0$
\item Resonanz (bis $\delta = 0,71 \omega_0$):\\
\emph{Resonanzfrequenz}: $\omega = \sqrt{\omega_0^2-2\delta^2}$\\
$\varphi = -90^\circ$, $M\to M_{max}$\\
Ab Dämpfung $\zeta = 0,7$ verschwindet die Resonanz beinahe vollständig.
\item hohe Anregerfrequenz $\omega \gg \omega_0$\\
$\varphi \to -180^\circ$, $M\to 0$
\item Bandbreite: Frequenzbereich mit $M$ zwischen $0,707\dots 1,414 (\pm 3dB)$
\end{itemize}

\chapter{Temperatursensoren}
Drucksensoren
Durchflusssensoren
Binäre Positionssensoren
Flüssigkeits-Chemosensoren
…
\section{Temperaturmessverfahren}
Alle physikalischen Eigenschaften sind mehr oder weniger stark temperaturabhängig:
\begin{itemize}
\item Gasthermometer (73-543 K)
\item Flüssigkeitstherm. (203-360 K)
\item Farbstofftherm. (303-1873 K)
\item Pt100 (53-1123 K)
\item Keramik NTC, PTC (193-523 K)
\item Thermoelemente (53-2773 K)
\item Si-Halbleiter (208-573 K)
\item IR-Sensoren (223-3473 K)
\end{itemize}
Hauptsächlich 2 Methodentypen:
\begin{itemize}
\item Kontaktthermometrie (misst im Prinzip allerdings nicht die Temperatur des Mediums, sondern seine eigene $\Rightarrow$ braucht Zeit um die Temperatur bei Kontakt anzugleichen $\to$ Zeitaufwändig)
\item Pyrometrie (Analysiert Wärmestrahlung)
\end{itemize}
Die für uns wichtigsten Kontaktthermometer sind:
\begin{itemize}
\item Thermoresistive Sensoren
\item Thermoelemente
\item Halbleiterfühler verschiedener Art (Sperrstrom oder Schwellspannung)
\end{itemize}
\section{Zeitverhalten eines Widerstandsensors}
ABB AS21\\
(Verhalten wie Sensor 1. Ordnung)
\begin{itemize}
\item Verzögerungszeit $t_V$
\item Ausgleichszeit $t_A = \tau$
\item Halbwertszeit $t_{50}$
\item Anstiegszeit $t_{90}$
\item Einstellzeit $t_E=3 \tau$
\end{itemize}
Luft: sehr langsame Anpassung der Temperatur am Sensor\\
Flüssigkeit: schnellere Anpassung (ca. 10 mal schneller als Luft)\\
Festkörper: etwas langsamere Anpassung

\section{Thermoresistive Sensoren}
ABB AS 22\\
Temperaturkoeffizient des Widerstands:
$$TK_R=\frac{1}{R}\cdot \frac{\diffd{R}}{\diffd{T}}$$
Unterscheidung der Sensoren:
\begin{itemize}
\item Kaltleiter $TK>0$ (PTC: positive temperature coefficient)\\
bspw. Metallwiderstände ($Ni$ und $Pt$)\\
Vorteil: in kleinen Temperaturbereichen ist Anstieg linear
\item Heißleiter $TK<0$ (NTC: negative temperature coefficient)
\end{itemize}

\subsection{Metallwiderstände}
Je nach Metall (Platin, Kupfer, Nickel) ist die Auslenkung unterschiedlich. Platin hat eine optimale Auslenkung, ist aber teuer. Etwas schlechter, aber billiger ist Kupfer. Nickel als alternative hat ab der Curie-Temperatur ($352 ^\circ C$) eine sehr schlechte Auslenkung.\\
Die Metalle sollten folgende Eigenschaften haben:
\begin{itemize}
\item korrosionsfest
\item alterungsfest
\item hohe $T_S$ (Schmelztemperatur)
\item hoher $\varrho$ (elektrischer Widerstand)\\
(da kleine Änderungen einer kleinen Größe gemessen werden soll)
\item duktil
\end{itemize}
\begin{tabular}{l l l}
Metall & $\varrho $ & TK\\
&$ 10^{-8}\Omega \mathrm{m}$&$10^{-3}K^{-1}$\\
\hline
Platin (Pt) & $10,6$ & $3,92$\\
Kupfer (Cu) & $1,67$ & $4,33$\\
Nickel (Ni) & $6,84$ & $6,75$
\end{tabular}\\
Man sieht: bei Metallen ist der Temperaturkoeffizient sehr klein.\\
Sensoren:
\begin{itemize}
\item Keramikkörper, in den Platinwendel (Drähte) lose in Löcher eingelegt werden
\item Keramikdorn, um den Platinwendel (Drähte) umgewickelt werden (abgedeckt mit Glasrohr)
\item alternative (zu den Draht-Lösungen): im Sensor wird eine Platindünnschicht auf Keramikträger aufgetragen (sehr wenig Platin $\to$ deutlich günstiger; Nachteil: dünne Platinschicht liegt dicht auf Keramik. Bei Temperaturänderungen dehnt sich Keramik und Platin unterschiedlich schnell, das führt zu mechanischen Spannung)
\end{itemize}

\subsubsection{Pt100}
gebräuchlichster Temperatursensor
\begin{itemize}
\item Spezifikation: DIN EN 60571
\item Temperaturbereich (Dauerbetrieb): $-7^\circ C$ bis $+550^\circ C$
\item Temperaturkoeffizient: $Tk=3850 \;\mathrm{ppm/K}$
\item Langzeitstabilität: Max. $R_0$-Drift: $0,04\%$ nach $1000h$ bei $500^\circ C$
\item Erschütterungsfestigkeit: Min. $40g$ Beschleunigung bei $10$ bis $2000\;\mathrm{Hz}$
\item Umgebungsbedingungen: Ungeschützt nur im trockener Umgebung einsetzbar
\item Isolationswiderstand: $>10 M\Omega$ bei $20^\circ C$, $>1 M\Omega$ bei $500^\circ C$
\item Selbsterwärumng: $0,2K/mW$ bei $0^\circ C$
\item Ansprechzeit: \\
Bewegtes Wasser ($v=0,4m/s$): $t_{0,5}=0,3s; t_{0,9}=0,8s$\\
Lufstrom ($v=1m/s$): $t_{0,5}=3,0s;t_{0,9}=9,0s$
\item Messstrom: $100 \Omega$: $0,1$ bis $0,4 mA$
\end{itemize}
Vorteile:
\begin{itemize}[label=$+$]
\item für Präzisionsmessungen geeignet ($\pm 0,3\%$)
\item sehr gute Langzeitstabilität
\item gute Linearität (bis $150^\circ C$)
\item PT-Sensoren: $-270^\circ C$ bis $+850^\circ C$\\
mech. sehr stabil, Dünnfilmsensoren
\item Ni-Sensoren: kostengünstige Dickschichtsensoren
\end{itemize}
Nachteile:
\begin{itemize}[label=$-$]
\item kostenaufwendig
\item wegen Gehäuse meist langsam
\item Spannungsversorgung erforderlich
\item Ni-Sensoren: $-60^\circ C$ bis $+180^\circ C$
\item keine punktförmige Messung
\item Selbsterwärmung
\end{itemize}
\subsubsection*{Schalttechniken}
\begin{itemize}
\item Zweileitertechnik:
Auslesen des Spannungsabfalls über Pt100-Sensor direkt\\
Problem: Widerstände der Drähte von/zum Pt100-Sensor können Signal verfälschen.
\item Vierleitertechnik:\\
Auslesen des Spannungsabfalls mit einer parallelen Leitung zusätzlich zur Versorgungsleitung des Pt100. Damit ist die Messung Stromfrei. \\
Nachteil: 4 Leiter mit Widerständen, die verfälchen können.\\
Lösung: Dreileitertechnik: nicht komplett parallele Spannungsmessung, sodass ein Widerstand weg fällt.
\end{itemize}
\subsection{Kalteiter (PTC)}
\subsubsection{Thermositoren} (temperaturabhängige Widerstände)
\begin{itemize}[label=$+$]
\item preiswert
\item großer Messwiderstand
\item $-30^\circ C$ bis $+350^\circ C$ ($\pm 5^\circ C$)
\item große Empfindlichkeit
\item ohne Gehäuse realisierbar $\Rightarrow$ kurze Ansprechzeit
\end{itemize}
\begin{itemize}[label=$-$]
\item stark nichtlineare Kennlinie
\item ohne Gehäuse mechanisch anfällig
\item Spannungsversorgung erforderlich
\item keine punktförmige Messung
\end{itemize}
\subsubsection{schaltende Sensoren}
\begin{itemize}
\item Überstrom-Überwachung
\item Übertemperatur-Überwachung
\item Niveaufühler für Flüssigkeiten
\end{itemize}
\subsubsection{Si-Temperatursensoren}
Widerstand steigt mit Temperatur $\to$ Kaltleiter.
\begin{itemize}[label=$+$]
\item preiswert
\item großer Messwiderstand
\item große Empfindlichkeit
\end{itemize}
\begin{itemize}[label=$-$]
\item keine lineare Kennlinie
\item DMS-Effekt (Dehnungs-Mess-Streifen)
\item Spannung erforderlich
\item Gehäuse $\Rightarrow$ langsam
\item keine punktförmige Messung
\end{itemize}

\subsubsection{Spreading-Resistance-Sensor}
$R=\frac{\varrho(T)}{\pi \cdot d}$\\
ABB AS23 (Ersatzschaldbild)

\subsubsection{Temperatur IC}
Diodengleichung: $I=I_0\cdot \left[ \exp\left(\frac{e \cdot U}{k \cdot T}\right) -1 \right]$ mit $I_0$: Sperrstrom

\subsection{Heißleiter (NTC)}
\begin{itemize}[label=$+$]
\item breites Spektrum an $R$
\item großer TK ($3 \dots 6 \% /K$)
\item[$-$] $-50^\circ C \dots +100^\circ C (\pm 5^\circ C)$
\end{itemize}
\begin{itemize}
\item Thermistoren: Bestehen aus Keramiken (oxyde).
\item Temperaturfühler
\item Zeitverzögerungsglied
\item Spannungsstabilisierung
\end{itemize}

\subsubsection{Thermoelemente}
ABB AS24\\
Beispiel: Draht führt durch zu Eis/Kerze (Kupfer), dazwischen ist dann Konstantan. Der Spannungsunterschied wird gemessen.\\
Seebeck-Effekt: $U_{th}=\alpha_{th} \cdot \Delta T$ mit $U_{th}$: Thermospannung und $\alpha_{th}$: Thermokraft

\subsubsection{Genormte Thermoelemente}
\begin{tabular}{L{0.2} | l L{0.1} | l | L{0.3}}
Thermopaar & Kurzzeichen & lin. Thermosp., $\mu V/K$ & T-Bereich ($^\circ C$) & Eigenschaften\\ \hline 
Eisen-Konstantan (Fe-CuNi) & Typ J (Typ L) & 53 & -250 bis 700 (900) & billig, rostet, empfindlich gegen schwefel- und kohlenwasserstoffhaltige Luft\\\hline
Kupfer-Konstantan (Cu-CuNi) & Typ T (Typ U) & 443 & -250 bis 300 & rostet nicht, Cu-CuNi verzundert\\\hline
Nichkelchrom-Nickel (NiCr-Ni) & Typ K (Typ N) & 41 & -200 bis 1000 (1300) & hohe Thermokraft, recht linear, empfindlich gegen S-haltiges Gas\\\hline
Platinrhodium-Platinum (PtRh-Pt; Pt/Rh=87/13) & Typ R & 6,5 & -50 bis 1500 & bei Rotglut einsetzbar, teuer, oxidationsbeständig\\\hline
Platinrhodium-Platinum (PtRh-Pt; Pt/Rh=90/10) & Typ S & 6,5 & -50 bis 1500 & wie Typ R (angesächsisch), aber mehr im dt. Raum\\\hline
PtRh-Schenkel mit verschiedenem Rh-Gehalt (70/30 - 94/6) & Typ B & 6,3 & 0 bis 1600 (1800) & unempfindlich gegen Verunreinigungen, T der Vergleichsstelle muss nicht konstant sein\\\hline
WRh-Schenkel mit verschiedenem Rh-Gehalt (75/25 - 97/3) & Typ G & 15 & 0 bis 2200 & T der Vergleichsstelle muss nicht konstant sein, nur in neutraler und reduzierender Atmosphäre
\end{tabular}

\subsubsection{Aufbau von Thermoelementen}
\begin{itemize}
\item Kabelthermoelemente
\begin{itemize}[label=$+$]
\item sehr schnell
\item[$-$] ungeschützt
\end{itemize}
\item Mantelthermoelemente (im Prinzip Kabel inkl. Mantel) $\to$ etwas langsamer, aber geschützter
\item Isolierte Messstelle (TI)
\begin{itemize}
\item[$+$] elektrische und galvanische Trennung
\item[$-$] lange Zeitkonstante
\end{itemize}
\item mit Mantel verbundene Messstelle (TM) [an Messstelle liegt Thermoelement „blank“]
\begin{itemize}
\item[$+$] kurze Zeitkonstante
\item[$-$] galvanische Verbindung
\end{itemize}
\end{itemize}
\subsubsection{Ansprechzeiten, Vorteile, Nachteile der Thermoelemente}
\begin{itemize}[label=$+$]
\item sehr großer T-Bereich
\item kleine Gesamtabmessungen
\item große Flexibilität
\item einfacher, sehr robuster Aufbau
\item unempfindlicher gegen Hitze, Druck und Vibration
\item kleinste Ansprechzeiten
\item praktisch punktförmige Messstellen
\end{itemize}
\begin{itemize}[label=$-$]
\item reduzierte Genauigkeit
\item sehr kleine Thermospannung
\item Temperatur der Anschlussstelle muss bekannt sein
\end{itemize}

\subsection{Bimetallthermometer}
Zwei unterschiedliche Metalle sind in einem Streifen miteinander verbunden. Dadurch, dass sie sich bei Temperatur unterschiedlich verformen, kann durch die Auslenkung die Temperatur berechnet werden.\\
$f$: Auslenkung der Spitze des Streifens,\\
$s$: Dicke des Streifens (beider Metalle),\\
$L$: Länge des Streifens\\
$\delta$: spezifische Ausbiegung
$$f=\frac{L^2}{s}\cdot \delta \cdot \Delta T$$
Einsatzbeispiel: Schalter wird durch den Bimetallstreifen betätigt, wenn die Temperatur bewirkt, dass er sich verformt: schaltende Sensoren (Überlastungsschutz, Temperaturregelung).

\section{Leitlinien für die Temperaturmessung}
\begin{itemize}
\item Messort
\item Gehäuseeinfluss
\item Ansprechzeit
\item Eigenerwärmung
\item Absolute Genauigkeit
\end{itemize}
Für hohe Ansprüche an die Auflösung und Genauigkeit: \fbox{Pt-Fühler}\\
Für kleinste Messstellen und schnelles Ansprechen: \fbox{Thermoelemente}
\paragraph{Messprobleme bei:}
\begin{itemize}
\item Dickem Gehäuse
\item Große Entfernung zwischen Objekt und Messstelle
\item Objekte geringer Masse (mit massiven Fühlern $\to$ Verfälschung der Messung durch Angleichung der zu messenden Temperatur an die Fühlertemperatur)
\item T-Messung bei Luftströmungen
\end{itemize}

\chapter{Druckmessung}
\section{Druck als Messgröße}
Normaldruck: $1013\;\mathrm{hPa}$\\
$\text{Druck}=\frac{\text{Kraft}}{\text{Fläche}} \quad [p]=1\;\mathrm{Pa}=\frac{1 \;\mathrm{N}}{1 \;\mathrm{m^2}} \quad 1 \;\mathrm{bar} = 10^5 \;\mathrm{Pa}$\\
\begin{tabular}{l | l}
Druckbereich & Anwendung\\\hline
$<40 \unit{mbar}$ & Füllstand in Waschmaschine, Geschirrspüler\\
$100\unit{mbar}$ & Staubsauger, Filterüberwachung, Durchflussmessung\\
$200\unit{mbar}$ & Blutdruckmessung\\
$1\unit{bar}$ & Barometer, Korrektur für Zündung und Einspritzung\\
$0\dots 1 \unit{bar}$ & Ansaugunterdruckmessung\\
$10\unit{bar}$ & Öldruck und Luftdruck für Bremsen, Kühlmaschinen\\
$50 \unit{bar}$ & Pneumatik, Industrieroboter\\
$500 \unit{bar}$ & Hydraulik, Baumaschinen\\
$3000 \unit{bar}$ & Hochdruckschneiden mit Wasser
\end{tabular}
\section{Arten der Druckmessung}
Absolutdruck: Ausgehend von keinem Druck ($0$)\\
Unterdruck/Überdruck: Ausgehend vom Relativdruck (festgelegt: $p_{atm}$, bspw. Atmosphärendruck)\\
Differenzdruck: Druck zwischen zwei Messungen $p_{diff}=p_1-p_2$
\begin{itemize}
\item Absolut-Drucksensor:\\
zu messender Druck auf einer Membran gegenüber dem Vakuum
\item Relativ-Drucksensor:\\
zu messender Druck auf eine Membran gegenüber dem Umgebungsdruck
\item Differenz-Drucksensor:\\
die beiden Drücke auf beiden Seiten der Membran
\end{itemize}
\section{Prinzipien der Druckmessung}
\begin{itemize}
\item Verformung unter Druck: magnetoelastische und Membransensoren
\item Druckabhängigkeit physikalischer Eigenschaften: piezoelektrische Kristallsensoren
\end{itemize}
Störeinflüsse:
\begin{itemize}
\item mechanische Spannungen
\item Temperatureffekte
\end{itemize}
\subsection{Membransensoren}
Membran biegt sich unter Druck $p$ (mit einer Durchbiegung $\Delta x$)

\subsubsection{Druckschalter}
unter Druck wird Schalter betätigt. Abhängig von den eingebauten Membranen und Druckfedern.
\begin{itemize}
\item[$+$] Messbereich: $\unit{mbar}\dots\unit{kbar}$
\item[$-$] relativ große Hysterese (Einschalt- und Ausschaltdruck unterscheiden sich)
\end{itemize}
\subsubsection{Kapazitiver Drucksensor}
Auf Membran ist Messelektrode angebracht. Durch die Verformung der Membran variiert der Abstand zwischen der Messelektrode und der Referenzelektrode. Dadurch verändert sich die Spannung.\\
$C=\varepsilon\varepsilon_0\frac{A}{d}$\\
Typische Abmessungen:
\begin{itemize}
\item Äußerer Druckmesser: $32,4\unit{mm}$
\item Dicke des Grundkörpers: $5\unit{mm}$
\item Dicke der Membran (je nach Messbereich): $0,2\dots 2,8 \unit{mm}$
\item Abstand zwischen Membran und Grundkörper: $40\unit{\mu m}$
\item Durchbiegung bei Nenndruck: max. $8 \unit{\mu m}$
\end{itemize}
Eigenschaften:
\begin{itemize}[label=$+$]
\item sehr empfindlich
\item geringer Energieverbrauch (da nicht generierender, sondern modulierender Sensor)
\item geringe T-Abhängigkeit
\item sehr einfach, robust
\item nicht empfindlich für Überdruck
\item besonders für niedrige $\unit{P}$: $10^{-7}\dots 1 \unit{bar}$
\item gasartunabhängig
\item Genauigkeit: $0,2\%\text{ (Ker)}\dots2\%\text{ (Me)}$ vom Messwert
\item lineare Kennlinie
\end{itemize}
\begin{itemize}[label=$-$]
\item Kapazitätsänderung: Bruchteile von $\unit{pF}$
\item Streukapazitäten-, staub- und feuchteempfindlich
\item relativ komplexe Elektronik
\end{itemize}

\subsubsection{Kapazitive MEMS-Sensoren}
Mikro-Elektro-Mechanische Systeme: Ähnlich wie kapazitive Drucksensoren, bloß deutlich kleiner.
\begin{itemize}
\item Drucksensor
\item Beschleunigungssensor
\end{itemize}

\subsubsection{Dehnungsmessstreifen (DMS)}
Messstreifen mit einer Länge $l$ und Durchschnitt $A$:\\
$R=\rho \frac{l}{A}$\\
Da sich der Durchschnitt kaum ändert: $\frac{\Delta R}{R}=\tred{k}\cdot \frac{\Delta l}{l}$\\
Der \tred{k}-Faktor:\\
\begin{tabular}{l | c}
Werkstoff & K-Faktor\\\hline
Konstantan & 2,0\\
Nichrom & 2,2\\
FeNi & 2,5\\
„Iso-Elastic“ & 3,6\\
PtW & 4,0
\end{tabular}\\
Der k-Faktor ist sehr stark von der Temperatur abhängig!

\subsubsection*{Membran-DMS-Sensorelement}
Widerstand-Anordnung an der Membran ist zu vergleichen mit einer Wheatstone-Brücke (mit $R_1$ und $R_2$ parallel zu $R_3$ und $R_4$), wobei $R_1=R_2=R_3=R_4$.\\
$R$: $120\dots5000\unit{\Omega}$\\
$\Delta R \sim 0,5\dots 5\%$
\begin{itemize}
\item Metall-Dünnfilm-DMS
\item polySi-Membran-DMS
\end{itemize}


%\newpage
%\printbibliography

\end{document}