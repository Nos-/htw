\newcommand{\customDir}{../}
\RequirePackage{ifthen,xifthen}

% Input inkl. Umlaute, Silbentrennung
\RequirePackage[T1]{fontenc}
\RequirePackage[utf8]{inputenc}

% Arbeitsordner (in Abhängigkeit vom Master) Standard: .LateX_master Ordner liegt im Eltern-Ordner
\providecommand{\customDir}{../}
\newcommand{\setCustomDir}[1]{\renewcommand{\customDir}{#1}}
%%% alle Optionen:
% Doppelseitig (mit Rand an der Innenseite)
\newboolean{twosided}
\setboolean{twosided}{false}
% Eigene Dokument-Klasse (alle KOMA möglich; cheatsheet für Spicker [3 Spalten pro Seite, alles kleiner])
\newcommand{\customDocumentClass}{scrreprt}
\newcommand{\setCustomDocumentClass}[1]{\renewcommand{\customDocumentClass}{#1}}
% Unterscheidung verschiedener Designs: htw, fjs
\newcommand{\customDesign}{htw}
\newcommand{\setCustomDesign}[1]{\renewcommand{\customDesign}{#1}}
% Dokumenten Metadaten
\newcommand{\customTitle}{}
\newcommand{\setCustomTitle}[1]{\renewcommand{\customTitle}{#1}}
\newcommand{\customSubtitle}{}
\newcommand{\setCustomSubtitle}[1]{\renewcommand{\customSubtitle}{#1}}
\newcommand{\customAuthor}{}
\newcommand{\setCustomAuthor}[1]{\renewcommand{\customAuthor}{#1}}
%	Notiz auf der Titelseite (A: vor Autor, B: nach Autor)
\newcommand{\customNoteA}{}
\newcommand{\setCustomNoteA}[1]{\renewcommand{\customNoteA}{#1}}
\newcommand{\customNoteB}{}
\newcommand{\setCustomNoteB}[1]{\renewcommand{\customNoteB}{#1}}
% Format der Signatur in Fußzeile:
\newcommand{\customSignature}{\ifthenelse{\equal{\customAuthor}{}} {} {\footnotesize{\textcolor{darkgray}{Mitschrift von\\ \customAuthor}}}}
\newcommand{\setCustomSignature}[1]{\renewcommand{\customSignature}{#1}}
% Format des Autors auf dem Titelblatt:
\newcommand{\customTitleAuthor}{\textcolor{darkgray}{Mitschrift von \customAuthor}}
\newcommand{\setCustomTitleAuthor}[1]{\renewcommand{\customTitleAuthor}{#1}}
% Standard Sprache
\newcommand{\customDefaultLanguage}[1]{}
\newcommand{\setCustomDefaultLanguage}[1]{\renewcommand{\customDefaultLanguage}{#1}}
% Folien-Pfad (inkl. Dateiname ohne Endung und ggf. ohne Nummerierung)
\newcommand{\customSlidePath}{}
\newcommand{\setCustomSlidePath}[1]{\renewcommand{\customSlidePath}{#1}}
% Folien Eigenschaften
\newcommand{\customSlideScale}{0.5}
\newcommand{\setCustomSlideScale}[1]{\renewcommand{\customSlideScale}{#1}}

%\setboolean{twosided}{true}
%\setCustomDocumentClass{scrartcl}
%\setCustomDesign{htw}
\setCustomTitle{Grundlagen der Informatik}
\setCustomSubtitle{Vorlesungsskript}
\setCustomAuthor{Falk-Jonatan Strube}
%\setCustomNoteA{Titlepage Note before Author}
\setCustomNoteB{Vorlesung von Dr. Boris Hollas}

%\setcustomSignature{\footnotesize{\textcolor{darkgray}{Mitschrift von\\ \customAuthor}}	% Formatierung der Signatur in der Fußzeile
%\setcustomTitleAuthor{\textcolor{darkgray}{Mitschrift von #1}}	% Formatierung des Autors auf dem Titelblatt

%-- Prüfen, ob Beamer
\ifthenelse{\equal{\customDocumentClass}{beamer}}{
%%% TODO: andere Layouts für Beamer außer HTW
	\documentclass[ignorenonframetext, 11pt, table]{beamer}
	
	\usenavigationsymbolstemplate{}
	\setbeamercolor{author in head/foot}{fg=black}
	\setbeamercolor{title}{fg=black}
	\setbeamercolor{bibliography entry author}{fg=htworange!70}
	%\setbeamercolor{bibliography entry title}{fg=blue} 
	\setbeamercolor{bibliography entry location}{fg=htworange!60} 
	\setbeamercolor{bibliography entry note}{fg=htworange!60}  
	
	\setbeamertemplate{itemize item}{\color{black}$\bullet$}
	\setbeamertemplate{itemize subitem}{\color{black}--}
	\setbeamertemplate{itemize subsubitem}{\color{black}$\bullet$}
	\makeatother
	\setbeamertemplate{footline}
	{
	\leavevmode
	\def\arraystretch{1.2}
	\arrayrulecolor{gray}
	\begin{tabular}{ p{0.167\textwidth} | p{0.491\textwidth} | p{0.089\textwidth} | p{0.103\textwidth}}
	\hline
	\strut\insertshortauthor & \insertshorttitle & Slide \insertframenumber{}% / \inserttotalframenumber{}
	 & May 4, 2016\\
	\end{tabular}
	}
	\setbeamertemplate{headline}
	{
	\leavevmode
	\setlength{\arrayrulewidth}{1pt}
	\hspace*{2em}	
	\begin{tabular}{p{0.63\textwidth}}
	\rule{0pt}{3em}\normalsize{\textbf{\insertsection\strut}}\\
	\arrayrulecolor{htworange}
	\hline
	\end{tabular}
	\begin{tabular}{l}
	\rule{0pt}{4em}\includegraphics[width=3.25cm]{\customDir .LaTeX_master/HTW_GESAMTLOGO_CMYK.eps}\\
	\end{tabular}
	}
	\makeatletter	
}{	
	%-- Für Spicker einiges anders:
	\ifthenelse{\equal{\customDocumentClass}{cheatsheet}}{
		\documentclass[a4paper,10pt,landscape]{scrartcl}
		\usepackage{geometry}
		\geometry{top=2mm, bottom=2mm, headsep=0mm, footskip=0mm, left=2mm, right=2mm}
		
		% Für Spicker \spsection für Section, zur Strukturierung \HRule oder \HDRule Linie einsetzen
		\usepackage{multicol}
		\newcommand{\spsection}[1]{\textbf{#1}}	% Platzsparende "section" für Spicker
	}{	%-- Ende Spicker-Unterscheidung-if
		%-- Unterscheidung Doppelseitig
		\ifthenelse{\boolean{twosided}}{
			\documentclass[a4paper,11pt, footheight=26pt,twoside]{\customDocumentClass}
			\usepackage[head=23pt]{geometry}	% head=23pt umgeht Fehlerwarnung, dafür größeres "top" in geometry
			\geometry{top=30mm, bottom=22mm, headsep=10mm, footskip=12mm, inner=27mm, outer=13mm}
		}{
			\documentclass[a4paper,11pt, footheight=26pt]{\customDocumentClass}
			\usepackage[head=23pt]{geometry}	% head=23pt umgeht Fehlerwarnung, dafür größeres "top" in geometry
			\geometry{top=30mm, bottom=22mm, headsep=10mm, footskip=12mm, left=20mm, right=20mm}
		}
		%-- Nummerierung bis Subsubsection für Report
		\ifthenelse{\equal{\customDocumentClass}{report} \OR \equal{\customDocumentClass}{scrreprt}}{
			\setcounter{secnumdepth}{3}	% zählt auch subsubsection
			\setcounter{tocdepth}{3}	% Inhaltsverzeichnis bis in subsubsection
		}{}
	}%-- Ende Spicker-Unterscheidung-else
	
	\usepackage{scrlayer-scrpage}	% Kopf-/Fußzeile
	\renewcommand*{\thefootnote}{\fnsymbol{footnote}}	% Fußnoten-Symbole anstatt Zahlen
	\renewcommand*{\titlepagestyle}{empty} % Keine Seitennummer auf Titelseite
	\usepackage[perpage]{footmisc}	% Fußnotenzählung Seitenweit, nicht Dokumentenweit
}

% Input inkl. Umlaute, Silbentrennung
\RequirePackage[T1]{fontenc}
\RequirePackage[utf8]{inputenc}
\usepackage[english,ngerman]{babel}
\usepackage{csquotes}	% Anführungszeichen
\RequirePackage{marvosym}
\usepackage{eurosym}

% Style-Aufhübschung
\usepackage{soul, color}	% Kapitälchen, Unterstrichen, Durchgestrichen usw. im Text
%\usepackage{titleref}
\usepackage[breakwords, fit]{truncate}	% Abschneiden von Sätzen
\renewcommand{\TruncateMarker}{\,…}

% Mathe usw.
\usepackage{amssymb}
\usepackage{amsthm}
\ifthenelse{\equal{\customDocumentClass}{beamer}}{}{
\usepackage[fleqn,intlimits]{amsmath}	% fleqn: align-Umgebung rechtsbündig; intlimits: Integralgrenzen immer ober-/unterhalb
}
%\usepackage{mathtools} % u.a. schönere underbraces
\usepackage{xcolor}
\usepackage{esint}	% Schönere Integrale, \oiint vorhanden
\everymath=\expandafter{\the\everymath\displaystyle}	% Mathe Inhalte werden weniger verkleinert
\usepackage{wasysym}	% mehr Symbole, bspw \lightning
% Auch arcus-Hyperbolicus-Funktionen
\DeclareMathOperator{\arccot}{arccot}
\DeclareMathOperator{\arccosh}{arccosh}
\DeclareMathOperator{\arcsinh}{arcsinh}
\DeclareMathOperator{\arctanh}{arctanh}
\DeclareMathOperator{\arccoth}{arccoth} 
%\renewcommand{\int}{\int\limits}
%\usepackage{xfrac}	% mehr fracs: sfrac{}{}
\let\oldemptyset\emptyset	% schöneres emptyset
\let\emptyset\varnothing
%\RequirePackage{mathabx}	% mehr Symbole
\mathchardef\mhyphen="2D	% Hyphen in Math

% tikz usw.
\usepackage{tikz}
\usepackage{pgfplots}
\pgfplotsset{compat=1.11}	% Umgeht Fehlermeldung
\usetikzlibrary{graphs}
%\usetikzlibrary{through}	% ???
\usetikzlibrary{arrows}
\usetikzlibrary{arrows.meta}	% Pfeile verändern / vergrößern: \draw[-{>[scale=1.5]}] (-3,5) -> (-3,3);
\usetikzlibrary{automata,positioning} % Zeilenumbruch im Node node[align=center] {Text\\nächste Zeile} automata für Graphen
\usetikzlibrary{matrix}
\usetikzlibrary{patterns}	% Schraffierte Füllung
\usetikzlibrary{shapes.geometric}	% Polygon usw.
\tikzstyle{reverseclip}=[insert path={	% Inverser Clip \clip
	(current page.north east) --
	(current page.south east) --
	(current page.south west) --
	(current page.north west) --
	(current page.north east)}
% Nutzen: 
%\begin{tikzpicture}[remember picture]
%\begin{scope}
%\begin{pgfinterruptboundingbox}
%\draw [clip] DIE FLÄCHE, IN DER OBJEKT NICHT ERSCHEINEN SOLL [reverseclip];
%\end{pgfinterruptboundingbox}
%\draw DAS OBJEKT;
%\end{scope}
%\end{tikzpicture}
]	% Achtung: dafür muss doppelt kompliert werden!
\usepackage{graphpap}	% Grid für Graphen
\tikzset{every state/.style={inner sep=2pt, minimum size=2em}}
\usetikzlibrary{mindmap, backgrounds}
%\usepackage{tikz-uml}	% braucht Dateien: http://perso.ensta-paristech.fr/~kielbasi/tikzuml/

% Tabular
\usepackage{longtable}	% Große Tabellen über mehrere Seiten
\usepackage{multirow}	% Multirow/-column: \multirow{2[Anzahl der Zeilen]}{*[Format]}{Test[Inhalt]} oder \multicolumn{7[Anzahl der Reihen]}{|c|[Format]}{Test2[Inhalt]}
\renewcommand{\arraystretch}{1.3} % Tabellenlinien nicht zu dicht
\usepackage{colortbl}
\arrayrulecolor{gray}	% heller Tabellenlinien
\usepackage{array}	% für folgende 3 Zeilen (für Spalten fester breite mit entsprechender Ausrichtung):
\newcolumntype{L}[1]{>{\raggedright\let\newline\\\arraybackslash\hspace{0pt}}m{\dimexpr#1\columnwidth-2\tabcolsep-1.5\arrayrulewidth}}
\newcolumntype{C}[1]{>{\centering\let\newline\\\arraybackslash\hspace{0pt}}m{\dimexpr#1\columnwidth-2\tabcolsep-1.5\arrayrulewidth}}
\newcolumntype{R}[1]{>{\raggedleft\let\newline\\\arraybackslash\hspace{0pt}}m{\dimexpr#1\columnwidth-2\tabcolsep-1.5\arrayrulewidth}}
\usepackage{caption}	% Um auch unbeschriftete Captions mit \caption* zu machen

% Nützliches
\usepackage{verbatim}	% u.a. zum auskommentieren via \begin{comment} \end{comment}
\usepackage{tabto}	% Tabs: /tab zum nächsten Tab oder /tabto{.5 \CurrentLineWidth} zur Stelle in der Linie
\NumTabs{6}	% Anzahl von Tabs pro Zeile zum springen
\usepackage{listings} % Source-Code mit Tabs
\usepackage{lstautogobble} 
\ifthenelse{\equal{\customDocumentClass}{beamer}}{}{
\usepackage{enumitem}	% Anpassung der enumerates
%\setlist[enumerate,1]{label=(\arabic*)}	% global andere Enum-Items
\renewcommand{\labelitemiii}{$\scriptscriptstyle ^\blacklozenge$} % global andere 3. Item-Aufzählungszeichen
}
\newenvironment{anumerate}{\begin{enumerate}[label=(\alph*)]}{\end{enumerate}} % Alphabetische Aufzählung
\usepackage{letltxmacro} % neue Definiton von Grundbefehlen
% Nutzen:
%\LetLtxMacro{\oldemph}{\emph}
%\renewcommand{\emph}[1]{\oldemph{#1}}
\RequirePackage{xpatch}	% ua. Konkatenieren von Strings/Variablen (etoolbox)


% Einrichtung von lst
\lstset{
basicstyle=\ttfamily, 
%mathescape=true, 
%escapeinside=^^, 
autogobble, 
tabsize=2,
basicstyle=\footnotesize\sffamily\color{black},
frame=single,
rulecolor=\color{lightgray},
numbers=left,
numbersep=5pt,
numberstyle=\tiny\color{gray},
commentstyle=\color{gray},
keywordstyle=\color{green},
stringstyle=\color{orange},
morecomment=[l][\color{magenta}]{\#}
showspaces=false,
showstringspaces=false,
breaklines=true,
literate=%
    {Ö}{{\"O}}1
    {Ä}{{\"A}}1
    {Ü}{{\"U}}1
    {ß}{{\ss}}1
    {ü}{{\"u}}1
    {ä}{{\"a}}1
    {ö}{{\"o}}1
    {~}{{\textasciitilde}}1
}
\usepackage{scrhack} % Fehler umgehen
\def\ContinueLineNumber{\lstset{firstnumber=last}} % vor lstlisting. Zum wechsel zum nicht-kontinuierlichen muss wieder \StartLineAt1 eingegeben werden
\def\StartLineAt#1{\lstset{firstnumber=#1}} % vor lstlisting \StartLineAt30 eingeben, um bei Zeile 30 zu starten
\let\numberLineAt\StartLineAt

% BibTeX
\usepackage[backend=bibtex8, bibencoding=ascii,
%style=authortitle, citestyle=authortitle-ibid,
%doi=false,
%isbn=false,
%url=false
]{biblatex}	% BibTeX
\usepackage{makeidx}
%\makeglossary
%\makeindex

% Grafiken
\usepackage{graphicx}
\usepackage{epstopdf}	% eps-Vektorgrafiken einfügen
%\epstopdfsetup{outdir=\customDir}

% pdf-Setup
\usepackage{pdfpages}
\ifthenelse{\equal{\customDocumentClass}{beamer}}{}{
\usepackage[bookmarks,%
bookmarksopen=false,% Klappt die Bookmarks in Acrobat aus
colorlinks=true,%
linkcolor=black,%
citecolor=red,%
urlcolor=green,%
]{hyperref}
}

%-- Unterscheidung des Stils
\newcommand{\customLogo}{}
\newcommand{\customPreamble}{}
\ifthenelse{\equal{\customDesign}{htw}}{
	% HTW Corporate Design: Arial (Helvetica)
	\usepackage{helvet}
	\renewcommand{\familydefault}{\sfdefault}
	\renewcommand{\customLogo}{HTW-Logo}
	\renewcommand{\customPreamble}{HTW Dresden}
}{
% \renewcommand{\customLogo}{HTW-Logo.eps}
}

% Nach Dokumentenbeginn ausführen:
\AtBeginDocument{
	% Autor und Titel für pdf-Eigenschaften festlegen, falls noch nicht geschehen
	\providecommand{\pdfAuthor}{John Doe}
	\ifdefempty{\customAuthor} {} {\renewcommand{\pdfAuthor}{\customAuthor}}
	\providecommand{\pdfTitle}{}
	\providecommand{\pdfTitleA}{}
	\providecommand{\pdfTitleB}{}
	\providecommand{\pdfTitleC}{}	
	\ifdefempty{\pdfTitle}{
		\ifdefempty{\customPreamble} {} {\renewcommand{\pdfTitleA}{\customPreamble{} | }}
		\ifdefempty{\customTitle} {\renewcommand{\pdfTitleB}{No Title}} {\renewcommand{\pdfTitleB}{\customTitle}}
		\ifdefempty{\customSubtitle} {} {\renewcommand{\pdfTitleC}{ - \customSubtitle}}
	}{}
	
	\newcommand{\customLogoLocation}{\customDir .LaTeX_master/\customLogo}
	\hypersetup{
		pdfauthor={\pdfAuthor},
		pdftitle={\pdfTitleA\pdfTitleB\pdfTitleC},
	}
	\ifthenelse{\equal{\customDocumentClass}{beamer}}{
		\title{\customTitle}
		\author{\customAuthor}
	}{
		\automark[section]{section}
		\automark*[subsection]{subsection}
		\pagestyle{scrheadings}
		\ifthenelse{\equal{\customDocumentClass}{report} \OR \equal{\customDocumentClass}{scrreprt}}{
		\renewcommand*{\chapterpagestyle}{scrheadings}
		}{}
		%\renewcommand*{\titlepagestyle}{scrheadings}
		\ihead{\includegraphics[height=1.7em]{\customLogoLocation}}
		%\ohead{\truncate{4cm}{\customTitle}}
		\chead{\truncate{.5\textwidth}{\headmark}}
		\ohead{\customTitle}
		\cfoot{\pagemark}
		\ofoot{\customSignature}
		% Titelseite
		\title{
		\includegraphics[width=0.35\textwidth]{\customDir .LaTeX_master/\customLogo}\\\vspace{0.5em}
		\Huge\textbf{\customTitle}
		\ifdefempty{\customSubtitle} {} {\\\vspace*{0.7em}\Large \customSubtitle}
		\\\vspace*{5em}}
		\author{
		\ifdefempty{\customNoteA} {} {\customNoteA \vspace*{1em}}\\ 
		\ifdefempty{\customAuthor} {} {\customTitleAuthor}
		\ifdefempty{\customNoteB}{}{\vspace*{1em}\\\customNoteB}
		}
		
		\ifthenelse{\equal{\customDocumentClass}{cheatsheet}}{
			\pagestyle{empty}
			\setlist{nolistsep}
	%		\usepackage{parskip}	% Aufzählung Abstand
	%		\setlength{\parskip}{0em}
			\lstset{
	    belowcaptionskip=0pt,
	    belowskip=0pt,
	    aboveskip=0pt,
			tabsize=2,
			frame=none,
			numbers=none,
			showspaces=false,
			showstringspaces=false,
			breaklines=true,
			}
		}{}
	}
}

% Unterabschnitte
%\newtheorem{example}{Beispiel}%[section]
%\newtheorem{definition}{Definition}[section]
%\newtheorem{discussion}{Diskussion}[section]
%\newtheorem{remark}{Bemerkung}[section]
%\newtheorem{proof}{Beweis}[section]
%\newtheorem{notation}{Schreibweise}[section]
\RequirePackage{xcolor}

% Horizontale Linie:
\newcommand{\HRule}[1][\medskipamount]{\par
  \vspace*{\dimexpr-\parskip-\baselineskip+#1}
  \noindent\rule[0.2ex]{\linewidth}{0.2mm}\par
  \vspace*{\dimexpr-\parskip-.5\baselineskip+#1}}
% Gestrichelte horizontale Linie:
\RequirePackage{dashrule}
\newcommand{\HDRule}[1][\medskipamount]{\par
  \vspace*{\dimexpr-\parskip-\baselineskip+#1}
  \noindent\hdashrule[0.2ex]{\linewidth}{0.2mm}{1mm} \par
  \vspace*{\dimexpr-\parskip-.5\baselineskip+#1}}
% Mathe in Anführungszeichen:
\newsavebox{\mathbox}\newsavebox{\mathquote}
\makeatletter
\newcommand{\mq}[1]{% \mathquotes{<stuff>}
  \savebox{\mathquote}{\text{"}}% Save quotes
  \savebox{\mathbox}{$\displaystyle #1$}% Save <stuff>
  \raisebox{\dimexpr\ht\mathbox-\ht\mathquote\relax}{"}#1\raisebox{\dimexpr\ht\mathbox-\ht\mathquote\relax}{''}
}
\makeatother

% Paragraph mit Zähler (Section-Weise)
\newcounter{cparagraphC}
\newcommand{\cparagraph}[1]{
\stepcounter{cparagraphC}
\paragraph{\thesection{}-\Roman{cparagraphC} #1}
%\addcontentsline{toc}{subsubsection}{\thecparagraphC{} #1}
\label{\thesection-\thecparagraphC}
}
\makeatletter
\@addtoreset{cparagraphC}{section}
\makeatother


% (Vorlesungs-)Folien einbinden:
% Folien von einer Datei skaliert
\newcommand{\slideScale}[2][0.5]{\begin{center}
\includegraphics[page=#2, scale=#1]{\customSlidePath .pdf}
\end{center}}
% Folien von einer Datei mit fester Höhe
\newcommand{\slideHeight}[2][9.63cm]{\begin{center}
\includegraphics[page=#2, height=#1]{\customSlidePath .pdf}
\end{center}}
% Folien von einer Datei mit fester Breite
\newcommand{\slideWidth}[2][12.8cm]{\begin{center}
\includegraphics[page=#2, width=#1]{\customSlidePath .pdf}
\end{center}}
% Folien von mehreren nummerierten Dateien skaliert
\newcommand{\slidesScale}[3][0.5]{\begin{center}
\includegraphics[page=#3, scale=#1]{\customSlidePath #2.pdf}
\end{center}}
% Folien von mehreren nummerierten Dateien mit fester Höhe
\newcommand{\slidesHeight}[3][9.63cm]{\begin{center}
\includegraphics[page=#3, height=#1]{\customSlidePath #2.pdf}
\end{center}}
% Folien von mehreren nummerierten Dateien mit fester Breite
\newcommand{\slidesWidth}[3][12.8cm]{\begin{center}
\includegraphics[page=#3, width=#1]{\customSlidePath #3.pdf}
\end{center}}


%% EINFACHE BEFEHLE

% Abkürzungen Mathe
\newcommand{\EE}{\mathbb{E}}
\newcommand{\QQ}{\mathbb{Q}}
\newcommand{\RR}{\mathbb{R}}
\newcommand{\CC}{\mathbb{C}}
\newcommand{\NN}{\mathbb{N}}
\newcommand{\ZZ}{\mathbb{Z}}
\newcommand{\PP}{\mathbb{P}}
\renewcommand{\SS}{\mathbb{S}}
\newcommand{\cA}{\mathcal{A}}
\newcommand{\cB}{\mathcal{B}}
\newcommand{\cC}{\mathcal{C}}
\newcommand{\cD}{\mathcal{D}}
\newcommand{\cE}{\mathcal{E}}
\newcommand{\cF}{\mathcal{F}}
\newcommand{\cG}{\mathcal{G}}
\newcommand{\cH}{\mathcal{H}}
\newcommand{\cI}{\mathcal{I}}
\newcommand{\cJ}{\mathcal{J}}
\newcommand{\cM}{\mathcal{M}}
\newcommand{\cN}{\mathcal{N}}
\newcommand{\cP}{\mathcal{P}}
\newcommand{\cR}{\mathcal{R}}
\newcommand{\cS}{\mathcal{S}}
\newcommand{\cZ}{\mathcal{Z}}
\newcommand{\cL}{\mathcal{L}}
\newcommand{\cT}{\mathcal{T}}
\newcommand{\cU}{\mathcal{U}}
\newcommand{\cV}{\mathcal{V}}
\renewcommand{\phi}{\varphi}
\renewcommand{\epsilon}{\varepsilon}

% Farbdefinitionen
\definecolor{red}{RGB}{180,0,0}
\definecolor{green}{RGB}{75,160,0}
\definecolor{blue}{RGB}{0,75,200}
\definecolor{orange}{RGB}{255,128,0}
\definecolor{yellow}{RGB}{255,245,0}
\definecolor{purple}{RGB}{75,0,160}
\definecolor{cyan}{RGB}{0,160,160}
\definecolor{brown}{RGB}{120,60,10}

\definecolor{itteny}{RGB}{244,229,0}
\definecolor{ittenyo}{RGB}{253,198,11}
\definecolor{itteno}{RGB}{241,142,28}
\definecolor{ittenor}{RGB}{234,98,31}
\definecolor{ittenr}{RGB}{227,35,34}
\definecolor{ittenrp}{RGB}{196,3,125}
\definecolor{ittenp}{RGB}{109,57,139}
\definecolor{ittenpb}{RGB}{68,78,153}
\definecolor{ittenb}{RGB}{42,113,176}
\definecolor{ittenbg}{RGB}{6,150,187}
\definecolor{itteng}{RGB}{0,142,91}
\definecolor{ittengy}{RGB}{140,187,38}

\definecolor{htworange}{RGB}{249,155,28}

% Textfarbe ändern
\newcommand{\tred}[1]{\textcolor{red}{#1}}
\newcommand{\tgreen}[1]{\textcolor{green}{#1}}
\newcommand{\tblue}[1]{\textcolor{blue}{#1}}
\newcommand{\torange}[1]{\textcolor{orange}{#1}}
\newcommand{\tyellow}[1]{\textcolor{yellow}{#1}}
\newcommand{\tpurple}[1]{\textcolor{purple}{#1}}
\newcommand{\tcyan}[1]{\textcolor{cyan}{#1}}
\newcommand{\tbrown}[1]{\textcolor{brown}{#1}}

% Umstellen der Tabellen Definition
\newcommand{\mpb}[1][.3]{\begin{minipage}{#1\textwidth}\vspace*{3pt}}
\newcommand{\mpe}{\vspace*{3pt}\end{minipage}}

\newcommand{\resultul}[1]{\underline{\underline{#1}}}
\newcommand{\parskp}{$ $\\}	% new line after paragraph
\newcommand{\corr}{\;\widehat{=}\;}
\newcommand{\mdeg}{^{\circ}}

\newcommand{\nok}[2]{\binom{#1}{#2}}	% n über k BESSER: \binom{n}{k}
\newcommand{\mtr}[1]{\begin{pmatrix}#1\end{pmatrix}}	% Matrix
\newcommand{\dtr}[1]{\begin{vmatrix}#1\end{vmatrix}}	% Determinante (Betragsmatrix)
\renewcommand{\vec}[1]{\underline{#1}}	% Vektorschreibweise
\newcommand{\imptnt}[1]{\colorbox{red!30}{#1}}	% Wichtiges
\newcommand{\intd}[1]{\,\mathrm{d}#1}
\newcommand{\diffd}[1]{\mathrm{d}#1}
% für Module-Rechnung: \pmod{}

\bibliography{\customDir .Literatur/HTW_Literatur.bib}

\begin{document}

%\selectlanguage{english}
\maketitle
\newpage
\tableofcontents
\newpage

\chapter*{Allgemeine Informationen}

Zugelassene Hilfsmittel Klausur: A-4 Blatt (doppelseitig, handbeschrieben)

Prüfungsvorleistung: alle paar Woche eine Lernabfrage in der Vorlesung (Bestanden wenn insgesamt im Schnitt 50\%)

Grundlage der Vorlesung: Grundkurs theoretische Informatik \cite{hollas2007grundkurs}

Lernkontrolle ab 23.10.2015 alle zwei Wochen.

\chapter{Aussagenlogik}

Mit der Aussagenlogik lassen sich Aussagen formulieren, die entweder wahr oder falsch sind. Aussagen sind atomare Aussagen wie „die Straße ist nass“ oder mit Hilfe von logischen Operatoren zusammengesetzte Aussagen.

\section{Syntax und Semantik}

\paragraph{Definition:} Die \emph{Formeln der Aussagenlogik} sind induktiv definiert.

\begin{itemize}
\item Jede atomare Aussage ist eine Formel der Aussagenlogik. Diese heißen Atomformeln oder Variablen. 
Atomformeln bezeichnen wir mit Kleinbuchstaben oder durch Wörter in Kleinbuchstaben.
\item Wenn $F$, $G$ Formeln der Aussagenlogik sind, dann auch $(F \wedge G)$, $(F\vee G)$, $(\neg F)$.
\end{itemize}

\subparagraph{Bsp.:} Formeln der Aussagenlogik sind $x$, $y$, $x\wedge y$, $( x\wedge (y \wedge z)) \vee (\neg x \wedge (y \wedge \neg z))$, $regnet$, $regnet \wedge nass$, da sie jeweils aus atomaren Aussagen die nach der Definition zusammensetzen lassen bestehen.\\
Keine Formeln der Aussagenlogik sind $x\wedge$, $\vee x$, $x\wedge \vee y$.\\
Um Klammern zu sparen, legen wir Prioritäten fest:\\
\begin{tabular}{c | c}
Operator & Priorität \\
\hline
$\neg$ & höchste\\
$\wedge ,\; \vee$ & \\
$\rightarrow , \; \leftrightarrow$ & niedrigste\\
\end{tabular}

\paragraph{Definition:} Eine \emph{Belegung} einer Formel $F$ der Aussagenlogik ist eine Zuordnung von Wahrheitswerten „wahr“ (1) oder „falsch“ (0) zu den Atomarformeln in $F$.
Daraus ergibt sich der \emph{Wahrheitswert} einer Formel:
\begin{itemize}
\item Eine Atomformel ist genau dann wahr, wenn sie mit „wahr“ belegt ist.
\item Die Formel $F\wedge G $ ist genau dann wahr, wenn $F$ „wahr“ ist und $G$ „wahr“ ist. \\
$F\vee G$ ist wahr, wenn $F$ wahr ist oder $G$ wahr.\\
$\neg F$ ist wahr, wenn $F$ falsch ist.\\
\begin{tabular}{c c | c c c c c}
$F$&$G$&$F\wedge G$&$F\vee G$&$\neg F$&$F\rightarrow G$& $F\leftrightarrow G$\\
\hline
0&0&0&0&1&1&1\\
0&1&0&1&1&1&0\\
1&0&0&1&0&0&0\\
1&1&1&1&0&1&1\\
\end{tabular}
\end{itemize}

\subparagraph{Bsp.:} Wenn $regnet$ bedeutet: „Es regnet.“\\
Wenn $nass$ bedeutet: „Die Straße ist nass.“\\
Dann bedeutet $regnet \wedge nass$: „Es regnet und die Straße ist nass.“\\
„Wenn es regnet, dann ist die Straße nass“ ($regnet \rightarrow nass$). Es muss nur der Fall ausgeschlossen werden, der nicht eintreffen kann: $\neg (regnet \wedge \neg nass)$ $\Rightarrow$ Folgendes darf nicht eintreffen: „Es regnet und die Straße ist nicht nass“. 

Alles andere („Es regnet nicht und die Straße ist nicht nass“, „Es regnet nicht und die Straße ist nass“ und „Es regnet und die Straße ist nass“) darf eintreffen.\\
\begin{tabular}{c c | c}
$regnet$ & $nass$ & $\neg (regnet \wedge \neg nass) = \neg regnet \vee nass$\\
\hline
0&0&1\\
0&1&1\\
1&0&0\\
1&1&1\\
\end{tabular}

\paragraph{Definition:} Die Operatoren $\rightarrow$ (\emph{Implikation}) und $\leftrightarrow$ (\emph{Äquivalenz}) sind definiert durch:
\begin{itemize}
\item $F\rightarrow G = \neg F \vee G$
\item $F\leftrightarrow G = (F \rightarrow G) \wedge (G \rightarrow F)$
\end{itemize}
(Siehe Tabelle oberhalb)

\subparagraph{Bsp.:} Berechnen des Betrags $y$ einer Zahl $x$:
\begin{lstlisting}[language=C]
if (x>= 0)
	y=x;
else 
	y=-x;
\end{lstlisting}
Dargestellt als Formel der Assagenlogik: $((x\geq 0) \rightarrow y=x)\wedge(\neg (x\geq 0) \rightarrow y=-x)$

\paragraph{Definition:} Eine Formel $F$ der Aussagenlogik heißt
\begin{itemize}

\item \emph{erfüllbar}, wenn es eine Belegung gibt, sodass $F$ wahr ist, sonst \emph{unerfüllbar}. Mit $\bot$ bezeichnen wir eine unerfüllbare Formel (Widerspruch).
\item \emph{Tautologie} oder \emph{gültig}, wenn $F$ für jede Belegung wahr ist. Bezeichnung: $\top$
\end{itemize}

\subparagraph{Bsp.:}
\begin{itemize}
\item $x\wedge y$ ist erfüllbar.
\item $((\neg x \wedge y)\vee (x\wedge \neg y)) \wedge \neg (x\vee y)$ ist unerfüllbar (linke Seite: entweder x oder y falsch - rechte Seite: x oder y falsch)
\item $x \vee \neg x$ ist eine Tautologie
\end{itemize}

\paragraph{Definition:} Wir schreiben $F \equiv G$ („$F$ ist äquivalent zu $G$“), wenn für jede Belegung gilt: $F \leftrightarrow G$ wahr (d.h., $F\leftrightarrow G$ ist gültig).

\section{Rechenregeln}
siehe Mathematik I

\chapter{Beweistechniken}

\begin{tikzpicture} [scale=.4]

\draw  (-6,6) ellipse (6 and 3) node {Als wahr bekannte Aussage};
\draw  (-16,1) ellipse (4 and 1) node{Folgerung};
\draw  (4,1) ellipse (4 and 1) node{Folgerung};
\draw [double, ->] (-12,4) -- (-14,3);
\draw [double, ->] (0,4) -- (2,3);
\draw  (-6,2) ellipse (16 and 9);
\node[right] at (7,-4) {neue wahre Aussage};
\end{tikzpicture}\bigskip\\
\begin{tikzpicture} [scale=.4]

\draw  (-11,2) ellipse (7 and 1) node{als wahr bekannte Aussage};

\draw[double, ->] (-3,2) -- (-1,2);
\draw[double, ->] (4,2) -- (6,2);
\draw[double, ->] (11,2) -- (13,2);
\node[right] at (-1,2) {Folgerung};
\node[right] at (6,2) {Folgerung};
\draw  (18,2) ellipse (4 and 1) node{Behauptung};
\draw[green]  (-2,4) rectangle (12,0) node[below left]{Beweis};
\end{tikzpicture}

\paragraph{Direkter Beweis}
\subparagraph{Bsp.:} Wenn $a \in \mathbb{Z}$ gerade ist, dann ist auch $a^2$ gerade.\\
$(a \in \mathbb{Z} \text{ gerade } \Rightarrow a^2 \text{ gerade})$

\emph{Beweis:} 
\begin{itemize}
\item Wenn $a$ gerade ist, gibt es ein $n$ mit $a=2\cdot n$.
\item Dann gilt $a^2=4\cdot n^2=2\cdot 2 n^2$, 
\item woraus $a^2$ gerade folgt.
\end{itemize}

\paragraph{Indirekter Beweis} Mit einem indirekten Beweis wird $A\Rightarrow B$ bewiesen, indem die äquivalente Aussage $\neg B \Rightarrow \neg A$ bewiesen wird.

\subparagraph{Bsp.:} Wenn $a^2$ gerade ist, dann auch $a$.\\
$(a^2 \text{ gerade } \Rightarrow a \text{ gerade})$

\emph{Beweis:} Wir zeigen: Wenn $a$ ungerade ist, dann auch $a^2$.
\begin{itemize}
\item Aus $a$ ungerade folgt $a=2n-1$ für ein $n \in \mathbb{N}$. 
\item Dann ist $a^2=4 n^2 - 4 n + 1 = \underbrace{4( n^2-n)}_{gerade} + \underbrace{1}_{ungerade}$, 
\item Aus gerade + ungerade folgt ungerade, woraus $a^2$ ungerade folgt.
\end{itemize}

\paragraph{Beweis durch Widerspruch} Mit einem Beweis durch Widerspruch wird eine Aussage $A$ bewiesen, indem gezeigt wird, dass die Annahme „$A$ ist falsch“ zu einem Widerspruch führt.\\
(D.h., es wird $\neg A \rightarrow \bot$ gezeigt)

\subparagraph{Bsp.:} $\sqrt{2}$ ist irrational. Siehe Mathematik I.

\paragraph{Vollständige Induktion} Mit einer vollständigen Induktion lassen sich Aussagen der Art „für alle $n \in \mathbb{N}$ gilt …“ beweisen.

Prinzip: Gegeben eine Aussage der Form „für alle $n \in \mathbb{N}$ gilt $A(n)$“

\begin{itemize}
\item \emph{Induktionsanfang}: Man zeigt, die Wahrheit der Aussage für $n=1$ (mit anderen Worten: Man zeigt, dass $A(1)$ wahr ist) [$1$: die kleinste mögliche Zahl $\Rightarrow$ kann auch 0 oder eine andere sein]
\item \emph{Induktionsvorussetzung}: Die Aussage ist für $n$ wahr.
\item \emph{Induktionsschritt}: Wenn IV wahr ist, dann ist die Aussage auch für $n+1$ wahr.
\end{itemize}
In Formeln: Man zeigt
\begin{itemize}
\item IA: $A(1)$
\item IV: $A(n)$
\item IS: für alle $n$: $A(n)\Rightarrow A(n+1)$
\end{itemize}
Beispiel Dominosteine: Wenn der erste Stein fällt, fällt auch der Zweite. Und wenn der $n$-te Stein fällt, fällt auch der $n+1$-te:
\begin{itemize}
\item IA: 1. Umstoßen
\item IV: Wenn der vorherige Stein umfällt, fällt auch der nächste
\item IS: Wenn $n$-ter umgestoßen wird, dann auch $n+1$-ter
\end{itemize}
\begin{tikzpicture} [scale=.4]
\draw (-16,4) -- (-16,1);
\draw (-15,4) -- (-15,1) ;
\draw (-14,4) -- (-14,1)node[below]{n};
\draw (-19,4) -- (-19,1)node[below]{1};
\draw (-18,4) -- (-18,1)node[below]{2};
\draw (-17,4) -- (-17,1);
\draw (-10,4) -- (-10,1);
\draw (-9,4) -- (-9,1);
\draw (-8,4) -- (-8,1);
\draw (-13,4) node[above]{n+1} -- (-13,1);
\draw (-12,4) -- (-12,1);
\draw (-11,4) -- (-11,1);
\end{tikzpicture}
\subparagraph{Bsp.:} Für alle $n\geq 1$ gilt $\sum_{k=1}^{n}k=\frac{n(n+1)}{2}$

\emph{Beweis (Induktion):}
\begin{itemize}
\item[IA] $n=1$: $1=\frac{1 \cdot 2}{2}$ ist wahr.
\item[IV] Es gelte $\sum_{k=1}^{n}k=\frac{n(n+1)}{2}$ ist wahr.
\item[IS] $n \rightarrow n+1$: Zu zeigen: $\sum_{k=1}^{n+1}k=\frac{(n+1)(n+2)}{2}$ Es gilt:
\begin{align*}
\sum_{k=1}^{n+1}k&=\left(\sum_{k=1}^{n}k\right)+n+1\\
&\overset{IV}{=}\frac{n(n+1)}{2}+n+1\\
&=...=\frac{(n+1)(n+2)}{2}
\#
\end{align*}
\end{itemize}

\chapter{Elementare Kombinatorik}

Kreuzprodukt:\\
$A\times B = \{(a,b)|a \in A, b \in B\}$\\
$A^n=\underbrace{A\times ... \times A}_{n}$\\
Die \emph{Potenzmenge} einer Menge $M$ ist die Menge aller Teilmengen von $M$:
$\mathcal{P}(M)=\{A|A\subseteq M\}$
\subparagraph{Bsp.:} $\mathcal{P}(\{1,2\})=\{\emptyset, \{1\}, \{2\}, \{1,2\}\}$

\paragraph{Definition:} Die Mächtigkeit einer Menge $A$ ist die Anzahl ihrer Elemente. Notation: $|A|$

\paragraph{Satz:} Es gilt $|A^n|=|A|^n$
\subparagraph{Beweis:} 
Nach Def. ist $A^n=\{(a_1,...,a_n)|a_1,...,a_n \in A\}$. Um das n-Tupel $(a_1,...,a_n)$ zu erzeugen, gibt es $|A|$ viele Möglichkeiten. Insgesamt gibt es daher $|A|^n$ Möglichkeiten das n-Tupel $(a_1,...,a_n)$ auszuwählen.

\subparagraph{Bsp.:} Eine PIN bestehe aus 6 Ziffern. Mit $A=\{0,...,9\}$ ist $A^6$ die Menge aller PINs. Mit obigen Satz folgt: Die Anzahl aller PINs ist $|A^6|=|A|^6 = 10^6$

\subparagraph{Bsp.:} In dem Programm
\begin{lstlisting}[language=C]
for (i=1 to n)
	for (j=1 to n)
		a[i][j]=i+j;
\end{lstlisting}
werden alle Paare (i,j) erzeugt. Die Anzahl der Paare ist $|\{1,...,n\}^2|=|\{1,...,n\}|^2=n^2$. Es gibt daher $n^2$ Schleifendurchläufe.

\paragraph{Satz:} $|\mathcal{P}(M)|=2^{|M|}$
\subparagraph{Beweis:} Für $M=\{m_1, ...,m_n\}$ identifizieren wir eine Teilmenge $A\subseteq M$ durch das n-Tupel $(a_1, ..., a_n)$ mit $a_k\begin{cases}0\text{ für }M_k \not \in A\\ 1\text{ für }m_k \in A\end{cases}$. Nach obigen Satz gibt es $|\{0,1\}^n|=2^n=2^{|M|}$ derartige Tupel.

\paragraph{Definition:} Für eine n-elementige Menge ist $\nok{n}{k}$ die Anzahl ihrer k-elementigen Teilmengen $(n\geq k \geq 0)$.

\subparagraph{Bsp.:} \parskp
$\nok{n}{0}=1$, da $\emptyset$ die einzige 0-elementige Teilmenge ist.\\
$\nok{n}{n}=1$, da es nur eine n-elementige Teilmenge gibt (die Menge selber).\\
$\nok{n}{1}=n$, da es $n$ 1-elementige Teilmengen gibt.\\
$\nok{n}{2}=\frac{n(n-1)}{2}$, denn für das 1. Element gibt es $n$ Möglichkeiten, für das 2. Element $n-1$ Möglichkeiten. Da das Element $\{a,b\}=\{b,a\}$ hierbei doppelt gezählt wird, müssen wir durch 2 teilen.

\paragraph{Definition:} Eine Permutation der Folge $1,...,n$ ist eine neue Anordnung dieser Folge.

\subparagraph{Bsp.:} Alle Permutationen von $1,2,3$ sind $1,2,3$; $1,3,2$; $2,1,3$; $2,3,1$; $3,1,2$; $3,2,1$.

\paragraph{Definition:} $n!=1\cdot ... \cdot n \qquad 0! =1$.
\paragraph{Satz:} Es gibt $n!$ Permutationen von $n$ Zahlen.

\subparagraph{Beweis:} Für die 1. Stelle gibt es $n$ Möglichkeiten, für die 2. Stelle $n-1$ usw. Für die letzte Stelle nur noch eine Möglichkeit. Insgesamt also $n\cdot ... \cdot 1=n!$ Möglichkeiten.

\paragraph{Satz:} $\nok{n}{k}= \frac{n!}{k! (n-k)!}$ \footnote{$\sum_{k=0}^{n}\nok{n}{k}=2^n$}
\subparagraph{Beweis:} Um aus einer $n$-elementigen Menge $k$ Elemente auszuwählen, gibt es $n$ Möglichkeiten, um das erste Element auszuwählen, für das zweite Element $n-1$ Möglichkeiten, …, für das $k$. Element $n-k+1$ Möglichkeiten, insgesamt daher $n \cdot ...\cdot (n-k+1)$ Möglichkeiten. Da die Reihenfolge, in der diese $k$ Elemente ausgewählt werden, keine Rolle spielt, muss dieses Produkt durch $k!$ geteilt werden. \\
Daher erhalten wir $\nok{n}{k}= \frac{n\cdot ...\cdot (n-k+1)}{k!}= \frac{n!}{k! (n-k)!}$

\chapter{O-Notation}
Mit Hilfe der O-Notation lassen sich obere Schranken für die Laufzeit eines Algorithmus angeben (Abschätzung mit $\leq$, die die maximale Laufzeit eines Algorithmus angibt, bspw. $\leq c \cdot n^2$). Um die Laufzeit eines Algorithmus zu messen, bestimmen wir die Anzahl Schritte und geben mit Hilfe der O-Notation deren Größenordnung in Abhängigkeit der Länge der Eingabe an.

\subparagraph{Beispiel:} lineare Suche
\begin{lstlisting} [language=C]
int lsearch (int a[], int n, int k) {
	int i;
	for (i=0; i<n; i++)
		if ( a[i] == k) return 1;	//gefunden
	return 0;	// nicht gefunden
}
\end{lstlisting}
Laufzeit dieser Funktion:\\
$\leq \textcolor{blue}{\underbrace{\textcolor{black}{c_1+ n\cdot c_2+c_3}}_{g(n)}} \overset{\text{Abschätzung}}{\leq}(c_1+c_2+c_3)\cdot n= c \cdot n$\\
$c_1$ … Deklarierung von $i$\\
$c_2$ … Vergleich der Werte in der Schleife (in $n$ Schleifedurchläufen)\\
$c_3$ … Ausführung return\\
(Durch den Worst-Case von annähernd unendlich vielen Durchläufen spielen die Konstanten, egal wie groß, keine besonder Rolle mehr und können, wie in der Abschätzung zu sehen, zusammengefasst werden).\\
Die Laufzeit der linearen Suche liegt in $O(n)$

\paragraph{Definition:} Für eine Funktion $f>0$ ist $O(f)$ die Menge aller Funktionen $g$, für die gilt:\\
$g(n)\leq c\cdot f(n)$ für ein $c>0$ für alle großen $n$.\\
\begin{tikzpicture}[scale=3]
\draw[-latex] (0,-0.1) -- (0,1.1);
\draw[-latex] (-0.1,0) -- (1.1,0) node[right]{$n$};

\pattern [pattern=north west lines,pattern color=lightgray] (0,0) -- plot[domain=0:1] (\x, {(\x*\x)}) -- (1,1) -- (1,0);

\draw plot[domain=0:1] (\x, {(\x*\x)}) node[below right] {$c\cdot f(n)$};
\draw (0.75,0.25) node{$O(f)$};
\end{tikzpicture}\\
$
2 log (n^2+1)\\
\leq 2 log ((1+1)n^2)\\
=2 log (2n^2)\\
= 2 log (2) + 2 log( n^2)\\
\leq (2log2+2)log(n^2)\\
= 2 (log2+1)log(n)
$

\subparagraph{Bsp.:} $2n^3-n+5\overset{1.)}{\leq} 2n^3+5 \leq 7n^3 \in O(n^3)$
\begin{enumerate}
\item $-n$ ist kleiner Null, deswegen ist die rechte Seite ohne $-n$ nachgewiesener Maßen größer (Vorgehensweise Ungleichung aufstellen (siehe auch folgende): weg lassen, was kleiner Null ist; mit $n^x$ o.ä. erweitern, um auszuklammern).
\end{enumerate}

\subparagraph{Bsp.:}$ $
\begin{lstlisting}[language=C]
for (i=0; i<n-1; i++)
	for (j=i+1 ; j<n ; j++)
		if( a[i] == a[j] ) return 1;
return 0;
\end{lstlisting}
Die if-Anweisung wird höchstens $\nok{n}{2}$ mal durchlaufen. Die Laufzeit ist daher $\leq c_1 \cdot \nok{n}{2} + c_2 \leq (c_1+c_2) \cdot \nok{n}{2} = \frac{c_1+c_2}{2}\cdot n \cdot (n-1)\leq \frac{c_1 + c_2}{2}\cdot n^2 \in O(n^2)$ (mit $c_i$ … Zeiteinheiten für Rechenaufwand).
\subparagraph{Bsp.:} $2\cdot log(n^2+1)\\
\leq 2 \cdot log (n^2(1+1))\\
=2log(2n^2)=2(log(2)+log(n^2)) \\
\leq 2(log(n)+2log(n))\\
\leq 6 log(n) \in O(log(n))$\\
Schneller mit:\\
$n^2+1\leq n^3 \Leftrightarrow \\
\frac{1}{n}+\frac{1}{n^3}\leq 1 \Rightarrow \\
0\leq 1 \qquad \text{ für } n \rightarrow\infty$\\
$2\cdot log(n^2+1)\leq 2 log (n^3)=6log(n)\in O(log(n))$

\chapter{Graphen}

\begin{tikzpicture}[scale=0.5,remember picture]
\begin{scope}
\begin{pgfinterruptboundingbox}
%\draw [clip] plot[smooth, tension=0.4, smooth cycle, rotate=-15] coordinates {(-0.6,0.8) (0.4,1.8) (1.7,1.1) (2.4,-0.2) (1.4,-1.2) (0.6,0.1)}[reverseclip];
%\draw [clip] plot[smooth, tension=0.4, smooth cycle, rotate=-50] coordinates {(1,4.4) (2,5.4) (3.3,4.7) (4,3.4) (3,2.4) (2.2,3.7)}[reverseclip];
%\draw [clip] plot[smooth, tension=0.4, smooth cycle, rotate=-45] coordinates {(5.2,1) (6.2,2) (7.5,1.3) (8.2,0) (7.2,-1) (6.4,0.3)}[reverseclip];
%\draw [clip] plot[smooth, tension=0.4, smooth cycle, rotate=-65] coordinates {(3,-0.2) (4,0.8) (5.3,0.1) (6,-1.2) (5,-2.2) (4.2,-0.9)}[reverseclip];
%\draw [clip] plot[smooth, tension=0.4, smooth cycle, rotate=35] coordinates {(4,-6) (5,-5) (6.3,-5.7) (7,-7) (6,-8) (5.2,-6.7)}[reverseclip];
%\draw [clip] plot[smooth, tension=0.4, smooth cycle, rotate=-55] coordinates {(4.6,10.2) (5.6,11.2) (6.9,10.5) (7.6,9.2) (6.6,8.2) (5.8,9.5)}[reverseclip];
%\draw [clip] plot[smooth, tension=0.4, smooth cycle, rotate=-35] coordinates {(10.8,3.4) (11.8,4.4) (13.1,3.7) (13.8,2.4) (12.8,1.4) (12,2.7)}[reverseclip];
\draw [rotate=-10, clip]  (4.1,2.8) rectangle (5.6,-0.2)[reverseclip];
\draw [rotate=30, clip] (0.5,1.4) rectangle (2,-1.6)[reverseclip];
\draw [rotate=10, clip] (3.5,-3.8) rectangle (5,-6.8)[reverseclip];
\draw [rotate=-10, clip] (10.5,4.2) rectangle (12,1.2)[reverseclip];
\draw [rotate=15, clip] (9.3,-6.2) rectangle (10.8,-9.2)[reverseclip];
\draw [rotate=91, clip] (-3,-6.6) rectangle (-1.5,-9.6)[reverseclip];
\draw [rotate=-20, clip] (2.1,-2.2) rectangle (3.6,-5.2)[reverseclip];
\end{pgfinterruptboundingbox}
\draw [blue] plot[smooth, tension=.7] coordinates {(-8,-2) (-4,-1) (-1,0) (3,1) (8,1) (13,1) (15,1)};
\draw [blue] plot[smooth, tension=.7] coordinates {(-8,-3) (-3,-3) (0.8,-4.6) (7,-5) (14,-5) (15,-5)};
\draw [blue] plot[smooth, tension=.7] coordinates {(-2,-2) (0,-1) (3,0) (7,-1) (7,-3) (4,-4) (-1,-3) (-2,-2)};
\draw [blue] plot[smooth, tension=.7] coordinates {(15,0) (10,0) (9,-2) (10,-4) (14,-4) (15,-4)};
\end{scope}

\draw [thick, green]  plot[smooth, tension=.7] coordinates {(5.2,4.2) (0.8,3.4) (0.4,1.2) (1.8,-0.6) (3.2,-2) (4.8,-2.6) (5.4,-4.6) (5,-8.4)};
\draw [thick, green]  plot[smooth, tension=.7] coordinates {(5.2,4.2) (4.8,2.6) (5,1) (4.4,-1.2) (3.2,-2) (1.8,-2.6) (1.2,-4.2) (0.8,-6.4) (3,-7.6) (5,-8.4)};
\draw [thick, green]  plot[smooth, tension=.7] coordinates {(5.2,4.2) (10.6,3.2) (11.8,2) (11.4,-0.4) (11.2,-2.4) (11.6,-3.8) (12,-7.2) (7.4,-7.8) (5,-8.4)};
\draw [thick, green]  plot[smooth, tension=.7] coordinates {(3.2,-2) (5,-2) (8,-1.8) (10.2,-2.4) (11.2,-2.4)};

\node [green] at (3.2,-2) [circle] {$\bullet$};
\node [green]at (11.2,-2.4) [circle] {$\bullet$};
\node [green]at (5,-8.4) [circle] {$\bullet$};
\node [green]at (5.2,4.2) [circle] {$\bullet$};
\end{tikzpicture}\\
Gibt es einen Rundweg über alle Brücken?\\
$\curvearrowright$ Abstraktion als Graph:\\
\begin{tikzpicture}[scale=0.5]
\node at (0,4) [circle] {$\bullet$};
\node at (2,2) [circle] {$\bullet$};
\node at (0,0) [circle] {$\bullet$};
\node at (0,2) [circle] {$\bullet$};
\draw  plot[smooth, tension=1] coordinates {(2,2) (1,2.3) (0,2)};
\draw  plot[smooth, tension=1] coordinates {(2,2) (1,1.7) (0,2)};
\draw  plot[smooth, tension=1] coordinates {(0,2) (0.3,1) (0,0)};
\draw  plot[smooth, tension=1] coordinates {(0,4) (0.3,3) (0,2)};
\draw  plot[smooth, tension=1] coordinates {(0,4) (-0.3,3) (0,2)};
\draw  plot[smooth, tension=1] coordinates {(0,2) (-0.3,1) (0,0)};
\end{tikzpicture}\\
$\curvearrowright$ Es gibt keinen Rundweg ($\Rightarrow$ Euler)!
\paragraph{Definition:} Ein (ungerichteter) Graph ist ein Paar $G=(V,E)$, wobei
\begin{itemize}
\item $V$ die Menge der Knoten und
\item $E$ die Menge der Kanten ist, die aus ungeordneten Paaren $\{u,v\}$ von Knoten besteht (also ungerichtet).
\end{itemize}

\subparagraph{Bsp.:} \parskp
\begin{tikzpicture}[scale=0.3]
\draw  (-3.5,3.5) ellipse (1 and 1) node{1};
\draw  (4.5,1.5) ellipse (1 and 1) node{4};
\draw  (-0.5,-3) ellipse (1 and 1) node{3};
\draw  (-7,-1.5) ellipse (1 and 1) node{2};

\draw (-4.4,2.4) -- (-6.2,-0.4);
\draw (-5.8,-2) -- (-1.8,-2.8);
\draw (-3,2.4) -- (-1,-1.8);
\draw (-2.2,3.4) -- (3.4,2);
\draw (3.6,0.6) -- (0.4,-2.2);
\end{tikzpicture}\\
$(\{1,2,3,4\},\{\{1,2\},\{1,3\}, \{2,3\}, \{1,4\}, \{3,4\}\})$\\
Beispiel:\\
\begin{tikzpicture}[scale=0.3]
\draw  (0,0)node[circle] {$\bullet$}  rectangle (5,5);
\draw (5,0) node[circle] {$\bullet$} -- (6.5,1.5)node[circle] {$\bullet$} ;
\draw (5,5) node[circle] {$\bullet$} -- (6.5,6.5);
\draw (0,5) node[circle] {$\bullet$} -- (1.5,6.5);
\draw (1.5,6.5)node[circle] {$\bullet$}  -- (6.5,6.5);
\draw (6.5,6.5) node[circle] {$\bullet$} -- (6.5,1.5);
\draw [dashed] (0,0) -- (1.5,1.5)node[circle] {$\bullet$} ;
\draw [dashed] (1.5,1.5) -- (1.5,6.5);
\draw [dashed] (1.5,1.5) -- (6.5,1.5);
\end{tikzpicture}\\
Die Punkte und Kanten eines 3D-Objektes werden auf einen Graph abgebildet $\Rightarrow$\\
\begin{tikzpicture}[scale=0.2]
\draw  (0,0) rectangle (5,5);
\draw  (-2,-2) rectangle (7,7);
\draw (-2,7)node[circle] {$\bullet$} -- (0,5)node[circle] {$\bullet$};
\draw (7,-2)node[circle] {$\bullet$} -- (5,0)node[circle] {$\bullet$};
\draw (-2,-2)node[circle] {$\bullet$} -- (0,0)node[circle] {$\bullet$};
\draw (5,5)node[circle] {$\bullet$} -- (7,7)node[circle] {$\bullet$};
\end{tikzpicture}

\paragraph{Definition:} Ein Graph heißt \emph{vollständig}, wenn alle Knoten paarweise verbunden sind. \bigskip\\
Ein vollständiger Graph mit $n$ Knoten besitzt genau $\nok{n}{2}$ Kanten (jeder Knoten hat den Grad $n-1$).

\paragraph{Definition:} Ein Knoten $v$ hat den Grad $k$, wenn $v$ mit genau $k$ anderen Knoten verbunden ist. \\
Notation: $deg(v)=k$

\subparagraph{Satz:} Für jeden Graphen gilt $\sum_{v\in V}deg(v)=2|E|$ (Sprich: Die Summe der Grade aller Knoten ist die zweifache Kanten-Anzahl).

\subparagraph{Beweis:} Wenn wir jede Kante in der Mitte durchschneiden, ist jeder Knoten mit genau $deg(v)$ Hälften verbunden. Die Summe der Knotengrade ist dann die Anzahl der Kantenhälften, und diese ist $2|E|$.

\paragraph{Definition:} Ein \emph{Weg} ist eine Folge von Knoten $v_1, ... , v_k$ mit $\{v_l,v_{l+1}\}\in E$ für $l=1, ..., k-1$. Die Länge dieses Weges ist $k-1$. Ein Weg heißt Kreis, wenn $v_1=v_k$.\\
\begin{tikzpicture}[scale=0.3]
\draw  (0,0) ellipse (1 and 1) node{$v_1$};
\draw  (5,0) ellipse (1 and 1) node{$v_2$};
\draw  (13,0) ellipse (1 and 1) node{$v_k$};

\draw (1.2,0) -- (3.8,0);
\draw (6.2,0) -- (8.2,0);
\draw (9.8,0) -- (11.8,0);

\node at (9,0) {...};
\end{tikzpicture}\\
Beispiele:\\
\begin{tabular}{r l}
\begin{tikzpicture}[scale=0.3]
\draw  (0,0) ellipse (1 and 1) node{1};
\draw  (5,0) ellipse (1 and 1) node{2};
\draw  (2.5,-3.5) ellipse (1 and 1) node{3};
\draw  (0,-7.5) ellipse (1 and 1) node{4};
\draw  (5,-7.5) ellipse (1 and 1) node{5};

\draw (1.2,0) -- (3.8,0);
\draw (0.6,-1.2) -- (1.6,-2.6);
\draw (3.4,-2.6) -- (4.4,-1);
\draw (1.2,-7.6) -- (3.8,-7.6);
\draw [dashed, thick, green] (0.6,-6.4) -- (1.7,-4.5);
\draw [dashed]  (2.5,-3.5) ellipse (6 and 6.5);
\end{tikzpicture} & \mpb[0.6] ist ein Graph: $(\{1,2,3,4,5\},\{\{1,2\},\{1,3\},\{2,3\},\{4,5\}\}$\\
(der grüne Pfad würde ihn zusammenhängend machen). \mpe\\
\begin{tikzpicture}[scale=0.3]
\draw  (-0.5,0.5) ellipse (1 and 1) node{1};
\draw  (4.5,-0.5) ellipse (1 and 1) node{2};
\draw  (-1,-4.5) ellipse (1 and 1) node{3};
\draw  (4.5,-4) ellipse (1 and 1) node{4};

\draw [dashed]  (2,-2.5) ellipse (5 and 6);
\end{tikzpicture} &ist ein Graph: $(\{1,2,3,4\},\emptyset\}$\\
\end{tabular}\\
Ein Graph heißt \emph{zusammenhängend}, wenn es für alle Paare von Knoten $u,v$ einen Weg von $u$ nach $v$ gibt.\\
Ein \emph{Pfad} ist ein Weg, der keinen Knoten mehrfach enthält.

\section{Bäume}
\begin{tikzpicture}[scale=0.3]
\draw [green] (-2,6) node (v2) {} ellipse (0.5 and 0.5);
\draw  (0,6) node (v4) {} ellipse (0.5 and 0.5);
\draw [green] (2,6) node (v3) {} ellipse (0.5 and 0.5);
\draw  (0,9) node (v1) {} ellipse (0.5 and 0.5);
\draw [green] (-2,0) node (v6) {} ellipse (0.5 and 0.5);
\draw [green] (0,0) node (v7) {} ellipse (0.5 and 0.5);
\draw [green] (2,0) node (v8) {} ellipse (0.5 and 0.5);
\draw  (0,3) node (v5) {} ellipse (0.5 and 0.5);
\draw  (-3,12) node (v9) {} ellipse (0.5 and 0.5);
\draw [green] (-7.5,12.5) node (v11) {} ellipse (0.5 and 0.5);
\draw [green] (-4.5,15) node (v10) {} ellipse (0.5 and 0.5);
\draw [green] (3,13) node (v12) {} ellipse (0.5 and 0.5);
\draw (v1) -- (v2);
\draw (v1) -- (v3);
\draw (v1) -- (v4) -- (v5) -- (v6);
\draw (v5) -- (v7);
\draw (v5) -- (v8);
\draw (v1) -- (v9) -- (v10);
\draw (v9) -- (v11);
\draw (v1) -- (v12);
\node at (0,-1.5) {(ein Baum)};
\node [green] at (-6,8) {Blätter};
\end{tikzpicture}
\begin{tikzpicture}[scale=0.3]
\draw (-2,6) node (v2) {} ellipse (0.5 and 0.5);
\draw (2,6) node (v3) {} ellipse (0.5 and 0.5);
\draw  (0,9) node (v1) {} ellipse (0.5 and 0.5);
\draw (1,3) node (v6) {} ellipse (0.5 and 0.5);
\draw (3,3) node (v7) {} ellipse (0.5 and 0.5);
\draw  (2,6) node (v5) {} ellipse (0.5 and 0.5);
\draw  (0,0) node (v9) {} ellipse (0.5 and 0.5);
\draw  (2,0) node (v10) {} ellipse (0.5 and 0.5);
\draw  (-1,3) node (v8) {} ellipse (0.5 and 0.5);
\draw  (-3,3) node (v4) {} ellipse (0.5 and 0.5);
\draw (v1) -- (v2);
\draw (v1) -- (v3);
\draw (v5) -- (v6);
\draw (v5) -- (v7);
\node at (0,-1.5) {(kein Baum)};
\draw (v4) -- (v2) -- (v8);
\draw (v9) -- (v6) -- (v10);
\draw (v8);
\draw (v4) -- (v8);
\end{tikzpicture}
\begin{tikzpicture}[scale=0.3]
\draw (-2,6) node (v2) {} ellipse (0.5 and 0.5);
\draw (2,6) node (v3) {} ellipse (0.5 and 0.5);
\draw   node (v1) {} ellipse (0.5 and 0.5);
\draw (1,3) node (v6) {} ellipse (0.5 and 0.5);
\draw (3,3) node (v7) {} ellipse (0.5 and 0.5);
\draw  (2,6) node (v5) {} ellipse (0.5 and 0.5);
\draw  (0,0) node (v9) {} ellipse (0.5 and 0.5);
\draw  (2,0) node (v10) {} ellipse (0.5 and 0.5);
\draw  (-1,3) node (v8) {} ellipse (0.5 and 0.5);
\draw  (-3,3) node (v4) {} ellipse (0.5 and 0.5);

\draw (v5) -- (v6);
\draw (v5) -- (v7);
\node at (0,-1.5) {(kein Baum)};
\node at (0,-2.75) {$\rightarrow$ 2 Bäume};
\draw (v4) -- (v2) -- (v8);
\draw (v9) -- (v6) -- (v10);
\draw (v8);
\end{tikzpicture}
\paragraph{Definition:} Ein Baum ist ein zusammenhängender Graph der keine Kreise enthält. Ein Blatt ist ein Knoten $v$ mit $deg(v)\leq 1$ (dem Grad 1, also nur eine Kante hat).\\
Anmerkung: Auch ein Graph mit nur einem Knoten ist ein Baum - ein Baum der keine Blätter hat.

\paragraph{Satz:} Sei $B=(V,E)$ ein Baum. Dann gilt $|E|=|V|-1$.

\subparagraph{Beweis:} (Indkution)
\begin{itemize}
\item[IA:] $|V|=1$: Ein Baum mit nur einem Knoten enthält keine Kanten.
\item[IV:] $|E|=|V|-1$.
\item[IS:]$|V| \rightarrow |V| +1$: Sei $B$ ein Baum mit $|V|+1$ Knoten. $B$ besitzt ein Blatt (siehe Übung). Indem wir dieses Blatt zusammen mit der zugehörogien Kante entfernen, erhalten wir ein Baum $B'$ mit $|V|$ Kanten und nach Induktionsvoraussetzung $|V|-1$ Kanten. Damit besitzt $B$ $(|V|+1)-1$ Kanten.
\end{itemize}

\paragraph{Definiton:} Ein \emph{Wurzelbaum} ist ein Baum mit einem als Wurzel ausgezeichnetem Knoten.

\paragraph{Definition:} Ein \emph{binärer Wurzelbaum} ist ein Wurzelbaum, in dem jeder Knoten, der kein Blatt ist, genau zwei Nachfolger besitzt.\\
\begin{tikzpicture}[scale=0.3]
\draw (2,0) node (v2) {} ellipse (0.5 and 0.5);
\draw (4,9) node (v3) {} ellipse (0.5 and 0.5);
\draw (1,3) node (v6) {} ellipse (0.5 and 0.5);
\draw (3,3) node (v7) {} ellipse (0.5 and 0.5);
\draw  (2,6) node (v5) {} ellipse (0.5 and 0.5);
\draw  (0,0) node (v9) {} ellipse (0.5 and 0.5);
\draw  (2,0) node (v10) {} ellipse (0.5 and 0.5);
\draw  (3,-3) node (v8) {} ellipse (0.5 and 0.5);
\draw  (1,-3) node (v4) {} ellipse (0.5 and 0.5);
\draw (6,6) node (v11) {} ellipse (0.5 and 0.5);
\draw (5,3) node (v12) {} ellipse (0.5 and 0.5);
\draw (7,3) node (v13) {} ellipse (0.5 and 0.5);

\draw (v5) -- (v6);
\draw (v5) -- (v7);
\draw (v4) -- (v2) -- (v8);
\draw (v9) -- (v6) -- (v10);
\draw (v8);
\draw (v3) -- (v5);
\draw (v3) -- (v11) -- (v12);
\draw (v11) -- (v13);
\draw [red, dashed]  (1.5,-1.5) ellipse (3 and 3);
\draw [green]  (8,2) rectangle (0,10);
\draw [red]  (-2,10.5) rectangle (8.5,-5);
\node [right, green] at (8.5,7.5) {perfekt balanciert};
\node [right, red, align=left] at (8.5,0) {nicht balanciert,\\da eine Wurzel länger};
\end{tikzpicture}
\paragraph{Definition:}  (induktiv)\\
\begin{tikzpicture}[scale=0.3]
\draw  (1,6) node (v5) {} ellipse (0.5 and 0.5);
\draw (7,6) node (v11) {} ellipse (0.5 and 0.5);
\draw (4,8.5) node (v13) {} ellipse (0.5 and 0.5);

\draw (v5) -- (v13) -- (v11) ;
\draw [scale=2] (0.5,3) -- (-1,-1) -- (2,-1) -- cycle;
\draw [scale=2](3.5,3) -- (2.5,0) -- (4.5,0) --cycle;
\node at (1,0.5) {$W_1$};
\node at (7,1.5) {$W_2$};
\end{tikzpicture}
\begin{itemize}
\item Ein einzelner Knoten ist ein binärer Wurzelbaum
\item Wenn $W_1, W_2$ binäre Wurzelbäume sind, dann erhalten wir einen neuen Wurzelbaum, indem die Wurzeln von $W_1,W_2$ mit einer neuen Wurzel verbunden werden.
\end{itemize}
\paragraph{Satz:} Ein balancierter binärer Wurzelbaum mit Tiefe $d$ (d.h. alle Pfade von Wurzel zu einem Blatt haben die Länge $d$) besitzt genau $2^d$ Blätter.\\
\begin{tikzpicture}[scale=0.3]
\draw (1,3) node (v6) {} ellipse (0.5 and 0.5);
\draw (3,3) node (v7) {} ellipse (0.5 and 0.5);
\draw  (2,6) node (v5) {} ellipse (0.5 and 0.5);
\draw (6,6) node (v11) {} ellipse (0.5 and 0.5);
\draw (5,3) node (v12) {} ellipse (0.5 and 0.5);
\draw (7,3) node (v14) {} ellipse (0.5 and 0.5);
\draw (4,8.5) node (v13) {} ellipse (0.5 and 0.5);

\draw (v6) -- (v5) -- (v13) -- (v11) -- (v14);
\draw (v5) -- (v7);
\draw (v12) -- (v11);
\draw (8.5,8.5) -- (9,8.5) -- (9,3) -- (8.5,3);
\node at (9,6) [right] {Tiefe $d$};
\end{tikzpicture}
\subparagraph{Beweis:} (Indkution)
\begin{itemize}
\item[IA:] $d=0$: Ein binärer Wurzelbaum, der nur aus der Wurzel besteht, enthält $2^0=1$ Blätter.
\item[IV:] $|\text{Blätter}| =2^d$
\item[IS:] $d\rightarrow d+1$: Ein binärer Wurzelbaum der Tief $d+1$ enthält zwei binäre Wurzelbäume (laut vorhergehender Definition) der Tiefe $d$.\\
\begin{tikzpicture}[scale=0.3]
\draw  (1,6) node (v5) {} ellipse (0.5 and 0.5);
\draw (7,6) node (v11) {} ellipse (0.5 and 0.5);
\draw (4,8.5) node (v13) {} ellipse (0.5 and 0.5);

\draw (v5) -- (v13) -- (v11) ;
\draw [scale=2](0.5,3) -- (-0.5,0) -- (1.5,0) --cycle;
\draw [scale=2](3.5,3) -- (2.5,0) -- (4.5,0) --cycle;

\draw (10,6) -- (10.5,6) -- (10.5,0) -- (10,0);
\draw (13,8.5) -- (13.5,8.5) -- (13.5,0) -- (13,0);
\node at (10.5,3) [right] {$d$};
\node at (13.5,4) [right] {$d+1$};
\end{tikzpicture}\\
Diese enthalten nach Induktionsvoraussetzung jeweils $2^d$ Blätter. Folglich besitzt der binäre Wurzelbaum der Tiefe $d+1$ genau $2\cdot 2^d=2^{d+1}$ Blätter.
\end{itemize}

\section{Datenstrukturen zur Repräsentation}
Es gibt zwei Möglichkeiten, um Graphen darzustellen:
\paragraph{Adjazenzmatrix}
Für einen Graphen $G=(V,E)$ ist die Adjazenzmatrix eine $|V|\times |V|$-Matrix ($a_{uv}$) mit
$a_{uv}=\begin{cases}
1 \qquad \text{für }\{u,v\}\in E\\
0 \qquad \text{sonst}
\end{cases}$
\subparagraph{Bsp.:} \parskp
\begin{tikzpicture}[scale=0.5]
\draw  (1,6) node {3} ellipse (0.5 and 0.5);
\draw  (7,8.5) node {2} ellipse (0.5 and 0.5);
\draw (7,6) node {4} ellipse (0.5 and 0.5);
\draw (1,8.5) node {1} ellipse (0.5 and 0.5);

\draw (1.8,8.4) -- (6.4,8.4);
\draw (7,7.8) -- (7,6.6);
\draw (1,7.8) -- (1,6.6);
\draw (1.6,8) -- (6.4,6.2);
\end{tikzpicture} \qquad 
$\begin{pmatrix}
0 & 1 & 1 & 1\\
1 & 0 & 0 & 1\\
1 & 0 & 0 & 0\\
1 & 1 & 0 & 0
\end{pmatrix}$
\paragraph{Adjazenzliste} Ein Array hat den Nachteil, dass es nicht in der Länge geändert werden kann. Der Vorteil ist allerdings, dass auf Elemente des Arrays in kurzer Zeit zugegriffen werden kann.\\
Eine  Liste kann wachsen und schrumpfen. Jedes Glied einer Liste verweist auf das nächste. Der 
Nachteil ist, dass der Zugriff auf Elemente aus der Liste nicht so schnell und einfach ist.\smallskip\\
Die Adjazenzliste ist ein Array, das an jeder Position $v$ eine Liste der mit $v$ verbundenen Knoten enthält.
\subparagraph{Bsp.:} \parskp
\begin{tikzpicture}[scale=0.3]

\draw  (-2.5,1) rectangle (2,-2.5);
\draw  (-2.5,-2.5) rectangle (2,-6);
\draw  (-2.5,-6) rectangle (2,-9.5);
\draw  (-2.5,-9.5) rectangle (2,-13);

\draw  (4.5,0.5) rectangle (7.5,-2);
\draw  (9,0.5) rectangle (12,-2);
\draw  (13.5,0.5) rectangle (16.5,-2);

\draw  (4.5,-3) rectangle (7.5,-5.5);
\draw  (9,-3) rectangle (12,-5.5);

\draw  (4.5,-6.5) rectangle (7.5,-9);

\draw  (4.5,-10) rectangle (7.5,-12.5);
\draw  (9,-10) rectangle (12,-12.5);
\node at (6,-0.75) {2};
\node at (10.4,-0.75) {3};
\node at (15,-0.75) {4};
\node at (6,-4.25) {1};
\node at (10.4,-4.25) {4};
\node at (6,-7.75) {1};
\node at (6,-11.25) {1};
\node at (10.4,-11.25) {2};

\node at (-3.5,-0.75) {1};
\node at (-3.5,-4.25) {2};
\node at (-3.5,-7.75) {3};
\node at (-3.5,-11.25) {4};
\draw [-latex] (2,-0.75) -- (4.5,-.75);
\draw [-latex] (7.5,-0.75) -- (9,-0.75);
\draw [-latex] (12,-0.75) -- (13.5,-0.75);

\draw [-latex] (2,-4.25) -- (4.5,-4.25);
\draw [-latex] (7.5,-4.25) -- (9,-4.25);

\draw [-latex] (2,-7.75) -- (4.5,-7.75);

\draw [-latex] (2,-11.25) -- (4.5,-11.25);
\draw [-latex] (7.5,-11.25) -- (9,-11.25);
\end{tikzpicture}\\
Bäume, insbesondere Binärbäume, lassen sich noch einfacher darstellen: Jeder Knoten wird dargestellt durch eine Datenstruktur, die einen Verweis auf die Nachfolger enthält.
\subparagraph{Bsp.:}\parskp
\begin{tikzpicture}[scale=0.5]
\draw (-0.5,3) node (v6) {4} ellipse (0.5 and 0.5);
\draw (1.5,3) node (v7) {5} ellipse (0.5 and 0.5);
\draw  (0.5,6) node (v5) {3} ellipse (0.5 and 0.5);
\draw (4.5,6) node (v11) {2} ellipse (0.5 and 0.5);
\draw (2.5,8.5) node (v13) {1} ellipse (0.5 and 0.5);

\draw (v6) -- (v5) -- (v13) -- (v11);
\draw (v5) -- (v7);

\draw  (11,9) rectangle (12,8);
\draw  (12,9) rectangle (13,8);
\draw  (13,9) rectangle (14,8);
\draw  (9.5,6.5) rectangle (10.5,5.5);
\draw  (10.5,6.5) rectangle (11.5,5.5);
\draw  (11.5,6.5) rectangle (12.5,5.5);
\draw  (13.5,6.5) rectangle (14.5,5.5);
\draw  (14.5,6.5) rectangle (15.5,5.5);
\draw  (15.5,6.5) rectangle (16.5,5.5);
\draw  (8,3.5) rectangle (9,2.5);
\draw  (9,3.5) rectangle (10,2.5);
\draw  (10,3.5) rectangle (11,2.5);
\draw  (12,3.5) rectangle (13,2.5);
\draw  (13,3.5) rectangle (14,2.5);
\draw  (14,3.5) rectangle (15,2.5);

\draw[-latex] (12.5,8.5) -- (11,6.5);
\draw[-latex] (13.5,8.5) -- (15,6.5);
\draw[-latex] (11,6) -- (9.5,3.5);
\draw[-latex] (12,6) -- (13.5,3.5);
\draw[-latex] (9,8.5) -- (11,8.5);
\end{tikzpicture}\\
Dieser Graph:\\
\begin{tikzpicture}[scale=0.5]
\draw  (1,6) node {3} ellipse (0.5 and 0.5);
\draw  (7,8.5) node {2} ellipse (0.5 and 0.5);
\draw (7,6) node {4} ellipse (0.5 and 0.5);
\draw (1,8.5) node {1} ellipse (0.5 and 0.5);

\draw (1.8,8.4) -- (6.4,8.4);
\draw (7,7.8) -- (7,6.6);
\draw (1,7.8) -- (1,6.6);
\draw (1.6,8) -- (6.4,6.2);
\end{tikzpicture}\\
würde dann als $Array(List(2,3,4), List(1,4), List(1), List(1,2))$ dargestellt werden.


\section{Grundlegende Graphalgorithmen}

\subsection{Breitensuche}
\begin{tikzpicture}[scale=0.5]
\draw (4,9) node (v3) {1} ellipse (0.5 and 0.5);
\draw  (2,6) node (v5) {2} ellipse (0.5 and 0.5);
\draw (1,3) node (v6) {4} ellipse (0.5 and 0.5);
\draw (3,3) node (v7) {5} ellipse (0.5 and 0.5);
\draw (6,6) node (v11) {3} ellipse (0.5 and 0.5);
\draw (5,3) node (v12) {6} ellipse (0.5 and 0.5);
\draw (7,3) node (v13) {7} ellipse (0.5 and 0.5);
\draw  (0,0) node (v9) {8} ellipse (0.5 and 0.5);

\draw (v5) -- (v6);
\draw (v5) -- (v7);
\draw (v9) -- (v6);
\draw (v3) -- (v5);
\draw (v3) -- (v11) -- (v12);
\draw (v11) -- (v13);
\draw (v5) -- (v12);
\draw (v5) -- (v11);

\node [right, purple] at (8.75,8.5) {1. Schritt};
\node [right, green] at (8.75,7.5) {2. Schritt};
\node [right, orange] at (8.75,6.5) {3. Schritt};
\node [right, align=left, blue] at (8.75,4.5) {4. Schritt\\ $\curvearrowright$ 8 gefunden};
\draw [dashed, purple]  (4,9) ellipse (0.8 and 0.8);
\draw [dashed, green]  (2,6) node (v1) {} ellipse (0.8 and 0.8);
\draw [dashed, orange]  (6,6) node (v2) {} ellipse (0.8 and 0.8);
\draw [dashed, blue]  (1,3) ellipse (0.8 and 0.8);
\draw [-latex, purple] (3.25,8.25) -- (2.75,7.5);
\draw [-latex, purple] (4.75,8.25) -- (5.25,7.5);
\draw [-latex, green] (2.5,5.25) -- (2.75,4.5);
\draw [-latex, green] (1.5,5.25) -- (1.25,4.5);
\draw [-latex, green] (2.75,5.5) -- (3.25,5);
\draw [-latex, green] (3,6.25) -- (3.75,6.25);
\draw [-latex, orange] (6.5,5.25) -- (6.75,4.5);
\draw [-latex, blue] (0.5,2.25) -- (0.25,1.5);
\end{tikzpicture}\\
Mit der Breitensuche kann ein Graph systematisch durchsucht werden. Von einem Startknoten ausgehend, besucht die Breitensuche zuerst die dem Startknoten benachbarten Knoten. Anschließend werden die noch nicht besuchten Nachbarn dieser Knoten besucht, usw., bis das Ziel gefunden wurde oder alle Knoten besucht wurden.

\begin{lstlisting}[language=C]
boolean bfs (node start, node goal){
	for (v $\in$ V){
		discovered[v] = false
	}
	queue.enqueue(start)
	discovered[start] = true
	while ($\neg$ queue.isEmpty){
		u = queue.dequeue()
		if (u = goal){
			return true
		}
		else{
			for (v $\in$ adj[u]){
				if ($\neg$discovered[v]){
					queue.enqueue(v)
					discovered[v] = true
				}
			}
		}
	}
	return false
}
\end{lstlisting}
\subparagraph{Bsp.:} \parskp
\begin{tikzpicture}[scale=0.5]
\draw (5,11) node (v0) {0} ellipse (0.5 and 0.5);
\draw (4,8.5) node (v1) {1} ellipse (0.5 and 0.5);
\draw  (6,8.5) node (v2) {2} ellipse (0.5 and 0.5);
\draw (5,6) node (v4) {4} ellipse (0.5 and 0.5);
\draw (5,3.5) node (v5) {5} ellipse (0.5 and 0.5);
\draw (3,6) node (v3) {3} ellipse (0.5 and 0.5);
\draw (v0) -- (v1);
\draw (v0) -- (v2);
\draw (v2) -- (v4);
\draw (v1) -- (v4);
\draw (v1) -- (v3);
\draw (v4) -- (v5);
\draw (v1) -- (v2);

\begin{scope}[scale=0.75, shift={(0.2,2)}]
\draw  (11,10.5) rectangle (13,8.5);
\node at (12,9.5) {t};
\node[above] at (12,10.5) {0};
\draw  (13,10.5) rectangle (15,8.5);
\node at (14,9.5) {f};
\node[above] at (14,10.5) {1};
\draw  (15,10.5) rectangle (17,8.5);
\node at (16,9.5) {f};
\node[above] at (16,10.5) {2};
\draw  (17,10.5) rectangle (19,8.5);
\node at (18,9.5) {f};
\node[above] at (18,10.5) {3};
\draw  (19,10.5) rectangle (21,8.5);
\node at (20,9.5) {f};
\node[above] at (20,10.5) {4};
\draw  (21,10.5) rectangle (23,8.5);
\node at (22,9.5) {f};
\node[above] at (22,10.5) {5};
\node [right] at (11,12.5) {discovered:};

\draw [purple] (11,3.5) rectangle (13,1.5);
\node at (12,2.5) {1};
\draw  [purple] (13,3.5) rectangle (15,1.5);
\node at (14,2.5) {2};
\draw [green] (15,3.5) rectangle (17,1.5);
\node at (16,2.5) {3};
\draw [green] (17,3.5) rectangle (19,1.5);
\node at (18,2.5) {4};
\draw [orange] (19,3.5) rectangle (21,1.5);
\node at (20,2.5) {5};
\node [right] at (11,4.5) {queue:};
\node [right, purple] at (25.5,12) {1.};
\node [right, green] at (25.5,10) {2.};
\node [right, orange] at (25.5,8) {3.};
\node [right, cyan] at (25.5,6) {4.};
\node [right, red] at (25.5,4) {5.};
\node [right, blue] at (25.5,2) {6. $\checkmark$};

\node [purple] at (14.6,9) {t};
\node [purple] at (16.6,9) {t};
\node [purple, opacity=0.6] at (14,9.5) {X};
\node [purple, opacity=0.6] at (16,9.5) {X};
\node [green] at (18.6,9) {t};
\node [green] at (20.6,9) {t};
\node [green, opacity=0.6] at (18,9.5) {X};
\node [green, opacity=0.6] at (20,9.5) {X};
\node [orange] at (22.4,9) {t};
\node [orange, opacity=0.6] at (22,9.5) {X};

\node [green, opacity=0.8] at (12,2.5) {X};
\node [orange, opacity=0.8] at (14,2.5) {X};
\node [cyan, opacity=0.8] at (16,2.5) {X};
\node [red, opacity=0.8] at (18,2.5) {X};
\node [blue, opacity=0.8] at (20,2.5) {X};
\end{scope}
\end{tikzpicture}

\subsection{Tiefensuche}
Die Tiefensuche lässt sich implementieren…
\begin{enumerate}
\item wie die Breitensuche, aber mit einem Stack anstelle einer Warteschlange
\item rekursiv.
\end{enumerate}
Eine Warteschlange ist eine FIFO (first in, first out) Datenstruktur, die sich implementieren lässt mit einer verketteten Liste, die einen Zeiger auf das letzte Element besitzt.\\
\begin{tikzpicture}[scale=0.2]
\draw  (0.5,0) rectangle (4.5,3);
\draw [-latex] (-3.5,1.5) -- (0.5,1.5);
\draw  (8.5,0) rectangle (12.5,3);
\draw [-latex] (4.5,1.5) -- (8.5,1.5);
\draw  (16.5,0) rectangle (20.5,3);
\draw [-latex] (12.5,1.5) -- (16.5,1.5);
\draw  (24.5,0) rectangle (28.5,3);
\draw [-latex] (20.5,1.5) -- (24.5,1.5);
\draw [thick, -latex] (26.5,-2.5) -- (26.5,0);
\node [above] at (1,3) {vorne};
\node [above] at (28,3) {hinten};
\end{tikzpicture}\\
Ein Stack (auch „Keller“) ist eine LIFO (last in, first out) Datenstruktur, die sich durch eine verkettete Liste implementieren lässt.\\
\begin{tikzpicture}[scale=0.2]
\draw  (0.5,-0.5) rectangle (4.5,2.5) node[above right] {oben};
\draw [-latex] (2.5,6) -- (2.5,2.5);
\draw  (0.5,-7) rectangle (4.5,-4);
\draw [-latex] (2.5,-0.5) -- (2.5,-4);
\draw  (0.5,-13.5) rectangle (4.5,-10.5);
\draw [-latex] (2.5,-7) -- (2.5,-10.5);
\draw  (0.5,-20) rectangle (4.5,-17);
\draw [-latex] (2.5,-13.5) -- (2.5,-17);


\draw [thick, -latex] (7,1) -- (4.5,1);

\end{tikzpicture}
\begin{lstlisting}[language=C]
boolean tfs (node start, node goal){
	for (v $\in$ V){
		discovered[v] = false
	}
	stack.push(start)
	discovered[start] = true
	while ($\neg$stack.isEmpty){
		u = stack.pop()
		if (u = goal){
			return true
		}
		else{
			for (v $\in$ adj[u]){
				if ($\neg$discovered[v]){
					stack.push(v)
					discovered[v] = true
				}
			}
		}
	}
	return false
}
\end{lstlisting}

\subparagraph{Bsp.:} \parskp
\begin{tikzpicture}[scale=0.5]
\draw (5,11) node (v0) {0} ellipse (0.5 and 0.5);
\draw (4,8.5) node (v1) {1} ellipse (0.5 and 0.5);
\draw  (6,8.5) node (v2) {2} ellipse (0.5 and 0.5);
\draw (5,6) node (v4) {4} ellipse (0.5 and 0.5);
\draw (5,3.5) node (v5) {5} ellipse (0.5 and 0.5);
\draw (3,6) node (v3) {3} ellipse (0.5 and 0.5);
\draw (v0) -- (v1);
\draw (v0) -- (v2);
\draw (v2) -- (v4);
\draw (v1) -- (v4);
\draw (v1) -- (v3);
\draw (v4) -- (v5);
\draw (v1) -- (v2);

\node at (9.5,11) {Stack:};

\draw  (8.5,5) rectangle (10.5,6.5);
\draw  (10.5,6.5) rectangle (8.5,8);
\draw (8.5,9.5) -- (8.5,8);
\draw (10.5,9.5) -- (10.5,8);
\draw  (12,5) rectangle (14,6.5);
\draw  (14,6.5) rectangle (12,8);
\draw (12,9.5) -- (12,8);
\draw (14,9.5) -- (14,8);
\draw  (15.5,5) rectangle (17.5,6.5);
\draw  (17.5,6.5) rectangle (15.5,8);
\draw (15.5,9.5) -- (15.5,8);
\draw (17.5,9.5) -- (17.5,8);
\draw  (19,5) rectangle (21,6.5);
\draw  (21,6.5) rectangle (19,8);
\draw (19,9.5) -- (19,8);
\draw (21,9.5) -- (21,8);
\draw  (22.5,5) rectangle (24.5,6.5);
\draw  (24.5,6.5) rectangle (22.5,8);
\draw (22.5,9.5) -- (22.5,8);
\draw (24.5,9.5) -- (24.5,8);
\node at (9.5,5.75) {2};
\node at (9.5,7.25) {1};
\node at (13,5.75) {2};
\node at (13,7.25) {4};
\node at (13,8.75) {3};
\node at (16.5,5.75) {2};
\node at (16.5,7.25) {4};
\node at (20,7.25) {5};
\node at (20,5.75) {2};
\node at (23.5,5.75) {2};
\draw [dashed, green] (20,7.25) ellipse (0.5 and 0.5);
\end{tikzpicture}\\
Problem bei Breiten und Tiefensuche: Man braucht das Feld „discovered“. Das kann bei großer Anzahl von Knoten ein Problem sein $\rightarrow$ Speicheraufwändig

\subsection{Topologisches Sortieren}
\begin{tikzpicture}[scale=0.5]
\draw  (-6,4) ellipse (0.5 and 0.5);
\draw  (-2,6.5) ellipse (0.5 and 0.5);
\draw  (-2,2) ellipse (0.5 and 0.5);
\draw  (2,6.5) ellipse (0.5 and 0.5);
\draw  (6,6.5) ellipse (0.5 and 0.5);

\draw [-latex] (-5.4,4.4) -- (-2.6,6.2);
\draw [-latex] (-5.4,3.4) -- (-2.6,2.2);
\draw [-latex] (-1.4,6.4) -- (1.4,6.4);
\draw [-latex] (2.6,6.4) -- (5.4,6.4);
\draw [-latex] (-1.4,2.4) -- (1.8,5.8);
\draw [-latex] (-1.2,2.2) -- (5.4,6);
\node [above, purple] at (-6,4.6) {1};
\node [above, purple] at (-2,7) {2};
\node [above, purple] at (2,7) {4};
\node [above, purple] at (6,7) {5};
\node [above, purple] at (-2,2.6) {3};
\node [purple] at (5.4,2) {topologische Sortierung};
\end{tikzpicture}
\paragraph{Def.:} Ein gerichteter Graph ist ein Paar $(V,E)$ mit $V\not = 0$ und $E \subseteq V \times V$. Die Begriffe Weg, Pfad, Kreis lassen sich entsprechend definieren.\\
\begin{tikzpicture}[scale=0.5]
\draw  (-9,4) node{u} ellipse (0.5 and 0.5);
\draw  (-5,4) node{v} ellipse (0.5 and 0.5);
\draw  (1,-1.5) node{u} ellipse (0.5 and 0.5);
\draw  (2.5,4) node{v} ellipse (0.5 and 0.5);
\draw  (-1.5,4) node{u} ellipse (0.5 and 0.5);

\node at (-7,5.5) {ungerichtet:};
\node at (0.5,5.5) {gerichtet:};
\draw (-8.4,4) -- (-5.6,4);
\draw [-latex] (-0.8,4) -- (1.8,4);

\draw [-latex]  plot[smooth, tension=.7] coordinates {(1.1,-0.8) (1.1,-0.2) (1.7,-0.2) (1.9,-0.6) (1.5,-1)};
\node at (1,0.5) {auch möglich:};
\node at (-7,2.5) {$\{u,v\}$};
\node at (0.5,2.5) {$(u,v)\in E$};
\node at (2.5,-2.5) {$(u,u)$};
\end{tikzpicture}
\paragraph{Def.:} Sei $G=(V,E)$ ein gerichteter Graph. Eine \emph{topologische Sortierung} von G ist eine Abbildung $t: V\rightarrow \mathbb{N} $ mit $ (u,v) \in E \Rightarrow t(u) < t(v)$\\
\begin{tikzpicture}[scale=0.5]
\draw  (-6,4) node{1} ellipse (0.5 and 0.5);
\draw  (-2,6.5) node{2} ellipse (0.5 and 0.5);
\draw  (-2,2) node {3} ellipse (0.5 and 0.5);
\draw  (2,6.5) node{4} ellipse (0.5 and 0.5);

\draw [-latex] (-5.4,4.4) -- (-2.6,6.2);
\draw [latex-] (-5.4,3.4) -- (-2.6,2.2);
\draw [-latex] (-1.4,6.4) -- (1.4,6.4);
\draw [-latex] (-2,2.6) -- (-2,5.8);
\node [above, purple] at (-6,4.6) {2};
\node [above, purple] at (-2,7) {3};
\node [above, purple] at (2,7) {4};
\node [right, purple] at (-1.4,2) {1};
\node [purple] at (4,4.2) {topologische Sortierung};
\end{tikzpicture}\\
$V= \{1,2,3,4\}\\
E=\{(1,2), (2,4), (3,1), (3,2)\}
$\\
\begin{tabular}{c | c c c c}
v & 1 & 2 & 3 & 4 \\
\hline
t & 2 & 3 & 1 & 4
\end{tabular}\medskip\\
Für einen Kreis oder einen Graph mit einer Schlinge existieren keine topologische Sortierungen.\medskip\\
Eine topologische Sortierung kann durch eine Tiefensuche bestimmt werden.\\
Alogrithmus TopSort:
\begin{lstlisting}[language=C]
for v $\in$ V
	markiere v mit weiß
for v $\in$ V
	tiefensuche(v)
	
tiefensuche(v){
	v grau: Fehler, Kreis vorhanden
	v weiß: markieren v mit grau
		for( u $|$ (v,u) $\in$ E )
			tiefensuche (v)
		markiere v mit schwarz und füge v an den Kopf einer Liste an
}
\end{lstlisting}
\begin{tikzpicture}[scale=0.3]
\pattern [pattern=north west lines,pattern color=ittenb] (0,0)ellipse (1 and 1);
\draw [thick,ittenr] (0,0)ellipse (1 and 1);
\draw (0,0) node {0};
\pattern [pattern=north west lines,pattern color=ittenp] (8,0)ellipse (1 and 1);
\draw [thick,ittenr] (8,0) ellipse (1 and 1);
\draw (8,0) node {1};
\pattern [pattern=north west lines,pattern color=itteng] (16,0)ellipse (1 and 1);
\draw [thick,ittenr] (16,0) ellipse (1 and 1);
\draw (16,0) node {2};
\draw (8,-8) node{3} ellipse (1 and 1);
\draw (4,-16) node{4} ellipse (1 and 1);
\draw (12,-16) node{5} ellipse (1 and 1);
\draw [-latex] (1.2,0) -- (6.8,0);
\draw [-latex] (9.2,0) -- (14.8,0);
\draw [-latex] (8,-6.8) -- (8,-1.2);
\draw [-latex] (11.4,-15) -- (8.6,-9);
\draw [-latex] (7.4,-9) -- (4.8,-15);
\draw [-latex] (12.2,-14.8) -- (8.8,-1);

\node at (14,-20) {2};
\node at (12,-20) {1};
\node at (10,-20) {0};
\draw [-latex, itteng] (14,-18) -- (14,-19);
\draw [-latex, ittenp] (12,-18) -- (12,-19);
\draw [-latex, ittenb] (10,-18) -- (10,-19);

\node [right, ittenr] at (22,-12) {1. grau};
\node [right, itteng] at (22,-13.5) {2. schwarz};
\node [right, ittenp] at (22,-15) {3. schwarz (durch rekursion)};
\node [right, ittenb] at (22,-16.5) {4. schwarz (durch rekursion)};
\end{tikzpicture}\\
\begin{tikzpicture}[scale=0.3]
\pattern [pattern=north west lines,pattern color=lightgray] (20,0)ellipse (1 and 1);
\draw  (20,0) node{0} ellipse (1 and 1);
\pattern [pattern=north west lines,pattern color=lightgray] (28,0)ellipse (1 and 1);
\draw  (28,0) node{1} ellipse (1 and 1);
\pattern [pattern=north west lines,pattern color=lightgray] (36,0)ellipse (1 and 1);
\draw  (36,0) node{2} ellipse (1 and 1);
\pattern [pattern=north west lines,pattern color=ittenp] (28,-8)ellipse (1 and 1);
\draw [thick,ittenr] (28,-8)ellipse (1 and 1);
\draw (28,-8) node {3};
\pattern [pattern=north west lines,pattern color=ittenb] (24,-16)ellipse (1 and 1);
\draw [thick,ittenr] (24,-16)ellipse (1 and 1);
\draw (24,-16) node {4};
\draw  (32,-16) node{5} ellipse (1 and 1);
\draw [-latex] (21.2,0) -- (26.8,0);
\draw [-latex] (29.2,0) -- (34.8,0);
\draw [-latex] (28,-6.8) -- (28,-1.2);
\draw [-latex] (31.4,-15) -- (28.6,-9);
\draw [-latex] (27.4,-9) -- (24.8,-15);
\draw [-latex] (32.2,-14.8) -- (28.8,-1);

\node at (34,-20) {2};
\node at (32,-20) {1};
\node at (30,-20) {0};
\node at (28,-20) {4};
\node at (26,-20) {3};
\draw [-latex, itteng] (28,-18) -- (28,-19);
\draw [-latex, ittenp] (26,-18) -- (26,-19);
\end{tikzpicture}\\
\begin{tikzpicture}[scale=0.3]
\pattern [pattern=north west lines,pattern color=lightgray] (40,0)ellipse (1 and 1);
\draw  (40,0) node{0} ellipse (1 and 1);
\pattern [pattern=north west lines,pattern color=lightgray] (48,0)ellipse (1 and 1);
\draw  (48,0) node{1} ellipse (1 and 1);
\pattern [pattern=north west lines,pattern color=lightgray] (56,0)ellipse (1 and 1);
\draw  (56,0) node{2} ellipse (1 and 1);
\pattern [pattern=north west lines,pattern color=lightgray] (48,-8)ellipse (1 and 1);
\draw  (48,-8) node{3} ellipse (1 and 1);
\pattern [pattern=north west lines,pattern color=lightgray] (44,-16)ellipse (1 and 1);
\draw  (44,-16) node{4} ellipse (1 and 1);
\pattern [pattern=north west lines,pattern color=ittenb] (52,-16)ellipse (1 and 1);
\draw [thick,ittenr] (52,-16)ellipse (1 and 1);
\draw (52,-16) node {5};
\draw [-latex] (41.2,0) -- (46.8,0);
\draw [-latex] (49.2,0) -- (54.8,0);
\draw [-latex] (48,-6.8) -- (48,-1.2);
\draw [-latex] (51.4,-15) -- (48.6,-9);
\draw [-latex] (47.4,-9) -- (44.8,-15);
\draw [-latex] (52.2,-14.8) -- (48.8,-1);

\node at (54,-20) {2};
\node at (52,-20) {1};
\node at (50,-20) {0};
\node at (48,-20) {4};
\node at (46,-20) {3};
\node at (44,-20) {5};
\draw [-latex, itteng] (44,-18) -- (44,-19);
\end{tikzpicture}\\
Endgültige topologische Sortierung: \begin{tabular}{c c c c c c}
5 & 3 & 4 & 0 & 1 & 2 \\
$\downarrow$ &$\downarrow$ &$\downarrow$ &$\downarrow$ &$\downarrow$ &$\downarrow$\\
\color{ittenb}0 & \color{ittenb}1 & \color{ittenb}2 & \color{ittenb}3 & \color{ittenb}4 & \color{ittenb}5
\end{tabular}\\
Laufzeit dieser topologischen Suche: $O(|V|+|E|)$\\
Sowohl die Tiefensuche als auch die Breitesuche besitzen eine Laufzeit in $O(|V|+|E|)$.

\subsection{Suche}
\paragraph{Lineare Suche} Laufzeit: $O(n)$
\paragraph{Binäre Suche} Voraussetzung: Sortiertes Array\\
\begin{tikzpicture}[scale=0.3]
\draw  (0,0) rectangle (4,4);
\draw  (4,0) rectangle (8,4);
\draw  (8,0) rectangle (12,4);
\draw  (12,0) rectangle (16,4);
\draw  (16,0) rectangle (20,4);
\draw  (20,0) rectangle (24,4);

\node at (2,2) {1};
\node at (6,2) {3};
\node at (10,2) {5};
\node at (14,2) {6};
\node at (18,2) {10};
\node at (22,2) {17};
\node [above] at (2,4) {0};
\node [above] at (6,4) {1};
\node [above] at (10,4) {2};
\node [above] at (14,4) {3};
\node [above] at (18,4) {4};
\node [above] at (22,4) {5};
\node[right] [orange, align=left] at (26,4) {Suche nach 4:\\in der Mitte starten (abrunden)};
\draw [orange, thick] (8.4,-0.4) rectangle (11.6,4.2);
\node [below, align=center] at (10,-0.4) {$=4?$\\$<4?$\\\color{orange}$>4?$\\ $>4$, also weiter links gucken};
\end{tikzpicture}\\
\begin{tikzpicture}[scale=0.3]
\draw  (1,-14) rectangle (5,-10);
\draw  (5,-14) rectangle (9,-10);

\node at (3,-12) {1};
\node at (7,-12) {3};
\node [above] at (3,-10) {0};
\node [above] at (7,-10) {1};

\draw [orange, thick] (1.4,-14.4) rectangle (4.6,-9.8);
\node [below, align=center] at (3,-14.4) {$=4?$\\\color{orange}$<4?$\\$>4?$\\ $>4$, also weiter rechts gucken};
\end{tikzpicture}\\
\begin{tikzpicture}[scale=0.3]
\draw  (5,-28) rectangle (9,-24);

\node at (7,-26) {3};
\node [above] at (7,-24) {1};

\draw [orange, thick] (5.4,-28.4) rectangle (8.6,-23.8);
\node [below, align=center] at (7,-28.4) {$=4?$\\$<4?$\\$>4?$\\ nicht gefunden!};
\end{tikzpicture}
\subparagraph{Vorgehen:} 
Es wird die Mitte des Arrays bestimmt (Länge/2 [abgerundet]) und der gesuchten Wert mit dem Wert an dieser Stelle verglichen. Dabei gibt es drei Möglichkeiten:
\begin{itemize}
\item Gleichheit: Wert gefunden.
\item Gesuchter Wert kleiner als Wert an der Stelle im Array: Auf gleiche Weise weitersuchen in der linken Hälfte (ausschließlich des bereits betrachteten Elements).
\item Gesuchter Wert größer: Auf gleiche Weise weitersuchen in der rechten Hälfte
\end{itemize}
Algorithmus wird beendet, wenn der Wert gefunden wurde oder die zu durchsuchende Arrayhälfte keine Elemente mehr enthält.
\subparagraph{Laufzeit:}\parskp
\begin{tikzpicture}[scale=0.4]

\draw  (-14.5,13) ellipse (0.5 and 0.5);
\draw  (-12.5,13) ellipse (0.5 and 0.5);
\draw  (-10.5,13) ellipse (0.5 and 0.5);
\draw  (-8.5,13) ellipse (0.5 and 0.5);
\draw  (-6.5,13) ellipse (0.5 and 0.5);
\draw  (-4.5,13) ellipse (0.5 and 0.5);
\draw  (-2.5,13) ellipse (0.5 and 0.5);
\draw  (-0.5,13) ellipse (0.5 and 0.5);
\draw  (1.5,13) ellipse (0.5 and 0.5);
\draw  (3.5,13) ellipse (0.5 and 0.5);
\draw  (5.5,13) ellipse (0.5 and 0.5);
\draw  (7.5,13) ellipse (0.5 and 0.5);
\draw  (-12.5,16) ellipse (0.5 and 0.5);
\draw  (-6.5,16) ellipse (0.5 and 0.5);
\draw  (-9.5,16) ellipse (0.5 and 0.5);
\draw  (-0.5,16) ellipse (0.5 and 0.5);
\draw  (5.5,16) ellipse (0.5 and 0.5);
\draw  (2.5,16) ellipse (0.5 and 0.5);
\draw  (-9.5,19) ellipse (0.5 and 0.5);
\draw  (2.5,19) ellipse (0.5 and 0.5);
\draw  (-3.5,19) ellipse (0.5 and 0.5);
\draw  (-3.5,22) ellipse (0.5 and 0.5);
\draw (-4,21.5) -- (-8.75,19.25);
\draw (-3.5,21.25) -- (-3.5,19.75);
\draw (-3,21.5) -- (1.75,19.25);
\draw (2.5,18.25) -- (2.5,16.75);
\draw (2,18.5) -- (0,16.5);
\draw (3,18.5) -- (5,16.5);
\draw (-9.5,18.25) -- (-9.5,16.75);
\draw (-9,18.25) -- (-7,16.5);
\draw (-10,18.25) -- (-12,16.5);
\draw (-13,15.5) -- (-14,13.75);
\draw (-12.5,15.25) -- (-12.5,13.75);
\draw (-12,15.5) -- (-11,13.75);
\draw (-7,15.5) -- (-8,13.75);
\draw (-6.5,15.25) -- (-6.5,13.75);
\draw (-6,15.5) -- (-5,13.75);
\draw (-1,15.5) -- (-2,13.75);
\draw (-0.5,15.25) -- (-0.5,13.75);
\draw (0,15.5) -- (1,13.75);
\draw (5,15.5) -- (4,13.75);
\draw (5.5,15.25) -- (5.5,13.75);
\draw (6,15.5) -- (7,13.75);

\node [above left] at (-6.25,20.5) {$<$};
\node [left] at (-3.5,20.5) {$=$};
\node [above right] at (-0.75,20.5) {$>$};

\node [above left] at (-11.25,17.25) {$<$};
\node [left] at (-9.5,17.25) {$=$};
\node [above right] at (-7.75,17.25) {$>$};
\node [above left] at (0.75,17.25) {$<$};
\node [left] at (2.5,17.25) {$=$};
\node [above right] at (4.25,17.25) {$>$};

\node [above left] at (-13.5,14.5) {$<$};
\node [left] at (-12.25,14.25) {$=$};
\node [above right] at (-11.5,14.5) {$>$};
\node [above left] at (-7.5,14.5) {$<$};
\node [left] at (-6.25,14.25) {$=$};
\node [above right] at (-5.5,14.5) {$>$};
\node [above left] at (-1.5,14.5) {$<$};
\node [left] at (-0.25,14.25) {$=$};
\node [above right] at (0.5,14.5) {$>$};
\node [above left] at (4.5,14.5) {$<$};
\node [left] at (5.75,14.25) {$=$};
\node [above right] at (6.5,14.5) {$>$};

\end{tikzpicture}\\
Zur Analyse der Laufzeit ändern wir den Algorithmus so, dass nur Vergleiche $\leq$ und $>$ vorgenommen werden. Ferner sei die Länge des Arrays eine Zweierpotenz und das gesuchte Element nicht vorhanden (worst-case).\\
\begin{tikzpicture}[scale=0.4]
\draw  (-12.5,16) ellipse (0.5 and 0.5);
\draw  (-6.5,16) ellipse (0.5 and 0.5);
\draw  (-0.5,16) ellipse (0.5 and 0.5);
\draw  (5.5,16) ellipse (0.5 and 0.5);
\draw  (-9.5,19) ellipse (0.5 and 0.5);
\draw  (2.5,19) ellipse (0.5 and 0.5);
\draw  (-3.5,22) ellipse (0.5 and 0.5);
\draw (-4,21.5) -- (-8.75,19.25);
\draw (-3,21.5) -- (1.75,19.25);
\draw (2,18.5) -- (0,16.5);
\draw (3,18.5) -- (5,16.5);
\draw (-9,18.25) -- (-7,16.5);
\draw (-10,18.25) -- (-12,16.5);

\node [above left] at (-6.25,20.5) {$\leq$};
\node [above right] at (-0.75,20.5) {$>$};
\node [above left] at (-11.25,17.25) {$\leq$};
\node [above right] at (-7.75,17.25) {$>$};
\node [above left] at (0.75,17.25) {$\leq$};
\node [above right] at (4.25,17.25) {$>$};
\end{tikzpicture}\\
In diesem Fall lässt sich das Verhalten des Algorithmus als vollständiger binärer Wurzelbaum darstellen.\\
\begin{tikzpicture}[scale=0.4]
\draw  (-14.5,13) ellipse (0.5 and 0.5);
\draw  (-10.5,13) ellipse (0.5 and 0.5);
\draw  (-8.5,13) ellipse (0.5 and 0.5);
\draw  (-4.5,13) ellipse (0.5 and 0.5);
\draw  (-2.5,13) ellipse (0.5 and 0.5);
\draw  (1.5,13) ellipse (0.5 and 0.5);
\draw  (3.5,13) ellipse (0.5 and 0.5);
\draw  (7.5,13) ellipse (0.5 and 0.5);
\draw  (-12.5,16) ellipse (0.5 and 0.5);
\draw  (-6.5,16) ellipse (0.5 and 0.5);
\draw  (-0.5,16) ellipse (0.5 and 0.5);
\draw  (5.5,16) ellipse (0.5 and 0.5);
\draw  (-9.5,19) ellipse (0.5 and 0.5);
\draw  (2.5,19) ellipse (0.5 and 0.5);
\draw  (-3.5,22) ellipse (0.5 and 0.5);
\draw (-4,21.5) -- (-8.75,19.25);
\draw (-3,21.5) -- (1.75,19.25);
\draw (2,18.5) -- (0,16.5);
\draw (3,18.5) -- (5,16.5);
\draw (-9,18.25) -- (-7,16.5);
\draw (-10,18.25) -- (-12,16.5);
\draw (-13,15.5) -- (-14,13.75);
\draw (-12,15.5) -- (-11,13.75);
\draw (-7,15.5) -- (-8,13.75);
\draw (-6,15.5) -- (-5,13.75);
\draw (-1,15.5) -- (-2,13.75);
\draw (0,15.5) -- (1,13.75);
\draw (5,15.5) -- (4,13.75);
\draw (6,15.5) -- (7,13.75);

\node [above left] at (-6.25,20.5) {$\leq$};
\node [above right] at (-0.75,20.5) {$>$};
\node [above left] at (-11.25,17.25) {$\leq$};
\node [above right] at (-7.75,17.25) {$>$};
\node [above left] at (0.75,17.25) {$\leq$};
\node [above right] at (4.25,17.25) {$>$};
\end{tikzpicture}\\
Wenn $n=2^k$ die Länge des Arrays ist, dann besitzt dieser Wurzelbaum genau $2^k$ Blätter (die Vergleichen in einen 1-elementigen Array entsprechen). Dieser Binärbaum besitzt daher die Tiefe $k=log_2(n)$. Die Laufzeit der binären Suche liegt daher in $O(log\;n)$ (gilt auch, wenn $n$ keine Zweierpotenz ist).\\
Um auch dynamische Datenstrukturen effizient durchsuchen zu können, lassen sich binäre Suchbäume nutzen.
\subparagraph{Suchbaum}
Ein \emph{Suchbaum} ist ein binärer Wurzelbaum, in dem jeder linke Teilbaum eines Knotens kleinere Wert und jeder rechte Teilbaum größere Wert als der Vorgängerknoten besitzt.\\
Beispiel:\\
\begin{tikzpicture}[scale=0.5]
\draw  (-8.5,16) node{1}ellipse (0.5 and 0.5);
\draw  (-4.5,16) node{4} ellipse (0.5 and 0.5);
\draw  (-2.5,16) node{6} ellipse (0.5 and 0.5);
\draw  (1.5,16) node{9} ellipse (0.5 and 0.5);
\draw  (-6.5,19) node{2} ellipse (0.5 and 0.5);
\draw  (-0.5,19) node{8} ellipse (0.5 and 0.5);
\draw  (-3.5,22) node{5} ellipse (0.5 and 0.5);
\draw (-4,21.5) -- (-6,19.5);
\draw (-3,21.5) -- (-1,19.5);
\draw (-1,18.5) -- (-2,16.5);
\draw (2,15.5) -- (3,13.5);
\draw (-6,18.25) -- (-5,16.5);
\draw (-7,18.25) -- (-8,16.5);
\draw (0,18.5) -- (1,16.5);
\draw  (3.5,13) node{10} ellipse (0.5 and 0.5);
\end{tikzpicture}\\
(muss nicht zwangsläufig balanciert sein oder 2 Nachfolger haben)\\
Ein Suchbaum lässt sich ähnlich der binären Suche rekursiv durchsuchen.\\
Der worst-case ist ein linear entarter Suchbaum:\\
\begin{tikzpicture}[scale=0.5]
\draw  (1.5,18) node{3} ellipse (0.5 and 0.5);
\draw  (-1,20) node{2} ellipse (0.5 and 0.5);
\draw  (-3.5,22) node{1} ellipse (0.5 and 0.5);
\draw (-3,21.5) -- (-1.5,20.5);
\draw (2,17.5) -- (3.5,16.5);
\draw (-0.5,19.5) -- (1,18.5);
\draw  (4,16) node{4} ellipse (0.5 and 0.5);
\end{tikzpicture}\\
Die Laufzeit der Suche ist $O(log(n))$, wenn der Baum vollständig balanciert ist und $O(n)$, wenn er linear entartet ist.

\paragraph{Hashing}
Prinzip: Mit Hilfe einer Hashfunktion $h$ werden Schlüssel auf eine Position in einem Array (Hashtabelle) abgebildet.\\
Beispiel für eine Hashfunktion:
\begin{lstlisting}[language=C]
h(s) = s mod m
\end{lstlisting}
wobei $m$ die Größe der Hashtabelle ist.\\
Problem: Es können Kollisionen auftreten, d.h. Schlüssel $s_1,s_2$ mit $h(s_1)=h(s_2)$. \\
Lösung: Überlauflisten:\\
An Position $h(s)$ wird eine Liste aller Elemente gespeichert, die diesen Hashwert besitzen.\\
Unter geeigneten Voraussetzungen besitzt Hashing eine Laufzeit von $O(1)$.

\subsection{Sortierverfahren}

\paragraph{Quicksort} 
partitioniert das zu sortierende Feld anhand eines Pivot-Elements und sortiert rekursiv die dadurch entstandenen Teilfelder.\medskip\\
\begin{tikzpicture}[scale=0.37]
\draw  (-7,7.5) rectangle (-5,5.5);
\draw  (-5,7.5) rectangle (7.5,5.5);
\draw [-latex] (-6,9) node[above]{Pivotelement} -- (-6,7.5);
\draw  (-1,1.5) rectangle (1,-0.5);
\draw  (1,1.5) rectangle (7.5,-0.5);
\draw [-latex] (0,3) node[above]{Pivotelement} -- (0,1.5);
\draw  (-1,-0.5) rectangle (-7,1.5);
\node at (-4,0.5) {$\leq$ Pivotel.};
\node at (4,0.5) {$>$ Pivotel.};
\draw (-7,-1) -- (-7,-1.5) -- (-1,-1.5) -- (-1,-1);
\draw (1,-1) -- (1,-1.5) -- (7.5,-1.5) -- (7.5,-1);
\draw (4,-1.5) -- (1,-3.5);
\draw (-4,-1.5) -- (-1,-3.5);
\node at (0,-4) {rekursiver Aufruf von Quicksort};
\end{tikzpicture}
\subparagraph{Bsp.:} \parskp
\begin{tikzpicture}[scale=0.4]
\draw  (-9,7) rectangle (-7,5);
\draw  (-7,7) rectangle (-5,5);
\draw  (-5,7) rectangle (-3,5);
\draw  (-3,7) rectangle (-1,5);
\draw  (-1,7) rectangle (1,5);
\draw  (1,7) rectangle (3,5);
\node at (-8,6) {5};
\node at (-6,6) {9};
\node at (-4,6) {1};
\node at (-2,6) {3};
\node at (0,6) {2};
\node at (2,6) {4};
\draw [-latex] (-8,8) -- (-8,7);
%\end{tikzpicture}\\
%\begin{tikzpicture}[scale=0.5]
\draw  (-9,3) rectangle (-7,1);
\draw  (-7,3) rectangle (-5,1);
\draw  (-5,3) rectangle (-3,1);
\draw  (-3,3) rectangle (-1,1);
\draw  (-1,3) rectangle (1,1);
\draw  (1,3) rectangle (3,1);
\node at (-8,2) {1};
\node at (-6,2) {3};
\node at (-4,2) {2};
\node at (-2,2) {4};
\node at (0,2) {5};
\node at (2,2) {9};
\draw [-latex] (0,4) -- (0,3);
\node at (0,0.5) {$\checkmark$};
\node at (2,0.5) {$\checkmark$};
%\end{tikzpicture}\\
%\begin{tikzpicture}[scale=0.5]
\draw  (-9,-1) rectangle (-7,-3);
\draw  (-7,-1) rectangle (-5,-3);
\draw  (-5,-1) rectangle (-3,-3);
\draw  (-3,-1) rectangle (-1,-3);
\node at (-8,-2) {1};
\node at (-6,-2) {3};
\node at (-4,-2) {2};
\node at (-2,-2) {4};
\draw [-latex] (-8,0) -- (-8,-1);
\node at (-8,-3.5) {$\checkmark$};
%\end{tikzpicture}\\
%\begin{tikzpicture}[scale=0.5]
\draw  (-7,-5) rectangle (-5,-7);
\draw  (-5,-5) rectangle (-3,-7);
\draw  (-3,-5) rectangle (-1,-7);
\node at (-6,-6) {3};
\node at (-4,-6) {2};
\node at (-2,-6) {4};
\draw [-latex] (-6,-4) -- (-6,-5);
%\end{tikzpicture}\\
%\begin{tikzpicture}[scale=0.5]
\draw  (-7,-9) rectangle (-5,-11);
\draw  (-5,-9) rectangle (-3,-11);
\draw  (-3,-9) rectangle (-1,-11);
\node at (-6,-10) {2};
\node at (-4,-10) {3};
\node at (-2,-10) {4};
\draw [-latex] (-4,-8) -- (-4,-9);
\node at (-6,-11.5) {$\checkmark$};
\node at (-2,-11.5) {$\checkmark$};
%\end{tikzpicture}\\
%\begin{tikzpicture}[scale=0.5]
\draw  (-9,-12.5) rectangle (-7,-14.5);
\draw  (-7,-12.5) rectangle (-5,-14.5);
\draw  (-5,-12.5) rectangle (-3,-14.5);
\draw  (-3,-12.5) rectangle (-1,-14.5);
\draw  (-1,-12.5) rectangle (1,-14.5);
\draw  (1,-12.5) rectangle (3,-14.5);
\node at (-8,-13.5) {1};
\node at (-6,-13.5) {2};
\node at (-4,-13.5) {3};
\node at (-2,-13.5) {4};
\node at (0,-13.5) {5};
\node at (2,-13.5) {9};
\end{tikzpicture}\\
Struktur der rekursiven Aufrufe:\smallskip\\
\begin{tikzpicture}[scale=0.35]
\draw  (-9,7) rectangle (-7,5);
\draw  (-7,7) rectangle (-5,5);
\draw  (-5,7) rectangle (-3,5);
\draw  (-3,7) rectangle (-1,5);
\draw  (-1,7) rectangle (1,5);
\draw  (1,7) rectangle (3,5);
\node at (-8,6) {5};
\node at (-6,6) {9};
\node at (-4,6) {1};
\node at (-2,6) {3};
\node at (0,6) {2};
\node at (2,6) {4};
\draw  (-10,3) rectangle (-8,1);
\draw  (-8,3) rectangle (-6,1);
\draw  (-6,3) rectangle (-4,1);
\draw  (-4,3) rectangle (-2,1);
\node at (-9,2) {1};
\node at (-7,2) {3};
\node at (-5,2) {2};
\node at (-3,2) {4};
\draw  (0,3) rectangle (2,1);
\node at (1,2) {9};
\draw  (-7,-1) rectangle (-5,-3);
\draw  (-5,-1) rectangle (-3,-3);
\draw  (-3,-1) rectangle (-1,-3);
\draw  (-8,-5) rectangle (-6,-7);
\draw  (-2,-5) rectangle (0,-7);
\node at (-6,-2) {3};
\node at (-4,-2) {2};
\node at (-2,-2) {4};
\node at (-7,-6) {2};
\node at (-1,-6) {4};
\draw (-6,3.25) -- (-3.25,4.75);
\draw (-2.75,4.75) -- (1,3.25);
\draw (-6.25,0.75) -- (-9.25,-0.75);
\draw (-5.75,0.75) -- (-4,-0.75);
\draw (-4.25,-3.25) -- (-7,-4.75);
\draw (-3.75,-3.25) -- (-1,-4.75);
\end{tikzpicture}\\
in Scala:
\begin{lstlisting}[language=scala]
package alg

object Quicksort{
	def qs(l: List[Int]): List[Int] = l match) {
		case Nil => Nil
		case h::t =>
			val (li, re) = t.partition(x=> x <= h)
			qs(li) ::: h :: qs(re)
	}
	
	def main(args: Array[String]) {
		// println(List(423,2,3,4,67,8,7,12,3,4)))
		// wird folgende Zeile verwendet (zufällige Folge von 10000 Zahlen), 
		// ist der Ablauf relativ schnell
		// val l = (for(i <- 0 until 100000) yield scala.util.Random.nextInt).toList
		// sortiert man mit folgender Zeile eine sortierte Liste, 
		// dauert es sehr lange oder führt zu einem Fehler (aufgrund zu vieler Rekursionen)
		val l = (for(i <- 0 until 100000) yield i).toList
		val x = qs(l)
	}
}
\end{lstlisting}
Die Laufzeitanalyse ist schwierig, weil die Länge der Teillisten vom Pivotelement abhängt. Wir betrachten stattdessen ein ähnlichens Verfahren: Mergesort.\\
Hierbei werden die Listen halbiert, rekursiv sortiert und dann zusammengefügt.

\paragraph{Mergesort} \parskp
\begin{tikzpicture}[scale=0.35]
\draw  (-0.5,0) rectangle (1.5,-2) node[pos=.5] {5};
\draw  (1.5,0) rectangle (3.5,-2) node[pos=.5] {6};
\draw  (3.5,0) rectangle (5.5,-2) node[pos=.5] {3};
\draw  (5.5,0) rectangle (7.5,-2) node[pos=.5] {2};
\draw  (7.5,0) rectangle (9.5,-2) node[pos=.5] {1};
\draw  (9.5,0) rectangle (11.5,-2) node[pos=.5] {4};
\draw  (11.5,0) rectangle (13.5,-2) node[pos=.5] {7};
\draw  (-2.5,-3.5) rectangle (-0.5,-5.5) node[pos=.5] {5};
\draw  (-0.5,-3.5) rectangle (1.5,-5.5) node[pos=.5] {6};
\draw  (1.5,-3.5) rectangle (3.5,-5.5) node[pos=.5] {3};
\draw  (7.5,-3.5) rectangle (9.5,-5.5) node[pos=.5] {2};
\draw  (9.5,-3.5) rectangle (11.5,-5.5) node[pos=.5] {1};
\draw  (11.5,-3.5) rectangle (13.5,-5.5) node[pos=.5] {4};
\draw  (13.5,-3.5) rectangle (15.5,-5.5) node[pos=.5] {7};
\draw  (-3,-7) rectangle (-1,-9) node[pos=.5] {5};
\draw  (0.5,-7) rectangle (2.5,-9) node[pos=.5] {6};
\draw  (2.5,-7) rectangle (4.5,-9) node[pos=.5] {3};
\draw  (6.5,-7) rectangle (8.5,-9) node[pos=.5] {2};
\draw  (8.5,-7) rectangle (10.5,-9) node[pos=.5] {1};
\draw  (12.5,-7) rectangle (14.5,-9) node[pos=.5] {4};
\draw  (14.5,-7) rectangle (16.5,-9) node[pos=.5] {7};
\draw  (-3,-10.5) rectangle (-1,-12.5) node[pos=.5] {5};
\draw  (0,-10.5) rectangle (2,-12.5) node[pos=.5] {6};
\draw  (3,-10.5) rectangle (5,-12.5) node[pos=.5] {3};
\draw  (6,-10.5) rectangle (8,-12.5) node[pos=.5] {2};
\draw  (9,-10.5) rectangle (11,-12.5) node[pos=.5] {1};
\draw  (12,-10.5) rectangle (14,-12.5) node[pos=.5] {4};
\draw  (15,-10.5) rectangle (17,-12.5) node[pos=.5] {7};

\draw [dotted, thick,red] (5.5,0.5) -- (5.5,-2.5);
\draw [dotted, thick,red] (-0.5,-3) -- (-0.5,-6);
\draw [dotted, thick,red] (2.5,-6.5) -- (2.5,-9.5);
\draw [dotted, thick,red] (11.5,-3) -- (11.5,-6);
\draw [dotted, thick,red] (8.5,-6.5) -- (8.5,-9.5);
\draw [dotted, thick,red] (14.5,-6.5) -- (14.5,-9.5);
\draw [latex-] (-4,-12) -- (-4,-0.5);
\end{tikzpicture}
\qquad
\begin{tikzpicture}[scale=0.35]
\draw  (-0.5,0) rectangle (1.5,-2) node[pos=.5] {1};
\draw  (1.5,0) rectangle (3.5,-2) node[pos=.5] {2};
\draw  (3.5,0) rectangle (5.5,-2) node[pos=.5] {3};
\draw  (5.5,0) rectangle (7.5,-2) node[pos=.5] {4};
\draw  (7.5,0) rectangle (9.5,-2) node[pos=.5] {5};
\draw  (9.5,0) rectangle (11.5,-2) node[pos=.5] {6};
\draw  (11.5,0) rectangle (13.5,-2) node[pos=.5] {7};
\draw  (-2.5,-3.5) rectangle (-0.5,-5.5) node[pos=.5] {3};
\draw  (-0.5,-3.5) rectangle (1.5,-5.5) node[pos=.5] {5};
\draw  (1.5,-3.5) rectangle (3.5,-5.5) node[pos=.5] {6};
\draw  (7.5,-3.5) rectangle (9.5,-5.5) node[pos=.5] {1};
\draw  (9.5,-3.5) rectangle (11.5,-5.5) node[pos=.5] {2};
\draw  (11.5,-3.5) rectangle (13.5,-5.5) node[pos=.5] {4};
\draw  (13.5,-3.5) rectangle (15.5,-5.5) node[pos=.5] {7};
\draw  (-3,-7) rectangle (-1,-9) node[pos=.5] {5};
\draw  (0.5,-7) rectangle (2.5,-9) node[pos=.5] {3};
\draw  (2.5,-7) rectangle (4.5,-9) node[pos=.5] {6};
\draw  (6.5,-7) rectangle (8.5,-9) node[pos=.5] {1};
\draw  (8.5,-7) rectangle (10.5,-9) node[pos=.5] {2};
\draw  (12.5,-7) rectangle (14.5,-9) node[pos=.5] {4};
\draw  (14.5,-7) rectangle (16.5,-9) node[pos=.5] {7};
\draw  (-3,-10.5) rectangle (-1,-12.5) node[pos=.5] {5};
\draw  (0,-10.5) rectangle (2,-12.5) node[pos=.5] {3};
\draw  (3,-10.5) rectangle (5,-12.5) node[pos=.5] {6};
\draw  (6,-10.5) rectangle (8,-12.5) node[pos=.5] {2};
\draw  (9,-10.5) rectangle (11,-12.5) node[pos=.5] {1};
\draw  (12,-10.5) rectangle (14,-12.5) node[pos=.5] {4};
\draw  (15,-10.5) rectangle (17,-12.5) node[pos=.5] {7};
\draw [-latex] (-4,-12) -- (-4,-0.5);
\end{tikzpicture}
\\
Struktur der rekursiven Aufrufe:\\
\begin{tikzpicture}[scale=0.35]
\draw  (-0.5,0) rectangle (1.5,-2) node[pos=.5] {5};
\draw  (1.5,0) rectangle (3.5,-2) node[pos=.5] {6};
\draw  (3.5,0) rectangle (5.5,-2) node[pos=.5] {3};
\draw  (5.5,0) rectangle (7.5,-2) node[pos=.5] {2};
\draw  (7.5,0) rectangle (9.5,-2) node[pos=.5] {1};
\draw  (9.5,0) rectangle (11.5,-2) node[pos=.5] {4};
\draw  (11.5,0) rectangle (13.5,-2) node[pos=.5] {7};
\draw  (-2.5,-3.5) rectangle (-0.5,-5.5) node[pos=.5] {5};
\draw  (-0.5,-3.5) rectangle (1.5,-5.5) node[pos=.5] {6};
\draw  (1.5,-3.5) rectangle (3.5,-5.5) node[pos=.5] {3};
\draw  (7.5,-3.5) rectangle (9.5,-5.5) node[pos=.5] {2};
\draw  (9.5,-3.5) rectangle (11.5,-5.5) node[pos=.5] {1};
\draw  (11.5,-3.5) rectangle (13.5,-5.5) node[pos=.5] {4};
\draw  (13.5,-3.5) rectangle (15.5,-5.5) node[pos=.5] {7};
\draw  (-3,-7) rectangle (-1,-9) node[pos=.5] {5};
\draw  (0.5,-7) rectangle (2.5,-9) node[pos=.5] {6};
\draw  (2.5,-7) rectangle (4.5,-9) node[pos=.5] {3};
\draw  (6.5,-7) rectangle (8.5,-9) node[pos=.5] {2};
\draw  (8.5,-7) rectangle (10.5,-9) node[pos=.5] {1};
\draw  (12.5,-7) rectangle (14.5,-9) node[pos=.5] {4};
\draw  (14.5,-7) rectangle (16.5,-9) node[pos=.5] {7};
\draw  (0,-10.5) rectangle (2,-12.5) node[pos=.5] {6};
\draw  (3,-10.5) rectangle (5,-12.5) node[pos=.5] {3};
\draw  (6,-10.5) rectangle (8,-12.5) node[pos=.5] {2};
\draw  (9,-10.5) rectangle (11,-12.5) node[pos=.5] {1};
\draw  (12,-10.5) rectangle (14,-12.5) node[pos=.5] {4};
\draw  (15,-10.5) rectangle (17,-12.5) node[pos=.5] {7};

\draw  (21.5,0) rectangle (23.5,-2) node[pos=.5] {1};
\draw  (23.5,0) rectangle (25.5,-2) node[pos=.5] {2};
\draw  (25.5,0) rectangle (27.5,-2) node[pos=.5] {3};
\draw  (27.5,0) rectangle (29.5,-2) node[pos=.5] {4};
\draw  (29.5,0) rectangle (31.5,-2) node[pos=.5] {5};
\draw  (31.5,0) rectangle (33.5,-2) node[pos=.5] {6};
\draw  (33.5,0) rectangle (35.5,-2) node[pos=.5] {7};
\draw  (19.5,-3.5) rectangle (21.5,-5.5) node[pos=.5] {3};
\draw  (21.5,-3.5) rectangle (23.5,-5.5) node[pos=.5] {5};
\draw  (23.5,-3.5) rectangle (25.5,-5.5) node[pos=.5] {6};
\draw  (29.5,-3.5) rectangle (31.5,-5.5) node[pos=.5] {1};
\draw  (31.5,-3.5) rectangle (33.5,-5.5) node[pos=.5] {2};
\draw  (33.5,-3.5) rectangle (35.5,-5.5) node[pos=.5] {4};
\draw  (35.5,-3.5) rectangle (37.5,-5.5) node[pos=.5] {7};
\draw  (19,-7) rectangle (21,-9) node[pos=.5] {5};
\draw  (22.5,-7) rectangle (24.5,-9) node[pos=.5] {3};
\draw  (24.5,-7) rectangle (26.5,-9) node[pos=.5] {6};
\draw  (28.5,-7) rectangle (30.5,-9) node[pos=.5] {1};
\draw  (30.5,-7) rectangle (32.5,-9) node[pos=.5] {2};
\draw  (34.5,-7) rectangle (36.5,-9) node[pos=.5] {4};
\draw  (36.5,-7) rectangle (38.5,-9) node[pos=.5] {7};

\draw (0.5,-3) -- (6,-2.5);
\draw (7,-2.5) -- (11.5,-3);
\draw (0,-6) -- (-2,-6.5);
\draw (1,-6) -- (2.5,-6.5);
\draw (2,-9.5) -- (1,-10);
\draw (3,-9.5) -- (4,-10);
\draw (11,-6) -- (8.5,-6.5);
\draw (12,-6) -- (14.5,-6.5);
\draw (8,-9.5) -- (7,-10);
\draw (9,-9.5) -- (10,-10);
\draw (14,-9.5) -- (13,-10);
\draw (15,-9.5) -- (16,-10);
\draw (20,-6.5) -- (22,-6);
\draw (23,-6) -- (24.5,-6.5);
\draw (30.5,-6.5) -- (33,-6);
\draw (34,-6) -- (36.5,-6.5);
\draw (22.5,-3) -- (28,-2.5);
\draw (29,-2.5) -- (33.5,-3);

\draw [color=red] (18.5,0.5) rectangle (39,-15.5) node[below left]{passiert eigentlich in linkem};
\draw (-2,-9.5) -- (-2,-14);
\draw (1,-13) -- (1,-14);
\draw (4,-13) -- (4,-14);
\draw (7,-13) -- (7,-14);
\draw (10,-13) -- (10,-14);
\draw (13,-13) -- (13,-14);
\draw (16,-13) -- (16,-14);
\draw (20,-10.5) -- (20,-9.5);
\draw (24,-9.5) -- (23,-10.5);
\draw (25,-9.5) -- (26,-10.5);
\draw (30,-9.5) -- (29,-10.5);
\draw (31,-9.5) -- (32,-10.5);
\draw (36,-9.5) -- (35,-10.5);
\draw (37,-9.5) -- (38,-10.5);
\end{tikzpicture}\\
Laufzeitanalyse: Wir stellen das Verhalten von Mergesort für $n=2^k$ durch einen Binärbaum dar.\\
\begin{tikzpicture}[scale=0.8]
\node at (0,0) [circle] {$\bullet$};
\node at (1,0) [circle] {$\bullet$};
\node at (2,0) [circle] {$\bullet$};
\node at (3,0) [circle] {$\bullet$};
\node at (0.5,1) [circle] {$\bullet$};
\node at (2.5,1) [circle] {$\bullet$};
\node at (1.5,2) [circle] {$\bullet$};
\node at (0,0) [left] {$\tfrac{n}{4}$};
\node at (1,0) [left] {$\tfrac{n}{4}$};
\node at (2,0) [left] {$\tfrac{n}{4}$};
\node at (3,0) [left] {$\tfrac{n}{4}$};
\node at (0.5,1) [left] {$\tfrac{n}{2}$};
\node at (2.5,1) [left] {$\tfrac{n}{2}$};
\node at (1.5,2) [left] {$n$};
\draw (1.5,2) -- (0.5,1) -- (0,0);
\draw (1,0) -- (0.5,1);
\draw (1.5,2) -- (2.5,1) -- (2,0);
\draw (3,0) -- (2.5,1);
\node [rotate=90] at (0,-0.5) {...};
\node [rotate=90] at (1,-0.5) {...};
\node [rotate=90] at (2,-0.5) {...};
\node [rotate=90] at (3,-0.5) {...};
\end{tikzpicture}\\
Zum erzeugen der Hälften und dem Zusammenfügen fällt der Aufwand O(|linke Liste| + |rechte Liste|). Dies ist $O(n)$ auf jeder Ebene des Baums. Der Baum hat $n=2^k$ Blätter und daher die Tiefe $k$. Die Laufzeit von Mergesort ist daher $O(n\;log(n))$ (Anzahl der Ebenen $n$ mit LZ auf jeder Ebene).\\
Mergesort besitzt immer die LZ $O(n\; log(n))$. Dies ist nahe am Optimum. Quicksort besitzt eine durchschnittliche LZ in $O(n\; log(n))$ und eine Worst-Case-LZ in $O(n^2)$.\\
Da Quicksort mit vorsortierten Listen nicht so gut umgehen kann (wird linear entartet), ist es kein sehr gutes Suchverfahren.

\chapter{Codierungstheorie}

\paragraph{Gruppen}
Erläuternde Beispiele:\\
$(\mathbb{Z}_2,+)$ (Elemente dieser Menge werden im folgenden \emph{Bits} genannt)\\
$a+b:=a+b\; mod\; 2$\\
$\Rightarrow 1+1\equiv 0 (mod\; 2)$\\
inverses Element: $a+a^{-1}=0 \rightarrow$ hier also: zu $1$ ist $1$ das inverse Element.\medskip\\
$(\mathbb{Z}_P,+)$ ($\mathbb{Z}_P$: alle Primzahlen)\\
$7+4\equiv 0 (mod\; 11)$\\
hier bspw. $4$ das inverse Element zu $7$.

\paragraph{Ausgangslage Codierungstheorie}\parskp
Problem: Daten können bei der Übertragung verändert werden. Wie können diese Fehler erkannt und ggf. korrigiert werden?\\
\begin{tikzpicture}[scale=0.8]
\draw  (0,0) rectangle (3,1) node [pos=0.5]{Sender};
\draw  (9.5,0) rectangle (12.5,1) node [pos=0.5]{Empfänger};

\draw [-latex] (3,0.5) -- (9.5,0.5) node [pos=0.5, above]{\lightning\;Störung};
\draw [-latex] (3,0.5) -- (9.5,0.5) node [pos=0.25, below]{Kanal};
\end{tikzpicture}\\
Häufige Lösung zur Fehlererkennung: Prüfsummen.
\section{Paritätsprüfung}
Für eine Folge von Bits $b_1,...,b_{n-1}\in \{0,1\}$ wird das \emph{Paritätsbit} $b_n= \left( \sum_{i=1}^{n-1}b_i\right) mod \; 2$ berechnet. Das enthaltene Codewort ist $b_1...b_n$
\subparagraph{Bsp.:} Für die Folge von Bits $0,1,1,0,1,0$ ist $1$ (Anzahl aller $1$er-Bits $mod\;2$) das Paritätsbit. Das Codewort ist also $0,1,1,0,1,0,1$.\medskip\\
Aus $\sum_{i=1}^{n-1}b_i\equiv b_n (mod\; 2)$ folgt daher (indem auf beiden Seiten $+b_n$ rechnet) $\sum_{i=1}^{n}b_i \equiv 0 (mod\;2)$.\\
Die Menge der Wörter des \emph{Parity-Check-Code} lässt sich damit durch $\{(b_1,...,b_n)\in \{0,1\}^n\|\sum_{i=1}^{n}b_i\equiv 0 (mod\; 2)\}$ darstellen. $|P_n|=2^{n-1}$ bezeichnet die Mächtigkeit dieser Menge (alle Bits, bis auf das letzte, können frei gewählt werden).
\subparagraph{Bsp.:} Für $n=8$ ist $00110011$ (4 Einsen $\Rightarrow 4mod \;2=0 \Rightarrow$OK) ein Element des Parity-Check-Code, $01110011$ kein Element des Parity-Check-Code.\\
\begin{tikzpicture}[scale=0.8]
\def \nx {8}
\def \ny {5}

\draw  (0,0)node [below right]  {alle Wörter in $\{0,1\}^n$} rectangle (\nx+1,\ny+1);
\draw (\nx,\ny) node [above right]  {\color{purple}$\curvearrowright$ Codewörter};
\foreach \x in {1,...,\nx}
{
	\foreach \y in {1,...,\ny}{
		\node[circle, purple] at (\x, \y) {$\bullet$};
  	}
}
\end{tikzpicture}
\paragraph{Satz:} Der Parity-Check-Code ist 1-fehlererkennend.
\subparagraph{Beweis:} Seien $b_1...b_n$ ein Codewort und $b_1'...b_n'$ ein Wort, das sich an Stelle $k$ von $b_1...b_n$ unterscheidet. Angenommen, $b_1'...b_n'$ ist ein Codewort. Dann gilt, $0\equiv \sum_{i=1}^{n}b_i'\equiv \underbrace{\sum_{i=1}^{n}b_i}_{=0}+\underbrace{(b_k'-b_k)}_{=1 \text{(da ungleich)}}\equiv 1$, Widerspruch. Also ist $b_1'...b_n'$ kein Codewort. Dies kann der Empfänger erkennen.\\
Bemerkung: Der Parity-Check-Code ist nicht 2-fehlerkorrigierend.\\
Anwendungen: 
\begin{itemize}
\item Speicherkontroller, Festplattenkontroller
\item Netzwerkprotokolle
\item 7-bit-ASCII (Bit 8 als Parität)
\end{itemize}
Ferner kann das Verfahren zur Rekonstruktion eines verloren gegangenen Bits verwendet werden, wenn die restlichen Bits fehlerfrei sind. Denn aus $0\equiv \sum_{i=1}^{n} b_i (mod\;2)$ folgt für $1\leq k \leq n$: \\
$b_k\equiv \sum_{i=1,\, i\not =k}^{n} b_i (mod\;2)$. Damit kann Bit $b_k$ aus den anderen Bits rekonstruiert werden.\\
Anwendung: RAID4, RAID5 -- Daten und Parität werden auf $n$ Festplatten verteilt. Beim Ausfall einer Platte können die Daten rechnerisch rekonstruiert werden.\\
\begin{tikzpicture}[scale=0.3]
\draw  (0,0) node{$D_1$} ellipse (3 and 1);
\draw (-3,0) -- (-3,-1.5);
\draw (3,0) -- (3,-1.5);
\draw  plot[smooth, tension=1.35] coordinates {(-3,-1.5) (0,-2.5) (3,-1.5)};
\draw  (14.5,0) node{$D_{n-1}$} ellipse (3 and 1);
\draw (11.5,0) -- (11.5,-1.5);
\draw (17.5,0) -- (17.5,-1.5);
\draw  plot[smooth, tension=1.35] coordinates {(11.5,-1.5) (14.5,-2.5) (17.5,-1.5)};
\draw  (21.5,0) node{$D_n$} ellipse (3 and 1);
\draw (18.5,0) -- (18.5,-1.5);
\draw (24.5,0) -- (24.5,-1.5);
\draw  plot[smooth, tension=1.35] coordinates {(18.5,-1.5) (21.5,-2.5) (24.5,-1.5)};
\draw (0,-2.5) -- (0,-4.5) -- (22.5,-4.5) -- (22.5,-2.5);
\draw (14.5,-2.5) -- (14.5,-4.5);

\node at (7,-1) {…};
\end{tikzpicture}
\subparagraph{Bsp.:} $0100x110$ wird korrigiert zu $01001110$ (da Parity-Check-Code $mod\;2=0$ ergeben muss).

\subsection{ISBN-Code}
Der ISBN-Code enthält eine Prüfsumme.
\subparagraph{Bsp.:} $\underbrace{382741826}_{\text{9-stellige Buchnummer}} \underbrace{7}_{\text{Prüfziffer}}$\\
Für die Prüfziffer gilt:\\
$z_{10}=\left(\sum_{i=1}^9 i\cdot z_i\right) mod\; 11$\\
Dabei wird $X$ für den Wert 10 verwendet. Wegen $10+1\equiv0 (mod\; 11)$ ist $10$ das inverse Element zu $1$ bezüglich $+$ (d.h. $10$ entspricht $-1$). Aus obiger Gleichung folgt damit:\\
$0\equiv \sum_{i=1}^{10} i\cdot z_i(mod\; 11)$\\
Die Menge der Codewörter ist daher: \\
$\left\lbrace z_1,...,z_{10})|z_1,...,z_9\in \{0,...,9\}, z_{10} \in \{0,...,X\},\sum_{i=1}^{10} i\cdot z_i\equiv 0 (mod\; 11)\right\rbrace$\\
$(\mathbb{Z}_{11},+,\cdot)$ ist ein Köper.\\
Beispiele:\\
$1+10\equiv 0$\\
$2\cdot 6 \equiv 1 \qquad (6\corr 2^{-1})$\\
$6\cdot 2 \equiv 1 \qquad (2 \corr 6^{-1})$\\
damit: $6 \equiv x \underset{\text{auf beiden Seiten mit 6 dividieren}}{\Leftrightarrow} 1 \equiv 2x$
\paragraph{Satz:} Der ISBN-Code ist 1-fehler-erkennend.\\
Beweis: Seien $z_1...z_{10}$ ein Codewort und $z_1'...z_{10}'$ ein Wort, das sich an Stelle $k$ von $z_1...z_{10}$ unterscheidet. Angenommen $z_1'...z_{10}'$ ist ein Codewort. Dann gilt $0 (mod\; 11)\equiv \sum_{i=1}^{10} i\cdot z_i'$. Da sich $z_1'...z_{10}'$ an Stelle $k$ von $z_1...z_{10}$ unterscheidet, folgt:\\
$0 \equiv \sum_{i=1}^{10} i\cdot z_i+k(z_k'-z_k) \Leftrightarrow\\
0\equiv k(z_k'-z_k)$\\
Da $k\not = 0$, besitzt $k$ ein bezüglich $\cdot$ inverses Element $k^{-1}$, womit folgt:\\
$0 \equiv z_k'-z_k$\\
und damit\\
$z_k \equiv z_k'$. Widerspruch!
\paragraph{Satz:} Der ISBN-Code erkennt Vertauschungen von Ziffern (Zahlendreher).\\
Beweis: Seien $z_1...z_{10}$ ein Codewort und $z_1'...z_{10}'$ ein daraus erhaltenes Wort, in dem die Stellen $k,l (k<l)$ vertauscht wurden. Angenommen  $z_1'...z_{10}'$ ist ein Codewort.  Dann gilt: \\
$0 \equiv \sum_{i=1}^{10} i\cdot z_i'\\
\equiv 1 \cdot z_1'+...+k\cdot z_k'+...+l\cdot z_l'+...+10\cdot z_{10}'\\
\equiv 1 \cdot z_1'+...+k\cdot z_l+...+l\cdot z_k+...+10\cdot z_{10}'\\
\equiv 1 \cdot z_1+...+k\cdot z_l+...+l\cdot z_k+...+10\cdot z_{10}\\
\equiv \sum_{i=1}^{10} i\cdot z_i + k(z_l-z_k)+l(z_k-z_l)\\
\equiv k(z_l-z_k)+l(z_k-z_l)\\
\equiv k(z_l-z_k)-l(z_k-z_k) \Leftrightarrow\\
k(z_l-z_k)\equiv l (z_k-z_k)$ \\
Da für $z_l \not= z_k$ das Element $(z_l-z_k)\not = 0 $ und daher $(z_l-z_k)^{-1}$ existiert, folgt 
$k\equiv l$. Widerpsruch zu der Annahme $k<l$!

\section{Fehlerkorrigierende Codes}

Idee: Für ein empfangenes Wort $w$ suchen wir ein Codewort $v$, so dass der Abstand $d(v,w)$ minimal ist (nearest neighbor decoding).\\
\begin{tikzpicture}[scale=0.5]
\def \nx {4}
\def \ny {3}

\draw  (1,1)node [below right]  {alle Wörter in $\{0,1\}^n$} rectangle (\nx*2+1,\ny*2+1);
\draw (\nx*2,\ny*2) node [above right]  {\color{purple}$\curvearrowright$ Codewörter $v_i$};
\draw (\nx*2+.7,\ny*2-2) node [right, align=right, green]  {$\curvearrowright$ Bereich, in dem Wort einem Code-\\wort zugeordnet werden kann};
\foreach \x in {1,...,\nx}
{
	\foreach \y in {1,...,\ny}{
		\node[circle, purple] at (\x*2, \y*2) {$\bullet$};
		\draw [green] (\x*2,\y*2) ellipse (0.75 and 0.75);
  	}
}
\node [circle] at (7.8,3.6) {$\bullet$};
\node [below left]at (7.8,3.6) {$w$};
\end{tikzpicture}
\paragraph{Def.:} Für Wörter $v,v'\in \{0,1\}^n$ ist der \emph{Hamming}-Abstand $d(v,v')$ die Anzahl Stellen, in denen sich $v,v'$ unterscheiden. Der \emph{Minimalabstand} eines Codes $C$ ist $d(C)=min\{d(v,v')|v,v'\in C, \; v\not = v'\}$\footnote{$d(u,v)=\sum d(u_i, v_i)=d(u-v,0)$}.
\subparagraph{Bsp.:} Parity-Check-Code\\
$d(0101,1010)=4$ (Codewörter unterscheiden sich in allen 4 Stellen)\\
$d(0101,0110)=2$ (Codewörter unterscheiden sich an 2 Stellen)\\
Der Minimalabstand beim Parity-Check-Code ist 2, da beides laut Def. ungleiche Codewörter sind. Der Abstand zweier gültiger Parity-Check-Codewörter ist mindestens 2, somit ist auch der Minimalabstand 2.

\paragraph{Satz:}
\begin{enumerate}
\item Ein Code ist $k$-fehlererkennend gdw. sein Minimalabstand mindestens $k+1$ ist.
\item Ein Code ist $k$-fehlerkorrigierend gdw. sein Minimalabstand mindestens $2k+1$ ist.
\end{enumerate} 

\subparagraph{Bsp.:} Der Parity-Check-Code besitzt den Minimalabstand 2 und ist daher 1-fehlererkennend und 0-fehlerkorrigierend\footnote{ein Fehler kann nur korrigiert werden, wenn die Position des fehlerhaften Bits bekannt ist. \\
Deswegen im Allgemein nur 0-fehlerkorrigierend}.\\
\begin{tikzpicture}[scale=0.8]
\node[circle] at (0,0.4) {$\bullet$};
\node[circle] at (1,-0.6) {$\bullet$};
\node[circle] at (-1,-0.6) {$\bullet$};
\node[circle] at (0,-1.6) {$\bullet$};
\node[circle] at (0,-0.6) {$\bullet$};
\draw [green] (0,-0.6) ellipse (1 and 1);
\node[circle] at (2,-0.6) {$\bullet$};
\draw [green] (2,-0.6) ellipse (1 and 1);
\end{tikzpicture}\\
(Kreise überlappen sich. Deswegen können Fehler nicht eindeutig einem Codewort zugewiesen werden)\\
Beweis:
\begin{enumerate}
\item Fehlererkennung:\\
$(\Rightarrow)$ Sei $C$ k-fehlererkennend. Seien ferner $c\in C$ und $w \in \{0,1\}^n$ mit $d(c,w)\leq k$. Da $C$ k-fehlererkennund ist, muss für jedes $c' \in C, c'\not = c$ gelten: $w \not = c'$. Daraus folgt $d(c,c') \geq k+1$, woraus folgt, dass der Minimalabstand von $C$ mindestens $k+1$($\geq k+1$) ist.\\
\begin{tikzpicture}[scale=.5]
\node[circle, purple] at (0,0) {$\bullet$};
\draw [green] (0,0) ellipse (2 and 2);
\node [circle] at (0.6,0) {$\bullet$};
\node [circle] at (1.2,0) {$\bullet$};
\node [circle] at (1.8,0) {$\bullet$};
\node [above] at (1.8,0) {$w$};
\node [circle] at (3,0) {$\bullet$};
\node [circle, blue] at (2.4,0) {$\bullet$};
\draw [decorate, decoration={brace, amplitude=5pt}] (1.8,-0.2) --(0,-0.2);
\node [below] at (0.8,-0.4) {$\leq k$};
\node [above, purple] at (0,0) {$c$};
\node [above, blue] at (2.4,0) {$c'$};
\end{tikzpicture}\\
$(\Leftarrow)$ Sei der Minimalabstand von $C$ mindestens $k+1$. Seien ferner $c\in C$ und $w \in \{0,1\}^n$ mit $d(c,w)\leq k$. Da $C$ den Minimalabstand $\geq k+1$ besitzt, kann $w$ kein Codewort $\not = c$ sein ($w$ liegt innerhalb des Radius $k$). Daher ist $C$ k-fehlererkennend. 

\item Fehlerkorrektur:\\
$(\Rightarrow)$ Wenn der Code $k$-fehlerkorrigierend ist, darf es nur ein Codewort $v$ mit $d(v,w)\leq k$ für ein empfangenes Wort $w$ geben.\\
\begin{tikzpicture}[scale=.5]
\node[circle] at (0,0) {$\bullet$};
\draw [green] (0,0) ellipse (2 and 2);
\node[circle] at (3.2,0) {$\bullet$};
\draw [green] (3.2,0) ellipse (2 and 2);
\node [circle] at (1.6,0) {$\bullet$};
\node at (1.6,0) [above] {$w$};

\draw [decorate, decoration={brace, amplitude=5pt}] (1.55,-0.2) --(0,-0.2);
\draw [decorate, decoration={brace, amplitude=5pt}] (3.2,-0.2) --(1.65,-0.2);
\node [below] at (0.8,-0.4) {$\leq k$};
\node [below] at (2.4,-0.4) {$\leq k$};
\end{tikzpicture}\\
(Die Kugeln dürfen sich nicht überschneiden. Es muss eindeutig bleiben.)\\
Folglich muss $d(v',w)\geq k+1$ für alle Codewörter $v'\not = v$ gelten, woraus $d(v,v')\geq 2k+1$ folgt.\\
\begin{tikzpicture}[scale=.5]
\node[circle] at (0,0) {$\bullet$};
\draw [green] (0,0) ellipse (2 and 2);
\node[circle] at (4.6,0) {$\bullet$};
\draw [green] (4.6,0) ellipse (2 and 2);
\node [circle] at (0.6,0) {$\bullet$};
\node [circle] at (1.2,0) {$\bullet$};
\node [circle] at (1.8,0) {$\bullet$};
\node [above] at (1.8,0) {$w$};

\node [circle] at (4,0) {$\bullet$};
\node [circle] at (3.4,0) {$\bullet$};
\node [circle] at (2.8,0) {$\bullet$};

\draw [decorate, decoration={brace, amplitude=5pt}] (1.8,-0.2) --(0,-0.2);
\draw [decorate, decoration={brace, amplitude=5pt}] (4.6,-0.2) --(2.8,-0.2);
\node [below] at (0.8,-0.4) {$k$};
\node [below] at (3.8,-0.4) {$k$};
\node [purple, above left] at (0,0) {$v$};

\node [purple, above right] at (4.6,0) {$v'$};
\end{tikzpicture}\\
$(\Leftarrow)$ Für ein Codewort $v$ sei $S(v)=\{w|d(v,w)\leq k\}$. Zu zeigen: aus $v,v'$ Codewörter mit $v\not = v'$ folgt $S(v) \cap S(v')=\emptyset$. Angenommen, es gibt ein $w \in S(v)\cap S(v')$. Dann gilt $d(v,w)\leq k$ und $d(w,v') \leq k$.\\
\begin{tikzpicture}[scale=.5]
\node[circle] (v1) at (0,0) {$\bullet$};
\node[circle] (v3) at (3,0) {$\bullet$};
\node[circle] (v2) at (1.5,1.5) {$\bullet$};
\node[circle, left] at (0,0) {$a$};
\node[circle,right] at (3,0) {$b$};
\node[circle, above] at (1.5,1.5) {$c$};
\draw (v1) -- (v2) -- (v3) -- (v1);
\end{tikzpicture}\\
$d(a,b)\leq d(a,c) + d(c,b)$\\
Daraus folgt mit der Dreiecksungleichung $d(v,v')\leq d(v,w)+d(w,v')=2k$. Dies ist ein Widerspruch, da der Minimalabstand des Codes mindestens $2k+1$ ist.\\
\begin{tikzpicture}[scale=.5]
\node[circle] at (0,0) {$\bullet$};
\draw [green] (0,0) ellipse (2 and 2);
\node [circle] at (1.8,0) {$\bullet$};
\node [above] at (1.8,0) {$w$};
\draw [decorate, decoration={brace, amplitude=5pt}] (1.8,-0.2) --(0,-0.2);
\node [below] at (0.8,-0.4) {$k$};
\node [above] at (0,0) {$v$};
\end{tikzpicture}(Wort $w$ kann eindeutig zum Codewort $v$ dekodiert werden.)
\end{enumerate}
Naiver Ansatz zur Fehlerkorrektur: Nachricht mehrfach senden.
\paragraph{Bsp.:} $0 \rightarrow 000$, $1\rightarrow 111$\\
Der Code $\{000,111\}$ hat Minimalabstand 3 (Nachricht wurde 3 mal gesendet) und ist daher 1-fehlerkorrigierend.\\
Nachteil: Platzverschwendung.\medskip\\
Effizienter sind lineare Codes.
\subsection{Lineare Codes}
Die Decodierung durch eine Nearest-Neighbor-Suche im Coderaum ist ineffizient. Um ein effizienteres Verfahren zu erhalten, beschreiben wir Codes durch Matrix-Vektor-Operationen.
\subparagraph{Bsp.:} Mit $A = (\underbrace{1 ... 1}_{n})$ lässt sich der Parity-Check-Code beschreiben durch $\{w\in \{0,1\}^n| A\cdot w^T=0\}$ ($w,A$: Zeilenvektoren) [Veranschaulichung: $(1111)\cdot(0101)^T=0$]

\paragraph{Def.:} Ein Code $C$ heißt \emph{linear}, wenn es eine Matrix $A$ gibt, sodass $C=\{w|A\cdot w^T=0\}$.\\
Die Matrix $A$ heißt \emph{Parity-Check-Matrix}.

\paragraph{Satz:} Ein binärer linearer Code $C$ ist ein Vektorraum über $\mathbb{Z}_2(+,\cdot)$.\\
Beweis(Skizze):
\begin{enumerate}
\item Abgeschlossenheit der Vektoraddition:\\
Zu zeigen: $w_1,w_2\in C \Rightarrow w_1+w_2 \in C$ ($+$: Kompomentenweise Addition im Vektorraum)\\
Aus $w_1,w_2\in C$ folgt $Aw_1^T=0, Aw_2^T=0$ ($0$ ist der Nullvektor) und daraus $0=Aw_1^T+Aw_2^T=A(w_1^T+w_2^T)=A(w_1+w_2)^T$ und daraus $w_1,w_2 \in C$.
\item Abgeschlossenheit der Mulitiplikation (mit Skalaren):\\
Zu zeigen: $\alpha \in \mathbb{Z}_2, w \in C \Rightarrow \alpha \cdot w \in C$\\
Dies gilt, da $\alpha\cdot w =\begin{cases}
0 & \text{für }\alpha =0\\
w & \text{für } \alpha =1
\end{cases}$ und $0,w \in C$.
\item[$\bullet$] Andere Vektoraxiome folgen entsprechend.
\end{enumerate}
Da ein linearer Code ein Vektorraum ist, besitzt er eine Basis\footnote{Eine Menge von linear unabhängigen Vektoren, mit der sich alle anderen Punkte erzeugen lassen}.

\paragraph{Def.:} Sei $C$ ein linearer Code. Eine Matrix $G$, deren Zeilen eine Basis von $C$ bilden heißt \emph{Generatormatrix}.

\subparagraph{Bsp.:} Sei $G=\begin{pmatrix}
1 & 0 & 0 & 1\\
0 & 1 & 0 & 1 \\
0 & 0 & 1 & 1
\end{pmatrix}$. Da die Zeilen von $G$ linear unabhängig sind \footnote{Nachweis lineare Unabhängigkeit (allgemein): \\
$\sum a_i v_i=0$. Vektoren sind dann unabhängig, wenn Gleichung nur lösbar, wenn alle $a_i=0$.}, bilden  sie eine Basis eines Vektorraumes. Folglich ist $G$ eine Generatormatrix.\\
Es gilt $\begin{pmatrix}
b_1 & b_2 & b_3
\end{pmatrix}\cdot \begin{pmatrix}
1 & 0 & 0 & 1\\
0 & 1 & 0 & 1 \\
0 & 0 & 1 & 1
\end{pmatrix}= \begin{pmatrix}
b_1 & b_2 & b_3 & \sum_{i=1}^3 b_i
\end{pmatrix}$ (alle Linearkombinationen der Matrix $G$). \\
Der durch $G$ induzierte Code ist daher $C=\left\lbrace\begin{pmatrix}
b_1 & b_2 & b_3 & \sum_{i=1}^3 b_i
\end{pmatrix}|b_1,b_2,b_3 \in \{0,1\}\right\rbrace$. \\
Dies ist der Parity-Check-Code der Länge 4.\bigskip\\
Wenn $C$ ein linearer Code mit Parity-Check-Matrix $A$ (Achtung $C\not = A$) und Generatormatrix $G$ ist, dann können wir $G$ zum Codieren und $A$ zum Decodieren verwenden.

\subparagraph{Bsp.:} Wir betrachten den Parity-Check-Code der Länge 4. \\
Wir wollen die Nachricht $(1\;0\;1)$ codieren:\\
Das entsprechende Codewort ist $\begin{pmatrix}
1 & 0 & 1
\end{pmatrix}\cdot \begin{pmatrix}
1 & 0 & 0 & 1\\
0 & 1 & 0 & 1 \\
0 & 0 & 1 & 1
\end{pmatrix}= \begin{pmatrix}
1 & 0 & 1 & 0
\end{pmatrix}$. \\
Zum Dekodieren berechnen wir $\begin{pmatrix}
1 & 1 & 1 & 1
\end{pmatrix}\cdot \begin{pmatrix}
1\\
0\\
1\\
0
\end{pmatrix}= 0$.\\
In diesem Fall wurde die Nachricht fehlerfrei übertragen.\bigskip\\
Vorteil dieses Ansatzes: Mit Hilfe einer Generatormatrix lassen sich beliebige weitere Codes definieren. Dies wird insbesondere zur Konstruktion von fehlerkorrigierenden Codes genutzt.\medskip\\
Eine auf diese Weise systematisch konstruierbare Klasse von Codes sind die Hemming-Codes.\\
Die Generatormatrix eines Hemming-Codes besteht aus allen Vektoren in $\{0,1\}^n$ außer dem Nullvektor. Auf diese Weise lassen sich fehlerkorrigierende Codes erzeugen.\\
Die Codierung und Decodierung (einschließlich der Fehlerkorrektur) lässt sich für Hamming-Codes sehr effizient durch Matrix-Vektor-Operationen realisieren. Es gibt dazu eine umfangreiche Theorie.\\
Die Hamming-Codes sind eine weit verbreitete und genutzte Klasse von fehlerkorrigierenden Codes. Hamming-Codes sind perfekte Codes.
\bigskip\\
Anhand der Parity-Check Generatormatrix lässt sich die Struktur einer Generatormatrix erkennen: In den ersten Spalten steht die Einheitsmatrix, um im Codewort wiederden Input wieder zu bekommen. Alle weiteren Spalten sind zum verifizieren da. \\
$\begin{pmatrix}
1 & 0 & 0 &| 1\\
0 & 1 & 0 &| 1\\
0 & 0 & 1 &| 1
\end{pmatrix}$. Dem enstprechend auch dieser (Hemming-)Code $\begin{pmatrix}
1 & 0 & 0 &| 0 & 1 & 1 & 1\\
0 & 1 & 0 &| 1 & 0 & 1 & 1\\
0 & 0 & 1 &| 1 & 1 & 0 & 1
\end{pmatrix}$\\
\begin{tikzpicture}[scale=.3]
\draw  (-6.5,12) rectangle (-3.5,9);
\draw  (-3.5,12) rectangle (-0.5,9);
\draw  (-6.5,9) rectangle (-3.5,6);
\draw  (-3.5,9) rectangle (-0.5,6);
\node at (-5,10.5) {$\bullet$};
\node at (-2,10.5) {$\bullet$};
\node at (-2,7.5) {$\bullet$};
\node at (-5,7.5) {$\bullet$};
\node [right] at (1,9) {perfekter Code};

\draw (-5,3) ellipse (1.4 and 1.4);
\draw (-2,3) ellipse (1.4 and 1.4);
\draw (-2,0) ellipse (1.4 and 1.4);
\draw (-5,0) ellipse (1.4 and 1.4);
\node at (-5,3) {$\bullet$};
\node at (-2,3) {$\bullet$};
\node at (-2,0) {$\bullet$};
\node at (-5,0) {$\bullet$};
\node [right] at (1,1.5) {nicht perfekter Code};
\draw  (-6.5,4.5) rectangle (-0.5,-1.5);
\node [red] at (-3.5,1.5) {$\bullet$};
\draw [-latex, red] (-3.5,-2.5) node[below]{nicht erfasst} -- (-3.5,1);
\end{tikzpicture}

\chapter{Ausblick: Klassische Kryptoverfahren}
\section{Caesar-Chiffre}
Verschiebechiffre:\\
Verschiebetabelle, z.B.\\
\begin{tabular}{c c c c c c c c c c c c c c c c c c c c c c c c c c c}
A&B&C&D&E&F&...&U&V&W&X&Y&Z\\
X&Y&Z&A&B&C&...&R&S&T&U&V&W\\
\end{tabular}\\
Chiffrierung: Aus DRESDEN wird AOBPABK\\
Die Chiffrierung und Dechiffrierung lasst sich in der Gruppe $\mathbb{Z}_{26}(+)$ wie folgt darstellen. Dazu werden die Buchstaben von $0$ bis $25$ nummeriert. Der Schlüssel $k$ ist ein Element aus $\mathbb{Z}_{26}$.\\
Chiffrierung: $E_k(x)=(x+k)mod\;26$ \footnote{E wie encrypt, D wie decrypt}\\
Dechiffrierung: $D_k(x)=(x-k)mod\;26$\\
Wesentliche Schwäche des Verfahrens: Es gibt nur 26 Schlüssel. Sicherheit des Verfahrens beruht darauf, dass der Angreifer das Verfahren nicht kennt (Prinzip „Security by Obscurity“).
\section{Vigenère-Verfahren}
Idee: Mehrere Verschiebechiffren mit unterschiedlichen Schlüsseln kombinieren. Der Schlüssel für das Vigenère-Verfahren ist ein Schlüsselwort (z.B. DRESDEN), wobei jeder Buchstabe des Schlüsselwortes der Verschiebeschlüssel des Caesar-Verfahrens ist.\\
Verschiebetabelle dazu:\\
\begin{tabular}{ c| c c c c c c c c c c c c c c c c c c c c c c c c c c c c c c}
A&A&B&C&D&E&F&...&U&V&W&X&Y&Z\\
\hline
B & B & C & D & E &F& G& ...   & W & X & Y & Z &A\\
C & C &D&E&F&G&H&...&V&W&X&Y&Z&A&B\\
D & ...\\
...
\end{tabular}
\subparagraph{Bsp.:} \parskp
\begin{tabular}{l c c c c c c c c c c c c c c }
Klartext & A&T&T&A&C&K&A&T&D&A&W&N\\
Schlüssel & L&E&M&O&N&L&E&M&O&N&L&E\\
Geheimtext & L&X&F&O&P&V&E&F&R&N&H&R
\end{tabular}\\
Mathematische Beschreibung durch Operation in $\mathbb{Z}_{26}(+)$:\\
Sei $K_1, ... ,K_n$ der mehrfach aneinander gelegte Schlüssel, wobei $n$ die Länge der Nachricht ist.\\
Chiffrierung: $E_k(M_i)=(M_i+K_i)mod\;26$\\
Dechiffrierung: $D_k(C_i)=(C_i-K_i)mod \; 26$\\
wobei $M_1, ..., M_n$ die Nachricht und $C_1, ..., C_n$ der Geheimtext sind.\\
Bestimmung der Schlüssellänge: Dazu gibt es verschiedene Verfahren:
\begin{itemize}
\item Folgende Beobachtung: Die gleichen Buchstaben werden mit dem gleichen Verschiebeschlüssel auf den gleichen Geheimbuchstaben abgebildet. In einem genügend langen Text gibt es gleiche Folgen von Klartextbuchstaben, die mit den gleichen Verschiebeschlüsseln auf die gleichen Folgen von Geheimbuchstaben abgebildet werden.
\subparagraph{Bsp.:} \parskp
\begin{tabular}{l c c c c c c c c c c c c c c c c c c c c c c  c c c c c c c c c }
Klartext 		& \tred{CRYPTO}&ISSHORTFOR&\tred{CRYPTO}&GRAPHY\\
Schlüssel 	& \torange{ABCDAB}&CDABCDABCD&\torange{ABCDAB}&CDABCD\\
Geheimtext 	& \tred{CSASTP}&KVSIQUTGQU&\tred{CSASTP}&IUAQJB\\
\end{tabular}\\
Idee: Im Geheimtext nach lägeren Vorkommen gleicher Folgen suchen, Abstände bestimmen, kgV davon bilden.\\
Damit lässt sich die Chiffre wie folgt brechen: Schreiben Geheimtext in eine Tabelle der Breite Schlüsselwortlänge\\
\begin{tabular}{c c c c}
A &B &C&D\\
\hline
C & S & A & S\\
T & P & K & V\\
S & I & Q & U
\end{tabular}\\
Jede Spalte der Tabelle ist dann eine Verschiebechiffre. In deutschen Texten ist das E der häufigtste Buchstabe. Durch eine Analyse der Häufigkeiten der Buchstaben in jeder Spalte lässt sich der Schlüssel bestimmen.
\end{itemize}

\chapter{Wiederholung}
\section{Beweistechniken}
\subsection{Induktion}
Anwendbar für Aussagen wie $\forall_n A(n)$
\begin{itemize}
\item Induktions-Anfang: $A(1)$ muss wahr sein!
\item Induktions-Voraussetzung: Es gelte $A(n)$.
\item Induktions-Schritt: $A(n)\to A(n+1)$
\end{itemize}
Also muss gelten $A(1)\vee (A(n)\to A(n+1)) = \top$
\subsection{Beweis durch Widerspruch}
$A \Leftrightarrow \neg A \to \bot$\\
Anfang: Angenommen $A$ gilt nicht. Dann Widerspruch suchen.\\
Negation von $A$? $\rightarrow$ aus $\exists$ wird $\forall$ und anders herum. \\
Die Beweisführung muss nachvollziehbar sein, sie müssen also auf bekannten Aussagen (aus Vorlesung oder Übung) fundieren.
\begin{quote}
„Es ist nicht wie in Gemeinschaftskunde, wo gilt, dass man bei 3 Seiten Text auf jeden Fall Punkte bekommt. Man bekommt nur Punkte für richtige Folgerungen, egal wie lang der Beweis ist.“ (Hollas, 2016)
\end{quote}

\section{Kombinatorik}
$|A^n|=|A|^n$ (Bspw.: Möglichkeiten ein Zahlenschloss zu öffnen. Herangehensweise: Anzahl Kombinationen=Kreuzprodukt einer Menge)\\
Bspw.: $|\{0,..,9\}^4|=10^4$\\
Visualisierung: als Baum. Jeder Knoten hat 10 Knoten/Blätter. Die Gesamtmenge ist dann die Summe aller Blätter\\
Anderes Beispiel: Zahlenschloss mit 1*alphabetisch und 1*Zahlen:\\
$|\{a,...,z\}\times \{0,...,9\}= |\{a,...,z\}| \cdot |\{0,...,9\}| =26\cdot 10$ \bigskip\\
Anders: Auswahl aus einer Menge:\\
Lotto: $\nok{49}{6}$\\
Vollständiger Graph: $V=\{1,...,n\}$\\
Mögliche Anzahl von Knoten: $\nok{|v|}{2}$\\
Möglichkeiten Anziehsachen: $n+m+1$ (n: Wintermäntel, m: Sommermäntel, 1: gar nichts) (was lässt sich frei/nicht frei kombinieren) vgl. Übungsaufgabe für Kombination von Modulen\\
$\sum_{k=0}^{n}\nok{n}{k}=2^n$ (Anzahl aller Teilmengen einer n-elementigen Mengen)\medskip\\
Möglichkeiten Jacken an Kleiderhaken zu hängen (nicht alle Jacken müssen aufgehangen werden): $n$ Kleiderjacken, $k$ Jacken $\Rightarrow$ $(k+1)^n$. Veranschaulichungen: Belegung der Gaderobe als Menge: $\{(h_1,...,h_n)|h_i \in J \cup \{leer\}\}$. Die Mächtigkeit daven ist $(J\cup \{leer\})^n=(|J|+1)^n=(k+1)^n$.\\
Möglichkeiten 3 Studenten auf 6 Plätze zu verteilen (jeder Student ist einzigartig und muss einen Sitzplatz haben). Möglichkeiten für den ersten Studenten sich einen Sitz auszusuchen: 6. Möglichkeiten für den 2.: 5, für den 3.: 4. Also $6\cdot 5 \cdot 4$





\printbibliography
\end{document}