\section{Überblick}
\subsection{FEC-Verfahren (Forward Error Correction)}
\slides{20-sicherungsprotokolle_print}{3}
Potentielle als Echtzeitanwendung.

\subsection{ARQ-Verfahren (Automatic Repeat Request)}
\slides{20-sicherungsprotokolle_print}{4}
Nicht gut für Echtzeitanwendungen, da Verzögerung.

\subsection{Hybride FEC/ARQ-Systeme}
\slides{20-sicherungsprotokolle_print}{5}

\subsection{Coderate}
\slides{20-sicherungsprotokolle_print}{6}

\subsection{Details zum ARQ-Verfahren}
\subsubsection{Fehlerereignisse}
\slides{20-sicherungsprotokolle_print}{7}
\subsubsection{Paketfehlerrate}
Bei der Paketfehlerrate muss berücksichtigt werden, wie oft das Paket erfolgreich als fehlerhaft erkannt wird und erneut gesendet wird:
\slides{20-sicherungsprotokolle_print}{8}
\subsubsection{Durchsatz}
\slides{20-sicherungsprotokolle_print}{9}
\subsubsection{Beispiel}
\slides{20-sicherungsprotokolle_print}{10}
D.h. maximal $84\%$ sind als Durchsatz zu erwarten!
\subsubsection{Unterteilung}
\slides{20-sicherungsprotokolle_print}{11}

\section{Stop-and-Wait}
\slides{20-sicherungsprotokolle_print}{12}
\subsection{Durchsatz}
\slides{20-sicherungsprotokolle_print}{13}
$T_W=2T_{AB}+T_{ACK}$
\subsection{Beispiel Satellitenkanal}
\slides{20-sicherungsprotokolle_print}{14}
Sehr ungünstig für Stop-and-Wait Protokoll, da das Verhältnis der Paketgröße zur Wartezeit zu ungünstig ist (zu kleine Pakete/zu lange Wartezeit) $\Rightarrow$ die Übertragungszeit ist zu lang.
\subsection{Beispiel Rechnernetz}
Abschätzung des Durchsatzes:
\slides{20-sicherungsprotokolle_print}{15}

\section{Go-Back-N-Protokoll (GBN)}
\slides{20-sicherungsprotokolle_print}{16}
\subsection{Optimale Fenstergröße}
\slides{20-sicherungsprotokolle_print}{17}
Hinweise Notation: $\lceil a \rceil \Rightarrow a$ wird aufgerundet 
\subsection{Sendefenster der Größe N}
\slides{20-sicherungsprotokolle_print}{18}
\subsection{Durchsatz}
\slides{20-sicherungsprotokolle_print}{19}
Ist zwar langsamer als folgendes (SR), dafür braucht Empfänger aber keinen Puffer!

\section{Selective Repeat-Protokoll (SR)}
\slides{20-sicherungsprotokolle_print}{20}
\subsection{Sendefenster}
\slides{20-sicherungsprotokolle_print}{21}
\subsection{Empfangsfenster}
\slides{20-sicherungsprotokolle_print}{22}
\subsection{Durchsatz}
\slides{20-sicherungsprotokolle_print}{23}

\section{Ergänzungen}
\subsection{Paketnummerierung}
\slides{20-sicherungsprotokolle_print}{24}
\subsection{Bidirektionale Datenübertragung}
\slides{20-sicherungsprotokolle_print}{25}
\subsection{Vergleich der Sicherungsprotokolle}
\slides{20-sicherungsprotokolle_print}{26}
\subsection{Beispiel mit verschiedenen ARQ-Protokollen}
(Richtfunkanlage)
\slides{20-sicherungsprotokolle_print}{27}
\subsection{Zusammenfassung}
\slides{20-sicherungsprotokolle_print}{28}
\subsection{Beleg Stop-and-Wait}
Zustände Sender:
\begin{itemize}
\item Wait for Data
\item Wait for ACK
\end{itemize}
Übergange:\\
Wait for Data $\to$ SW called send / send packet 0/1 $\to$ Wait for ACK\\
Wait for ACK $\to$ received ACK /  $\to$ Wait for Data\\
Wait for ACK $\to$ timeout / resend packet || stop $\to$ Wait for ACK \\
Wait for ACK $\to$ wrong ACK || wrong Session ID / $\to$ Wait for ACK