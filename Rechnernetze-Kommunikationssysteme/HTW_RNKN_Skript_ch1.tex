\section{Moderne Kommunikation}
\slides{05-Ueberblick}{11}
Bandbreite: zur Zeit noch beschränkt, denkbar auch, dass jeder auf allen Bandbreiten. WLAN und Bluetooth sind schon auf einer Bandbreite und bekommen es auch hin.\\
Sendeleistung: Nicht „zu laut reden“, damit sich andere auch noch verstehen.

\section{Internet}
\slides{05-Ueberblick}{15}
\subsection{Dienste}
\slides{05-Ueberblick}{16}
\subsection{Rechnernetz/Verteiltes System}
\slides{05-Ueberblick}{18}
\subsection{ISO OSI 7-Schichtenmodell}
\slides{05-Ueberblick}{22}
\subsubsection{Beispiel}
\slides{05-Ueberblick}{23}
\subsection{Anwendungsschicht}
\subsubsection*{Sockets}
\slides{05-Ueberblick}{25}
\subsection{Transportschicht}
\subsubsection*{Sicherungsprotokolle}
\slides{05-Ueberblick}{27}
\subsection{Vermittlungsschicht}
\subsubsection*{Vermittlungstechnik}
\slides{05-Ueberblick}{29}
\subsubsection*{Paketvermittlung: statisches Multiplexing}
\slides{05-Ueberblick}{30}
\subsubsection*{Quellen der Verzögerung}
\slides{05-Ueberblick}{31}
\slides{05-Ueberblick}{32}
\subsection{Durchsatz}
\slides{05-Ueberblick}{35}