% Header aus der Vorlage
\documentclass[a4paper,11pt, footheight=26pt
%,twoside
]{scrreprt}
\usepackage[head=23pt]{geometry}	% head=23pt umgeht Fehlerwarnung, dafür größeres "top" in geometry
\geometry{a4paper, top=30mm, bottom=22mm,headsep=10mm, footskip=12mm
, left=20mm, right=20mm
%, inner=27mm, outer=13mm
}

\setcounter{secnumdepth}{3}	% zählt auch subsubsection
\setcounter{tocdepth}{3}	% Inhaltsverzeichnis bis in subsubsection

% Zeile 2 (,twoside) und 7 (inner=...) für eine Druckversion (doppelseitig) ent-kommentieren (Rand für Hefter)

% Input inkl. Umlaute, Silbentrennung
\usepackage[T1]{fontenc}
\usepackage[utf8]{inputenc}
\usepackage[ngerman]{babel}
\usepackage{csquotes}	% Anführungszeichen
\usepackage{eurosym}

% HTW Corporate Design: Arial (Helvetica)
\usepackage{helvet}
\renewcommand{\familydefault}{\sfdefault}

% Style-Aufhübschung
\usepackage{soul, color}	% Kapitälchen, Unterstrichen, Durchgestrichen usw. im Text
\usepackage{scrlayer-scrpage}	% Kopf-/Fußzeile
%\usepackage{titleref}
\usepackage[perpage]{footmisc}	% Fußnotenzählung Seitenweit, nicht Dokumentenweit
\renewcommand*{\thefootnote}{\fnsymbol{footnote}}	% Fußnoten-Symbole anstatt Zahlen
\renewcommand*{\titlepagestyle}{empty} % Keine Seitennummer auf Titelseite

% Mathe usw.
\usepackage{amssymb}
\usepackage[fleqn]{amsmath}	% fleqn: align-Umgebung rechtsbündig
\usepackage{xcolor}
\usepackage{esint}	% Schönere Integrale, \oiint vorhanden
\everymath=\expandafter{\the\everymath\displaystyle}	% Mathe Inhalte werden weniger verkleinert
\usepackage{wasysym}	% mehr Symbole, bspw \lightning
% Auch arcus-Hyperbolicus-Funktionen
\DeclareMathOperator{\arccot}{arccot}
\DeclareMathOperator{\arccosh}{arccosh}
\DeclareMathOperator{\arcsinh}{arcsinh}
\DeclareMathOperator{\arctanh}{arctanh}
\DeclareMathOperator{\arccoth}{arccoth} 
% Mathe in Anführungszeichen:
\newsavebox{\mathbox}\newsavebox{\mathquote}
\makeatletter
\newcommand{\mq}[1]{% \mathquotes{<stuff>}
  \savebox{\mathquote}{\text{"}}% Save quotes
  \savebox{\mathbox}{$\displaystyle #1$}% Save <stuff>
  \raisebox{\dimexpr\ht\mathbox-\ht\mathquote\relax}{"}#1\raisebox{\dimexpr\ht\mathbox-\ht\mathquote\relax}{''}
}
\makeatother

% tikz usw.
\usepackage{tikz}
\usepackage{pgfplots}
\pgfplotsset{compat=1.11}	% Umgeht Fehlermeldung
\usetikzlibrary{graphs}
%\usetikzlibrary{through}	% ???
\usetikzlibrary{arrows}
\usetikzlibrary{arrows.meta}	% Pfeile verändern / vergrößern: \draw[-{>[scale=1.5]}] (-3,5) -> (-3,3);
\usetikzlibrary{automata,positioning} % Zeilenumbruch im Node node[align=center] {Text\\nächste Zeile} automata für Graphen
\usetikzlibrary{matrix}
\usetikzlibrary{patterns}	% Schraffierte Füllung
\tikzstyle{reverseclip}=[insert path={	% Inverser Clip \clip
	(current page.north east) --
	(current page.south east) --
	(current page.south west) --
	(current page.north west) --
	(current page.north east)}
% Nutzen: 
%\begin{tikzpicture}[remember picture]
%\begin{scope}
%\begin{pgfinterruptboundingbox}
%\draw [clip] DIE FLÄCHE, IN DER OBJEKT NICHT ERSCHEINEN SOLL [reverseclip];
%\end{pgfinterruptboundingbox}
%\draw DAS OBJEKT;
%\end{scope}
%\end{tikzpicture}
]	% Achtung: dafür muss doppelt kompliert werden!
\usepackage{graphpap}	% Grid für Graphen
\tikzset{every state/.style={inner sep=2pt, minimum size=2em}}

% Tabular
\usepackage{longtable}	% Große Tabellen über mehrere Seiten
\usepackage{multirow}	% Multirow/-column: \multirow{2[Anzahl der Zeilen]}{*[Format]}{Test[Inhalt]} oder \multicolumn{7[Anzahl der Reihen]}{|c|[Format]}{Test2[Inhalt]}
\renewcommand{\arraystretch}{1.3} % Tabellenlinien nicht zu dicht
\usepackage{colortbl}
\arrayrulecolor{gray}	% heller Tabellenlinien

% Nützliches
\usepackage{verbatim}	% u.a. zum auskommentieren via \begin{comment} \end{comment}
\usepackage{tabto}	% Tabs: /tab zum nächsten Tab oder /tabto{.5 \CurrentLineWidth} zur Stelle in der Linie
\NumTabs{6}	% Anzahl von Tabs pro Zeile zum springen
\usepackage{listings} % Source-Code mit Tabs
\usepackage{lstautogobble} 
\usepackage{enumitem}	% Anpassung der enumerates
\setlist[enumerate,1]{label=\arabic*.)}	% global andere Enum-Items
\renewcommand{\labelitemiii}{$\scriptscriptstyle ^\blacklozenge$} % global andere 3. Item-Aufzählungszeichen
\usepackage{letltxmacro} % neue Definiton von Grundbefehlen
% Nutzen:
%\LetLtxMacro{\oldemph}{\emph}
%\renewcommand{\emph}[1]{\oldemph{#1}}

% Einrichtung von lst
\lstset{
basicstyle=\ttfamily, 
mathescape=true, 
%escapeinside=^^, 
autogobble, 
tabsize=2,
basicstyle=\footnotesize\sffamily\color{black},
frame=single,
rulecolor=\color{lightgray},
numbers=left,
numbersep=5pt,
numberstyle=\tiny\color{gray},
commentstyle=\color{gray},
keywordstyle=\color{green},
stringstyle=\color{orange},
morecomment=[l][\color{magenta}]{\#}
%showspaces=false,
showstringspaces=false,
breaklines=true,
literate=%
    {Ö}{{\"O}}1
    {Ä}{{\"A}}1
    {Ü}{{\"U}}1
    {ß}{{\ss}}1
    {ü}{{\"u}}1
    {ä}{{\"a}}1
    {ö}{{\"o}}1
    {~}{{\textasciitilde}}1
}
\usepackage{scrhack} % Fehler umgehen
\def\ContinueLineNumber{\lstset{firstnumber=last}} % vor lstlisting. Zum wechsel zum nicht-kontinuierlichen muss wieder \StartLineAt1 eingegeben werden
\def\StartLineAt#1{\lstset{firstnumber=#1}} % vor lstlisting \StartLineAt30 eingeben, um bei Zeile 30 zu starten
\let\numberLineAt\StartLineAt

% BibTeX
\usepackage[backend=bibtex, bibencoding=ascii]{biblatex}	% BibTeX
\usepackage{makeidx}
%\makeglossary
%\makeindex

% Grafiken
\usepackage{graphicx}
\usepackage{epstopdf}	% eps-Vektorgrafiken einfügen

% pdf-Setup
\usepackage{pdfpages}
\usepackage[bookmarks,%
bookmarksopen=false,% Klappt die Bookmarks in Acrobat aus
colorlinks=true,%
linkcolor=black,%
citecolor=red,%
urlcolor=green,%
]{hyperref}

% Titel, Autor usw. werden vor dem Anfang des Dokuments in einem Rutsch definiert…
\newcommand{\DTitel}[1]{\newcommand{\Dokumententitel}{#1}}
\newcommand{\DUntertitel}[1]{\newcommand{\Dokumentenuntertitel}{#1}}
\newcommand{\DAutor}[1]{\newcommand{\Dokumentenautor}{#1}}
\newcommand{\DNotiz}[1]{\newcommand{\Dokumentennotiz}{#1}}
\newcommand{\DSign}[1]{\newcommand{\Dokumentensignatur}{#1}}
\DSign{\footnotesize{\textcolor{darkgray}{Mitschrift von\\ \Dokumentenautor}}}
\newcommand{\Autorformat}[1]{\textcolor{darkgray}{Mitschrift von #1}}
% … Deswegen folgendes erst Nach Dokumentenbeginn ausführen:
\AtBeginDocument{
	\hypersetup{
		pdfauthor={\Dokumentenautor},
		pdftitle={HTW Dresden | \Dokumententitel - \Dokumentenuntertitel},
	}
	\automark[section]{section}
	\automark*[subsection]{subsection}
	\pagestyle{scrheadings}
	\ihead{\includegraphics[height=1.7em]{\workingdir LaTeX_master/HTW-Logo.eps}}
	\ohead{\Dokumententitel}
	\cfoot{\pagemark}
	\ofoot{\Dokumentensignatur}
	% Titelseite
	\title{\includegraphics[width=0.35\textwidth]{\workingdir LaTeX_master/HTW-Logo.eps}\\\vspace{0.5em}
	\Huge\textbf{\Dokumententitel} \\\vspace*{0,5cm}
	\Large \Dokumentenuntertitel \\\vspace*{4cm}}
	\author{\Autorformat{\Dokumentenautor} \vspace*{1cm}\\\Dokumentennotiz}
}

%% EIGENE BEFEHLE

% Arbeitsordner (in Abhängigkeit vom Master)
\newcommand{\workingdir}{../}

% Farbdefinitionen
\definecolor{red}{RGB}{180,0,0}
\definecolor{green}{RGB}{75,160,0}
\definecolor{blue}{RGB}{0,75,200}
\definecolor{orange}{RGB}{255,128,0}
\definecolor{yellow}{RGB}{255,245,0}
\definecolor{purple}{RGB}{75,0,160}
\definecolor{cyan}{RGB}{0,160,160}
\definecolor{brown}{RGB}{120,60,10}

\definecolor{itteny}{RGB}{244,229,0}
\definecolor{ittenyo}{RGB}{253,198,11}
\definecolor{itteno}{RGB}{241,142,28}
\definecolor{ittenor}{RGB}{234,98,31}
\definecolor{ittenr}{RGB}{227,35,34}
\definecolor{ittenrp}{RGB}{196,3,125}
\definecolor{ittenp}{RGB}{109,57,139}
\definecolor{ittenpb}{RGB}{68,78,153}
\definecolor{ittenb}{RGB}{42,113,176}
\definecolor{ittenbg}{RGB}{6,150,187}
\definecolor{itteng}{RGB}{0,142,91}
\definecolor{ittengy}{RGB}{140,187,38}

% Textfarbe ändern
\newcommand{\tred}[1]{\textcolor{red}{#1}}
\newcommand{\tgreen}[1]{\textcolor{green}{#1}}
\newcommand{\tblue}[1]{\textcolor{blue}{#1}}
\newcommand{\torange}[1]{\textcolor{orange}{#1}}
\newcommand{\tyellow}[1]{\textcolor{yellow}{#1}}
\newcommand{\tpurple}[1]{\textcolor{purple}{#1}}
\newcommand{\tcyan}[1]{\textcolor{cyan}{#1}}
\newcommand{\tbrown}[1]{\textcolor{brown}{#1}}

% Umstellen der Tabellen Definition
\newcommand{\mpb}[1][.3]{\begin{minipage}{#1\textwidth}\vspace*{3pt}}
\newcommand{\mpe}{\vspace*{3pt}\end{minipage}}

\newcommand{\resultul}[1]{\underline{\underline{#1}}}
\newcommand{\parskp}{$ $\\}	% new line after paragraph
\newcommand{\corr}{\;\widehat{=}\;}
\newcommand{\mdeg}{^{\circ}}

\newcommand{\nok}[2]{\begin{pmatrix}#1\\#2\end{pmatrix}}	% n über k
\newcommand{\mtr}[1]{\begin{pmatrix}#1\end{pmatrix}}	% Matrix
\newcommand{\dtr}[1]{\begin{vmatrix}#1\end{vmatrix}}	% Determinante (Betragsmatrix)
\renewcommand{\vec}[1]{\underline{#1}}	% Vektorschreibweise
\newcommand{\imptnt}[1]{\colorbox{red!30}{#1}}	% Wichtiges

%\bibliography{../Literatur/HTW_Literatur.bib}

% Definition von Titel, Autor usw.
\DTitel{tikz Prefabs}
\DUntertitel{}
\DAutor{Falk Jonatan Strube}
\DNotiz{}

\begin{document}

\pagenumbering{Roman}

%\maketitle
%\newpage
\tableofcontents
\newpage

\pagenumbering{arabic}

\section{TU Mathe 3}
Integrationsbereich y 
\begin{tikzpicture}[scale=0.75]
%Punkte zeichnen
\fill[black] (2,4) circle (2pt);
\fill[black] (-1,1) circle (2pt);
% Füllung
%\fill[yellow] plot[domain=-1:2] (\x, {\x+2}) -- plot[domain=-1:2] (\x,{\x*\x});
\pattern [pattern=north west lines,pattern color=yellow] plot[domain=-1:2] (\x, {\x+2}) -- plot[domain=-1:2] (\x,{\x*\x});
%Graphen
\draw [gray,thick, domain=-2.1:2.1] plot[smooth] (\x,{\x*\x});
\draw [gray,thick, domain=-2.2:2.2] plot[smooth] (\x,{\x+2});
%Graphen Integrationsbereich
\draw [orange,thick, domain=-1:2] plot[smooth] (\x,{\x*\x});
\draw [purple,thick, domain=-1:2] plot[smooth] (\x,{\x+2});
% Achsen zeichnen, zuerst hier, da clip die Zeichenflaeche beschneidet
\draw[->,thick] (-2.6,0) -- (2.4,0) node[right] {$x$};
\draw[->,thick] (0,-.2) -- (0,4.3) node[above right] {$y$};
% Achsen beschriften
\foreach \x in {-2,-1,0,1,2}
\draw (\x,-.1) -- (\x,.1) node[below=5pt] {$\scriptstyle\x$};
\foreach \y in {1,2,3,4}
\draw (-.1,\y) -- (.1,\y) node[left=5pt] {$\scriptstyle\y$};
% Grid einzeichnen
\draw [dashed, cyan] (-1, -.3) -- (-1,2);
\draw [dashed, cyan] (2, -.3) -- (2,4.5);
%Pfeil von unten kommend: Integrationsrichtung inneres Integral
\draw [->,thick, brown] (-2, .6) -- (-2,1.6);
\end{tikzpicture} \\
%
Integrationsbereich x1 
\begin{tikzpicture}[scale=0.5]
%Punkte zeichnen
\fill[black] (2,4) circle (2pt);
\fill[black] (-1,1) circle (2pt);
% Füllung
%\fill[yellow] plot[domain=-1:2] (\x, {\x+2}) -- plot[domain=-1:2] (\x,{\x*\x});
\pattern [pattern=north east lines,pattern color=yellow] plot[domain=-1:1] (\x, {1}) -- plot[domain=-1:1] (\x,{\x*\x});
\pattern [pattern=north west lines,pattern color=yellow] plot[domain=-1:2] (\x, {\x+2}) -- plot[domain=-1:1] (\x, {1}) -- plot[domain=1:2] (\x,{\x*\x});
%Graphen
\draw [gray,thick, domain=-2.1:2.1] plot[smooth] (\x,{\x*\x});
\draw [gray,thick, domain=-2.2:2.2] plot[smooth] (\x,{\x+2});
%Graphen Integrationsbereich
\draw [orange,thick, domain=-1:0] plot[smooth] (\x,{\x*\x});
\draw [purple,thick, domain=0:1] plot[smooth] (\x,{\x*\x});
\draw [purple!75,thick, domain=1:2] plot[smooth] (\x,{\x*\x});
\draw [orange!75,thick, domain=-1:2] plot[smooth] (\x,{\x+2});
% Achsen zeichnen, zuerst hier, da clip die Zeichenflaeche beschneidet
\draw[->,thick] (-2.6,0) -- (2.4,0) node[right] {$x$};
\draw[->,thick] (0,-.2) -- (0,4.3) node[above right] {$y$};
% Achsen beschriften
\foreach \x in {-2,-1,0,1,2}
\draw (\x,-.1) -- (\x,.1) node[below=5pt] {$\scriptstyle\x$};
\foreach \y in {1,2,3,4}
\draw (-.1,\y) -- (.1,\y) node[left=5pt] {$\scriptstyle\y$};
% Grid einzeichnen
\draw [dashed, cyan] (-1.5, 0) -- (1.5,0);
\draw [dashed, cyan] (-1.7, 1) -- (2, 1);
\draw [dashed, green] (-1.75, 1) -- (2, 1);
\draw [dashed, green] (-2.2, 4) -- (2.2, 4);
%Pfeil von links kommend; Integrationsrichtung inneres Integral
\draw [->,thick, brown] (-2.7,1.3) -- (-1.7,1.3);
\end{tikzpicture} \\
%
Integrationsbereich x-y 
\begin{tikzpicture}[scale=0.5]
%Punkte zeichnen
\fill[rotate=-90,black] (-2,4) circle (2pt);
\fill[rotate=-90,black] (1,1) circle (2pt);
% Füllung
%\fill[yellow] plot[domain=-1:2] (\x, {\x+2}) -- plot[domain=-1:2] (\x,{\x*\x});
\pattern [rotate=-90,pattern=north east lines,pattern color=yellow] plot[domain=-1:1] (\x, {1}) -- plot[domain=-1:1] (\x,{\x*\x});
\pattern [rotate=-90,pattern=north west lines,pattern color=yellow] plot[domain=-2:-1] (\x,{\x*\x}) -- plot[domain=-1:1] (\x, {1}) -- plot[domain=-2:1] (\x, {-\x+2});
%Graphen
\draw [rotate=-90,gray,thick, domain=-2.1:2.1] plot[smooth] (\x,{\x*\x});
\draw [rotate=-90,gray,thick, domain=-2.2:2.2] plot[smooth] (\x,{-\x+2});
%Graphen Integrationsbereic
\draw [rotate=-90,orange,thick, domain=0:1] plot[smooth] (\x,{\x*\x});
\draw [rotate=-90,purple,thick, domain=-1:0] plot[smooth] (\x,{\x*\x});
\draw [rotate=-90,purple!75,thick, domain=-2:-1] plot[smooth] (\x,{\x*\x});
\draw [rotate=-90,orange!75,thick, domain=-2:1] plot[smooth] (\x,{-\x+2});
% Achsen zeichnen, zuerst hier, da clip die Zeichenflaeche beschneidet
\draw[->,thick] (-.2,0) -- (4.3,0) node[right] {$y$};
\draw[->,thick] (0,-2.6) -- (0,2.4) node[above right] {$x$};
% Achsen beschriften
\foreach \x in {-2,-1,0,1,2}
\draw (-.1, \x) -- (.1, \x) node[left=5pt] {$\scriptstyle\x$};
\foreach \y in {1,2,3,4}
\draw (\y,-.1) -- (\y,.1) node[below=5pt] {$\scriptstyle\y$};
% Grid einzeichnen
\draw [rotate=-90,dashed, cyan] (-1.5,0) -- (1.5,0);
\draw [rotate=-90,dashed, cyan] (-1.7, 1) -- (2, 1);
\draw [rotate=-90,dashed, green] (-1.75, 1) -- (2, 1);
\draw [rotate=-90,dashed, green] (-2.2, 4) -- (2.2, 4);
%Pfeil von unten kommend; Integrationsrichtung inneres Integral
\draw [->,thick, brown] (3.5, -.7) -- (3.5,.3);
\end{tikzpicture} \\
%
Kartesisches Koordinatensystem
\begin{tikzpicture}[scale=1]
% Achsen zeichnen, zuerst hier, da clip die Zeichenfl"ache beschneidet
\draw[->,thick] (-0.5,0) -- (1,0) node[right] {$y$};
\draw[->,thick] (0,-0.5) -- (0,1) node[above] {$z$};
\draw[->,thick] (.5,.5) -- (-1,-1) node[left] {$x$};
\end{tikzpicture} \\
%
Polarkoordinaten
\begin{tikzpicture}[scale=1]
% Achsen zeichnen, zuerst hier, da clip die Zeichenfl"ache beschneidet
\draw[->,thick] (.5,.5) -- (.5,1) node[above] {$z$};
\draw[->,thick] (.5,.5) -- (-.5,-.5) node[left] {$r$};
\draw[->] (.5,-.3) arc (10:130:.5cm) node[right=25pt] {$\varphi$};
\end{tikzpicture} \\
%
Zylinderkoordinaten
\begin{tikzpicture}[scale=1]
% Achsen zeichnen, zuerst hier, da clip die Zeichenfl"ache beschneidet
\draw[->,thick] (.5,.5) -- (-.5,-.5) node[left] {$r$};
\draw[->] (0,0) arc (0:340:.2cm) node[left=10pt] {$\vartheta$};
\draw[->] (.75,-.1) arc (10:130:.5cm) node[right=25pt] {$\varphi$};
\end{tikzpicture} \\
%
Fläche von Funktionen begrenzt
\begin{tikzpicture}[scale=1]
% Achsen zeichnen, zuerst hier, da clip die Zeichenfl"ache beschneidet
\draw[->,thick] (-0.5,0) -- (2.5,0) node[right] {$x$};
\draw[->,thick] (0,-0.5) -- (0,2.5) node[above] {$y$};
%Füllung
\pattern [pattern=north west lines,pattern color=yellow] plot[domain=0.396:1] (\x,{sqrt(\x)*0.5}) -- plot[domain=0.629:1.587] (\x,{sqrt(\x)}) -- plot[domain=0.396:0.629] (\x, {\x*\x*2}) -- plot[domain=1:1.587] (\x,{\x*\x*0.5}) ;
%Graphen
\draw [thick, domain=0:1.1] plot[smooth] (\x,{\x*\x*2}) node[above] {b};
\draw [thick, domain=0:2.2] plot[smooth] (\x,{\x*\x*0.5}) node[above] {a};
\draw [thick, domain=0:2.2] plot[smooth] (\x,{sqrt(\x)*0.5}) node[above] {p};
\draw [thick, domain=0:2.2] plot[smooth] (\x,{sqrt(\x)}) node[above] {q};
%Bezeichnug
\draw (2,3.5) node {$\textcolor{green}{x^2=\begin{matrix}a \\ b \end{matrix}\cdot y}$};
\draw (3.5,1) node {$\textcolor{green}{y^2=\begin{matrix}p \\ q \end{matrix}\cdot x}$};
\end{tikzpicture} \\
%
Teilabschnitt einer Funktion
\begin{tikzpicture}[scale=1]
% Achsen zeichnen, zuerst hier, da clip die Zeichenfl"ache beschneidet
\draw[->,thick] (-0.5,0) -- (2.5,0) node[right] {$x$};
\draw[->,thick] (0,-0.5) -- (0,2.5) node[above] {$y$};
%Graphen
\draw [green,thick] plot[smooth] (0,2) cos (2.5,0);
\draw[red, thick] (1.15,1.5) -- (1.65,1);
\draw[->, purple, thick] (0,0) -- (1.15,1.5) node[left=10pt, below] {$\overrightarrow{r}$};
\draw[->, dashed, purple, thick] (0,0) -- (1.65,1);
% Graphen
\draw (1.15,-1) -- (1.15,2);
\draw (1.65,-1) -- (1.65,2);
\draw (-1,1) -- (2,1);
\draw (-1,1.5) -- (2,1.5);
%Bezeichnung
\draw[<->] (1.15,-.5) -- (1.65,-.5) node[left=7pt, below] {$\mathrm{dx}$};
\draw[<->] (-.5,1) -- (-.5,1.5) node[left=7pt, below] {$\mathrm{dy}$};
\coordinate (a) at (1.15,0);
\coordinate (b) at (1.65,0);
\fill[black] (a) circle (2pt) node[left=5pt, below] {$t_1$};
\fill[black] (b) circle (2pt) node[right=5pt, below] {$t_2$};
\end{tikzpicture} \\
%
Teil eines Kreises
\begin{tikzpicture}[scale=1]
% Achsen zeichnen, zuerst hier, da clip die Zeichenfl"ache beschneidet
\draw[->,thick] (-0.5,0) -- (3,0) node[right] {$x$};
\draw[->,thick] (0,-0.5) -- (0,3) node[above] {$y$};
%Graphen
%viertel Kreis
\draw (2,0) arc (0:90:2cm);
\draw[red] (1.8,.9) arc (26.5:63:2cm) node[right=20pt, fill=white] {wenn $\mathrm{d\varphi}$ ganz klein $\rightarrow$ Gerade};
% Graphen
\draw (0,0) -- (2,1) node[below] {r};
\draw (0,0) -- (1,2);
\draw[orange] (1.4,.7) -- (0.7,1.4) node[right=30pt, fill=white] {$r \sin d\varphi \approx rd\varphi$};
\draw[blue] (0.9,1.8) -- (0.7,1.4) node[above=15pt] {$r^\prime\mathrm{d\varphi}$};
\draw[purple, ->] (.8,.4) arc (20:53:1cm) node[right] {$\mathrm{d\varphi}$} ;
\draw[gray] (0.8,1.3) arc (0:-115:.15cm) node[left=15pt] {$\sphericalangle 90^\circ$};
\draw[red, ->] (1,0) arc (0:27:1cm) node[right] {$\varphi$};
\end{tikzpicture} \\
%
Punkte in/um/auf Strecken (Umlaufintegral)
\begin{tikzpicture}[scale=.8]
% Achsen 
\draw[->,thick] (-0.2,0) -- (7.5,0);
\draw[->,thick] (0,-0.2) -- (0,6.5);
% Prefab: to [out=, in=] (,)
% Umlaufintegral mit Singularitäten
\draw[->, thick, brown] (1,1) to [out=-60, in=178] (3.5,-.5) to [out=2, in=-100] (7,2) to [out=80, in=-25] (6,7) to [out=160, in=90] (-.25,3.5) to [out=-90, in=130] (.25,2) to [out=-50, in=115] (1.02,0.98) node[below left]{$c_4$}; 
% Umlaufintegral (ohne Singularität)
\draw[->, thick , red] (2,1) to [out=-40, in=-130] (4,3) node[below right]{$c_2$};
\draw[thick , red] (3.98,2.98) to [out=50, in=-5] (4,5) to [out=-175, in=70] (3.25,4) to [out=-110, in=120] (2,1);
% direkt mit Singularität
\draw[->, thick , green] (2,1) to [out=170, in=-90] (1.25,3.25) node[left]{$c_3$};
\draw[ thick , green] (1.25,3.15) to [out=80, in=-80] (1,4) to [out=85, in=-170] (2.5,5) to [out=15, in=90] (4,5);
% direkt
\draw[->, thick , cyan] (2,1) to [out=10, in=-145] (3.5,3.5)node[right]{$c_1$};
\draw[thick , cyan] (3.48,3.48) to [out=55, in=-90] (4,5);
% Punkte
\fill[black] (2,1) circle (2pt) node[below right=-3pt] {$z_1\! (t_1)$};
\fill[black] (1,4) circle (2pt) node[above] {$z_{S_1}$};
\fill[black] (4,5) circle (2pt) node[above right] {$z_2\! (t_2)$};
\fill[black] (6,3) circle (2pt) node[above] {$z_{S_3}$};
\fill[black] (4,2) circle (2pt) node[below right=-2pt] {$z_{S_2}$};
\end{tikzpicture} \\
%
Entscheidung Laurent
\begin{tikzpicture}[level distance=10mm]
\tikzstyle{level 1}=[sibling distance=40mm]
\node { $c_{-1}$: Koeff. des ersten Gliedes des Hauptteils der Laurentreihe}
	child {node {Tabelle}}
	child {node {Entwicklung der Laurent-Reihe}};
\end{tikzpicture}\\
%
Darstellungsformen Imaginärzahlen
\begin{tikzpicture}[scale=1]
\draw[->,thick] (-1.5,0) -- (1.5,0) node[right] {$Re$};
\draw[->,thick] (0,-1.5) -- (0,1.5) node[left] {$Im$};
\fill[black] (0,0) circle (2pt) node[left=7pt,below] {$z_0$};
\draw[dashed] (1,0) arc (0:360:1cm);
\draw[->] (0,0) -- (.71,.71) node[below] {r};
\end{tikzpicture}\\
%
Entscheidungen Formel
\begin{tikzpicture}[level distance=10mm]
\tikzstyle{level 1}=[sibling distance=40mm]
\node {$f=\frac{A}{B-z}$}
	child {node {$1>\lvert \frac{z}{B}\rvert \Leftarrow \frac{A}{B(1-\frac{z}{B})}$}
	child {node {$\lvert z \rvert < \lvert B \rvert$}}}
	child {node {$\frac{A}{z(\frac{z}{B}-1)} \Rightarrow \lvert \frac{B}{z}\rvert <1$}
	child {node {$\lvert z \rvert > \lvert B \rvert$}}};
\end{tikzpicture} 


\section{HTW Mathe 1}

A geschnitten B 
\begin{tikzpicture} [scale =0.25]
\node at (-7,3) {A};
\node at (1,3) {B};
\draw  (-5,3) ellipse (3 and 3);
\draw  (-1,3) ellipse (3 and 3);
\begin{scope} % lokal halten der Wirkung von \clip
\clip (-5,3) ellipse (3 and 3);
\draw[pattern= north east lines, pattern color=gray] (-1,3) ellipse (3 and 3);
\end{scope}
\end{tikzpicture} \\
%
A und B 
\begin{tikzpicture} [scale =0.25]
\draw [pattern= north east lines, pattern color=gray]  (-5,3) ellipse (3 and 3) (-1,3) ellipse (3 and 3);
\node at (-7,3) {A};
\node at (1,3) {B};
\end{tikzpicture} \\
%
B ohne A
\begin{tikzpicture} [scale =0.25, remember picture]
\draw  (-5,3) ellipse (3 and 3);
\draw  (-1,3) ellipse (3 and 3);
\begin{scope}
\begin{pgfinterruptboundingbox} % To make sure our clipping path does not mess up the placement of the picture
\draw [clip] (-5,3) ellipse (3 and 3) [reverseclip];
\end{pgfinterruptboundingbox}
\draw[pattern= north east lines, pattern color=gray] (-1,3) ellipse (3 and 3);
\end{scope}
\node at (-7,3) {A};
\node at (1,3) {B};
\end{tikzpicture}  \\
%
A negiert
\begin{tikzpicture} [scale =0.25, remember picture]
\draw  (-9,-1) rectangle (3,7);
\draw  (-5,3) ellipse (3 and 3);
\begin{scope}
\begin{pgfinterruptboundingbox} % To make sure our clipping path does not mess up the placement of the picture
\draw [clip]  (-5,3) ellipse (3 and 3) [reverseclip];
\end{pgfinterruptboundingbox}
\draw[pattern= north east lines, pattern color=gray](-9,-1) rectangle (3,7);
\end{scope}
\node at (-7,3) {A};
\node at (1,3) {E};
\end{tikzpicture} \\
%
Städte in Ländern
\begin{tikzpicture} [scale=0.25]
\draw  (0,2) rectangle (10,0) node[pos =.5]{Berlin};
\draw  (0,-1) rectangle (10,-3) node[pos =.5]{Dresden};
\draw  (0,-4) rectangle (10,-6) node[pos =.5]{Köln};
\draw  (0,-7) rectangle (10,-9) node[pos =.5]{Paris};
\draw  (0,-10) rectangle (10,-12) node[pos =.5]{Rom};
\draw  (0,-13) rectangle (10,-15) node[pos =.5]{Neapel};
\draw  (0,-16) rectangle (10,-18) node[pos =.5]{Oslo};

\draw  (18,2) rectangle (28,0) node[pos =.5]{D};
\draw  (18,-1) rectangle (28,-3) node[pos =.5]{F};
\draw  (18,-4) rectangle (28,-6) node[pos =.5]{B};
\draw  (18,-7) rectangle (28,-9) node[pos =.5]{P};
\draw  (18,-10) rectangle (28,-12) node[pos =.5]{I};
\draw  (18,-13) rectangle (28,-15) node[pos =.5]{N};

\node at (5,3) {S};
\node at (23,3) {L};
\draw[-latex] (10,1) -- (18,1);
\draw[-latex] (10,-2) -- (18,1);
\draw[-latex] (10,-5) -- (18,1);
\draw[-latex] (10,-8) -- (18,-2);
\draw[-latex] (10,-11) -- (18,-11);
\draw[-latex] (10,-14) -- (18,-11);
\draw[-latex] (10,-17) -- (18,-14);
\end{tikzpicture}  \\
%
gerichteter Graph
\begin{tikzpicture} [scale=0.35]
\draw (0,0) circle (1) node{x};
\draw [-latex] plot[smooth, tension=1.5] coordinates {(0.2,1) (0.5,2) (-0.5,2) (-0.2,1)};
\draw[-latex] (1,0) -- (3,0);
\draw (4,0) circle (1) node{y};
\draw[-latex] (5,0) -- (7,0);
\draw (8,0) circle (1) node{z};
\draw[-latex] (1,0) -- (3,0);
\draw [-latex] plot[smooth, tension=1] coordinates {(8,-1) (6,-2) (4,-1)};
\end{tikzpicture}  \\
%
Graph-Koordinatensystem
\begin{tikzpicture}[scale=0.5]
% Achsen zeichnen
\draw[thick] (0,0) -- (2,0);
\draw[thick] (0,0) -- (0,2);
\draw[dashed] (1,0) -- (1,2);
\draw[dashed] (2,0) -- (2,2);
\draw[dashed] (0,1) -- (2,1);
\draw[dashed] (0,2) -- (2,2);

\fill[black] (0,0) circle (0.1) node[below left]{x};
\fill[black] (1,0) circle (0.1) node[below]{y};
\fill[black] (2,0) circle (0.1) node[below]{z};
\fill[black] (0,1) circle (0.1) node[left]{y};
\fill[black] (0,2) circle (0.1) node[left]{z};

\draw (0,1) circle (0.2);
\draw (0,0) circle (0.2);
\draw (1,2) circle (0.2);
\draw (2,1) circle (0.2);
\end{tikzpicture}  \\
%
reflexive Menge
\begin{tikzpicture}[scale=0.5]
% Achsen zeichnen
\draw[thick] (0,0) -- (4,0);
\draw[thick] (0,0) -- (0,4);
\foreach \x in {0,1,2,3,4}
\draw[dashed] (\x,0) -- (\x,4);
\foreach \y in {0,1,2,3,4}
\draw[dashed] (0,\y) -- (4,\y);
%Achsen beschriften
\fill[black] (0,0) circle (0.1) node[below left]{$x_1$};
\foreach[count=\x] \i in {2,3,4,5}
\fill[black] (\x,0) circle (0.1) node[below]{$x_\i$};
\foreach[count=\y] \i in {2,3,4,5}
\fill[black] (0,\y) circle (0.1) node[left]{$x_\i$};

\draw (0,0) circle (0.2);
\draw (1,1) circle (0.2);
\draw (2,2) circle (0.2);
\draw (3,3) circle (0.2);
\draw (4,4) circle (0.2);
\draw (1,3) circle (0.2);
\draw (2,1) circle (0.2);
\draw (4,2) circle (0.2);
\draw (3,0) circle (0.2);

\fill[red] (0,0) circle (0.1);
\fill[red] (1,1) circle (0.1);
\fill[red] (2,2) circle (0.1);
\fill[red] (3,3) circle (0.1);
\fill[red] (4,4) circle (0.1) node[above right]{$I_M$};
\end{tikzpicture}  \\
%
symmetrische Menge
\begin{tikzpicture}[scale=0.5]
% Achsen zeichnen
\draw[thick] (0,0) -- (4,0);
\draw[thick] (0,0) -- (0,4);
\foreach \x in {0,1,2,3,4}
\draw[dashed] (\x,0) -- (\x,4);
\foreach \y in {0,1,2,3,4}
\draw[dashed] (0,\y) -- (4,\y);
%Achsen beschriften
\fill[black] (0,0) circle (0.1) node[below left]{$x_1$};
\foreach[count=\x] \i in {2,3,4,5}
\fill[black] (\x,0) circle (0.1) node[below]{$x_\i$};
\foreach[count=\y] \i in {2,3,4,5}
\fill[black] (0,\y) circle (0.1) node[left]{$x_\i$};

\draw (0,0) circle (0.2);
\draw (1,1) circle (0.2);
\draw (3,3) circle (0.2);
\draw (1,3) circle (0.2);
\draw (3,1) circle (0.2);
\draw (0,4) circle (0.2);
\draw (4,0) circle (0.2);
\draw (1,4) circle (0.2);
\draw (4,1) circle (0.2);
\end{tikzpicture}  \\
%
Projektion Menge
\begin{tikzpicture} [scale = 0.25]
%Grenze
\draw  (0,0) rectangle (10,8);
\draw (10,0) node [below left] {$u$};
\draw (0,8) node [below left] {$v$};
%Gebilde
\fill[pattern=north west lines,pattern color=gray] (3,2) to [out=-90, in=180] (4,1) to [out=0, in=-160] (7,3) to [out=20,in=-90] (8,4) to [out=90,in=0] (6,6) to [out=180, in=80] (5,4) to [out=-100, in=90] (3,2);
\draw (3,2) to [out=-90, in=180] (4,1) to [out=0, in=-160] (7,3) to [out=20,in=-90] (8,4) to [out=90,in=0] (6,6) to [out=180, in=80] (5,4) to [out=-100, in=90] (3,2);
\draw (6.5,4) node{$T$};

%Projektion
\draw[dashed, red] (3,2) -- (3,0);
\draw[dashed, red] (8,4) -- (8,0);
\draw[red] (3,-0.1)--(5.5, -0.1) node[below]{$proj_1(T)$}--(8,-0.1);

\draw[dashed, orange] (4,1) -- (10,1);
\draw[dashed, orange] (6,6) -- (10,6);
\draw[orange] (10.1,1)--(10.1, 3) node[right]{$proj_2(T)$}--(10.1,6);
\end{tikzpicture} \\
%
Projektion Koordinatensystem
\begin{tikzpicture} [scale=0.6]
% Bereich
\draw (0,0) rectangle (4,5);
\draw (0,5) node [above]{Fach};
\draw (4,0) node[right]{Student};
\foreach [count=\i] \x in {0,1,2,3,4}
\draw (\x,-.1) -- (\x,.1) node[below=5pt] {$\i$};
\draw (-.1,0) -- (.1,0) node[left=5pt] {$a$};
\draw (-.1,1) -- (.1,1) node[left=5pt] {$b$};
\draw (-.1,2) -- (.1,2) node[left=5pt] {$c$};
\draw (-.1,3) -- (.1,3) node[left=5pt] {$d$};
\draw (-.1,4) -- (.1,4) node[left=5pt] {$e$};
\draw (-.1,5) -- (.1,5) node[left=5pt] {$f$};

%Punkte
\draw (1,0) circle (0.2);
\draw (1,3) circle (0.2);
\draw (1,4) circle (0.2);

\draw (3,1) circle (0.2);
\draw (3,5) circle (0.2);

\draw (4,1) circle (0.2);
\draw (4,3) circle (0.2);
\draw (4,4) circle (0.2);
\draw (4,5) circle (0.2);

%Projektion
\draw[orange, latex-] (4.2,2.5) -- (5.2,2.5);
\draw[orange] (-0.6,0) circle (0.3);
\draw[orange] (-0.6,1) circle (0.3);
\draw[orange] (-0.6,3) circle (0.3);
\draw[orange] (-0.6,4) circle (0.3);
\draw[orange] (-0.6,5) circle (0.3);
\draw[orange] (-0.6, 2.5) node[left]{$proj_2(P)$};

\draw[red, latex-] (2,5.2) -- (2,6.2);
\draw[red] (1,-0.6) circle (0.3);
\draw[red] (3,-0.6) circle (0.3);
\draw[red] (4,-0.6) circle (0.3);
\draw[red] (2, -0.7) node[below]{$proj_1(P)$};
\end{tikzpicture} \\
%
transitive Hülle
\begin{tikzpicture}[scale=.5]
% Menge T
\draw (0,0) node{a};
\draw[-latex] (0.3,0.3) -- (1.7,1.7);
\draw (2,2) node{b};
\draw[-latex] (2.3,2) -- (3.7,2);
\draw (4,2) node{c};
\draw[-latex] (4.3,2) -- (5.7,2);
\draw (4,0) node{d};
\draw[-latex] (2,1.7) -- (3.7,0.3);
\draw (6,2) node{e};
\draw[-latex] (4.3,0.3) -- (5.7,1.7);
\draw (8,2) node{f};
\draw[-latex] (6.3,2) -- (7.7,2);

\draw [-latex, red] plot[smooth, tension=0.9] coordinates {(0,0.3) (1.3,2.7) (3.7,2.3)};
\draw [-latex, red] plot[smooth, tension=0.9] coordinates {(0.3,0) (2,-0.5) (3.7,0)};
\draw [-latex, red] plot[smooth, tension=0.9] coordinates {(2.3,2.3) (4,2.8) (5.7,2.3)};
\draw [-latex, red] plot[smooth, tension=0.9] coordinates {(4.3,2.3) (6,2.8) (7.7,2.3)};
\draw [-latex, red] plot[smooth, tension=0.9] coordinates {(4.3,0) (6,0.5) (8,1.7)};

\draw [-latex, green] plot[smooth, tension=0.9] coordinates {(0,0.3) (2.3,3) (5.8,2.5)};
\draw [-latex, green] plot[smooth, tension=0.9] coordinates {(2.3,1.7) (5,1.3) (7.7,1.7)};

\draw [-latex, orange] plot[smooth, tension=0.9] coordinates {(0.3,-0.3) (5,-0.7) (8.1,1.7)};
\end{tikzpicture} \\
%
reflexive Hülle
\begin{tikzpicture}[scale=.5]
% Menge T
\draw (0,0) node{a};
\draw[-latex] (0.3,0.3) -- (1.7,1.7);
\draw (2,2) node{b};
\draw[-latex] (2.3,2) -- (3.7,2);
\draw (4,2) node{c};
\draw[-latex] (4.3,2) -- (5.7,2);
\draw (4,0) node{d};
\draw[-latex] (2,1.7) -- (3.7,0.3);
\draw (6,2) node{e};
\draw[-latex] (4.3,0.3) -- (5.7,1.7);
\draw (8,2) node{f};
\draw[-latex] (6.3,2) -- (7.7,2);


\draw [-latex, orange] plot[smooth, tension=1.5] coordinates {(0.2,0.2) (0.5,1) (-0.5,1) (-0.2,0.2)};
\draw [-latex, orange] plot[smooth, tension=1.5] coordinates {(2.2,2.2) (2.5,3) (1.5,3) (1.8,2.2)}; 
\draw [-latex, orange] plot[smooth, tension=1.5] coordinates {(4.2,2.2) (4.5,3) (3.5,3) (3.8,2.2)};\draw [-latex, orange] plot[smooth, tension=1.5] coordinates {(4.2,0.2) (4.5,1) (3.5,1) (3.8,0.2)};
\draw [-latex, orange] plot[smooth, tension=1.5] coordinates {(6.2,2.2) (6.5,3) (5.5,3) (5.8,2.2)};
\draw [-latex, orange] plot[smooth, tension=1.5] coordinates {(8.2,2.2) (8.5,3) (7.5,3) (7.8,2.2)};
\end{tikzpicture} \\
%
symmetrische Hülle
\begin{tikzpicture}[scale=.5]
% Menge T
\draw (0,0) node{a};
\draw[-latex] (0.3,0.3) -- (1.7,1.7);
\draw (2,2) node{b};
\draw[-latex] (2.3,2) -- (3.7,2);
\draw (4,2) node{c};
\draw[-latex] (4.3,2) -- (5.7,2);
\draw (4,0) node{d};
\draw[-latex] (2,1.7) -- (3.7,0.3);
\draw (6,2) node{e};
\draw[-latex] (4.3,0.3) -- (5.7,1.7);
\draw (8,2) node{f};
\draw[-latex] (6.3,2) -- (7.7,2);

\draw[latex-, orange] (0.2,0.5) -- (1.6,2);
\draw[latex-, orange] (2.2,2.2) -- (3.6,2.2);
\draw[latex-, orange] (4.2,2.2) -- (5.6,2.2);
\draw[latex-, orange] (1.9,1.5) -- (3.3,0.3);
\draw[latex-, orange] (4.6,0.3) -- (5.9,1.5);
\draw[latex-, orange] (6.2,2.2) -- (7.6,2.2);
\end{tikzpicture} \\
%


\section{HTW Programmierung 1}
PAP
\begin{tikzpicture}[scale=0.3]
\draw (-3,6) ellipse (3 and 1) node{Start};
\draw[-latex] (-3,5) -> (-3,3);
\draw  (-7,3) rectangle (1,1) node[pos =.5]{Akk:=0};
\draw[-latex] (-3,1) -> (-3,-1);
\draw  (-7,-1) rectangle (1,-3) node[pos =.5]{Count:=0};
\draw[-latex] (-3,-3) -- (-3,-5);
\draw (-3,-5) -- (-10,-11) -- (-3,-17) -- (4,-11) -- cycle;
\draw  (-3,-11) node[align=center] {Gibt es eine\\ gültige Note an der\\ Stelle Count?};

\draw[-latex] (-3,-17) -- (-3,-19);
\draw (-2,-18) node{ja};
\draw  (-13,-19) rectangle (8,-21) node[pos =.5]{Akku := Akku + Noten[Count]};
\draw[-latex] (-3,-21) -- (-3,-23);
\draw (-10,-23) rectangle (5,-25) node[pos =.5]{Count := Count +1};
\draw[-latex] (-3,-25) -- (-3,-27) -- (10,-27) -- (10, -4) -- (-3,-4);

\draw[-latex] (-10,-11) -- (-19,-11) -- (-19,-13);
\draw (-19,-13) -- (-24,-17) -- (-19,-21) -- (-14,-17) -- cycle;
\draw  (-19,-17) node[align=center] {Count > 0?};
\draw[-latex] (-14,-17) -- (-12,-17);
\draw (-13,-16) node{nein};
\draw (-9,-17) ellipse (3 and 1) node {Fehler};
\draw[-latex] (-19,-21) -- (-19,-23);
\draw (-18,-22) node{ja};
\draw  (-27,-23) rectangle (-11,-25) node[pos =.5]{Akku := Akku/Count};
\draw[-latex] (-19,-25) -- (-19,-27);
\draw (-26,-27) -- (-10,-27) -- (-11,-29) -- (-27,-29) -- cycle;
\draw (-19,-28) node{Ausgabe Akku};
\draw[-latex] (-19,-29) -- (-19,-31);
\draw (-19,-32) ellipse (3 and 1) node{Erfolg};
\end{tikzpicture} \\
%
Struktogramm
\begin{tikzpicture} [scale = 0.3]
\draw  (0,0) rectangle (28,2);
\draw (0,1) node[anchor=west]{Akku = 0}; 
\draw  (0,-6) rectangle (28,0);
\draw (0,-1) node[anchor=west]{while (Noten[Count]!=0)};
\draw  (4,-4) rectangle (28,-2);
\draw (4,-3) node[anchor=west]{Akku = Akku + Noten[Count]}; 
\draw  (4,-6) rectangle (28,-4);
\draw (4,-5) node[anchor=west]{Count :=  Count + 1}; 
\draw  (0,-8) rectangle (28,-6);
\draw (0,-7) node[anchor=west]{if (Count>0)}; 
\draw  (0,-12) rectangle (28,-8);
\draw (0,-11) node[anchor=west]{ja}; 
\draw (28,-11) node[anchor=east]{nein};
\draw (0,-8) -- (14,-12) -- (28,-8);
\draw (0,-12) rectangle (14,-16);
\draw (0,-14) node[anchor=west, align=left]{Akku = Akku/Count\\ Ausgabe Akku}; 
\draw (14,-12) rectangle (28,-16);
\draw (14,-14) node[anchor=west, align=left]{Fehler}; 
\end{tikzpicture}  \\
%

\section{Allgemein}

Koordinatensystem
\begin{tikzpicture}[scale=0.5]
\def \xa {-1};
\def \xb {1};
\def \ya {0};
\def \yb {1};
% Achsen zeichnen
\draw[-latex,thick] (\xa-0.5,0) -- (\xb+0.5,0) node[right] {$x$};
\draw[-latex,thick] (0,\ya-.2) -- (0,\yb+0.5) node[above right] {$y$};
% Achsen beschriften
\foreach \x in {\xa,...,\xb}
\draw (\x,-.1) -- (\x,.1) node[below=5pt] {$\scriptstyle\x$};
\foreach \y in {1,...,\yb}
\draw (-.1,\y) -- (.1,\y) node[left=5pt] {$\scriptstyle\y$};
\end{tikzpicture} 

%\newpage
%\printbibliography
\end{document}